\lecture{39}{Mon 29 Nov 2021 11:29}{Polynomials (5)}
\begin{recall}
	We characterized the prime elements of \(R\left[ x \right] \) for some UFD \(R\) . Next, we show the final part of the theorem, that \(R\) is a UFD implies \(R\left[ x \right] \) is a UFD.
\end{recall}
\begin{proof}
	Let \(f \in R\left[ x \right] \) be nonzero. Clearly, \(f \in K\left[ x \right] \) with \(f = \cnt\left( f \right) \left( \prod_{i= 1}^{n} g_{i}  \right) \) where \(g_{i} \in K\left[ x \right] \) are irreducible polynomials. But, since \(R\) is a UFD, we can factor \(\cnt\left( f \right) \) into primes from \(R\). We know this factorization to also be primes in \(R\left[ x \right] \). Hence \(f\) can be factorized as the factorization of its content times a product of irreducible polynomials inn \(K\left[ x \right] \) which are also prime.\\
	Lastly, we need to show this factorization unique. This is essentially trivial as \(\cnt\left( f \right) \in R\) and \(\prod_{i= 1}^{n} g_{i} \in K\left[ x \right]  \), a UFD, so we see any factorization in \(R\left[ x \right] \) is the product of these unique factorizations, so it is unique.
\end{proof}
The converse can be proved directly by examining only constant polynomials.\\
Unfortunately, this conclusion does not extend to PIDs as we have already shown. However, we can extend this to multivariate polynomial rings to yield the following generalization.
\begin{corollary}
	If \(R\) is a UFD, then \(R\left[ x_1, \ldots, x_{n} \right] \) is a UFD.
\end{corollary}

Next class we will prove a few more theorems/methods about polynomials such as the rational root theorem, eisenstein criterion, and reduction of coefficients, and then review for the final.
