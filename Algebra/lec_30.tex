\lecture{30}{Wed 03 Nov 2021 11:32}{Ring Theory (5)}
Again, we suppose \(R\) to be commutative unless otherwise stated.
\begin{proposition}
	If \(R\) is an integral domain with \(p \in R\) being prime, then \(p\) is irreducible.
\end{proposition}
\begin{proof}
	We know \(p\) is nonzero and a non-unit. Then, suppose \(p = xy\)  \(x, y \in R\). Since \(p\) prime, we see \(p \mid xy\) implies \(p \mid x\) or \(p \mid y\). WLOG, suppose \(p \mid x\), then \(x \in\left( p \right) \) , so \(x = rp\) for an \(r \in R\). Then, we see \[
		p = xy = \left( rp \right) y = \left( ry \right) p
	.\]
	Canceling \(p\) yields \(1 = ry\), so \(y\) is a unit. Hence, \(p\) is irreducible.
\end{proof}
\begin{remark}
	Here are a few basic facts about principal ideals, prime ideals, etc. we have shown, compiled together:
	\begin{itemize}
		\item \(x \mid y \iff y \in \left( x \right) = Rx\).
		\item \(x \mid y\) and \(y \mid x \iff \left( x \right)  = \left( y \right) \).
		\item  If \(R\) is an integral domain with \(x \neq 0\) then \(\left( x \right)  = \left( y \right) \iff ux = y\) for a unit \(u\).
		\item \(\left( x \right)  = R \iff x\) is a unit.
		\item \(p \in R\) is prime implies \(\left( p \right)\)  is a prime ideal.
		\item \(\left( p \right) \) is a prime ideal and \(p \neq 0\) implies \(p \in R\) is prime.
		\item \(p \in R\) irreducible implies \(\left( p \right) \)  is maximal among all proper principal ideals.
		\item If \(R\) is an integral domain and \(p \neq 0\) , then \(\left( p \right) \subset R\) is maximal among principal ideals \(\iff p \in R\) is irreducible.
		\item If \(R\) is an integral domain with \(p \in R\) being prime then \(p\) is also irreducible.
	\end{itemize}
\end{remark}
\begin{definition}[Factorization]
If \(R\) is a commutative ring, a \textbf{factorization} of an element \(x \in R\) is an expression \[
x = u \prod_{i= 1}^{n} y_{i}
\] where \(u\) is a unit and \(y_1, \ldots, y_{n}\) are irreducibles.\\
The factorization is a \textbf{unique factorization} if for a second factorization \[
x = u^{\prime} \prod_{i= 1}^{n^{\prime}} y^{\prime}_{i}
\] we find \(n = n^{\prime}\) and there exists a permutation \(\pi\) of \(\{1, \ldots, n\} \) such that \(y_{\pi\left( i \right) }  =y^{\prime}_{i}\) up to units for all \(1 \le y \le n\).
\end{definition}
\begin{definition}[Unique Factorization Domain]
	A commutative ring \(R\) that is an integral domain in which every nonzero \(x \in R\) 	has a unique factorization is called a \textbf{Unique Factorization Domain (UFD)}.
\end{definition}

\begin{theorem}
	If \(R \) is a UFD, then  \(p \in R\)   is prime if and only if \(p\) is irreducible.
\end{theorem}
\begin{proof}
	Since \(R\) is a UFD, it is an integral domain, hence a prime is irreducible.\\
	Now, let \(p\) be irreducible, so \(p \neq 0\) and \(p\) is a non-unit. Suppose \(p \mid xy\) for some \(x,y \in R\). Then, we see \(xy = rp\) for some \(r \in R\), hence letting
	\begin{align*}
		x &= u_1 \prod_{i= 1}^{n} x_{i}   \\
		y&= u_2 \prod_{i= 1}^{m} y_{i}
	\end{align*} be the unique factorizations for \(x\) and \(y\) respectively yields a factorization \[
	xy = u_3 \prod_{i= 1}^{n} x_{i} \prod_{i= 1}^{m} y_{i}
	.\]
	Hence, \[
	rp = rxy = u_3 \prod_{i= 1}^{n} x_{i} \prod_{i= 1}^{m} y_{i} \cdot r
	.\]
	Hence, we find \[
	u_3 \prod_{i= 1}^{n} x_{i} \prod_{i= 1}^{m} y_{i}  \cdot r = r \cdot p
	.\]
	Hence, cancelling \(r\),  we must have \(p= x_{j}\) or \(y_{k}\) for some \(1 \le j \le n\) or \(1 \le k \le m\) as it is irreducible. So, \(p \mid x\) or \(p \mid y\), hence \(p\) is prime.
\end{proof}
It is of note that a factorization can contain multiple copies of a particular irreducible. Hence, we can also represent a factorization as a multi-set. That is, if \(x = u p_1^{\alpha_1} \ldots p_{n}^{\alpha_{n}}\), we can represent this as the multi-set \[
	\fac\left( x \right)  = \{\underbrace{p_1, \ldots, p_1}_{\alpha_1 \text{times}} , \underbrace{p_2, \ldots, p_2}_{\alpha_2 \text{times}},  \ldots, \underbrace{p_{n}, \ldots, p_{n}}_{\alpha_{n} \text{times}}  \}
.\]
Then, we can view the factorization of a product \(xy\) as the union of their respective factorization multisets, \(\fac\left( x \right) \cup \fac\left( y \right) = \fac\left( xy \right)  \).
\begin{definition}[Finitely Generated]
	An ideal \(I\)  is finitely generated if \(I = \left( x_1, x_2, \ldots, x_{n} \right) \) for a finite set \(\{x_1, x_2, \ldots, x_{n}\} \).
\end{definition}
\begin{definition}[Noetherian Ring]
	A commutative ring is \textbf{Noetherian} if it satisfies the \textbf{ascending chain condition (a.c.c.)} on ideals. That is, if \(I_1 \subseteq I_2 \subseteq I_3 \subseteq \ldots\) is an ascending chain for some ideals \(I_1, I_2, \ldots\), then there exists a \(m \ge 1\) such that \(I_{i} = I_{m}\) for all \(i \ge m\).\\
	More simply, a ring is Noetherian if all properly ascending chains of ideals are finite in lengths.
\end{definition}
This definition is rather clunky, so the following characterization is the more standard use case:
\begin{theorem}
\(R\) is a noetherian ring if and only if all ideals in \(R\) are finitely generated.
\end{theorem}
\begin{remark}
	A Noetherian ring which is also an integral domain is sometimes called a \textbf{Noetherian Domain}.
\end{remark}
Noetherian domains are a weaker class of rings than principal ideal domains, but they are more "resiliant" to algebraic operations. That is, most algebraic operations preserve Noetherian-ness even if they do not preserve the PID property.
