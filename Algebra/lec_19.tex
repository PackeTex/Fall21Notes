\lecture{19}{Wed 06 Oct 2021 11:33}{Free Groups (2)}
Recall we had a set of letters \(X = \{a, b, c, \ldots, a^{-1}, b^{-1}, c^{-1}, \ldots,  1 \}\). Then, we define a word on the alphabet \(X\) to be a string \(\omega = x_1^{\epsilon_1} x_2^{\epsilon_2} \ldots, x_{s}^{\epsilon_{s}}\) where \(x_1, x_2, \ldots, x_{s} \in X\) and \(\epsilon _{i} = \pm 1\). For example with \(X = \{a, b, c\} \) we have a word \(x_1 x_1 x_2 x_1^{-1} x_1 x_3\) for example. Then, define \(1\) to be the empty product, that being a string with no symbols. Now, we define an equivalence relation on the words to induce a group.\\
We say two words \(\omega_1 \sim \omega_2\) if we can transform \(\omega_1\) into \(\omega_2\) with a finite sequence of the following operations
\begin{itemize}
	\item Remove a sequential pair \(x x^{-1}\) or \(x^{-1} x\) from the string.
	\item Insert a substring \(x x^{-1}\) or \(x^{-1} x\) into the string.
\end{itemize}
So, we see \(x_1x_2x_3^{-1}x_4 \sim x_1x_2x_3^{-1}x_2x_2^{-1} x_1^{-1} x_1 x_4\) and so on. It is trivial to verify this to be an equivalence relation, so we omit the proof. Henceforth, we will denote the equivalence class of a word \(\omega\) by \(\left[ \omega \right] \). So, we see if \( \omega_1 \sim \omega_2\), we have \(\left[ \omega_1 \right]  = \left[ \omega_2 \right] \).\\
Now, let \(F\left( X \right) \) be the set of all equivalence classes on \(X\) and define \(\left[ \omega_1 \right] \left[ \omega_2 \right]  \coloneqq \left[ \omega_1\omega_2 \right]  \) with \(\omega_1 \omega_2\) simply being the concatenation of the two words. First, we verify this to be well-defined. Suppose \(w^{\prime} \sim w\) and \(v^{\prime} \sim v\) are \(4\) words. Hence, there is a simple sequence taking \(v \mapsto v^{\prime}\) and \(w \mapsto w^{\prime}\). It is easy to see then, that the same operations applied to their respective parts will take \(vw \mapsto v^{\prime} w^{\prime}\) and \(wv \mapsto w^{\prime} v^{\prime}\), hence \(\left[ vw \right]  = \left[ v^{\prime} w^{\prime} \right] \).
