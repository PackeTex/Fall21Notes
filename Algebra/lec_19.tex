\lecture{19}{Wed 06 Oct 2021 11:33}{Free Groups (2)}
Recall we had a set of letters \(X = \{a, b, c, \ldots, a^{-1}, b^{-1}, c^{-1}, \ldots,  1 \}\). Then, we define a word on the alphabet \(X\) to be a string \(\omega = x_1^{\epsilon_1} x_2^{\epsilon_2} \ldots, x_{s}^{\epsilon_{s}}\) where \(x_1, x_2, \ldots, x_{s} \in X\) and \(\epsilon _{i} = \pm 1\). For example with \(X = \{a, b, c\} \) we have a word \(x_1 x_1 x_2 x_1^{-1} x_1 x_3\) for example. Then, define \(1\) to be the empty product, that being a string with no symbols. Now, we define an equivalence relation on the words to induce a group.\\
We say two words \(\omega_1 \sim \omega_2\) if we can transform \(\omega_1\) into \(\omega_2\) with a finite sequence of the following operations
\begin{itemize}
	\item Remove a sequential pair \(x x^{-1}\) or \(x^{-1} x\) from the string.
	\item Insert a substring \(x x^{-1}\) or \(x^{-1} x\) into the string.
\end{itemize}
So, we see \(x_1x_2x_3^{-1}x_4 \sim x_1x_2x_3^{-1}x_2x_2^{-1} x_1^{-1} x_1 x_4\) and so on. It is trivial to verify this to be an equivalence relation, so we omit the proof. Henceforth, we will denote the equivalence class of a word \(\omega\) by \(\left[ \omega \right] \).
