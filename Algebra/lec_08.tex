\lecture{8}{Fri 10 Sep 2021 11:23}{Sylow Groups (3)}
\begin{recall}
	We proved Sylow's \(2\)nd theorem, that every \(p\)-group in \(G\) is contained within some \(p\)-group.
\end{recall}
\begin{proof}[3rd and 4th theorems]
	\begin{itemize}
		\item[3.] Recall we let \(G\) act on \(\Omega\), being the set of all subgroups conjugate to \(P\), be conjugation and we shoed any \(p\)-group \(P^{\prime} \le G\) has some \(p^{\prime}^{\prime} \in \Omega\) such that \(p^{\prime} \le p^{\prime}^{\prime}\).
			\\ Now, let \(p^{\prime} \) be an arbitrary sylow \(p\)-group. By the above we have the existence of a \(p^{\prime}^{\prime} \in \Omega\) such that \(p^{\prime} \le p^{\prime}^{\prime}\). But \(\left| p^{\prime}  \right| = \left| P^{\prime}^{\prime} \right|  = p^{n} \) as this is the maximum power of \(p\) dividing \(\left| G \right| \) by definition of sylow groups.\\
			Hence \(p^{\prime} = p^{\prime}^{\prime} \in \Omega\), so \(p^{\prime}\) is conjugate. Hence, every sylow \(p\)-group is conjugate to the fixed sylow \(p\)-group so they are all conjugate by transitivity.
		\item[4] Now that we know all sylow \(p\)-groups are conjugate, we know there is a \(n_{p} = \left| \Omega \right| \) with \(\Omega\) being a single orbit in the action of \(G\) on \(\Omega\).
			So, the orbit stabilizer lemma yields \begin{align*}
			n_{p} = \left| \Omega \right| &= \left| G : G_{p} \right| \) where \(G_{P} = \{x \in G: P^{x} = P\}
&= \{x\in G : xPx^{-1} = P\}  \\
&= N_{G} \left( P \right)  .
		\end{align*}
		Now, we restrict the action of \(G\) on \(\Omega\) to an action of \(P\) on \(\Omega\). Hence, \(P\) is a \(p\)-group, hence finite, acting on the finite set \(\Omega\). And, as we know the number of fixed points \(n_{p} = \left| \Omega \right| \left( \mod p \right) \).\\
		So, we must only examine the fixed points now. Let \(P^{\prime} \in \Omega\) be an arbitrary subgroup such that \(P^{\prime}\) is fixed by all \(x \in P\). That is, \(xP^{\prime} x^{-1} = P^{\prime}\). If \(P^{\prime} = P\) this is clearly true. By definition, we know \(P \in N_{G}\left( P^{\prime} \right) \), but by an earlier lemma, we know that \(P \le P^{\prime}\), both were \(p\)-groups of maximal cardinality so both sylow groups are equal. Hence, \(P^{\prime} = P\) is the only fixed point. This completes the proof as \(n_{p} \equiv 1 \left( \mod p \right) \).
	\end{itemize}
\end{proof}
\begin{theorem}
	Let \(G\) be a group with \(\left| G \right|  = p^2\) with \(p\) being prime. Then, \(G\) is abelian.
\end{theorem}
\begin{remark}
	This is a generalization of the theorem that every group of order \(p\) is cyclic, hence abelian.
\end{remark}
\begin{lemma}
	If \(G\) is a finite nontrivial \(p\)-group, then \(Z\left( G \right) \) is nontrivial.
\end{lemma}
\begin{proof}[Proof of lemma]
	By the class equation
	\[	\left| G \right|  &= \left| Z\left( G \right)  \right|  +\sum_{x \in I}^{} \left| G : Z_{G}\left( x \right)  \right| .\]
	But, as each \(Z_{G}\left( x \right) \) with \(x \in I\) has \(p \mid \left|Z_{G}\left( x \right) \right| \) hence \(p\mid \left| Z\left( G \right)  \right| \). We have actually already argued this same fact before, so the details are omitted. Hence, as \( p \mid \left| Z\left( G \right)  \right| \), then \(Z\left( G \right) \) is nontrivial.
\end{proof}
\begin{proof}[Proof of theorem]
	\(Z\left( G \right) \) is nontrivial by the lemma, hence \(\left| Z\left( G \right)  \right|  = p\) or \(\left| Z\left( G \right)  \right|  = p^2\) by lagrange's theorem. In the second case \(G\) is abelian hence we need only examine the case \(\left| Z\left( G \right)  \right| = p\). As groups of order \(p\) are cyclic, any nonidentity element \(x \in Z\left( G \right) \) will be a generator. Now, we know \(Z\left( G \right) \trianglelefteq G\) and \(\left| G / Z\left( G \right)  \right| = \frac{p^2}{p} = p\), so \(G / Z\left( G \right) \) is also a group of order \(p\), let it be generated by \(xZ\left( G \right) \), where \(x\in G\). Then, \(G = \left<Z\left( G \right) , x \right> \). So, any arbitrary element of \(g\) is a product \(xy\) with \(y \in Z\left( G \right) \), and as \(x\) commutes with everything in \(Z\left( G \right) \), we have \(xy=yx\).
\end{proof}
\begin{theorem}
	Suppose \(G\) is a group and \(\left| G \right|  = pq\) for distinct primes \(p < q\) with \(p \nmid q-1\). Then, \(G\) is abelian.
\end{theorem}
\begin{proof}
	Let \(P, Q\) be sylow \(p\)-groups and \(q\)-groups respectively. Let \(n_{p}\) to be the number of sylow \(p\)-groups in \(G\) and similarly for \(n_{q}\). By sylow's theorems, we know \(n_{p \mid \frac{\left| G \right| }{p}}\). So, \(n_{p} = 1\) or \(q\) and \(n_{p} \equiv 1 \left( \mod p \right) \). If \(n_{p} \equiv q \equiv 1 \left( \mod p \right) \), this is a contradiction as \(p | q-1\). \(\lightning\) Hence, \(n_{p}= 1\). Likewise, \(n_{q} | \frac{\left| G \right| }{q} = p\), so \(q \equiv 1\) or \(p \left( \mod p \right) \) and if \(n_{q} \equiv p \equiv 1 \left( \mod q \right)  = 1\), then \(p = xq + 1\) for some positive \(x\), hence \(p \ge q + 1\). \( \lightning\). So \(n_{q} = 1\).\\
	This means every \(g \in G\) fixes the unique sylow \(q\)-grop \(Q\) by conjugation (\(gQg^{-1} = Q\)), hence \(Q\trianglelefteq G\) and likewise \(P\trianglelefteq G\).\\
	Consider the subgroup \(PQ\). Since \(P, Q\) are normal \(P \le N_{G}\left( Q \right)  = G\) and \(Q \le N_{G} \left( P \right)  = G\), so \(PQ\) is a subgroup by the \(2\)nd homomorphism theorem. Furthermore, \(\left| P \right|  \mid \left| PQ \right| \) and \(\left| Q \right|  \mid \left| PQ \right| \). Hence, \(pq \mid \left| PQ \right| \le pq\). Thus, \(PQ = G\).\\
	Now, \(\left| P \right|  = P\), so \(P = \left<x \right> \) for some \(x \in G\) and \(\left| Q \right| = q\), so \(Q = \left<y \right> \) for some \(y \in C\). As \(p, q\) are prime these groups are cyclic hence abelan. So, we need only show \(xy=yx\). We see \(yxy^{-1} = x^{\prime} \in P\)  as \(P \trianglelefteq G\). Hence, \(yx=x^{\prime}y = y^{\prime}x^{\prime}\) for some \(y^{\prime} \in Q\) as \(Q \trianglelefteq G\). As \(PQ = G\) with \(\left| P \right| = p\) and \(\left| Q \right| = q\), hence \(\left| G \right|  = pq\) so each element \(x \in G\) has a unique expression \(x = ab\) with \(a \in P\) and \(b \in Q\). Hence \(x = x^{\prime}\) and \(y = y^{\prime}\), so \(xy=yx\).
\end{proof}
\begin{remark}
	It is a general technique that if a sylow group is unique, it is normal in \(G\).
\end{remark}
