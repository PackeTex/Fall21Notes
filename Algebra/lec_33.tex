\section{Ring Localization}
\lecture{33}{Wed 10 Nov 2021 17:33}{Localization of Rings}
\begin{recall}
	Recall \(R\) denotes a commutative ing. If \(S \subseteq R\) is a multiplicative subset, we see \(x, y \in S\) implies \(xy\in S\) and \(0 \not\in S\) but \(1 \in S\).\\
	Then, we define \(S^{-1} R = \{X / s : x\in R, s \in S\} \). Then, we see \(\frac{x_1}{s_1} = \frac{x_2}{s_2}\) if and only if there is an \(s \in S\) so that \(s\left( s_2 x_1 - s_1 x_2 \right) = 0\). Of course, if \(R\) is an integral domain we see this iplies \(s_2 x_1 - s_1 x_2 = 0\), the normal definition of fraction equality.
\end{recall}
Now, we turn this set into a ring. We define \(\frac{x_1}{s_1}\cdot \frac{x_2}{s_2} \coloneqq \frac{x_1x_2}{s_1s_2}\) and \(\frac{x_1}{s_2} + \frac{x_2}{s_2} = \frac{s_2x_1}{s_1s_2} + \frac{s_1 x_2}{s_1 s_2} = \frac{s_2x_1 + s_1x_2}{s_1 s_2}\). Now, we need to show that \(+, \cdot\) are well defined (meaining they do not vary for different representatives of a given equivalence class). This fact is easily checked by symbolic manipulation so we omit the proof. For the addition case suppose \(\frac{x_1}{s_1} = \frac{x_1^{\prime}}{s_1^{\prime}}\) and similairly for \(\frac{x_2}{s_2}\) then take the multiplicative representation of the fraction and multiply the \(\frac{x_1}{s_1}\) representation by \(-s_2 s_2^{\prime} t s\) and the \(\frac{x_2}{s_2}\) representation by \(-s_1 s_1^{\prime} s t\) and by adding together these representations we see terms cancel and we obtain that addition is in fact well defined. Moreover, it is trivial to check that the ring axioms hold.
\begin{definition}[Ring Localization]
We denote this new fraction ring \(S^{-1} R\) to be the \textbf{localization of \(R\) } with additive identity \(\frac{0}{1}\), multiplicative identity \(\frac{1}{1}\) and \(\frac{tx}{ts}= \frac{x}{s}\) for all \(t \in S\).
\end{definition}
Note that \(s \in S\) is nonzero by definition, so \(\frac{1}{s} \cdot \frac{s}{1} = \frac{1}{1} = 1_{S^{-1} R}\) , so every element has an inverse.
\begin{proposition}
	If \(R\) is a commutative ring with \(S \subseteq R\) being a multplicative subset. Then the map \begin{align*}
		\pi: R &\longrightarrow S^{-1}R \\
		x &\longmapsto \pi(x) = \frac{x}{1}
	\end{align*} is a ring homomorphism. Moreover, if \(S\) has no zero-divisors, then \(\pi\) is an injection.
\end{proposition}
\begin{proof}
	If \(x, y \in R\) then \(\pi\left( x \pm y \right) = \frac{x \pm y}{1} = \frac{x}{1} \pm \frac{y}{1} = \pi\left( x \right) \pm \pi\left( y \right) \). Furthermore \(\pi\left( 1 \right) = \frac{1}{1} = 1\).\\
	Lastly, \(\pi\left( xy \right) = \frac{xy}{1} = \frac{x}{1}\frac{y}{1} = \pi\left( x \right) \pi\left( y \right) \). Hence, \(\pi\) is a ring homomorphism. Now consider \(\ker \left( \pi \right) = \{x \in R : \frac{x}{1} = \frac{0}{1}\} \). We see this implies an \(s \in S\) so that \(s\left( 1x - 1\cdot 0 \right) = sx = 0 \), hence \(s\) is a zero divisor if \(x \neq 0\) \(\lightning\). So, the kernel is trivial.
\end{proof}
\begin{example}
	If \(R\) is a commutative ring and \(P \subseteq R\) is a prime ideal, then \(S \coloneqq R \setminus P\) is a multiplicative set. Moreover, \(0 \in P\) so \( 0 \not\in S\)  and \(P \subset R\) is proper, so \(1 \in S\).\\
	If \(x, y \in S\) with \(xy \not\in S\), then \(xy \in P\) so \(x \in P\) or \(y \in P\) \(\lightning\). So, \(S\) is closed under multiplication. Then the localization \(S^{-1} R\) is often denoted \(R_{P}\). This is the canonical example of localization which we will study more next class.
\end{example}
The use of this construction is that it allows us to embed an integral domain \(R\) in a field \(R_{\left( 0 \right) }\) called the \textbf{field of fractions}.
