\lecture{15}{Tue 28 Sep 2021 17:46}{Nilpotent Groups (2)}
\begin{lemma}
	If \(H, K\) are groups, then \(Z\left( H \times K \right) = Z\left( H \right)  \times Z\left( K \right) \).
\end{lemma}
\begin{proof}
	Let \(\left( x, y \right) \in H \times K\). If \(\left( x, y  \right) \in Z\left( H \times K \right) \) then \[
		\underbrace{\left( a, 1 \right) \left( x, y \right) \left( a, 1 \right) ^{-1}}_{= \left( axa^{-1}, 1 \right) } = \left( x,y \right)
	.\]
	Hence, \(x \in Z\left( H \right) \) and similarly, \(y \in Z\left( K \right) \). Hence, \(Z\left( H \times K \right) \subseteq Z\left( H \right) \times Z\left( K \right) \). The other direction of inclusion is trivial and left as an exercise.
\end{proof}
\begin{lemma}
	Let \(\phi: G \to G^{\prime}\) be a homomorphism with \(\ker \left( \phi \right) = K\) and \(H \le G\) such that \(K \le H\). Then, \(N_{G}\left( H \right) = f^{-1}\left( N_{G^{\prime}}\left( \phi\left( H \right)  \right)  \right) \).
\end{lemma}
\begin{proof}
Let \(x \in N_{G}\left( H \right) \), so \(xHx^{-1} = H\). Hence, \[\phi\left( H \right)  = \phi\left( xHx^{-1} \right) = \phi\left( x \right)  \phi\left( H \right)  \phi\left( x \right) ^{-1}.\]
Thus, \begin{align*}\phi\left( x \right) &\in N_{G^{\prime}}\left( \phi\left( H \right)  \right)\\
	\implies x &\in \phi^{-1}\left( N_{G^{\prime}}\left( \phi\left( H \right)  \right)  \right) \\
	\implies N_{G}\left( H \right) &\subseteq \phi^{-1}\left( N_{G^{\prime}}\left( \phi\left( H \right)  \right)  \right) .
\end{align*}
Conversely, let \(x \in \phi^{-1}\left( N_{G^{\prime}}\left( \phi\left( H \right)  \right)  \right) \), hence \(\phi\left( x \right) \in N_{G^{\prime}}\left( \phi(H) \right) \).\\
Then, we see \begin{align*}
\phi\left( H \right) &= \phi\left( x \right) \phi\left( H \right) \phi\left( x^{-1} \right)\\
		     &= \phi\left( xHx^{-1} \right)\\
		     \implies xHx^{-1} &\subseteq \phi^{-1}\left( \phi\left( H \right)   \right)\\
				       &= \left< H, \ker \left( \phi   \right) \right>\\
	     &=  H \text{ as \(\ker \left( \phi \right) \subseteq H\)}.\end{align*}
	     Hence, \(xHx^{-1} \subseteq H\), so \(x \in N_{G}\left( H \right) \). This concludes the proof.
\end{proof}
Now, recall that if \(G\) is a finite group with \(P\) being a sylow \(p\)-group, then TFAE
\begin{enumerate}
	\item P is unique.
	\item \(P \trianglelefteq G\).
	\item \(P\) is characteristic.
	\item Any subgroup generated by elements whose orders are powers of \(p\) is itself a \(p\)-group.
\end{enumerate}
\begin{theorem}
	If \(G\) is a finite group, then the following are equivalent:
	\begin{enumerate}
		\item \(G\) is nilpotent.
		\item \(H < G \implies H < N_{G}\left( H \right) \).
		\item All sylow \(p\)-groups are normal.
		\item \(G\) is the direct product of its sylow \(p\)-groups.
	\end{enumerate}
\end{theorem}
\begin{proof}
	\begin{itemize}
		\item \(\left( 2\implies 3 \right) \). Let \(P\) be a sylow \(p\)-group of \(G\). Assume \(P\) is not normal, then denote \(N = N_{G}\left( P \right) \subset G\). Hence, by the preceding lemma, \(P\) is characteristic in \(N\). Then, as \(N \trianglelefteq N_{G}\left( N \right) \), we see \(P \trianglelefteq N_G\left( N \right) \). But \(N = N_{G}\left( P \right) \) was the largest subgroup in which \(P\) was normal, hence \(N_{G}\left( P \right) = N_{G}\left( N \right) \). So, by contrapositive of the assumption, \((2)\), we have \(N  = N_{G}\left( N \right) \), so \(N = G\), hence \(P \trianglelefteq G\).
		\item \(\left( 3 \implies 4 \right) \).
		\item \(\left( 1\implies 2 \right) \). Let \(G\) be nilpotent. If \(G\) is abelian, then \(N_{G}\left( A \right)  = G\) for all \(A\le G\), hence any proper subgroup \(H < G\) has \( H < N_{G}\left( H \right)  = G\). Hence, assume \(G\) is non-abelian and proceed by induction on \(\left| G \right| \) with base case \(\left| G \right|  = p\) being already completed \(p\)-prime. Suppose indirectly that there is  an \(H < G\) such that \(H =  N_{G}\left( H \right) \).\\
			Now, we note that \(Z\left( G \right) \le N_{G}\left( H \right) = H \) by definition of \(Z\left( G \right) \). That is, \(Z\left( G \right)  \le H\). Let \(\phi: G \to G / Z\left( G \right) , \ x \mapsto \phi(x) = xZ\left( G \right) \). Since \(G\) is nilpotent, \(Z\left( G \right)  = 1 \iff G = 1\), but we assumed \(G\) to be nonabelian, so this is not the case. Hence, we can assume \(Z\left( G \right)  = \{1\} \), hence \(\left| G / Z\left( G \right)  \right|  < \left| G \right| \).\\
			As we know, \(G\) being nilpotent implies \(G / Z\left( G \right) \) is nilpotent. Lastly, we note that \(Z\left( G \right)  \le H < G\), so by the lattice theorem, we have \(H / Z\left( G \right)  < G / Z\left( G \right) \). Applying the induction hypothesis yields \(H / Z\left( G \right)  < N_{G / Z\left( G \right) }\left( H / Z\left( G \right)  \right) \). Recalling the lemma from last class, \(\phi^{-1}\left( N_{G / Z\left( G \right) }\left( H / Z\left( G \right)  \right)  \right) = N_{G}\left( H \right)  \). Then, we note \[
				\phi^{-1} \left( \phi\left( H \right)  \right)  < \phi^{-1}\left( N_{\phi\left( G \right) }\left( \phi\left( H \right)  \right)  \right) = N_{G}\left( H \right)
			.\]  And as \(\ker \left( \phi \right) = Z\left( G \right) \le H\), we have \(H < N_{G}\left( H \right) \).
	\end{itemize}


\end{proof}
