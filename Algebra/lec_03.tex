\lecture{3}{Fri 27 Aug 2021 11:31}{Group Actions}
\section{Group Actions}
\begin{recall}[The Lattice Theorem]
	Recall that if \(\phi: G \to G^{\prime}\) is a surjective homomorphism, then there is a bijective correspondance between subgroups of \(G\) which contain \(\ker \left( \phi \right) \) and subgroups of \(G^{\prime}\) which preserves normality.
\end{recall}
\begin{definition}[Permuatation Group]
	Recall \[\perm \left( \Omega \right) = \{f :  f: \Omega \to \Omega \text{ such that f is a bijection.} \} \] is the \textbf{permutation group} of \(\Omega\). This is essentially a shuffling of elements of \(\Omega\). If \(\left| \Omega \right| = n < \infty\), then \(\perm\left( \Omega \right) \simeq S_{n}\).
\end{definition}
\begin{definition}[Group Action]
	Let \(G\) be a group and \(\Omega\) to be a collection of elements of \(G\) (a set). Then a \textbf{group} action of \(G\) on \(\Omega\) is a homomoprhism \(\alpha: G \to \perm(\Omega)\). We say \(G\) acts on \(\Omega\).
\end{definition}
\begin{notation}
\begin{enumerate}
	\item We generally use the exponential notation \(x^{g}\coloneqq \left( \alpha \left( g \right)  \right) \left( x \right) \) for \(g \in G\) and \( x \in \Omega\)).
	\item Some authors, such as Dummit and Foote, use multiplicative notation \(gx\) or \(g \cdot x\) for the same action.
\end{enumerate}\end{notation}
\begin{intuition}
Our homomorphism \(\alpha\) essentially characterized how an element within \(G\) will "move around" the elements of \(\Omega\) in some way.
\end{intuition}
The defining property of a group action is that \(\left( x^{g} \right)^{h}=x^{hg} \) for all \(h, g \in G\) and \(x \in \Omega\). That is, group actions turn composition into multiplication. In the function notation this is,
\begin{align*}
	\left( x^{g} \right) ^{h} &= \left( \left( \alpha\left( g \right) \left( x \right)  \right)  \right) ^{h}\\
				  &= \alpha \left( h \right) \left( \alpha\left( g \right) \left( x \right)  \right) \\
				  &= \left( \alpha\left( h \right) \alpha\left( g \right)  \right) \left( x \right) \text{as \(\alpha \left( g \right) \left( x \right) \in \perm \left( \Omega \right) \).} \\
				  &= \left( \alpha\left( hg \right) \left( x \right)  \right) \text{ By \(\alpha\) being a homomorphism.}\\
				  &= x^{hg}
.\end{align*}
\begin{remark}
	We know \(x^{1}=x\) for \(x \in \Omega\) as \(\alpha\left( 1 \right) =1\) by homomorphism. This corresponds to the map which leaves all elements of \(\Omega\) in place.
\end{remark}
\begin{example}[Conjugation Map]
Let \(G\) act on itself by conjugation, that is \(\Omega = G\) and let  \begin{align*}
	\alpha: G &\longrightarrow  \perm\left( G \right) \\
	g	 &\longmapsto \alpha(g) = g x g^{-1} \in \aut\left( G \right) \le \perm\left( G \right)
.\end{align*}
We see this is simply the cnonjugation by \(g\) map. Let us verify this is a group action. \(x^{1}= 1x 1^{-1} = x\). Similarly,
\begin{align*}
	\left( x^{g} \right)^{h}&= \left( gxg^{-1} \right) ^{h}\\
				&= h\left( gxg^{-1} \right) h^{-1}\\
				&=\left( hg \right) xg^{-1}h^{-1}\\
				&= \left( hg \right) x \left( hg \right) ^{-1}\\
				&= x^{hg}
.\end{align*}
Hence, we have confirmed \(\alpha\) is a group action.
\end{example}
Now, let us examine \(\ker \left( \alpha \right) \trianglelefteq G\).\\
\begin{align*}
	\ker \left( \alpha \right)  &= \{g \in G : x^{g}= x \ \forall \ x \in G\} \\
		      &= \{g \in G: gxg^{-1} = x \ \forall \ x \in G\} \\
		&= \{g \in G : gx=xg \ \forall \ x \in G\} \text{ multiplying by \(g\) from the right}\\
		      &=C_{G}\left(  G \right) = Z_{G}\left( G \right) \text{, the center of \(G\)}
.\end{align*}
\begin{definition}[Inner Automorphisms]
	We call \(\alpha \left( G \right) \) the \textbf{inner automorphisms of \(G\)}.
\end{definition}
\begin{example}[Conjugation Map on Sets]
	Let \(G\) act on the subsets \(A \subseteq G\) by conjugation, that is \(\Omega = \{H : H \subseteq G\} \). For \(X \subseteq G\) and \(g \in G\),  let \begin{align*}
		X^{g}&= gXg^{-1}
		     &= \{gxg^{-1} : x \in X\}
	.\end{align*}
	Here, \(g\) is a bijection of the sets as the map \(g ^{-1}\) is an inverse map to \(g.\) (hence it is a permutation and thus a group action.). That is,  \[X \to_{g} X^{g}\to_{g^{-1}} \left( X^{g} \right)^{g^{-1}}= X \].
\end{example}
\begin{remark}[Permutations]
	The two properties \(\left( x^{g} \right)^{h}= x^{hg} \) and \(x^{1}= x\) completely characterizes a group action (and hence a permutation), but sometimes it is easier to check for an inverse map as we did in the example previous.
\end{remark}
In general, if \(G\) acts on \(\Omega\) and \(\Omega^{\prime} \subseteq \Omega\) is a subset which is closed (meaning \(x \in \Omega^{\prime}\), \(g \in G\) implies \(x^{g}\in \Omega ^{\prime}\)), then we can simply restrict the codomain of the group action, hence \(G\) can act on \(\Omega^{\prime}\) in exactly the same way.
\begin{example}[Left Multiplication]
	Let \(G\) act on itself by left multiplication. (right multiplication will be essentially equivalent). Hence \(\Omega = G\)	 and \(x^{g}\coloneqq gx\) for \(x, g \in G\). Of course, \(x^{1}= 1x = x\) and
	\begin{align*}
		\left( x^{g} \right)^{h}&= \left( gx \right) ^{h}  \\
		&= h\left( gx \right) \\
		&-\left( hg \right) x\\
		&= x^{hg}
	.\end{align*}
Hence, this is a group action, but it will not be an automorphism (as it is not necessarily a bijection). There is, however, an inverse map, simply multiplication by \(g^{-1}\), so we see it really does map to a permutation of \(G\).
\end{example}
