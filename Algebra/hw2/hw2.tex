\documentclass[a4paper]{article}
% Some basic packages
\usepackage[utf8]{inputenc}
\usepackage[T1]{fontenc}
\usepackage{textcomp}
\usepackage{url}
\usepackage{graphicx}
\usepackage{float}
\usepackage{booktabs}
\usepackage{enumitem}

\pdfminorversion=7

% Don't indent paragraphs, leave some space between them
\usepackage{parskip}

% Hide page number when page is empty
\usepackage{emptypage}
\usepackage{subcaption}
\usepackage{multicol}
\usepackage{xcolor}

% Other font I sometimes use.
% \usepackage{cmbright}

% Math stuff
\usepackage{amsmath, amsfonts, mathtools, amsthm, amssymb}
% Fancy script capitals
\usepackage{mathrsfs}
\usepackage{cancel}
% Bold math
\usepackage{bm}
% Some shortcuts
\newcommand\N{\ensuremath{\mathbb{N}}}
\newcommand\R{\ensuremath{\mathbb{R}}}
\newcommand\Z{\ensuremath{\mathbb{Z}}}
\renewcommand\O{\ensuremath{\varnothing}}
\newcommand\Q{\ensuremath{\mathbb{Q}}}
\newcommand\C{\ensuremath{\mathbb{C}}}
% Easily typeset systems of equations (French package)

% Put x \to \infty below \lim
\let\svlim\lim\def\lim{\svlim\limits}

%Make implies and impliedby shorter
\let\implies\Rightarrow
\let\impliedby\Leftarrow
\let\iff\Leftrightarrow
\let\epsilon\varepsilon
\let\nothing\varnothing

% Add \contra symbol to denote contradiction
\usepackage{stmaryrd} % for \lightning
\newcommand\contra{\scalebox{1.5}{$\lightning$}}

 \let\phi\varphi

% Command for short corrections
% Usage: 1+1=\correct{3}{2}

\definecolor{correct}{HTML}{009900}
\newcommand\correct[2]{\ensuremath{\:}{\color{red}{#1}}\ensuremath{\to }{\color{correct}{#2}}\ensuremath{\:}}
\newcommand\green[1]{{\color{correct}{#1}}}

% horizontal rule
\newcommand\hr{
    \noindent\rule[0.5ex]{\linewidth}{0.5pt}
}

% hide parts
\newcommand\hide[1]{}

% Environments
\makeatother
% For box around Definition, Theorem, \ldots
\usepackage{mdframed}
\mdfsetup{skipabove=1em,skipbelow=0em}
\theoremstyle{definition}
\newmdtheoremenv[nobreak=true]{definition}{Definition}
\newmdtheoremenv[nobreak=true]{eg}{Example}
\newmdtheoremenv[nobreak=true]{corollary}{Corollary}
\newmdtheoremenv[nobreak=true]{lemma}{Lemma}[section]
\newmdtheoremenv[nobreak=true]{proposition}{Proposition}
\newmdtheoremenv[nobreak=true]{theorem}{Theorem}[section]
\newmdtheoremenv[nobreak=true]{law}{Law}
\newmdtheoremenv[nobreak=true]{postulate}{Postulate}
\newmdtheoremenv{conclusion}{Conclusion}
\newmdtheoremenv{bonus}{Bonus}
\newmdtheoremenv{presumption}{Presumption}
\newtheorem*{recall}{Recall}
\newtheorem*{previouslyseen}{As Previously Seen}
\newtheorem*{interlude}{Interlude}
\newtheorem*{notation}{Notation}
\newtheorem*{observation}{Observation}
\newtheorem*{exercise}{Exercise}
\newtheorem*{comment}{Comment}
\newtheorem*{practice}{Practice}
\newtheorem*{remark}{Remark}
\newtheorem*{problem}{Problem}
\newtheorem*{solution}{Solution}
\newtheorem*{terminology}{Terminology}
\newtheorem*{application}{Application}
\newtheorem*{instance}{Instance}
\newtheorem*{question}{Question}
\newtheorem*{intuition}{Intuition}
\newtheorem*{property}{Property}
\newtheorem*{example}{Example}
\numberwithin{equation}{section}
\numberwithin{definition}{section}
\numberwithin{proposition}{section}

% End example and intermezzo environments with a small diamond (just like proof
% environments end with a small square)
\usepackage{etoolbox}
\AtEndEnvironment{example}{\null\hfill$\diamond$}%
\AtEndEnvironment{interlude}{\null\hfill$\diamond$}%

\AtEndEnvironment{solution}{\null\hfill$\blacksquare$}%
% Fix some spacing
% http://tex.stackexchange.com/questions/22119/how-can-i-change-the-spacing-before-theorems-with-amsthm
\makeatletter
\def\thm@space@setup{%
  \thm@preskip=\parskip \thm@postskip=0pt
}


% \lecture starts a new lecture (les in dutch)
%
% Usage:
% \lecture{1}{di 12 feb 2019 16:00}{Inleiding}
%
% This adds a section heading with the number / title of the lecture and a
% margin paragraph with the date.

% I use \dateparts here to hide the year (2019). This way, I can easily parse
% the date of each lecture unambiguously while still having a human-friendly
% short format printed to the pdf.

\usepackage{xifthen}
\def\testdateparts#1{\dateparts#1\relax}
\def\dateparts#1 #2 #3 #4 #5\relax{
    \marginpar{\small\textsf{\mbox{#1 #2 #3 #5}}}
}

\def\@lecture{}%
\newcommand{\lecture}[3]{
    \ifthenelse{\isempty{#3}}{%
        \def\@lecture{Lecture #1}%
    }{%
        \def\@lecture{Lecture #1: #3}%
    }%
    \subsection*{\@lecture}
    \marginpar{\small\textsf{\mbox{#2}}}
}



% These are the fancy headers
\usepackage{fancyhdr}
\pagestyle{fancy}

% LE: left even
% RO: right odd
% CE, CO: center even, center odd
% My name for when I print my lecture notes to use for an open book exam.
% \fancyhead[LE,RO]{Gilles Castel}

\fancyhead[RO,LE]{\@lecture} % Right odd,  Left even
\fancyhead[RE,LO]{}          % Right even, Left odd

\fancyfoot[RO,LE]{\thepage}  % Right odd,  Left even
\fancyfoot[RE,LO]{}          % Right even, Left odd
\fancyfoot[C]{\leftmark}     % Center

\makeatother




% Todonotes and inline notes in fancy boxes
\usepackage{todonotes}
\usepackage{tcolorbox}

% Make boxes breakable
\tcbuselibrary{breakable}

% Verbetering is correction in Dutch
% Usage:
% \begin{verbetering}
%     Lorem ipsum dolor sit amet, consetetur sadipscing elitr, sed diam nonumy eirmod
%     tempor invidunt ut labore et dolore magna aliquyam erat, sed diam voluptua. At
%     vero eos et accusam et justo duo dolores et ea rebum. Stet clita kasd gubergren,
%     no sea takimata sanctus est Lorem ipsum dolor sit amet.
% \end{verbetering}
\newenvironment{correction}{\begin{tcolorbox}[
    arc=0mm,
    colback=white,
    colframe=green!60!black,
    title=Opmerking,
    fonttitle=\sffamily,
    breakable
]}{\end{tcolorbox}}

% Noot is note in Dutch. Same as 'verbetering' but color of box is different
\newenvironment{note}[1]{\begin{tcolorbox}[
    arc=0mm,
    colback=white,
    colframe=white!60!black,
    title=#1,
    fonttitle=\sffamily,
    breakable
]}{\end{tcolorbox}}


% Figure support as explained in my blog post.
\usepackage{import}
\usepackage{xifthen}
\usepackage{pdfpages}
\usepackage{transparent}
\newcommand{\incfig}[2][1]{%
    \def\svgwidth{#1\columnwidth}
    \import{./figures/}{#2.pdf_tex}
}

% Fix some stuff
% %http://tex.stackexchange.com/questions/76273/multiple-pdfs-with-page-group-included-in-a-single-page-warning
\pdfsuppresswarningpagegroup=1
\binoppenalty=9999
\relpenalty=9999

% My name
\author{Thomas Fleming}

\usepackage{pdfpages}
\title{Algebraic Theory I: Homework II}
\date{Sun 26 Sep 2021 22:11}
\DeclareMathOperator{\SRG}{SRG}
\DeclareMathOperator{\cut}{Cut}
\DeclareMathOperator{\GF}{GF}
\DeclareMathOperator{\V}{V}
\DeclareMathOperator{\E}{E}
\DeclareMathOperator{\edg}{e}
\DeclareMathOperator{\vtx}{v}
\DeclareMathOperator{\diam}{diam}

\DeclareMathOperator{\tr}{tr}
\DeclareMathOperator{\A}{A}

\DeclareMathOperator{\Adj}{Adj}
\DeclareMathOperator{\mcd}{mcd}

\begin{document}
\maketitle
\begin{problem}[1]
	Let \(G_1\), \(G_2\) be finite groups with \(\gcd\left( \left| G_1 \right| , \left| G_2 \right|  \right) = 1\). Show that \(\aut\left( G_1 \times G_2 \right)  \simeq \aut\left( G_1 \right) \times \aut\left( G_2 \right) \).
\end{problem}
\begin{solution}
	We induce a bijective correspondence. Let \(\alpha \in \aut\left( G_1 \times G_2 \right) \) , \(x \in G_1\) and \(y \in G_2\). Then, let \(\alpha\left( x, 1 \right) = \left( a, b \right) \) and \(\alpha\left( 1, y \right)  = \left( c, d \right) \). We see,
\begin{align*}
	\alpha\left( \left( x, 1 \right) ^{\left| G_1 \right| } \right) &= \alpha\left( x^{\left| G_1 \right| }, 1 \right)  \\
	\alpha\left( \left( a, b \right) ^{\left| G_1 \right| } \right) &= \alpha\left( 1, 1 \right)  \\
	\alpha\left( a^{\left| G_1 \right| }, b^{\left| G_1 \right| } \right) &= \left( 1, 1 \right)  \\
									      &= \alpha\left( 1, b^{\left| G_1 \right| } \right)  \\
.\end{align*}
Hence, as \(\alpha\) is a bijection, we must have \(b^{\left| G_1 \right| } = 1\) and as \(\left| G_1 \right| , \left| G_2 \right|  \) are coprime this implies \( b = 1\). Similairly, we see \(c = 1\). Hence,
\begin{align*}
	\alpha\left( \left( x, 1 \right) \cdot \left( 1, y \right)  \right) &= \alpha\left( \left( x, 1 \right)  \right) \left( \alpha\left( \left( 1, y \right)  \right)  \right)  \\
	\alpha\left(  x, y   \right) &= \left( a, 1 \right) \cdot \left( 1, d \right)  \\
				     &=  \left( a, d \right) \\
.\end{align*}
Then, we note that as \(G_1 \simeq G_1 \times \{1\} \) and \(G_2 \simeq \{1\} \times G_2\), we have \[\alpha\left( x, 1 \right) \in \aut\left( G_1 \times \{ 1\}  \right) \simeq \aut\left( G_1 \right) \text{ and  }\alpha\left( 1, y \right) \in \aut\left( \{1\} \times G_2 \right) \simeq \aut\left( G_2 \right) \] Hence, let us define \(\alpha_1: G_1 \to G_1\) and \(\alpha_2 : G_2 \to G_2\) to simply be the projection of \(\alpha\) into their respective coordinates. We see by the preceding argument that \(\alpha_1 \in \aut\left( G_1 \right) \) and \(\alpha_2 \in \aut\left( G_2 \right) \).\\
Hence, let \(\Phi : \aut\left( G_1 \times G_2 \right) \to \aut\left( G_1 \right) \times \aut\left( G_2 \right), \ \alpha \mapsto (\alpha_1, \alpha_2) \). Let \(\alpha, \beta \in \aut\left( G_1 \times G_2 \right)  \)  and suppose \(\Phi\left( \alpha \right)  = \Phi\left( \beta \right) \). Then, we have \(\Phi\left( \alpha \right)  = \left( \alpha_1, \alpha_2 \right) = \left( \beta_1, \beta_2 \right) =\Phi\left( \beta \right) \), hence \(\alpha_1 = \beta_1\) and \(\alpha_2 = \beta_2\), so we have \[\alpha\left( x, y \right)  =\alpha\left( x, 1 \right) \cdot \alpha\left( 1, y \right)  = \left( \alpha_1\left( x \right) , \alpha_2\left( y \right)  \right) = \left( \beta_1\left( x \right), \beta_2\left( y \right)   \right) = \beta\left( x, 1 \right) \beta\left( 1, y \right) = \beta\left( x, y \right)  \] for all \(x \in G_1\), \(y \in G_2\), so \(\alpha  = \beta\) and \(\Phi\) is an injection. Now, let \(\left( \alpha_1, \alpha_2 \right)  \in \aut\left( G_1 \right) \times \aut\left( G_2 \right) \) and we define \(\alpha: G_1 \times G_2 \to G_1 \times G_2, \ \left( x, y \right) \mapsto \left( \alpha_1\left( x \right) , \alpha_2\left( y \right)  \right) \). We see \(\alpha_1, \alpha_2\) are bijective, hence \(\alpha\) is bijective. Furthermore,
\begin{align*}
	\alpha\left( \left( a, b \right) \left( c, d \right)  \right) &= \alpha\left( ac, bd \right)  \\
								      &= \left( \alpha_1\left( ac \right) , \alpha_2\left( bd \right)  \right)  \\
								      &= \left( \alpha_1\left( a \right) \alpha_1\left( c \right) , \alpha_2\left( b \right) \alpha_2\left( d \right)  \right)  \\
								      &= \left( \alpha_1\left( a \right) , \alpha_2\left( b \right)  \right) \left( \alpha_1\left( c \right) , \alpha_2\left( d \right)  \right)   \\
								      &= \alpha\left( a, b \right) \alpha\left( c, d \right)  \\
.\end{align*}
Hence, \(\alpha\) is a homomorphism, so \(\alpha \in \aut\left( G_1 \times G_2 \right) \). Hence, \(\Phi\) is a bijection. Lastly, we show \(\Phi\) is a homomorphism,
\begin{align*}
	\Phi\left( \alpha \beta \right) &= \left( \alpha_1 \beta_1, \alpha_2 \beta_2 \right) 	\\
					&= \left( \alpha_1, \alpha_2 \right) \left( \beta_1, \beta_2 \right)  \\
					&= \Phi\left( \alpha \right) \Phi\left( \beta \right)
.\end{align*}
So, \(\Phi\) is an isomorphism, so \(\aut\left( G_1 \times G_2 \right) \simeq \aut\left( G_1 \right) \times \aut\left( G_2 \right) \).
\end{solution}
\newpage
\begin{problem}[2]
	Let \(n \ge 1 \) be an integer. For \( x \in \Z\), denote \(\overline{x} = x + n\Z \in \Z / n\Z\) and let \(\left( \Z / n\Z \right)^{\times} = \{\overline{x} : x \in \Z, \gcd\left( x, n \right) = 1\}  \).
	\begin{enumerate}
		\item Show that \(\left( \Z / n\Z \right)^{\times} \) is an abelian multiplicative group.
		\item Show that \(\aut\left( \Z / n\Z \right) \simeq \left( \Z / n\Z \right)^{\times} \).
	\end{enumerate}
\end{problem}
\begin{solution}
	\begin{enumerate}
		\item First, we show multiplication is well defined. Let \(a, b \in \Z\), hence \(an, bn \in n\Z\) and we see for \(x, y \in \Z\), \(x + an \in \overline{x}\) and \(y + bn \in \overline{y}\). Then, we have
			\begin{align*}
				\left( x + an \right)  \cdot \left( y + bn \right) &=  xy + \left( ay + bx \right) n + abn^2  \\
										   &= xy + n\left( ay + bx + abn \right)  \\
										   &\in xy + n\Z \\
			.\end{align*}
			And, as \(x, y \) are coprime to \(n\), we see \(\gcd\left( xy, n \right) = 1\) hence we have \(\overline{xy} \in \left( \Z / n\Z \right)^{\times} \). Now, note that \(\overline{1} = 1 + n\Z \in \left( \Z / n\Z \right) \) as \(1\) is coprime to all numbers and \(\overline{1}\overline{x} = \overline{1x} =  \overline{x1} = \overline{x}\overline{1}= \overline{x}\), so \(\overline{1}\) is the identity. Now, recall that there is a linear combination \(ax + bn = \gcd\left( x, n \right) = 1\), hence we have that \(ax = xa = 1 - bn \in 1 + n\Z = \overline{1}\), hence \(\overline{a} = \overline{x}^{-1}\), we note that as \(a \mid 1-bn\), we have \(a \nmid bn\), hence \(a \nmid n\), so \(\gcd\left( a, n \right) = 1 \), so \(\overline{a} \in \left( \Z / n\Z \right)^{\times} \), hence inverses exist and are well defined. Next, we show associativity.			\begin{align*}
				\left( \overline{x} \cdot \overline{y} \right) \overline{z} &= \overline{xy} \cdot \overline{z}\\
				&= \overline{xyz} \\
				&= \overline{x} \cdot \overline{yz} \\
				&= \overline{x}\left( \overline{y} \cdot \overline{z} \right)
			.\end{align*}
			Lastly, let us determine commutativity,
			\begin{align*}
				\overline{x} \cdot \overline{y} &= \overline{xy}  \\
				&= xy + n\Z \\
				&= yx + n\Z \\
&= \overline{yx} \\
&= \overline{y} \cdot \overline{x} \\
			.\end{align*}
			Hence, \(\left( \Z /  n \Z \right) ^{ \times}\) is an abelian group under multiplication.
		\item Let \(x \in \Z / n\Z\) be a generator and \(\phi \in \aut\left( \Z / n\Z \right) \) be an automorphism. We wish to induce a correspondance between each \(\phi\) and each \(0 \le m < n\) such that \(\gcd\left( m, n \right)  = 1\), \(m\) being a congruence class in \(\left( \Z / nZ \right) ^{\times}\). First, note that all automorphisms of \(\Z / n\Z\) amount to fixing a generator and mapping it to each other generator. Hence a generator \(x \mapsto y = x^{a}\), \(y \in \Z / n\Z\) being another generator. We see \(\gcd\left( a, n \right)  = 1\), else \(y\) would not be a generator, hence we have each \(\phi\) corresponds to an \(a \nmid n\), denote these automorphisms by \(\phi_{a}\), \(0 \le a < n\), \(\gcd\left( a, n \right)  = 1\). Then, define a bijective correspondance \(\kappa : \aut\left( \Z / n\Z \right)  \to \left( \Z / n\Z \right) ^{\times}, \ \phi_{a} \mapsto \overline{a}\). First, we show this is a homomorphism,
			\begin{align*}
				\kappa\left( \phi_{a} \right) \kappa\left( \phi_{b} \right)  &= \overline{a} \cdot \overline{b}\\
				&= \overline{ab} \\
				&= \kappa\left( \phi_{ab} \right)  \\
				&= \kappa\left( x^{ab} \right)  \\
				&= \kappa\left( x^{a}x^{b} \right)  \\
				&= \kappa\left( \phi_{a}\phi_{b} \right)  \\
			.\end{align*}Next, we show bijection. As each \(\gcd\left( a, n \right) = 1\) yields an autmorphism, we see \(\kappa\) is surjective and as each automorphism is completely determined by \(a\), we see a given \(\phi_{a}\) corresponds to only one \(\overline{a} \in \left( \Z / n\Z \right) ^{\times}\) we have \(\kappa\) is injective. Thus, \(\kappa\) is an isomorphism, so we have \(\aut\left( \Z / n\Z \right) \simeq \left( \Z / n\Z \right) ^{\times}\)
\end{enumerate}
\end{solution}
\newpage
\begin{problem}[3]
	Let \(H = \left<x \right>  \simeq C_2\) and \(N = \left<y \right> \simeq C_{15}\) be cyclic groups generated by \(x \in H\) and \(y \in N\) respectively.
	\begin{enumerate}
		\item Show that \(\aut\left( C_{15} \right) \simeq C_2 \times C_4 \).
		\item Let \(\alpha: H \to \aut \left( N \right) \) be a homomorphism and let \(\alpha\left( x \right) \left( y \right)  = y^{r}\) with \(r \in \{ 0, 1, \ldots, 14\} \). What possible values can \(r\) take?
			\item For each possible value of \(\alpha\) from item \(2\) determine which of the following four groups is isomoprhic to \(N \rtimes_{\alpha} H\): \(C_{30}, D_{15}, C_3 \times D_5, C_5 \times S_3\).
	\end{enumerate}
\end{problem}
\begin{solution}
	\begin{enumerate}
		\item Note that as \(15 = 3\cdot 5\), we have \(C_{15} \simeq C_3 \times C_5\), so by problem \(1\), \(\aut \left( C_{15} \right)  = \aut \left( C_3 \right) \times \aut \left( C_5 \right)  = C_2 \times C_4\).
			\item Recall from problem \(2\) that all automorphisms of a cyclic group \(C_{n} = \Z / n\Z\) amount to mapping generators to generators \(x \mapsto y = x^{a}\), and we see as \(y\) is a generator that \(a \nmid n\). Hence, the only possible \(r\) values are those coprime to \(15\): \(r \in \{1, 2, 4,  7, 8, 11, 13, 14 \} \).
			\item If \(r = 1\), we see \(\alpha_1 \left( x \right)  = y^{1} = y\) is simply the identity automorphism, hence \(C_2 \rtimes_{\alpha} C_{15} = C_2 \times C_{15} = C_{30}\).\\
				If \(r = 14\), we see elements of the form \(\left( y^{a}, x \right) \) have \(\left( y^{a}, x \right) ^2 = \left( y^{15a}, 1 \right) = \left( 1, 1 \right)  \) and elements of the form \(\left( y^{a}, 1 \right) \) have \(\left( y^{a}, 1 \right) ^{15} = \left( y^{15a}, 1 \right)  = \left( 1, 1 \right) \). Lastly, we have
				\begin{align*}
					\left( y^{a}, x \right) \left( y^{b}, 1 \right) \left( y^{a}, x \right) ^{-1} &=  \left( y^{a}, x \right) \left( y^{b}, 1 \right)\left( y^{a}, x \right)    \\
														      &= \left( y^{a}, x \right)\left( y^{b + a}, x\right)   \\
														      &= \left( y^{a + 14\left( b + a \right)},1 \right)  \\
														      &= \left( y^{15b}y^{14a}, 1 \right)\\
														      &= \left( y^{14a}, 1 \right)  \\
														      &= \left( y^{a}, 1 \right)^{-1}  \\
				.\end{align*}
				Hence, when \(r = 14\), \( N \rtimes_{\alpha} H \simeq D_{15}\)\\

				Next, the case \(r = 2\). Note that \(C_5 \times S_3\) is the only nonabelian group with an element of order \(10\) out of the possibilities and as \(\ord\left( y, x \right) = 10 \) and \(\left( y^2, x \right) \left( y^3, 1 \right)  = \left( y^{8}, x \right) \neq \left( y^{5}, x \right)  = \left( y^3,1 \right), \left( y^2, x \right)   \) we have \(r = 2\) produces a nonabelian group, hence for \(r = 2\) we have \(N \rtimes_{\alpha} H \simeq C_5 \times S_3\).\\

				Similarly, for the case \(r = 8\) we have \(\ord\left( y, x \right)  = 10\) and \(\left( y, x \right) \left( y, 1 \right) = \left( y^{9}, x \right) \neq \left( y^2, x \right)  = \left( y, 1 \right) \left( y, x \right)  \) so \(r = 8\) produces a nonabelian group, hence \(N\rtimes_{\alpha} H \simeq C_5 \times S_3\).\\

				Again, for the case \(r = 11\) we have \(\ord\left( y, x \right)  = 10\) and \(\left( y, x \right) \left( y, 1 \right)  = \left( y^{12, x} \right) \neq \left( y^2, x \right)  = \left( y, 1 \right) \left( y, x \right)  \), hence \(r = 11\) produces a nonabelian group, so we have \(N \rtimes_{\alpha} H = C_5 \times S_3\).

				Now, for the case \(r = 4\) note that \(C_3 \times D_5\) is the only nonabelian group with an element of order \(6\) out of the possibilities and as \(\ord\left( y, x \right) =6\) and \(\left( y^2, x \right) \left( y^3, 1 \right) = \left( y^{14}, x \right) \neq \left( y^{5}, 1 \right) = \left( y^3, 1 \right) \left( y^2, x \right)   \) we see \(r = 4\) produces a nonabelian group, hence for \(r = 4\) we have \(N \rtimes_{\alpha} H \simeq C_3 \times D_5\).\\

				Similarly, we have for \(r = 7\), \(\ord\left( y^{5}, x \right) = 6 \) and \(\left( y, x \right) \left( y, 1 \right)  =\left( y^{8}, x \right)\neq \left( y^2, x \right)  = \left( y, 1 \right) \left( y ,x \right)  \). Hence, for \(r = 7\) \(N \rtimes_{\alpha} H \simeq C_3 \times D_5\).\\

				Lastly, note that when \(r = 13\), we have \(\ord\left( y, x \right)  = 30\) and as \(C_{30}\) is the only group under consideration of order \(30\), we have \(N\rtimes_{\alpha} H \simeq C_{30}\).
	\end{enumerate}
\end{solution}
\newpage
\begin{problem}[4]
	Show there is no simple group of order \(5103\).
\end{problem}
\begin{solution}
	First, let \(G\) be a group with \( \left| G \right| = 5103 = 3^{6} \cdot 7\) and denote \(n_3, n_7\) to be the number of sylow \(3\)-groups and \(7\)-groups in \(G\) respectively. Then, we note by sylows theorms that \(n_7 \mid 3^{6}\) and \(n_7 \equiv 1 \left( \mod 7 \right) \). Note that the only numbers dividing \(3^{6}\) are \(1, 3, 3^2, 3^3, 3^{4}, 3^{5}, \) and \(3^{6}\), with
	\begin{align*}
		1 &\equiv 1 \left( \mod 7 \right) \quad
		3 &\equiv 3 \left( \mod 7 \right) \quad
		3^2 &\equiv 2 \left( \mod 7 \right) \quad
		3^3 &\equiv 6 \left( \mod 7 \right) \\
		3^{4} &\equiv 4 \left( \mod 7 \right) \quad
		3^{5} &\equiv 5 \left( \mod 7 \right) \quad
		3^{6} &\equiv 1 \left( \mod 7 \right) \quad
		\end{align*}
		If \(n_7 = 1\), then there is a unique normal sylow \(7\)-group, so let us assume \(n_7 = 3^{6}\). Similarly, \(n_3 \mid 7\) and \(n_3 \equiv 1 \left( \mod 3 \right) \), hence \(n_3 = 1\) or \(7\). If \(n_3 = 1\)  then there is a unique normal sylow \(3\)-group, hence let us assume \(n_3 = 7\). Then, recall for two sylow \(7\) groups of order \(7\) , \(P_1, P_2\) we have \(P_1 = P_2\) or \(P_1 \cap P_2 = \{1\} \). Hence, as \(n_7 = 3^{6}\), we have \(7 \cdot \left( 3^{6} - 1 \right)  = 4368\) elements among sylow \(7\)-groups, excluding identity. Additionally, we have \(n_3 = 7\) and as each sylow \(3\)-group is distinct from each other we note there must be atleast \(3^{6} + 1 \underbrace{ - 1}_{\text{identity}} \) elements among the \(3\)-groups, excluding identity. Now, as each element of a sylow \(7\)-group has order \(7\), excluding identity, and each element of a sylow \(3\)-group has order \(3^{i} \nmid 7\), \(1  \le i \le 6\), excluding identity, hence the sylow \(3\)-groups and \(7\)-groups share no common elements, so their combined size is \(3^{6} + 4374 \underbrace{ + 1}_{\text{identity}}  > \left| G \right| \),\(\lightning\), hence either \(n_3\) or \(n_7 = 1\), so there is a normal subgroup (the unique sylow group), so \(G\) is not simple.
\end{solution}
\newpage
\begin{problem}[5]
	Show there is no simple group of order \(4851\).
\end{problem}
\begin{solution}
	We follow a similar argument. Let \(G\) be a group with \(\left| G \right|  = 4851 = 3^2 \cdot 7 ^2 \cdot 11\). Let \(n_3, n_7, n_{11}\) be the number of sylow \(3, 7, 11\)-groups respctively. Then, note that by sylows theorem we have \(n_7 \mid 3^2 \cdot 11\) and \(n_7 \equiv 1 \left( \mod 7 \right) \). We see the only factors with both properties are \(1\) and \(99\). If \(n_7 = 1\), we have a unique sylow group, so assume \(n_7 = 99\). Similarly note that \(n_{11} | 3^2 \cdot 7 ^2\) and \(n_{11} \mid 1 \left( \mod 11 \right) \) and the only numbers with both properties are \(1\) and \(441\), and if \(n_{11} = 1\) there would be contradiction, hence we assume \(n_{11} = 441.\)  Hence, we have two distinct sylow \(11\)-groups of order \(11\) have only trivial intersection, hence there are \(10 \cdot 441 = 4410\) unique elements among the sylow \(11\)-groups. Furthermore,  Thus, we have \(4410 + 4252 > \left| G \right| \) elements among  the sylow \(7\) and \(11\)-groups, so this is a contradiction \(\lightning\). Hence either \(n_7=1\) or \(n_{11} = 1\), so we have a unique sylow group, hence a normal subgroup, hence \(G\) is not simple.
\end{solution}
\end{document}
