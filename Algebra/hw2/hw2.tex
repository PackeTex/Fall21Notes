\documentclass[a4paper]{article}
\input{../../preamble.tex}
\usepackage{pdfpages}
\title{Algebraic Theory I: Homework II}
\date{Sun 26 Sep 2021 22:11}
\DeclareMathOperator{\SRG}{SRG}
\DeclareMathOperator{\GF}{GF}
\DeclareMathOperator{\V}{V}
\DeclareMathOperator{\E}{E}
\DeclareMathOperator{\edg}{e}
\DeclareMathOperator{\vtx}{v}
\DeclareMathOperator{\diam}{diam}

\DeclareMathOperator{\tr}{tr}
\DeclareMathOperator{\A}{A}

\DeclareMathOperator{\Adj}{Adj}
\DeclareMathOperator{\tr}{tr}
\DeclareMathOperator{\mcd}{mcd}

\begin{document}
\maketitle
\begin{problem}[1]
	Let \(G_1\), \(G_2\) be finite groups with \(\gcd\left( \left| G_1 \right| , \left| G_2 \right|  \right) = 1\). Show that \(\aut\left( G_1 \times G_2 \right)  \simeq \aut\left( G_1 \right) \times \aut\left( G_2 \right) \).
\end{problem}
\begin{solution}
\end{solution}
\newpage
\begin{problem}[2]
	Let \(n \ge 1 \) be an integer. For \( x \in \Z\), denote \(\overline{x} = x + n\Z \in \Z / n\Z\) and let \(\left( \Z / n\Z \right)^{\times} = \{\overline{x} : x \in \Z, \gcd\left( x, n \right) = 1\}  \).
	\begin{enumerate}
		\item Show that \(\left( \Z / n\Z \right)^{\times} \) is an abelian multiplicative group.
		\item Show that \(\aut\left( \Z / n\Z \right) \simeq \left( \Z / n\Z \right)^{\times} \).
	\end{enumerate}
\end{problem}
\begin{solution}
\end{solution}
\newpage
\begin{problem}[3]
	Let \(H = \left<x \right>  \simeq C_2\) and \(N = \left<y \right> \simeq C_{15}\) be cyclic groups generated by \(x \in H\) and \(y \in N\) respectively.
	\begin{enumerate}
		\item Show that \(\aut\left( C_{15} \right) \simeq C_2 \times C_4 \).
		\item Let \(\alpha: H \to \aut \left( N \right) \) be a homomorphism and let \(\alpha\left( x \right) \left( y \right)  = y^{r}\) with \(r \in \{ 0, 1, \ldots, 14\} \). What possible values can \(r\) take?
			\item For each possible value of \(\alpha\) from item \(2\) determine which of the following four groups is isomoprhic to \(N \rtimes_{\alpha} H\): \(C_{30}, D_{15}, C_3 \times D_5, C_5 \times S_3\).
	\end{enumerate}
\end{problem}
\begin{solution}
\end{solution}
\newpage
\begin{problem}[4]
	Show there is no simple group of order \(5103\).
\end{problem}
\begin{solution}
	First, note that \(5103 = 3^{6} \cdot 7\) and denote \(n_3, n_7\) to be the number of sylow \(3\)-groups and \(7\)-groups respectively. Then, we note by sylows theorms that \(n_7 \mid 3^{6}\) and \(n_7 \equiv 1 \left( \mod 7 \right) \). Note that the only numbers dividing \(3^{6}\) are \(1, 3, 3^2, 3^3, 3^{4}, 3^{5}, \) and \(3^{6}\), with
	\begin{align*}
		1 &\equiv 1 \left( \mod 7 \right) \quad
		3 &\equiv 3 \left( \mod 7 \right) \quad
		3^2 &\equiv 2 \left( \mod 7 \right) \quad
		3^3 &\equiv 6 \left( \mod 7 \right) \\
		3^{4} &\equiv 4 \left( \mod 7 \right) \quad
		3^{5} &\equiv 5 \left( \mod 7 \right) \quad
		3^{6} &\equiv 1 \left( \mod 7 \right) \quad
		\end{align*}
		If \(n_7 = 1\), then there is a unique normal sylow \(7\)-group, so let us assume \(n_7 = 3^{6}\). Similairly, \(n_3 \mid 7\) and \(n_3 \equiv 1 \left( \mod 3 \right) \), hence \(n_3 = 1\) or \(7\). If \(n_3 = 1\)  then there is a unique normal sylow \(3\)-group, hence let us assume \(n_3 = 7\). Then,
\end{solution}
\newpage
\begin{problem}[5]
	Show there is no simple group of order \(4851\).
\end{problem}
\begin{solution}
\end{solution}
\end{document}
