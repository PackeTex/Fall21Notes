\lecture{38}{Mon 22 Nov 2021 11:31}{Polynomials (4)}

\begin{recall}
	We found the content of a polynomial over a UFD, \(R\), and its quotient field \(K\),essentially being its generalized \(\gcd\) in order to reduce polynomials in \(K\) to polynomials in \(R\).\\
	Moreover, for \(f, g \in K\left[ x \right] \), then \(\cnt\left( f \right) \cdot \cnt\left( g \right) = \cnt\left( fg \right)  \).
\end{recall}
Now, let \(f \in R\left[ x \right] \) with \(f = gh\) for \(g, h \in K\left[ x \right] \), \(K\) being the quotient field of \(R\). Then, denote \(c_g = \cnt\left( g \right) \) and \(c_{h}= \cnt\left( h \right) \). Then, we find \(f = \left( c_{g}c_{h} \right) g_1 h_1 \) for some \(h_1, g_1 \in R\left[ x \right] \).\\
Then, we see \(\cnt\left( f \right)  = \cnt\left( h \right) = c_{g}c_{h}\). Since \(f \in R\left[ x \right] \), we see \(\cnt\left( f \right)  \in R\). This implies all factorizations over \(K\) admit a factorization over \(R\).\\
\\
Now, if \(f, g \in R\left[ x \right] \) with \(h \in K\left[ x \right] \) and \(f = gh\), then the same argument shows \(\cnt\left( f \right) = \cnt\left( g \right) \cnt\left( h \right) \). Hence if \(f, g\) are primitive, we find \(\cnt\left( h \right)  \in R\), so \(h \in R\left[ x \right] \).

\begin{theorem}
	Let \(R\) be a UFD with quotient field \(K\). Let \(f \in R\left[ x \right] \) (we will prove the case \(f\) primitive for simplicity, though the non-primitive case is completely analogous). Then, we find \(f\) is irreducible in \(R\left[ x \right] \) if and only if \(f\) is irreducible in \(K\left[ x \right] \).
\end{theorem}
\begin{proof}
	Suppose \(f\) irreducible in \(K\left[ x \right] \) but not in \(R\left[ x \right] \). Denote \(f = gh\) with \(g, h \in R\left[ x \right] \) being non-units (in \(R\left[ x \right] \)).\\
	We know \(\cnt\left( f \right) = \cnt\left( g \right) \cnt\left( h \right)  = 1\). \(f = gh\) is a factorization in \(K\) unless \(g\) or \(h\) is a unit. So, assume WLOG \(g\) is a unit in \(K\left[ x \right] \), hence \(g\) is constant and \(\cnt\left( g \right) = g\) hence \(g^{-1} = \cnt\left( h \right) \). So \(g\) is a unit in \(R\) \(\lightning\).\\
	Now, assume \(f\) irreducible in \(R\left[ x \right] \) but not in \(K\left[ x \right] \).\\
	Then \(f = gh\) for some \(g, h \in K\left[ x \right] \) being non-units in \(K\left[ x \right] \). Hence, we find \(g, h\) are nonconstant polynomials in \(K\). Denote \(c_{g} = \cnt\left( g \right) , c_{h}= \cnt\left( h \right) \) with \(g = c_{g}g_1\) and \(h = g_{h}h_1\) for \(g_1, h_1 \in R\left[ x \right] \) being primitive, Thus, \(f = \left( c_{g}c_{h} \right) g_1 h_1 \) with \(c_{g}, c_{h} = \cnt\left( f \right) \in R\left[ x \right]  \) by hypothesis. Since \(g, h\) are nonconstant, \(g_1, h_1\) are nonconstant, hence nonunits and nonzero, so this is a factorization of \(f\) over \(R\left[ x \right] \) \(\lightning\). So the claim is shown.
\end{proof}
\begin{theorem}
	A ring \(R\) is a UFD if and only if \(R\left[ x \right] \) is a UFD. Moreover if \(R\) be a UFD with quotient field \(K\) then \(f \in R\left[ x \right] \) is prime if and only if one of the following hold
	\begin{enumerate}
		\item \(f = p \in R\) is a constant with \(p\) being prime in \(R\), or
		\item \(f\) is irreducible over \(K\left[ x \right] \) with \(\cnt\left( f \right) = 1\).
	\end{enumerate}
\end{theorem}
\begin{proof}
	We begin by examining the prime elements of \(R\left[ x \right] \). First, we show constant polynomials with prime content are prime in \(R\left[ x \right] \).\\
	Let \(f = p \in R\left[ x \right] \) with \(p \in R\) being a prime in \(R\). To show \(f\) is prime in \(R\left[ x \right] \),  suppose \( p \mid gh \) with \(g, h \in R\left[ x \right] \). Then let \(c_{g} = \cnt\left( g \right) \) and \(c_{h} = \cnt\left( h \right) \) so \(g = c_{g}g_1\) nad \(h = c_{h }h_1\) for primitive \(g_1, h_1 \in R\left[ x \right] \). So, \(p \mid \left( c_{g}c_{h} \right) g_1 h_1 \), so \(p \mid c_{g}c_{h}\). So \(p \mid c_{g}\) or \(c_{h}\), WLOG suppose the case \(c_{g}\). Then, \(p \mid g\), so \(p\) is prime in \(R\left[ x \right] \).\\
	Now, suppose \(f \in R\left[ x \right] \) with \(f\) primitive and \(f\) irreducible over \(K\left[ x \right] \). Since \(K\) is a field, \(K\left[ x \right] \) is a PID, hence UFD, so primes are irreducible, hence \(f\) is prime in \(K\left[ x \right] \). Suppose \(f \mid gh\) (over \(R\)), sometimes denoted \(f \mid_{R} gh\), with \(g, h \in R\left[ x \right] \). Then, \(f \mid_{K\left[ x \right] } gh\), so \(f \mid_{K\left[ x \right] } g\) or \(h\). Assume WLOG the case \(g\) and suppose \(f = gt\) for some \(t \in K\left[ x \right] \). Since \(\cnt\left( g \right), \cnt\left( f \right)  \in R\) we see \(\cnt\left( t \right)  \in R\), hence \(t \in \left[ x \right] \), so \(f \mid_{R\left[ x \right] } g\), hence \(f\) is prime.\\
	Now, let \(f \in R\left[ x \right] \) be prime. First, suppose \(f = p \in R\) is a constant polynomial which is prime in \(R\left[ x \right] \). If \(p \mid_{R\left[ x \right] } ab\) with \(ab \in R\), then we see \(p \mid_{R} ab\) . So, \(pq = ab \in R\) for a polynomial \(q\) implies \(\deg \left( q \right) \le 1\). That is, \(p \mid_{R\left[ x \right] }ab\) and since \(p\) is prime in \(R\left[ x \right] \) we find WLOG \(p \mid_{R\left[ x \right] }a\). So, \(p \mid_{R} a\) by a similar argument, and we see \(p \in R\) is prime.\\
	Otherwise, suppose the prime \(f \in R\left[ x \right] \) has \(\deg \left( f \right) \ge 1\). We wish to show \(\cnt\left( f \right)  = 1\) and \(f\) irreducible over \(R\left[ x \right] \). But, \(f = \cnt\left( f \right) f_1\) with \(f_1 \in R\left[ x \right] \) being primitive and \(\deg \left( f \right)  = \deg \left( f_1 \right)  \ge 1\) implies \(f_1\) is a nonunit (in \(R\left[ x \right] \) and \(K\left[ x \right] \)). If \(\cnt\left( f \right)  = 1\) this is a contradiction as \(f\) is prime (hence irreducible) over \(R\left[ x \right] \) . So, \(\cnt\left( f \right)  =1\).\\
	Finally, we must show \(f\) irreducible over \(K\left[x  \right] \) but the preceding lemma handles precisely this case.\\
	Next class we show the final piece of the theorem, that \(R\) is a UFD if and only if \(R\left[ x \right] \) is a UFD.
\end{proof}
