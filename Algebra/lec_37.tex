\lecture{37}{Fri 19 Nov 2021 11:30}{Polynomials (3)}
\begin{theorem}
	Let \(K\) be a field, with \(U\) being a finite multiplicative subgroup. Then it is cyclic.
\end{theorem}
\begin{proof}
Since \(U\) is a finite additive group, we see \(U = \prod_{i= 1}^{n} P_{i} \) for some sylow \(p\) groups \(P_{i}\). It suffices to show that each subgroup is cyclic as the product of their generators will generate \(U\). Let \(x \in P_{i}\) be an element of maximal order \(p^{m}\) and let \(\left| P_{i} \right|= p^{n} \) for \(m \le n\). Then every \(y \in P_{i}\) has order \(\ord\left( y \right) \mid p^{m}\). Hence, they are all roots of \(f = x^{p^{m}} - 1\) which has at most \(p^{m}\) roots, so  \(p^{n} = \left| P_{i} \right|  \le p^{m}\), hence \(n \le m\) so equality holds. So, \(x\) has order \(p^{n}\) implying \(x\) generates \(P_{i}\).
\end{proof}
\begin{corollary}
	\(\left( \Z / p\Z \right)^{\times} \simeq \Z / \left( p-1 \right) \Z \).
\end{corollary}
\begin{definition}[Content of a Polynomial]
Let \(R\) be a UFD with its quotient field \(K\). Let \(x \in K\), then there is a unique (up to units) representation \(x = \frac{a}{b}\) with \(a, b \in R\) being coprime (no prime \(p\) has \(p \mid a\) and \(p \mid b\)). Then, for a prime \(p\), define \(V_{p}\left( \frac{a}{b} \right)  = V_{p}\left( a \right) - V_{p}\left( b \right) \) where \(V_{p}\left( x \right) \) is the power of \(p\) in the unique factorization of \(x\). We see one of \(V_{p}\left( a \right)\) or \(V_{p}\left( b \right) = 0\). Leaving results \(V_{p}\left( a \right) \) if \(p \mid a\) or \(-V_{p}\left( b \right) \) if \(p \mid b\). This is called the \textbf{\(p\)-adic} valuation of \(\frac{a}{b}\). Note \(V_{p}\left( 0 \right)  \coloneqq \infty\).  \\
Now, let \(f \in K\left[ x \right] \) with \[
f = \sum_{i= 0}^{n} a_{i}x^{i}
\] for some \(n \in \N\) and \(a_{i} \in K\). Then, we define \(V_{p}\left( f \right) = \inf \{ V_{p}\left( a_{i} \right)  : i \ge 0  \} \). With this, we define the \textbf{content} of \(f\) to be \[
\cnt \left( f \right)  = \prod_{p \text{ prime}}^{} p^{V_{p}\left( f \right) }
.\]
\end{definition}
\begin{remark}
	The notion of content essentially generalizes the GCD to fraction fields.
\end{remark}
\begin{example}
	Let \(R = \Z\) so \(K = \Q\), then \(V_{2}\left( \frac{2}{9} \right)  = 1\) and \(V_{3}\left( \frac{2}{9} \right) = -2 \) and \(V_{5}\left( \frac{2}{9} \right) = 0 \).\\
	Then, let \(f\left( x \right)  = \frac{3}{4}x^2 + 6x -3\), then \[
	\cnt \left( f \right) = 3 \cdot 2^{-2} = \frac{3}{4}
	.\]
	Since \(\cnt\left( f \right) \) will always contain all denominators, this allows us to reduce a polynomial over \(\Q\) to a rational times a polynomial, \(f_1 \in K\left[ x \right] \) having content \(\cnt\left( f_1 \right) = 1\) , hence \(f_1 \in R\left[ x \right] \).
\end{example}
\begin{lemma}
	If \(R\) is a UFD, with \(K\) its quotient field, and \(f \in K\left[ x \right] \), then \(\cnt\left( f \right)  = 1\) implies \(f \in R\left[ x \right] \).
\end{lemma}
\begin{remark}
	It is of note that the converse does not hold, take \(2x^2 + 4\).
\end{remark}
\begin{definition}
	For a UFD \(R\) and quotient field \(K\), we say \(f \in K\left[ x \right] \) is \textbf{primitive} if \(\cnt \left( f \right)  = 1\) (hence \(f \in R\left[ x \right] \)).
\end{definition}
