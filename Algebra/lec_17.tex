\lecture{17}{Fri 01 Oct 2021 11:28}{Nilpotent Groups \(\left( 3 \right)\) and Solvable Groups}
\begin{recall}
	We had a theorem that, for a finite group \(G\), implied \(G\) was nilpotent if and only if all maximal subgroups are normal.
\end{recall}
\begin{proof}
	\begin{enumerate}
		\item \(\left( \implies \right) \). Let \(M < G\) be a maximal subgroup, so \(M < N \le G\) implies \(N = G\). Let \(N_{g}\left( M \right) \) be the normalizer of \(M\)< then \(M < G\), hence \(M < N_{G}\left( P \right) \) by the earlier characterization of finite nilpotent groups. Hence, \(N_{G}\left( M \right) = G\). But \(M < N_{G}\left( M \right) \) and \(M\) ix maximal, hence \(N_{G}\left( M \right) \) if and only if \(M\) is normal.
		\item \(\left( \impliedby \right) \). Assume every maximal subgroup is normal. Note that it suffies to show that all sylow groups are normal in \(G\) by the earlier characterization. Let \(P \le G\) be an arbitrary sylow \(p\)-group and let \(N = N_{G}\left( P \right) \). Let \(M\) be a maximal subgroup containing \(N_{G}\left( P \right) \). We know such a group exists because if we assume indirectly that \(P\) is not normal, this implies \(N_{G}\left( P \right) < G\) as every proper subgroup of a finite group is contained in a maximal subgroup.\\
			We now have \(P \le N_{G}\left( P \right) \le M < G\) and by hypothesis, we know \(M \trianglelefteq G\). Since \(P\le M\)with \(P\) being a sylow group of \(G\) implies \(P\le M\) is a sylow group for \(M\). But now we can applying the frattini argument. We see \(G = N_{G}\left( P \right) M\) but \(N_{G}\left( P \right) \le M\), hence \(G \subseteq MM = M < G\). \(\lightning\).
	\end{enumerate}
\end{proof}
\begin{remark}
	If \(G\) is nilpotent, then recall \(Z_0\left( G \right)  < Z_1\left( G \right) < Z_2\left( G \right)  < \ldots < Z_{i}\left( G \right) \) is the upper central series where \(Z_0 \left( G \right)  = \{1\} \), \(Z_1\left( G \right)  = Z\left( G \right) \) and \(Z_{i}\left( G \right)  / Z_{i - 1}\left( G \right)  = Z\left( G / Z_{i-1}\left( G \right)  \right) \).\\
	There is an alternative characterization, let \(G^{0}  = G\), \(G^{1} = \left[ G, G \right] = \left<x^{-1}y^{-1}xy : x, y \in G \right>  \) and define recursively \(G^{i} = \left[ G, G^{i-1} \right] = \left<x^{-1}y^{-1}xy : x\in G , y \in G^{i-1}\right>  \) to be the lower central series. Then, \(G\) is nilpotent if and only if there is \(c \ge 0\) such that \(G^{c} = \{1\} \). Furthermore, we find \(G^{c- i} \le Z_{i}\left( G \right) \) for all \(0\le i \le c\), with the minimal constant \(c\) being the same in the upper and lower central series.
\end{remark}
\begin{definition}[Solvable Groups]
	A group \(G\) is \textbf{solvable} if there's a chain  of subgroups \[
	H_0 \triangleleft H_1 \triangleleft \ldots \triangleleft H_{n} = G
	\] such that \(H_{i} / H_{ i -1}\) are abelian for \(1 \le i \le n\).
\end{definition}
As it turns out there is an equivalent chain condition for solvability closed to our characterizations of nilpotence.
Define \(G^{\left( 0 \right) } = G\), \(G^{\left( 1 \right) } = \left[ G, G \right]  = G^{1}\), Now, define \(G^{\left( i \right) } = \left[ G^{\left( i-1 \right) }, G^{\left( i-1 \right) } \right] = \left<x^{-1}y^{-1}xy : x, y \in G^{\left( i-1 \right) } \right>  \). So, \(G^{\left( n \right) }\) is essentially the \(n\)-th iterated commutator of \(G\). Then, we obtain a chain \[
	G^{\left( 0 \right) }\ge G^{\left( 1 \right) } \ge \ldots \ge G^{\left( c \right) } \ge \ldots
.\]
If \(G^{\left( c \right) } = 1\) for some \(c \ge 1\), then \(G\) is solvable. We show these two conditions are equivalent. The proof will involve multiple invocations of the basic result that \(G / H\) is abelian if and only if \(\left[ G, G \right]  \le H\).
\begin{proof}
	Assume \(G\) is solvable, and the \(1\)st characterization is true with \(1 = H_0 \trianglelefteq H_1 \trianglelefteq \ldots \trianglelefteq H_{n} = G\) with \(H_{i} / H_{ i -1}\) being abelian for all \(1 \le i \le n\). We will show by induction that \(G^{\left( i \right) }\le H_{n - i}\) for all \(1\le i \le n\). For \( i = 0\) we have \(H_{n} = G\), hence \(G^{\left( 0 \right) = G}\) and \(G \le G\), so the claim holds for \(i = 0\). Now, note that
	\begin{align*}
		G^{\left( i \right) } &= \left[ G^{\left( i - 1 \right) }, G^{\left( i - 1 \right) } \right] \\
				      &\le \left[ H_{n - \left( i - 1 \right) }, H_{n - \left( i - 1 \right) } \right] \text{ by inductive hypothesis} \\
				      &= \left[ H_{ n - i + 1}, H_{n - i + 1} \right]  \\
	.\end{align*}
	We also know that \(H_{ n - i + 1} / H_{ n - i}\) is abelian, hence we have \(G^{\left( i \right) } \le \left[H_{ n - i + 1}, H_{n - i + 1}  \right] \le H_{n - i} \) by the preceding lemma. This completes the induction. But, we have \(G^{\left( n \right) }\le H_{n - n } = H_0 = \{1\} \), so \(G^{\left( n \right) }\) is trivial.
\end{proof}
