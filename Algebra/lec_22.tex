\lecture{22}{Fri 15 Oct 2021 11:21}{Free Groups (5)}
\begin{recall}
	Let \(G, H\) be groups with presentations \(\epsilon : F\to G\)  and \(\delta: F \to H\)  for some free group \(F\), If every relator of \(G\) is also a relator for \(H\), then there is a surjective homomorphism \(\phi: G \to H, \ \epsilon\left( x \right)\mapsto \delta\left( x \right) \).
\end{recall}
\begin{definition}[Reduced Word]
We define a word \(w\) to be \textbf{reduced} if no string \(x x^{-1}\) or \(x^{-1} x\) occurs within \(w\) for any \(x \in X\). We find any word is equivalent to some reduced word by applying our relations.
\end{definition}
\begin{theorem}
	Every word is equivalent to a unique reduced word.
\end{theorem}
\begin{proof}
	We proceed fancily (he really said this). Let \(R\) be the set of reduced words on the alphabet \(X\). For each \(m \in X\), define a map \[m^{\prime} : R \to R, \ x_1^{\epsilon_1}\ldots x_{\ell}^{\epsilon_\ell} \mapsto \left \{
		\begin{array}{11}
			mx_1^{\epsilon_1}\ldots x_{\ell}^{\epsilon_{\ell}}, & \quad m \neq x_1^{-\epsilon_1} \\
			x_2^{\epsilon_2} \ldots x_{\ell}^{\epsilon_{\ell}}, & \quad m = x_1^{-\epsilon_1}
		\end{array}
\right.\] We see \(m^{\prime}\)  is a bijection as \(\left( m^{-1} \right) ^{\prime} = m^{\prime} ^{-1}\). Hence, \(m^{\prime}\)  is simply a permutation of the set \(R\).\\
Now, using the universal mapping property on \(F\left( X \right) \), we define a homomorphism \begin{align*}
	\theta: F\left( X \right)   &\longrightarrow \SYM\left( R \right)  \\
	 \left[ m \right] &\longmapsto m^{\prime}
\end{align*}
where \(\SYM \left( R \right) \)  is simply the set of all permutations of \(R\). Now, suppose \(w = x_1^{\epsilon_1} \ldots x_{\ell}^{\epsilon_{\ell}}\)  and \(w^{\prime} = y_1^{\delta_1} \ldots y_{s}^{\delta_{s}}\) are two reduced words that are equivalent, that is \(\left[ w \right]  = \left[ w^{\prime} \right] \). Then, we have \(\theta \left( \left[ w \right]  \right) = \left( x_1^{\prime} \right) ^{\epsilon_1} \ldots \left( x_{\ell}^{\prime} \right)^{\epsilon_{\ell}} \).Then, we see \(\theta\left( \left[ w \right]  \right) \left( 1 \right)  = w\). Hence, \(\theta\left( \left[ w^{\prime} \right]  \right) = \theta\left( \left[ w \right]  \right) = y_1^{\delta_1} \ldots y_{s}^{\delta_{s}}\). Hence, we see \(x_1^{\epsilon_1} \ldots x_{\ell}^{\epsilon_{\ell}} = y_1^{\delta_1} \ldots y_{s}^{\delta_{s}}\) as words. Hence, there is at most one distinct reduced word in \(\left[ w \right] \). And, as there is always atleast \(1\) reduced word, we see this completes the proof.

\end{proof}
\begin{remark}
	We define \(x^{n} = \underbrace{x\ldots x}_{n \text{ times}} \) and \(x^{-n} = \underbrace{x^{-1} x^{-1} \ldots x^{-1}}_{n \text{ times}} \). Then, we see any reduced word has the form \(x_1^{\ell_1} \ldots x_{s}^{\ell_{s}}\) with \(\ell_{i} \in \Z \setminus \{0\} \)  and \(x_{i} \neq x_{i-1}\) for all \(1 \le i \le s\). This is called the normal form of a word.
\end{remark}
\begin{definition}
	With the normal form of a word, we define a \textbf{multiplicity function}. For \(x \in X\) and a word \(w = x_1^{\ell_1} \ldots x_{s}^{\ell_{s}}\) we define \(V_{x}\left( w \right)  = \sum_{x_{j} = x}^{} \ell_{j}\).
\end{definition}
\begin{definition}[Rank]
	Recall that if \(\left| X \right|  = \left| Y \right| \), we had \(F\left( X \right)  \simeq F\left( Y \right) \). We define \(\rank \left( F\left( X \right)  \right)  = \left| X \right| \). We have yet to show this is well defined, but the next theorem will take care of this.
\end{definition}
\begin{theorem}
	If \(X\)  and \(Y\) are sets with \(F\left( X \right)  \simeq F\left( Y \right) \), then \(\left| X \right|  = \left| Y \right| \).
\end{theorem}
We will prove this claim next class.
