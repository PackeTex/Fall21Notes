\lecture{16}{Wed 29 Sep 2021 11:25}{Nilpotent Groups (3)}
\begin{corollary}
	A finite abelian group is the direct product of its sylow groups.
\end{corollary}
This follows directly from the theorem from last class.
\begin{corollary}
	If \(G\) is a finite group such that for all \(n \mid \left| G \right| \) such that there are at most \(n\) elements \(x \in G\) with \(x^{n}= 1\), then \(G\) is cyclic.
\end{corollary}
\begin{proof}
	Let \(p\) be an arbitrary prime with \(p \mid \left| G \right| \). Let \(P\) be a sylow \(p\)-group with \(\left| P \right| = p^{\alpha}\). We know for any \(x \in P\), we have \(x^{\left| P \right| } = 1\), hence there are \(\left| P \right|  = p^{\alpha}\) elements \(x \in P\) such that \(x^{p^{\alpha}} = 1\). By hypothesis there is infact equality. If there was another distinct sylow \(p\)-group we would have elements \(y \not\in P\) such that \(y^{p^{\alpha}} = 1\). Hence, \(P\) is unique. Hence, as every \(p\)-group is unique, so normal, we see \(G\) is the product of its \(P\)-groups.\\
	Denote \(G = P_1 \times P_2 \times  \ldots P_{t}\) with the \(P_{i}\)s being the distinct sylow \(p_{i}\\)-groups of \(G\). Also, if \(\left| P_1 \right|  = p_1^{\alpha_1}\), then all \(x \in P_1\) have \(\ord\left( x \right) \mid p_1^{\alpha_1}\) and there are at most \(p_1^{\alpha_1 -1} < p_1^{\alpha_1}\) such \(x\) with \(\ord\left( x \right)  \mid p_1^{\alpha_1 - 1}\). Since \(\left| P \right|  < p_1^{\alpha_1 -1}\) we see there is an \(x \in P_1\) with \(\ord\left( x \right) = p_1^{\alpha_1} = \left| P \right| \), hence \(\left<x \right>  = P_1\). So, \(P_1\) is cyclic. Likewise, all other \(P_{i}\) are shown cyclic by the same argument, with \(P_{i} = \left<x_{i} \right> \). Then, the element \(x = \prod_{i= 1}^{t} x_{i}\) is a generator of \(G\), so \(G\) is cyclic.
\end{proof}
\begin{theorem}[Frattini's Argument]
	Let \(G\) be a finite group , \(H \trianglelefteq G\), \(P \le H\) being a sylow \(p\)-group in \(H\). Then, \[
		G = HN_{G}\left( P \right) \text{ and } \left| G : H \right|  \mid \left| N_{G}\left( P \right)  \right|
	.\]
\end{theorem}
\begin{proof}
	Let \(g \in G\), we wish to show \(g \in HN_{G}\left( P \right) \). We know this to be a subgroup as \(H \trianglelefteq G\). Let \(G\) act by conjugation on its sets. Now
	\begin{align*}
		P^{g} &= gPg^{-1}\\
		      &\le H^{g}\\
		      &= gHg^{-1} \\
		      &= H \text{ by normality}
	.\end{align*}
	Then, we see as \(\left| P^{g} \right|= \left| P \right|  \), then \(P^{g}\) is another sylow \(p\)-group in \(H\). And, as we know all sylow \(p\)-groups are conjugate. Hence, there is an \(h \in H\) such that \(P^{h} = P^{g}\). Hence, \(P = P^{h^{-1} g}\), hence \(h^{-1} g \in N_{G}\left( P \right) \). Then, we see \(g \in hN_{G}\left( P \right) \subseteq  HN_{G}\left( P \right)  \). So, we see \(G = HN_{G}\left( P \right) \)\\
Now, we show the other result. Note that by the second isomorphism theorem, we have \[
	G / H = \left( HN_{G}\left( P \right)\right) / H \simeq \frac{ N_{G}\left( P \right) }{H \cap N_{G }\left( P \right) }
.\]
Thus, \(\left| G : H \right|  = \left| N_{G}\left( P \right)  : H \cap N_{G}\left( P \right)  \right| \). As we know this divides \(\left| N_{G}\left( P \right)  \right| \), hence \(\left| G  : H\right|  \mid \left| N_{G}\left( P \right)  \right| \) .
\end{proof}
\begin{theorem}
	if \(G\) is a finite group, then \(G\) is nilpotent if and only if every maximal subgroup in \(G\) is normal in \(G\).
\end{theorem}
