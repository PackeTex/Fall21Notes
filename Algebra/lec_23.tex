\lecture{23}{Mon 18 Oct 2021 11:26}{Free Groups (6)}
Recall, we defined the rank of a free group to be the size of its underlying alphabet. In order to ensure this was well defined, we needed to prove the following claim
\begin{proposition}
	If \(F\left( X \right)  \simeq F\left( Y \right) \) via the isomoprhism \(\phi\), then \(\left| X \right|  = \left| Y \right| \) .
\end{proposition}
\begin{proof}
	Denote \(G = F\left( X \right) \) and \(G^{\prime} = F\left( Y \right) \) and let \(H = \left<g^2 : g \in F\left( X \right)  \right> \). We know this to be a characteristic subgroup by the homework problem. Hence, we have \(H \trianglelefteq F\left( X \right) \). Consider \(G / H\)  and note that \(\phi \left( H \right) = H^{\prime} = \{h^2 : h \in F\left( Y \right) \} \). Since, \(\phi\left( H \right)  = \{\phi\left( g^2 \right)  = \phi\left( g \right) ^2 : g \in F\left( X \right) \}  = \{h^2 : h \in \phi\left( F\left( X \right)  \right)  = F\left( Y \right) \} \).
	Hence, \(G / H \simeq \phi\left( G \right)  / \phi\left( H \right)  \simeq G^{\prime} / H^{\prime}\)  as \(\phi\)  is an isomorphism. We show that \(G / H \simeq \underbrace{\Z / 2\Z + \ldots + \Z / 2\Z}_{\left| X \right|  \text{ times}} \simeq \left( \Z / 2\Z \right) ^{\left| X \right| }\).\\
	First, note \(xyxy = \left( xy \right) ^2 = 1 \) in \(G / H\) for all \(x, y \in G / H\)  by definition. Hence, \(xyx^{-1}y^{-1} = xyxy\) as \(x^2 = y^2= 1\) for every \(x, y \in G / H\). Hence, \(xyx^{-1} = y\), so \(G / H\) is an abelian \(2\)-group. Now, note that \(\left<xH : x\in X \right> = G / H \) and denote \(xH = \overline{x}\) for each \(x\in G\). Then \(G / H = \{\overline{x} :  x\in X\} \). Note that an element \(g \in G / H\) has \[
	\overline{x_1}\overline{x_2}\ldots \overline{x_{\ell}}
	\]  with all \(\overline{x_1}, \ldots, \overline{x_{\ell}}\)  being distinct.\\
	Suppose \(\overline{x_1} \ldots \overline{x_{\ell}} = \overline{y_1} \ldots \overline{y_{s}}\). We claim that \(\ell = s\) and there is a permutation such that \(x_{i} = y_{i}\) for all \(i\). Suppose the contrary, so WLOG \(x_1 \not\in \{y_1, \ldots, y_{\ell}\} \). Hence, \(w = \overline{x_1} \ldots \overline{x_{\ell}} \overline{y_{s}} \ldots \overline{y_1} = 1\), so \(w \in H\). Furthermore, we find \(V_{x_1} \left( w \right)  = 1\). But, for any generator \(g^2 \in H\), we have \(V_{x_1}\left( g^2 \right)  = 2n\) for some \(n \ge 0\). So, we must have \(V_{x_1} \left( w \right)  = \sum_{i= 1}^{m} V_{x_1}\left( g_i^2 \right) = 2\hat{n}\) for generators \(g_{i}\) and some \(\hat{n} \ge 0\).     \(\lightning\). Hence there is a unique representation in \(G / H\).\\
	This shows that
	\begin{align*}
		G / H &=  \left<\overline{x} : x \in X \right>  \\
		&= \bigoplus _{x \in X} \left<x \right>
	\end{align*}
	with each \(\left<\overline{x} \right> \in \Z / 2\Z\)  as \(\ord\left( \overline{x} \right) = 2\). Hence, \[
	G / H = \sum_{i= 1}^{\left| X \right| } \Z / 2\Z
	.\]
	We know this to be a vector space over a \(2\) element field, \(\mathbb{F}_2\), consisting of elements \(\left( \epsilon_{x} \right) _{x \in X} \mapsto \prod_{x \in X}^{} \overline{x}^{\epsilon_{x}} \) with almost all (finitely many) \(\epsilon_{x} = 0\) and \(\dim_{\mathbb{F}_2} \left( G / H \right) = \left| X \right| \) as \(\overline{X}\) is a basis for \(G / H\). As \(G / H \simeq G^{\prime} / H^{\prime}\), we see \(\dim_{\mathbb{F}_2} \left( G^{\prime} / H^{\prime} \right) = \left| X \right| \). But by the same argument, we see \(\dim_{\mathbb{F}_2} \left( G^{\prime} / H^{\prime} \right) = \left| Y \right| \) as well. Hence, \(\left| X \right|  = \left| Y \right| \).
\end{proof}
\begin{remark}
	If \(F \simeq F\left( X \right) \) is free and \(H \le F\), then \(H\) is free. Similairly, if \(\left| F : H \right| = m  < \infty\) then \(\rank \left( H \right)  = \rank\left( F \right) \cdot m + (1-m)\) for some \( m \ge 0\).
\end{remark}
\begin{note}{Midterm}
	The test Wednesday will be proofs of \(\sim 4\) (choose \(2\) out of \(4\)) theorems, propositions, lemmas we proved in class. There will be a second part consisting of short answers consisting of applying theorems, lemmas, \(\ldots\) from class to prove simple or concrete results.
\end{note}
