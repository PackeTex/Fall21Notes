\lecture{28}{Fri 29 Oct 2021 11:31}{Ring Theory (3)}
Recall \(R\) will be a commutative ring unless otherwise noted.
\begin{definition}[Prime Ideal]
	Recall an ideal \(P \subseteq R\) is a \textbf{prime ideal} when \(xy \in P\) implies one of \(x \in P\) or \(y \in P\). This is equivalent to the statement that \(R / P\) is an integral domain.
\end{definition}
\begin{definition}[Maximal Ideals]
	A proper ideal \(M \subseteq R\) is maximal if it is not strictly contained in any other proper ideal. That is, the only ideals containing \(M\) are \(M\) and \(R\). Equivalently, an ideal \(I\) is maximal if and only if \(R / I\)  is a field.
\end{definition}
We prove these two definitions to be equivalent.
\begin{proof}
	First, assume \(I\) maximal. Then, note that an ideal in \(R / I\) has the form \(J / I\) with \(I \subseteq J \subseteq R\) and \(J\) being an ideal in \(R\). Hence, as \(I\) is maximal, we find \(J = I\) or \(J = R\). Hence, \(R / I\) is a field by  prior characterization.\\
	Now assume \(R / I\) is a field for some ideal \(I\). Then, the only ideals of \(R / I\) are \(\{0\} \) and \(R / I\). Suppose \(I\) nonmaximal, then we find a \(I \subset J \subset R\) corresponding to a proper nontrivial ideal \(J / I \subseteq R / I\), \(\lightning\) as \(R / I\) is a field.
\end{proof}
\begin{proposition}
	In a commutative ring \(R\) any maximal ideal is prime.
\end{proposition}
\begin{proof}
	Since \(M \subset R\) and \(R / M\) is a field (hence integral domain), we find \(M \) to be a prime ideal by the quotient characterization.
\end{proof}
\begin{example}
	If \(R = \Z\), then \(\left( 0 \right) \) is a prime ideal, but it is obviously not maximal.
\end{example}
In order to prove some theorems concerning maximal ideals, we need to state some results from basic set theory.
\begin{definition}
	If \(\left( X, \preceq \right) \) is a poset (partially ordered set), with a totally ordered subset \(Y \subseteq X\), then an \textbf{upper bound} of \(Y\) is an element \( x \in X\) so that \(y \le x\) for all \(y \in Y\). A \textbf{maximal element} of \(X\) is a \(x \in X\) so that for all \(y \in X\), \(x \le y\) implies \(x = y\).
\end{definition}
\begin{law}[Zorn's Lemma]
	If \(\left( X, \preceq \right) \) is a nonempty poset, with every totally ordered subset having an upper bound, then we find a maximal element \(x \in X\).
\end{law}
Of course, this is equivalent to axiom of choice, so we must take it as an axiom. Using Zorn's lemma, we find that every ideal is contained in a maximal ideal (as with subgroups).
\begin{theorem}
	If \(R\) is a commutative ring with \(I \subset R\) being a proper ideal. Then there is a maximal ideal \(M \subset R\) with \(I \subseteq M\).
\end{theorem}
\begin{proof}
	Let \(\left( X, \subseteq \right) \) be the set of all proper ideals of \(R\) which contain \(I\) partially ordered by inclusion. As \(I\) is proper, we see \(I \subseteq I\) hence \(I \in X\), so \(X \neq \O\). Any maximal element \(m \in X\) will be a maximal ideal of \(R\) containing \(I\). Hence, we need only show the existence of a maximal element.\\
	Let \(\left( I_{\alpha} \right)_{\alpha \in \Omega} \) by a nonempty totally ordered subset of \(X\). Hence, each \(I_{\alpha}\) is a proper ideal containing \(I\) with either \(I \subseteq I_{\alpha} \subseteq I_{\beta}\) or \(I \subseteq I_{\beta} \subseteq I_{\alpha}\) for all \(\alpha, \beta \in \Omega\). Let \(J = \bigcup_{\alpha \in \Omega} I_{\alpha}\), clearly, \(I_{\alpha}\subseteq J\) for all \(\alpha \in \Omega\), so we need only show \(J \in X\). Clearly, \(I \subseteq I_{\alpha} \subseteq J\), so \(J\) is nonempty and contains \(I\). Now, let \(x, y \in J\) with \(x \in I_{\alpha}\), \(y \in I_{\beta}\). By total ordering WLOG, let \(I_{\alpha} \subseteq I_{\beta}\). Hence, \(x, y \in I_{\beta}\). Hence, \(x - y \in I_{\beta} \subseteq J\) as this is an ideal and \(rx \in I_{\beta} \subseteq J \) for all \(r \in R\), hence \(J\) is an ideal. Finally, suppose \(J = R\), then \(1 \in J\) , so \(1 \in I_{\alpha}\) for some \(\alpha \in \Omega\) \(\lightning\), as \(I_{\alpha}\) is assumed proper. Hence, \(J \in X\) is an upper bound of \(\left( I_{\alpha} \right)_{\alpha \in \Omega} \), so there is a maximal element \(M \in X\) which is clearly a maximal ideal.
\end{proof}
