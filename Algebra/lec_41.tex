\section{Review of Ring Theory}
%\lecture{41}{}{Review of Ring Theory}

\begin{definition}[Rings]
	A \textbf{Ring} is a set and two operations, \(+, \cdot\).\\
	A \textbf{Unit} is an element with multiplicative inverse.\\
	A \textbf{Field} is a commutative ring with all nonzero elements units.\\
	An \textbf{Integral Domain} is a Ring with the zero product property.\\
	A \textbf{Division Ring} is a noncommutative field.\\
	A \textbf{Ring Homomorphism} respects \(+\) and \(\cdot\).\\
	An \textbf{Ideal} is a subset of \(R\) which is a subgroup under addition and has absorbtion property.\\
	A \textbf{Quotient Ring} Is simply the set of additive cosets of a given ideal.\\
	\(\left( X \right) \) is the smallest ideal containing the set \(X\). Arbitrary elements are linear combinations of elements from \(X\) with elements from \(R\).\\
	A \textbf{Prime Ideal} has \(xy \in P \implies x \in P \text{ or } y \in P\). Alternatively, \(R / P\) is an ID.\\
	\textbf{Maximal Ideals} are maximal by containment. Equivalently \(R / I\) is a field \(\iff I\)  is maximal.\\
	A \textbf{Principal Ideal} is generated by \(1\) element.
	\(x \mid y\) if \(y = rx\) for \(r \in R\).\\
	Two elements are \textbf{Associate} if they are equal up to units.\\
	A \textbf{Principal Ideal Domain} is an ID where all ideals are principal.\\
	A \textbf{Euclidean Domain} is an ID with a norm and well defined division with remainders.\\
	An element is \textbf{Prime} if \(p \mid xy \implies p \mid x \text{ or } p \mid y\).\\
	An element is \textbf{Irreducible} if \(x = yz \implies y\) or \(z\) a unit.\\
	A \textbf{Factorization} is an equivalence to a unit times a product of irreducibles.\\
	A \textbf{UFD} is an ID with all nonzero elements having Unique factorization.\\
\end{definition}
\begin{proposition}[1st Isomorphism Theorem]
	A surjective homomorphism is an ideal.
\end{proposition}
\begin{theorem}
	All maximal ideals are prime.
\end{theorem}
\begin{proof}
	Maximal ideals induce a field, hence an integral domain, hence a prime ideal.
\end{proof}
\begin{definition}[Zorn's Lemma]
A non-empty partially ordered set with every totally ordered subset having an upper bound admits a maximal element.
\end{definition}
\begin{theorem}
All proper ideals are contained in a maximal ideal.
\end{theorem}
\begin{proof}
	Take set of all proper ideals containing \(I\) po'd by inclusion. It is nonempty and the union of nested ideals is itself an ideal and it is an upper bound, hence there is a maximal element by zorn's lemma.
\end{proof}
\begin{proposition}
	\(x \mid y\) and \(y \mid x\) iff \(\left( x \right)  = \left( y \right) \).\\
	If \(R\) is an integral domain, then \(x, y \)  are associate.
\end{proposition}
\begin{proposition}
\(p\) prime implies \(\left( p \right) \) 	prime.
\end{proposition}
\begin{theorem}
If \(p\) irreducible, then \( \left( p \right) \) is maximal by inclusion among proper PI's.
\end{theorem}
\begin{proof}
	If \(\left( p \right) \) is in a proper PI, then \(p = rx\) implying \(r\) is a unit, so \(p, x\) are associate \(\lightning\) .
\end{proof}
\begin{corollary}
	\(p\) irreducible implies \(\left( p \right) \) maximal.
\end{corollary}
\begin{theorem}
	If \(R\) is an ID, then maximal among PI's implies irreducible.
\end{theorem}
\begin{proof}
	If \(p = xy\), then \(p \in \left( x \right) \text{ and } \left( y \right) \), so \(\left( y \right)  = \left( p \right) \) or \(\left( y \right)  = R\). If \(\left( y \right)  = \left( p \right) \), then \(p, y\) are associate implying \(x\) a unit. Else \( \left( y \right)  = R\), so \(y\) is a unit.
\end{proof}
\begin{theorem}
	If \(R\) is an ID, prime implies irreducible.
\end{theorem}
\begin{proof}
	If \(p = xy\), then WLOG \(x \in \left( p \right) \), so \(x = rp\) hence \(p = rpy\) implying \(y\) a unit.
\end{proof}
\begin{theorem}
	In a UFD, prime iff irreducible.
\end{theorem}
\begin{proof}
Let \(p\) be irreducible with \(p \mid xy\). then \(xy = rp\), so setting up factorization yields \(r\fac\left( x \right) \fac\left( y \right) = rp \). Since its an ID, \(p \in \fac\left( x \right) \) WLOG, hence \(p \mid x\) so \(p\) prime.
\end{proof}
