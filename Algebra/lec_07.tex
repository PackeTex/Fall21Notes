\lecture{7}{Wed 08 Sep 2021 11:20}{Sylow Groups (2)}
\begin{recall}
	If \(G\) is a finite group, then a subgroup of \(G\) such that \(p^{n}\) is the maximal power of \(p\) such that \(p^{n} \mid \left| G \right| \), then \(H\) is a sylow \(p\)-group.
\end{recall}
\begin{theorem}
	If \(G\) is a finite group and \(p\) is a prime, then \(G\) has a sylow \(p\)-group.
\end{theorem}
\begin{proof}
	We will use induction. For the first cases, if \(\left| G \right|  = p^{n}\) then the subgroup \(H = G\) is a sylow \(p\)-group. Also if \(p \nmid \left| G \right| \), then the trivial subgroup is a sylow \(p\)-group. Hence, we can assume \(p \mid \left| G \right| \) with \(\hat{p} \mid \left| G \right| \) for some prime \(\hat{p} \neq p\).\\ \\
	First, recall the class equation, \(\left| G \right|  = \left| Z\left( G \right)  \right|  + \sum_{x \in I}^{}\left| G : G_{x} \right| \) where \(I\) is a set of representatives of each orbit of size \(\ge 2\) when \(G\) acts on itself by conjugation.
	\begin{observation}
		If \(K < G\), then we can assume \(p^{n} \nmid \left| K \right| \) else a sylow \(p-\)group for \(K\) would also be a sylow \(p\)-group for \(G\), which we would know to exist by induction hypothesis. Hence, we can assume \(p \mid \left| G : K \right| \).
	\end{observation}
	\\ Now, note that every \(G_{x}\) with \(x \in I\) has that \(G_{x} < G\), else its index would be \(1\) and \(x\) would not be in \(I\). Hence, we have \(p \mid \left| G : G_{x} \right| \) for all \(x \in I\). And, as  \( p \mid \left|  G \right| \), we see \(p \mid \left| Z\left( G \right)  \right| \) by the class equation. This implies the center is nontrivial.\\
	Hence, by cauchy's theorem, there is an \(x \in Z\left( G \right) \) such that \(\ord \left( x \right) = p\). Then, \(\left< x \right>  \le Z\left( G \right)  \trianglelefteq G\). Furthermore, every subgroup of \(Z\left( G \right) \) is normal by definition of the center, hence \(\left<x \right>  \trianglelefteq G\). \\
	Let us now examine \(G / \left<x \right> \). We see \(\left| G / \left<x \right>  \right| = \frac{\left| G \right| }{p}\), hence \(p^{n-1}\) is the highest power which divides \(G / \left<x \right> \). Using the induction hypothesis yields a sylow \(p\)-group of \(G / \left<x \right> \) and by the lattice theorem, we know the \(p\)-group has the form \(H / \left<x \right> \) for a subgroup \(H \le G\) such that \(\left<x \right> \le H\). Again, we see \(\left| H / \left<x \right>  \right| = \frac{\left| H \right| }{p} = p^{n-1} \implies \left| H \right|  = p^{n}\).
\end{proof}
\begin{lemma}
	If \(G\) is a \(p\)-group acting on the finite set \(\Omega\), then the number of fixed points in \(\Omega\), denoted \(n\), has \(n \equiv \left| \Omega \right| (\mod p)\)
\end{lemma}
\begin{proof}
	Recall \[
	\left| \Omega \right|  = \sum_{ x\in I}^{} \left| G : G_{x} \right|
	\]  where \(I\) is a set of representatives for the orbit of each action. As \(x\) is a fixed point, we see \(G_{x} = G\), hence let us separate the equation and define \(\mathscr{O}\) to be the set of representatives from each orbit of size \(\ge 2\) and \(n\) to be the aforementioned number of fixed points.. Then \[
	\left| \Omega \right|  = n \sum_{ x\in I}^{} \left| G : G_{x} \right|
	.\]
As \(G\) is a finite \(p\)-group, we know \(\left| G:G_{x} \right| \ge 2\), hence \(\left| G : G_{x} \right| = p^{m}\) for some \(m\), hence \(p \mid \left| G:G^{x} \right| \), so
\begin{align*}
	\left| \Omega \right|  &\equiv n + \sum_{x \in \mathscr{O}}^{} \left| G : G_{x} \right| \left( \mod p \right) \\
			       &\equiv n + 0 \left( \mod p \right) \\
			       &\equiv n \left( \mod p \right)
.\end{align*}
\end{proof}
\begin{lemma}
	Let \(G\) be  finite group, \(p\) be prime, \(P\) is a sylow \(p\)-group in \(G\). If \(H \le N_{G}\left( P \right) \) then \(H \le P\).
\end{lemma}
\begin{proof}
	Since \(H \le N_{G}\left( P \right) \)	 we must have \(HP \le G\) with \(P \trianglelefteq HP\). Hene \(\frac{HP}{P} \simeq \frac{H}{H \cap P}\) by the 2nd isomorphim theorem. Thus, \(\underbrace{\left| \frac{HP}{P} \right|}_{= \frac{\left| HP \right| }{\left| P \right| }}  = \left| \frac{H}{H \cap P} \right|  = \frac{\left| H \right| }{\left| H \cap P \right| }\). This yields \(\left| HP \right|  = \frac{\left| H \right| \cdot \left| P \right| }{\left| H \cap P \right| }\).\\
	Since \(\left| H \right| \) and \(\left| P \right| \) are both powers of \(p\), we have \(\left| H \right|  \cdot \left| P \right| \) is also a power of \(p\). By definition \(p^{n} = \left| P \right| \) is the maximum power of \(p\) dividing \(\left| G \right| \), so \(\left| HP \right|  \le p^{n} = \left| P \right| \) by lagranges theorem, but \(p \le HP\), so \(\left| P \right| \le \left| HP \right| \le \left| P \right| \), hence \(\left| P \right|  = \left| HP \right| \) and since there is only \(1\) \(P\)-coset, we see \(HP = P\) implies \(H \le P\).
\end{proof}
\begin{theorem}[Sylow Theorems]
	Let \(G\) be a finite group, \(p\) a prime with \(n_{p}\) being the number of sylow \(p\)-groups in \(G\).
	\begin{enumerate}
		\item \(n_{p}\ge 1\) for all \(p\).
			\item If \(H \le G\) is a \(p\)-group, then there exists a sylow \(p\)-group, \(P \le G\) with \(H \le P\).
				\item All sylow \(p\)-groups are conjugate.
				\item \(n_{p} \equiv 1 \left( \mod p \right) \) .
				\item \(n_{p} = \left| G:N_{G}\left( P \right)  \right| \) where \(P\) is a sylow \(p\)-group in \(G\). In partiuclar, \(n_{p} \mid \frac{ \left| G \right| }{p^{n}}\).
	\end{enumerate}
\end{theorem}
\begin{proof}
	\begin{enumerate}
		\item We have already proved this theorem
		\item Let \(P\) be a sylow \(p\)-group in \(G\) (which we know to exist). Let \(\Omega = \{A : A \text{ is a subgroup conjugate to \(P\)}\} \). Let \(G\) act by conjugation on \( \Omega\). Then, as \(\Omega\) is simply one orbit, \(\left| \Omega \right|  = \left| G:G_{P} \right| \) where \(G_{P} = \{g \in G : gPg^{-1} = P\} =  N_{G}\left( P \right) \). Hence, \(\left| \Omega \right| = \left| G : N_{G}\left( P \right)  \right| \). As \(P \le N_{G}\left( P \right) \) and \(\left| P \right|  = p^{ n}\) is the maximum power of \(p\) such that \(p^{n} \mid \left| G \right| \), then by definition of a sylow group, \(p \nmid \left| \Omega \right| = \left| G : N_{G}\left( P \right)  \right| \). Let \(H \le G\) be a \(p\)-group in \(G\). Then, restrict the action of \(G\) on \(\Omega\) to an action of \(H\) on \(\Omega\). By the previous lemma, we have the number of fixed points in \(\Omega\) under the action of \(H\), denoted \(m\) is \(m \nequiv 0 \left( \mod p \right) \).\\
			Thus, there is some \(P^{\prime} \in \Omega\) that is a fixed point for \(H\), meaning \(hP^{\prime}h^{-1} = P^{\prime}\) for all \(h \in H\), hence \(H\le N_{G}\left( P^{\prime} \right) \). Now, \(P^{\prime}\) is conjugate to \(P\) as \(P^{\prime} \in \Omega\), so \(P^{\prime} \simeq P\) with \(\left| P^{\prime} \right|  = \left| P \right|  = p^{n}\). So, \(P^{\prime}\) is also a sylow \(p\)-group in \(G\).\\
				Taking the preivous lemma and applying it to \(P^{\prime}\) yields \(H \le P^{\prime}\), so this completes the proof of (2).
\end{proof}
The rest of the proofs will be completed next lecture.
