\lecture{20}{Fri 08 Oct 2021 11:26}{Free Groups (3)}
\begin{recall}
	\(F\)  is a free group on the set \(X\)  when there is an injection \(\sigma: X \xhookrightarrow F\)  such that for all maps \(\alpha: X \to G\) , there is a homomorphism \(\beta: F \to G\)  such that \(\beta \circ\sigma = \alpha\).
\end{recall}
\begin{remark}
	\(F\)  is also a free group on \(\sigma\left( X \right) \subseteq F\) , using a similair inclusion map, so often we will assume \(X \subseteq F\).
\end{remark}
\begin{theorem}
	If \(F_1\)  is free on \(X_1\)  and \(F_2\)  is free on \(X_2\)  and \(\left| X_1 \right|  = \left| X_2 \right| \) , then \(F_1 \simeq F_2\).
\end{theorem}
\begin{proof}
	Since \(\left| X_1 \right|  = \left| X_2 \right| \)  we find a bijection \(\alpha : X_1 \to X_2\)  and we can assume WLOG that \(X_1 \subseteq F_1\)  and \(X_2 \subseteq F_2\). Then, the free property of \(F_1\)  implies there is a unique homomorphism \(\beta: F_1 \to F_2\)  such that \(\beta\left( x \right)  = \alpha\left( x \right) \) for all \(x \in X_1\) . Similairly, thee is a unique map \(\gamma : F_2 \to F_1\)  extending \(\alpha^{-1} : X_2 \to X_1\)  such that \(\gamma \left( y \right)  = \alpha^{-1}\left( y \right) \)  for all \(y \in X_2\) . So, we see \begin{align*}
		\beta \mid_{X_1}:X_1	  &\longrightarrow  X_2 \\
		x&\longmapsto \beta(x) = \alpha\left( x \right)
	\end{align*}
and \begin{align*}
	\gamma \mid_{X_2}: X_2  &\longrightarrow  X_1 \\
	y&\longmapsto \gamma(y) = \alpha^{-1}\left( y \right)
\end{align*}
are inverses.\\
Hence, we have \(\beta\) and \(\gamma\) are a pair of inverse homomorphisms as \(X_1\)  generates \(F_1\)  and likewise \(X_2\)  generates \(F_2\) .\\
Then, for an arbitrary element in \(F\)  of the form \(x = x_1^{\epsilon_1}\ldots x_{\ell}^{\epsilon_{\ell}}\)  with \(\epsilon_{i} \in \Z\)  and \(x_{i} \in X_1\) , then we see \(\gamma(\beta\left(  x \right)) =   x\) , hence this completes the proof.
\end{proof}
\begin{theorem}
	Let \(F\)  be a free group with \(H, G\)  being groups. Suppose \(\alpha: F \to H\)  is a homomorphism and \(\beta: G \to H\)  is a surjective homomorphism. Then, there is a \(\gamma : F \to G\)  such that \(\beta \gamma = \alpha\) .
\end{theorem}
\begin{proof}
	Let \(F\)  be free on \( X \subseteq F\) . Then, each \(x \in X\)  has \(\alpha\left( x \right) \in H = \IM\left( \beta \right) \) . Then, there is some \(g_{x} \in G\)  such that \(\beta\left( g_{x} \right) = \alpha\left( x \right)  \) . By the universal mapping property of \(F\) , we have the map \(X \to G, x \mapsto g_{x}\) extends to a homomorphism \begin{align*}
		\gamma: F &\longrightarrow G \\
		x &\longmapsto \gamma(x) = g_{x}
	.\end{align*}
	Then, for \(x \in X\) we see \(\beta\left( \gamma\left( x \right)  \right)  = \beta\left( g_{x} \right) = \alpha\left( x \right)  \)  , so \(\beta \circ \gamma = \alpha\)  on \(X\)  which generates \(F\) , so \(\beta \circ \gamma = \alpha\)  on \(F\)  as \(\beta \circ \gamma, \alpha\)  are homomorphisms.
\end{proof}
