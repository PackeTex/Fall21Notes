\lecture{25}{Fri 22 Oct 2021 11:31}{Review of Test and Intro to Ring Theory}
\begin{proof}[Proof of question 6]
	Let \(C_{105} \rtimes_{\alpha} C_5\) 	and define \(\alpha: C_5 \to \aut\left( C_{105} \right) \). Recall, we need only show \(\alpha\) is the trivial homomorphism. Recall \( \aut\left( C_{105} \right) = C_2 \times C_4 \times C_6  \). Hence, \(\left| \aut\left( C_{105} \right)  \right| = 2 \cdot 4 \cdot 6 \) and as \(5 \nmid 2 \cdot 4 \cdot 6\), we see every element must map to \(1\).
\end{proof}
\section{Intro to Ring Theory}
\begin{definition}[Ring]
A \textbf{ring} \(R\) is a set equipped with two closed operations \(+\) and \(\times\) obeying the following properties
\begin{enumerate}
	\item \((R, +)\) forms an abelian group with additive identity, \(0\) .
	\item There is a multiplicative identity, \(1\) .
	\item The multiplicative operation is associative : \(\left( xy \right) z = x\left(yz  \right) \) for all \(x, y, z \in R\).
	\item The distributive properties hold: \(x\left( y+z \right) = xy + xz\) and \(\left( x + y \right) z = xz + yz\) for all \(x, y, z \in R\).
\end{enumerate}
A ring for which the multiplication operation is also commutative: \(xy = yx\), will be called a \textbf{commutative ring}.\\
	In general not every element \(x \in R\) has a multiplicative inverse. We define the special class of elements with inverses the \textbf{units} of \(R\) and we denote \(x^{-1}\) to denote the unique inverse of a unit \(x\).\\
	A (not necessarily commutative) ring in which every nonzero element is a unit is a \textbf{division ring}.
	A commutative ring for which every nonzero element is a unit is a \textbf{field}.
\end{definition}
\begin{remark}
	Technically, a ring need not have a multiplicative identity, but almost all of them will be equipped with one. Sometimes we denote a ring without identity to be a rng (no i).
\end{remark}
\begin{example}

\end{example}
