\lecture{18}{Mon 04 Oct 2021 11:28}{Solvable Groups (2)}
\begin{recall}
	A group is solvable if there exists a chain of subgroups \[
	\{1\} \trianglelefteq H_0 \trianglelefteq H_1 \trianglelefteq \ldots \trianglelefteq H_{n} = G
	\]  such that \(H_{i} / H_{i - 1}\) is abelian.\\
	We had that this is equivalent to the condition that for \(G^{\left( n \right) } = 1\) where \(G^{\left( 0 \right) } = G\) and \(G^{\left( n \right) } = \left[ G^{n-1}, G^{n-1} \right] \). We showed the forward implication, so now we show the reverse implication.
\end{recall}
\begin{proof}
	Suppose \(G^{\left( n \right) } = 1\)	 for some \(n \ge 0\). Then, we have a chain \[
		G = G^{\left( 0 \right) } \trianglelefteq G^{\left( 1 \right) } \trianglelefteq \ldots \trianglelefteq G^{\left( n \right) } = \{1\}
	.\]
	So, we have
	\[
		\{1\}  = G^{\left( n \right) } \trianglerighteq G^{\left( n-1 \right) } \trianglerighteq \ldots \trianglerighteq G^{\left( 0 \right) } = G
	.\]
	Furthermore,  we know the commutator of \(G^{\left( i \right) }\)is a characteristic subgroup, hence it is normal.\\
	Then, define \(H_{i} = G^{\left( n - i \right) }\) for \(0 \le i \le n\). We need only show the quotients to be abelian. We see \(H_{i} / H_{i - 1} = G^{\left(n - i \right) } / G^{\left( n - i + 1 \right) }\). But, \(G^{\left( n - i + 1 \right) } = \left[ G^{\left( n-i \right) }, G^{\left( n-i \right) } \right] \) by definition. Hence, \(G^{\left( n - i \right) } / G^{\left( n - i + 1 \right) }\) is abelian by the lemma from last class. So, the chain condition holds and \(G\) is solvable.
\end{proof}
\begin{theorem}
	Let \(G\) be a solvable group with \(H\) being a subgroup. Then, \(H\) is solvable.
\end{theorem}
\begin{proof}
We simply show \(H^{\left( n \right) } \le G^{\left( n \right) }\) for all \(n\) by induction. For the base case we know \(H = H^{\left( 0 \right) } \le G^{\left( 0 \right) } = G\). Then, we note \(H^{\left( n \right) } = \left[ H^{\left( n-1 \right) }, H^{\left( n-1 \right) }} \right] \subseteq \left[ G^{\left( n-1 \right) }, G^{\left( n-1 \right) } \right] = G^{\left( n \right) } \) by inductive hypothesis. Since \(G\) is solvable, we find a \(n \ge 0\) such that \(G^{\left( n \right) } = \{1\} \). Then, \(H^{\left( n \right) } \le G^{\left( n \right) }= \{1\} \), so \(H^{\left( n \right) } = \{1\} \) hence \(H\) is solvable.
\end{proof}
\begin{theorem}
	If \(G\) is solvable and \(\phi: G \to G^{\prime}\) is a homomorphism, then \(\phi\left( G \right) \) is also solvable.
\end{theorem}
\begin{proof}
We see \(\phi\left( G^{\left( 0 \right) } \right) = \phi\left( G \right) ^{\left( 0 \right) } \). So, \(\phi\left( G^{\left(  0\right) } \right) = \phi\left( G \right) ^{\left( 0 \right) } \). We induce on \(n\). We see \begin{align*}\phi\left( G^{\left( n \right) } \right) &= \phi \left( \left[ G^{\left( n-1 \right) }, G^{\left( n-1 \right) } \right]  \right)\\
&= \phi\left( \left<x^{-1}y^{-1}xy : x, y \in G^{\left( n-1 \right) } \right>  \right)  \\
&= \left<\phi\left( x^{-1}y^{-1}xy : x, y \in G^{\left( n-1 \right) } \right)  \right>  \\
&= \left< \phi\left( x \right) ^{-1} \phi\left( y \right) ^{-1} \phi\left( x \right) \phi\left( y \right)  : x, y \in G^{\left( n-1 \right) } \right> \\
&= \left< \overline{x}^{-1} \overline{y}^{-1} \overline{x} \overline{y} : \overline{x}, \overline{y} \in \phi\left( G^{\left( n-1 \right) } \right) \right>  \\
&= \left<\overline{x}^{-1}\overline{y}^{-1}\overline{x}\overline{y} : \overline{x}, \overline{y} \in\phi \left( G \right) ^{\left( n-1 \right) } \right> \text{ by the inductive hypothesis.}\\
&= \left[ \phi\left( G \right) ^{\left( n-1 \right) }, \phi\left( G \right) ^{\left( n-1 \right) } \right]  \\
&= \phi\left( G \right)^{\left( n \right) } .
\end{align*}
Since \(G\) is solvable, we find an \(n \ge 0\) such that \(G^{\left( n \right) } = \{1\} \). Hence, \(\phi\left( G^{\left( n \right) } \right) = \phi\left( \{1\}  \right) = \{1\}  = \phi\left( G \right) ^{\left( n \right) }  \), so \(\phi\left( G \right) \) is solvable.
\end{proof}
\begin{theorem}
	If \(G\) is a  group with \(H \trianglelefteq G\), then \(G\) is solvable if and only if \(H\) and \(G / H\) are solvable.
\end{theorem}
\begin{proof}
	\(\left( \implies \right) \). We know all subgroups and homomorphic images to be solvable, hence this direction is already proven.\\
	\(\left( \impliedby \right) \). Assume \(H\) and \(G / H\) are solvable. As \(H\) is solvable it has a normal chain \[
	H_0 \trianglelefteq H_1 \trianglelefteq \ldots \trianglelefteq H_{n} = H
	\]  with \(H_{i} / H_{i - 1}\) is abelian for all \(1 \le i \le n\). Similarly, since \(G / H\) is solvable there is a normal chain \[
	\{1\}  = K_{n + 0} \trianglelefteq K_{n + 1} \trianglelefteq \ldots K_{n + s} = G / H
	\]
	With \(K_{n + i} / K_{n + i - 1}\) being abelian for all \(i \ge 1\). We know by the lattice theorem that there are groups \(H_{n + i}\) such that \(K_{ n + i} = H_{n + i} / H\) for some \(H _{n + i} \le G\) and \(H \le H_{n + i}\). Then, we have \[
	\{1\}  = H / H \trianglelefteq H_{n + 1} / H \trianglelefteq \ldots \trianglelefteq H_{n + s} / H = G / H
	.\]
	Then, we have \(H_{n} = H\) and \(H_{n + s} = G\) and, as each contains the kernel, this correspondance preserves normality, hence we have \[
	H_{n} = H\trianglelefteq H_{n + 1} \trianglelefteq H_{n + 2} \trianglelefteq \ldots H_{n + s}  = G
	.\]
	Then, note that \(H_{ n + i} / H_{ n + i - 1} = \left( H_{n + i} / H \right) / \left( H_{n + i - 1} / H \right) = K_{n + i} / K_{n + i - 1} \) which we know to be abelian. Hence all successive quotients are abelian. So, \[
	\{1\}  = H_0 \trianglelefteq H_1 \trianglelefteq \ldots \trianglelefteq H_{n} \trianglelefteq H_{n + 1} \trianglelefteq H_{ n + 2} \trianglelefteq \ldots H_{n + s} = G
	.\]
	with \(H_{i} / H_{ i - 1}\) being abelian, so \(G\) is solvable.
\end{proof}
\begin{remark}
	Subgroups and quotients of nilpotent groups are nilpotent, but this converse does not hold in general for nilpotent groups.
\end{remark}

\section{Free Groups}
\begin{recall}
	\(\left<\alpha, \tau : \alpha ^{n} = 1, \tau^2 = 1, \tau \alpha \tau = \alpha ^{-1} \right>  = D_{2n}\) is the dihedral group of order \(2n\). This is technically ill defined. In general, we have generators \(\alpha, \tau\) and a set of relations that allow us to say when products of generators are equal. Similairly, we find \(\left<\alpha : \alpha ^{n}=1, \alpha ^{n + 1} = 1 \right> = \{1\}  \). We have not, however, ensured that these form groups. This problem motivates the definition of free groups.
\end{recall}
	If \(S\) is a set, then we let \(S^{-1}\) be a disjoint set of formal symbols with \(x \mapsto x^{-1}\)	, so \(S = \{a, b, c\} \) and \(S^{-1} = \{a^{-1}, b^{-1}, c^{-1}\} \). Then, let \(F\left( S \right) \) to be the set of all formal products of elements from \(S \cup S^{-1} \cup \{1\} \). Next class we will define an equivalence relation which takes these products into a group.
