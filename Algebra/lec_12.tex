\lecture{12}{Mon 20 Sep 2021 11:13}{Classification of Finite Groups}
\begin{recall}
	We showed that for a finite group \(G\) we could exploit the number of sylow \(p\)-groups, \(n_{p}\) to set up a congruence system with the only solution being \(n_{p} = 1\) for some \(p\), hence \(G\) was not simple (as \(n_{p} = 1\) guarantees the corersponding \(p\)-group to be normal). Failing this, we found we could assume a sylow \(p\)-group of order \(p\) had only trivial intersection to attain a lower bound on the size of the group which was larger than \(\left| G \right| \), implying once again that \(n_{p} =1\) for a particular \(p\), so \(G\) was not normal.
\end{recall}
We wish to continue this example to classify all possible groups of \(\left| G \right|  = 30\).\\
We had that either a sylow \(3\)-group, denoted \(P\), or a sylow \(5\)-group, denoted \(Q\),  must be normal, hence either \(P \trianglelefteq G\) or \(N\trianglelefteq G\) (with \(Q \leN_{G}\left( P \right)  = G\) or \(P \le N_{G}\left( Q \right) = G\)). Hence \( PQ\) is a group by the \(2\)nd homomorphism theorem.\\
Hence as \( P, Q \le PQ\), we have \(\left| P \right|  = 3 \mid \left| PQ \right| \) and \(\left| Q \right|  = 5 \mid \left| PQ \right| \), so \(15 \mid \left| PQ \right| \). Furthermore, as \(P \cap Q = \{1\} \) (all nonidentity elements of \(P\) have order \(3\), and all or \(Q\) have order \(5\)). As \(3 \mid 5 - 1\), then we know by an earlier theorem (a group of order \(pq\) with \(p \not\mid q-1\) is abelian) we have an abelian group. Hence \(PQ \simeq C_{15}\). Using cauchy's theorem yields an element \(t\) or order \(2\), then we have \(t \not\in PQ\) as \(PQ\) had no elements of even order. Hence, \(\left<PQ, t \right>  = G\).\\
Let \(H = \left<t \right> \simeq C_2\)  and let \(N = PQ \simeq C_{15}\). Clearly, \(N \trianglelefteq G\) and \(H \cap N = \{1\} \).\\
By another theorem from class, we have that \(G = HN \simeq N \rtimes_{\alpha} H\) by some automorphism \(\alpha : C_2 \to \aut \left( C_{15} \right) \).It remains only to determine what automorphisms \(\alpha\) are possible in this case. As \(C_2 = \{1, x\} \) for some \(x\) of order \(2\), then we see \(\alpha\) is completely characterized by the value of \(\alpha \left( x \right) \) and as \[
	\underbrace{\alpha\left( t^2 \right) }_{= \alpha \left( 1 \right)  = 1}  = \left( \alpha \left( t \right)  \right) ^2
\]  we see \(\ord \left( \alpha \left( t \right)  \right)  \mid 2\).\\
Now note that
\begin{align*}
	\aut \left( C_{15} \right)  &= \aut \left( C_3 \times C_5 \right) \\
				    & \simeq \aut \left( C_3 \right)  \times \aut \left( C_5 \right) \\
				    &=  C_2 \times C_4 \\
\end{align*}
and as there are \(4\) elements in \(C_2 \times C_4\) of order \(1\) or \(2\), we have at most \(4\) possible automorphisms \(\alpha\) (though some could give rise to isomorphic groups). It turns out that there are \(4\) such automorphisms, yielding nonisomorphic groups \(C_{30}, D_{30}, C_3 \times D_{10}, C_5 \times S_3\).\\
We now introduce a second trick for inducing normal subgroups by exploiting low-index subgroups.\\
\begin{proof}
Assume \(G\) is finite and \(H \le G\)	 with \(\left| G : H \right|  = k\), \(k\) being sufficiently small. Let \(G\) act on the left \(H\)-cosets by left multiplications. This is of course transitive as \(aH \mapsto bH\) by \(ba^{-1}\).\\
Let \(\alpha : G \to S_{k}\) be the associated homomorphism. If \(\ker \left( \alpha \right)  = G\), then there is a \(g \in G\) such that \(x^{g} = 1\) hence \(k =1\) by transitivity, hence \(\ker \left( \alpha \right)  = G \iff H = G\).\\
Similairly, if \(\ker \left( \alpha \right)  = \{1\} \), then \(\alpha\) is an injection. Thus, \(G \le S_{k}\)  up to isomorphism. Hence, knowledge of the subgroups of \(S_{k}\)  may yield that \(G \trianglelefteq S_{k}\), hence a contradiction. If we have a contradiction, then \(\{1\}  < \ker \left( \alpha \right)  < G\), so we have a nontrivial normal subgroup.\\
One easy way to exploit this is to compare \(\left| G \right| \) and \(\left| S_{k} \right| = k!\). Clearly, \(\left| G \right|  \mid k!\) or \(G \not\le S_{k}\). So, if \(\left| G \right| \mid k!\) we have the kernel is nontrivial so there is a proper nontrivial subgroup \(K = \ker \left( \alpha \right)  \trianglelefteq G\).
\end{proof}
\begin{example}
	Recall that \(n_{p} = \left| G :N_{G}\left( P \right)  \right| \) where \(P\) is a sylow \(p\)-group. Hence, if \(n_{p}\) is small (but larger than \(1\)), we can use \(N_{G}\left( P \right) \) to be our group of small index.
\end{example}
