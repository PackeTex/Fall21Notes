\section{Polynomial Rings}

\lecture{35}{Mon 15 Nov 2021 11:32}{Polynomials}
\begin{definition}[Polynomial Ring]
	Let \(R\) be a commutative ring and we define \(R\left[ X \right] \) to be the ring of polynomials in the variable \(x\) with coefficients from \(R\) defined as follows.\\
	An element \(f \in R\left[ X \right] \) has the form \[
	f = a_0 + a_1x + \ldots + a_{n} x^{n}
	\] for some \(n \ge 0\) and each \(a_{i} \in R\). This is a formal sum in the sense that two polynomials
	\begin{align*}
		f&= a_0 + a_1x + \ldots + a_{n}x^{n} \\
		g&= b_0 + b_1 x + \ldots b_{m} x^{m}  \\
	\end{align*} have \(f = g\) if and only if \(a_{i} = b_{i}\) for every \(i\).\\
For the polynomial \(f\), we call \(a_0\) the \textbf{constant term} and \(a_{n}\) to be the \textbf{leading coefficient} and \(n\) to be the \textbf{degree}, denoted \(\deg \left( f \right) = n\) .\\
For the polynomial \(f = 0\), we specifically define \(\deg \left(  f \right) = -1\). For all other constant polynomials \(g\), we define \(\deg \left( g \right) = 0\).
\end{definition}
\begin{remark}
	Occasionally, we will write \(f = \sum_{i=0}^{\infty} a_{i} x^{i}\) with almost every \(a_{i} = 0\). With this form we see elements of \(R\left[ X \right] \) are in a bijective correspondence with finite support tuples from \(R^{\N}\).
\end{remark}
We see \(R\left[ X \right] \) forms a ring with two polynomials \(f, g \in R\left[ X \right] \) as defined earlier having sum \[
	\left( f + g \right)  = \sum_{i=0}^{\infty} \left( a_{i} + b_{i} \right) x^{i}
\] and \[
fg = \sum_{i=0}^{\infty} a_{i} x^{i} \sum_{j=0 }^{\infty} b_{j}x^{j} = \sum_{n=0}^{\infty} \sum_{\underset{i + j = n}{i ,j}}^{}a_{i}b_{j} x^{n}
.\]
\begin{definition}[Multivariate Polynomial Rings]
We define a \textbf{multivariate polynomial ring}	\(R \left[ x_1, \ldots, x_{n} \right] = \left( R\left[ x_1, \ldots, x_{n-1} \right]  \right) \left[ x_{n} \right]  \) with addition and multiplication defined similarly. It is worth noting that while degree and constants are well defined, the leading coefficient may be poorly defined without adding extra constraints.
\end{definition}
\begin{definition}[Projected Degree]
	For a multivariate polynomial \(f \in R\left[ x_1, \ldots, x_{n} \right] \) we define \(\deg \left(  f \right) _{x_{i}}\) to be the degree when considered only in the variable \(x_{i}\).
\end{definition}
\begin{remark}
	It is of note that polynomials are more formal objects and not necessarily functions. The distinction is mostly moot, but we can induce a function from a polynomial by defining a function \begin{align*}
		f: R &\longrightarrow R \\
		b &\longmapsto f(b) = \sum_{i=0}^{\infty} a_{i}b^{i}
	.\end{align*}
	The point of this distinction is that polynomials over finite (or otherwise nonstandard spaces) may not be distinct. For example \(x \mapsto x^{5} - x\) and \(x \mapsto 0\) are completely equivalent in \(\mathbb{F}_5\). This, of course, cannot happen over \(\R\) unless the coefficients are precisely equal.
\end{remark}
We can construct a function in a different way as follows:
\begin{definition}[Evaluation Map]
	Fixing \(b \in R\) we define the \textbf{evaluation map} on \(R\left[ x \right] \)  as \begin{align*}
		\ev_{b}: R\left[ x \right]   &\longrightarrow  R\\
		f &\longmapsto \ev_{b}(f) = f\left( b \right)
	.\end{align*}
	We find this map to be a ring homomorphism, essentially compressing \(R\left[ x \right] \) down into \(R\).
\end{definition}
\begin{lemma}[Gauss Lemma]
	Let \(R\) be a UFD with \(K\) its quotient field. If \(f, g \in K\left[ x \right] \), then \(\cnt\left( fg \right) = \cnt\left( f \right) \cnt\left( g \right) \).
\end{lemma}
\begin{proof}
	Let \(c_1 = \cnt\left( f \right) \) , \(c_2 =\cnt\left( g \right) \). Then, \(f = c_1 f_1\) and \(g = c_2 g_1\) for some \(f_1, g_1 \in R\left[ x \right] \) with \(\cnt\left( f_1 \right)  = \cnt\left( g_1 \right)  = 1\). So, \(fg = \cnt\left( f \right) \cnt\left( g \right) f_1g_1\).\\
	Thus, it suffices to show \(\cnt\left( f_1g_1 \right) = 1 \).
\end{proof}
