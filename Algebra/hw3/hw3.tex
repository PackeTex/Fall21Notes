\documentclass[a4paper]{article}
% Some basic packages
\usepackage[utf8]{inputenc}
\usepackage[T1]{fontenc}
\usepackage{textcomp}
\usepackage{url}
\usepackage{graphicx}
\usepackage{float}
\usepackage{booktabs}
\usepackage{enumitem}

\pdfminorversion=7

% Don't indent paragraphs, leave some space between them
\usepackage{parskip}

% Hide page number when page is empty
\usepackage{emptypage}
\usepackage{subcaption}
\usepackage{multicol}
\usepackage{xcolor}

% Other font I sometimes use.
% \usepackage{cmbright}

% Math stuff
\usepackage{amsmath, amsfonts, mathtools, amsthm, amssymb}
% Fancy script capitals
\usepackage{mathrsfs}
\usepackage{cancel}
% Bold math
\usepackage{bm}
% Some shortcuts
\newcommand\N{\ensuremath{\mathbb{N}}}
\newcommand\R{\ensuremath{\mathbb{R}}}
\newcommand\Z{\ensuremath{\mathbb{Z}}}
\renewcommand\O{\ensuremath{\varnothing}}
\newcommand\Q{\ensuremath{\mathbb{Q}}}
\newcommand\C{\ensuremath{\mathbb{C}}}
% Easily typeset systems of equations (French package)

% Put x \to \infty below \lim
\let\svlim\lim\def\lim{\svlim\limits}

%Make implies and impliedby shorter
\let\implies\Rightarrow
\let\impliedby\Leftarrow
\let\iff\Leftrightarrow
\let\epsilon\varepsilon
\let\nothing\varnothing

% Add \contra symbol to denote contradiction
\usepackage{stmaryrd} % for \lightning
\newcommand\contra{\scalebox{1.5}{$\lightning$}}

 \let\phi\varphi

% Command for short corrections
% Usage: 1+1=\correct{3}{2}

\definecolor{correct}{HTML}{009900}
\newcommand\correct[2]{\ensuremath{\:}{\color{red}{#1}}\ensuremath{\to }{\color{correct}{#2}}\ensuremath{\:}}
\newcommand\green[1]{{\color{correct}{#1}}}

% horizontal rule
\newcommand\hr{
    \noindent\rule[0.5ex]{\linewidth}{0.5pt}
}

% hide parts
\newcommand\hide[1]{}

% Environments
\makeatother
% For box around Definition, Theorem, \ldots
\usepackage{mdframed}
\mdfsetup{skipabove=1em,skipbelow=0em}
\theoremstyle{definition}
\newmdtheoremenv[nobreak=true]{definition}{Definition}
\newmdtheoremenv[nobreak=true]{eg}{Example}
\newmdtheoremenv[nobreak=true]{corollary}{Corollary}
\newmdtheoremenv[nobreak=true]{lemma}{Lemma}[section]
\newmdtheoremenv[nobreak=true]{proposition}{Proposition}
\newmdtheoremenv[nobreak=true]{theorem}{Theorem}[section]
\newmdtheoremenv[nobreak=true]{law}{Law}
\newmdtheoremenv[nobreak=true]{postulate}{Postulate}
\newmdtheoremenv{conclusion}{Conclusion}
\newmdtheoremenv{bonus}{Bonus}
\newmdtheoremenv{presumption}{Presumption}
\newtheorem*{recall}{Recall}
\newtheorem*{previouslyseen}{As Previously Seen}
\newtheorem*{interlude}{Interlude}
\newtheorem*{notation}{Notation}
\newtheorem*{observation}{Observation}
\newtheorem*{exercise}{Exercise}
\newtheorem*{comment}{Comment}
\newtheorem*{practice}{Practice}
\newtheorem*{remark}{Remark}
\newtheorem*{problem}{Problem}
\newtheorem*{solution}{Solution}
\newtheorem*{terminology}{Terminology}
\newtheorem*{application}{Application}
\newtheorem*{instance}{Instance}
\newtheorem*{question}{Question}
\newtheorem*{intuition}{Intuition}
\newtheorem*{property}{Property}
\newtheorem*{example}{Example}
\numberwithin{equation}{section}
\numberwithin{definition}{section}
\numberwithin{proposition}{section}

% End example and intermezzo environments with a small diamond (just like proof
% environments end with a small square)
\usepackage{etoolbox}
\AtEndEnvironment{example}{\null\hfill$\diamond$}%
\AtEndEnvironment{interlude}{\null\hfill$\diamond$}%

\AtEndEnvironment{solution}{\null\hfill$\blacksquare$}%
% Fix some spacing
% http://tex.stackexchange.com/questions/22119/how-can-i-change-the-spacing-before-theorems-with-amsthm
\makeatletter
\def\thm@space@setup{%
  \thm@preskip=\parskip \thm@postskip=0pt
}


% \lecture starts a new lecture (les in dutch)
%
% Usage:
% \lecture{1}{di 12 feb 2019 16:00}{Inleiding}
%
% This adds a section heading with the number / title of the lecture and a
% margin paragraph with the date.

% I use \dateparts here to hide the year (2019). This way, I can easily parse
% the date of each lecture unambiguously while still having a human-friendly
% short format printed to the pdf.

\usepackage{xifthen}
\def\testdateparts#1{\dateparts#1\relax}
\def\dateparts#1 #2 #3 #4 #5\relax{
    \marginpar{\small\textsf{\mbox{#1 #2 #3 #5}}}
}

\def\@lecture{}%
\newcommand{\lecture}[3]{
    \ifthenelse{\isempty{#3}}{%
        \def\@lecture{Lecture #1}%
    }{%
        \def\@lecture{Lecture #1: #3}%
    }%
    \subsection*{\@lecture}
    \marginpar{\small\textsf{\mbox{#2}}}
}



% These are the fancy headers
\usepackage{fancyhdr}
\pagestyle{fancy}

% LE: left even
% RO: right odd
% CE, CO: center even, center odd
% My name for when I print my lecture notes to use for an open book exam.
% \fancyhead[LE,RO]{Gilles Castel}

\fancyhead[RO,LE]{\@lecture} % Right odd,  Left even
\fancyhead[RE,LO]{}          % Right even, Left odd

\fancyfoot[RO,LE]{\thepage}  % Right odd,  Left even
\fancyfoot[RE,LO]{}          % Right even, Left odd
\fancyfoot[C]{\leftmark}     % Center

\makeatother




% Todonotes and inline notes in fancy boxes
\usepackage{todonotes}
\usepackage{tcolorbox}

% Make boxes breakable
\tcbuselibrary{breakable}

% Verbetering is correction in Dutch
% Usage:
% \begin{verbetering}
%     Lorem ipsum dolor sit amet, consetetur sadipscing elitr, sed diam nonumy eirmod
%     tempor invidunt ut labore et dolore magna aliquyam erat, sed diam voluptua. At
%     vero eos et accusam et justo duo dolores et ea rebum. Stet clita kasd gubergren,
%     no sea takimata sanctus est Lorem ipsum dolor sit amet.
% \end{verbetering}
\newenvironment{correction}{\begin{tcolorbox}[
    arc=0mm,
    colback=white,
    colframe=green!60!black,
    title=Opmerking,
    fonttitle=\sffamily,
    breakable
]}{\end{tcolorbox}}

% Noot is note in Dutch. Same as 'verbetering' but color of box is different
\newenvironment{note}[1]{\begin{tcolorbox}[
    arc=0mm,
    colback=white,
    colframe=white!60!black,
    title=#1,
    fonttitle=\sffamily,
    breakable
]}{\end{tcolorbox}}


% Figure support as explained in my blog post.
\usepackage{import}
\usepackage{xifthen}
\usepackage{pdfpages}
\usepackage{transparent}
\newcommand{\incfig}[2][1]{%
    \def\svgwidth{#1\columnwidth}
    \import{./figures/}{#2.pdf_tex}
}

% Fix some stuff
% %http://tex.stackexchange.com/questions/76273/multiple-pdfs-with-page-group-included-in-a-single-page-warning
\pdfsuppresswarningpagegroup=1
\binoppenalty=9999
\relpenalty=9999

% My name
\author{Thomas Fleming}

\usepackage{pdfpages}
\title{Algebraic Theory I: Homework III}
\date{Thu 14 Oct 2021 11:07}
\DeclareMathOperator{\SRG}{SRG}
\DeclareMathOperator{\cut}{Cut}
\DeclareMathOperator{\GF}{GF}
\DeclareMathOperator{\V}{V}
\DeclareMathOperator{\E}{E}
\DeclareMathOperator{\edg}{e}
\DeclareMathOperator{\vtx}{v}
\DeclareMathOperator{\diam}{diam}

\DeclareMathOperator{\tr}{tr}
\DeclareMathOperator{\A}{A}

\DeclareMathOperator{\Adj}{Adj}
\DeclareMathOperator{\mcd}{mcd}

\begin{document}
\maketitle
\begin{solution}[1]
\begin{enumerate}
	\item First, we note that if \(xK = yK\) for some \(x \neq y\)  , then \(\overline{\phi}\left( xK \right) = \overline{x}H\) and \(\overline{\phi}\left( yK \right) = \overline{y}H\), hence we need to show \(\overline{x}H = \overline{y}H\). We see \(x^{-1}y K = K\), hence \(\overline{\phi}\left( x^{-1}yK \right) = \overline{x^{-1}y}H = \overline{x}^{-1} \overline{y} H = \overline{\phi}\left( 1K \right) = 1 H\). So, \(\overline{x}^{-1}\overline{y} \in H\), hence \(\overline{y} \in \overline{x}H\) and similairly, \(\overline{x} \in\overline{y}H\). So, \(\overline{x}H = \overline{\phi}\left( xK \right) = \overline{y}H = \overline{\phi}\left( yK \right) \), so \(\overline{\phi}\)  is well defined.\\
		Now,
		\begin{align*}
			\overline{\phi}\left( xK yK \right) &= \overline{\phi}\left( xyKK \right)  \\
			&=  \overline{xy} H\\
			&= \phi\left( xy \right) H\\
			&= \phi\left( x \right) \phi\left( y \right) H \\
			&= \overline{x} \overline{y} H\\
			&= \overline{x}\overline{y} H H \text{ as \(H = H H\) by closure }\\
		&= \overline{x} H \overline{y} H \\
		&= \overline{\phi}\left( xK \right) \overline{\phi}\left( yK \right)
		.\end{align*}
		Furthermore, we see \(\overline{\phi}\left( 1K \right) = \overline{1}H = 1H\) as \(\phi\left( 1 \right)  = \overline{1} = 1\)  by homomorphism properties.

			\item  First, note that \(Z_0 \left( \overline{G} \right) = \{1\} = \overline{Z_0\left( G \right) }\). Now, we induce on \(n\) and we see \(Z_{n-1}\left( G \right) \trianglelefteq G\)  and \(Z_{n-1}\left( \overline{G} \right) \trianglelefteq \overline{G}\) with \(\overline{Z_{n-1}\left( G \right) } \le Z_{n-1}\left( \overline{G} \right) \) by inductive hypothesis,  so \(\overline{\phi}: Z_{n}\left( G \right) / Z_{n-1}\left( G \right)  \to Z_{n}\left( \overline{G} \right) / Z_{n-1}\left( \overline{G} \right) \) is a well defined homomorphism. Hence, letting \(\overline{x} \in \overline{Z_{n}\left( G \right) }\), hence \(x \in Z_{n}\left( G \right) \) and hence \(xZ_{n-1}\left( G \right)  \in Z_{n}\left( G \right) / Z_{n-1}\left( G \right) \)  implies \(\overline{\phi}xZ_{n-1}\left( G \right)  = \overline{x}Z_{n-1}\left( \overline{G} \right) \in Z_{n}\left( \overline{G} \right) / Z_{n-1}\left( \overline{G} \right)  \) . Hence, we find \(\overline{x} \in Z_{n}\left( \overline{G} \right) \). This completes the induction.
			\item Suppose \(G\) is nilpotent and let \(n\) be the nilpotence class of \(G\). Then, we see \(\overline{Z_{n}\left( G \right) } = \overline{G} \le Z_{n}\left( \overline{G} \right) \). Hence, \(\overline{G}\) is of nilpotence class at most \(n\), so we see \(\overline{G}\) is nilpotent.
			\item Let \(\overline{G} = H\) with \(\phi: G \to H\)  being the restriction to \(H\) homomoprhism. That being \(\phi\left( x \right)  = \left \{
				\begin{array}{11}
					x, & \quad x \in H \\
					1, & \quad x \not\in H
				\end{array}
				\right.\) It is clear this is well defined
				\item Suppose \(n\) is the nilpotence class of \(G\). Then \(Z_{n}\left( G \right) \cap H = G \cap H = H \le Z_{n}\left( H \right) \), so \(H\) is of nilpotence class at most \(n\), hence \(H\) is nilptent.
\end{enumerate}
\end{solution}
\newpage
\begin{lemma}
	Automorphisms preserve maximality of subgroups.\\
	Let \(\phi :G \to G\)  be an automorphism and let \(M < G\)  be a maximal subgroup. Suppose \(\phi\left( M \right) = M^{\prime} \) is not maximal. That is, there is a \(\overline{M}^{\prime}\)  such that \(M^{\prime} < \overline{M}^{\prime} < G\). Then, we find
	\begin{align*}
		\phi ^{-1}\left( \overline{M}^{\prime} \right) &=  \phi ^{-1} \left( M^{\prime} \cup \left( \overline{M}^{\prime} \setminus M^{\prime} \right)\right)  \\
							       &= \phi ^{-1}\left( M^{\prime} \right) \cup \phi ^{-1}\left( \overline{M}^{\prime} \setminus M^{\prime}\right)  \\
							       &= M \cup \{\phi^{-1}\left( m \right) : m \in \overline{M}^{\prime} \setminus M^{\prime}\}  \\
							       &> M
	.\end{align*}
Furthermore, \(\overline{M}^{\prime} < G\)  by assumption, hence \(M < \overline{M}^{\prime} < G\). \(\lightning\) .
\end{lemma}
\begin{solution}[2]
	\begin{proof}
		Now, let \(\alpha: G \to G\)  be an automorphism of \(G\) and denote \(\alpha \left( M \right)  = M^{\prime}\) . Then, we see
\begin{align*}
\alpha \left( \Phi \left( G \right)  \right)  &= \alpha\left( \bigcap_{\underset{M \text{ is maximal}}{M < G}} M \right)  \\
						      &= \bigcap_{ \underset{ M \text{ is maximal}}{M < G}} \alpha \left( M \right)   \\
						      &= \bigcap_{\underset{M \text{ is maximal}}{M < G}} M^{\prime} \\
.\end{align*}
Then, as \(M^{\prime}\) is maximal and \(\alpha\) is an injection, we see if \(N \neq M\) are both maximal subgroups, we have \(\alpha \left( N \right)  \neq \alpha \left( M \right) \), hence \[\{ M :  \underset{ M \text{ is maximal}}{M < G}\}  = \{M^{\prime} :  \underset{ M \text{ is maximal}}{M < G}\} .\] So, we have \[
	\alpha\left( \Phi \left( G \right)  \right)  = \bigcap_{ \underset{ M \text{ is maximal}}{M < G}} M^{\prime} = \bigcap_{ \underset{ M \text{ is maximal}}{M < G}} M = \Phi\left( G \right)
.\]
\end{proof}
\end{solution}
\newpage
\begin{solution}[3]

\end{solution}
\newpage
\begin{solution}[4]
	\begin{lemma}
		\(\left[ M, M \right] \)  and \(\left<x^{p} : x \in M \right> \)  are characteristic in \(M\).\\
		Let \(\alpha: M \to M\) be an automorphism. Then, denote \(\alpha\left( x \right)  = x^{\prime}\)  for \(x \in M\) and we see,
		\begin{align*}
			\alpha\left( \left[ M, M \right]  \right) &= \alpha\left( \left<xyx^{-1}y^{-1} : x, y \in M \right> \right)  \\
								  &= \left<\alpha\left( xyx^{-1}y^{-1} \right) x, y \in M \right>  \\
								  &= \left<\alpha\left( x \right) \alpha\left( y \right) \alpha\left( x \right) ^{-1} \alpha\left( y \right)^{-1} : x, y \in M  \right>  \\
								  &= \left<x^{\prime} y^{\prime} x^{\prime}^{-1} y^{\prime}^{-1} : x, y \in M \right>  \\
								  &\le \left<x^{\prime}y^{\prime}x^{\prime}^{-1}y^{\prime}^{-1} : x^{\prime}, y^{\prime} \in  M \right>  \\
								  &=  \left[ M, M \right]
		.\end{align*}
		Similairly,
		\begin{align*}
			\alpha \left( \left<x^{p} : x\in M \right>  \right) &= \left<\alpha\left( x^{p} \right): x \in M  \right>  \\
									    &= \left<\alpha\left( x \right) ^{p} : x \in M \right>  \\
									    &= \left<x^{\prime}^{p} : x \in M \right>  \\
									    &\le \left<x^{\prime}^{p} : x^{\prime} \in M \right>
									    &= \left<x^{p} : x \in M \right>  \\
		.\end{align*}
Then, we see as \(M \trianglelefteq G\) and these two groups are characteristic we also have \(\left<x^{p}: x\in M \right> \trianglelefteq G \) and \(\left[ M, M \right] \trianglelefteq G\). Furthermore, we note that as \(xyx^{-1}y^{-1} \in M\)  for \(x, y \in M\) we have \(\left\{ M, M \right\}  \le \left<x^{p}: x\in M \right> \). Now,
Suppose \(M\) is not an elementary abelian \(p\)-group. Then, we find either \(\left[ M, M \right] > 1\)  or there is an element \(x\) of order \(q \neq p\).
	\end{lemma}
\end{solution}
\newpage
\begin{solution}[5]


\end{solution}
\end{document}
