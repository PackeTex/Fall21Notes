\documentclass[a4paper]{article}
% Some basic packages
\usepackage[utf8]{inputenc}
\usepackage[T1]{fontenc}
\usepackage{textcomp}
\usepackage{url}
\usepackage{graphicx}
\usepackage{float}
\usepackage{booktabs}
\usepackage{enumitem}

\pdfminorversion=7

% Don't indent paragraphs, leave some space between them
\usepackage{parskip}

% Hide page number when page is empty
\usepackage{emptypage}
\usepackage{subcaption}
\usepackage{multicol}
\usepackage{xcolor}

% Other font I sometimes use.
% \usepackage{cmbright}

% Math stuff
\usepackage{amsmath, amsfonts, mathtools, amsthm, amssymb}
% Fancy script capitals
\usepackage{mathrsfs}
\usepackage{cancel}
% Bold math
\usepackage{bm}
% Some shortcuts
\newcommand\N{\ensuremath{\mathbb{N}}}
\newcommand\R{\ensuremath{\mathbb{R}}}
\newcommand\Z{\ensuremath{\mathbb{Z}}}
\renewcommand\O{\ensuremath{\varnothing}}
\newcommand\Q{\ensuremath{\mathbb{Q}}}
\newcommand\C{\ensuremath{\mathbb{C}}}
% Easily typeset systems of equations (French package)

% Put x \to \infty below \lim
\let\svlim\lim\def\lim{\svlim\limits}

%Make implies and impliedby shorter
\let\implies\Rightarrow
\let\impliedby\Leftarrow
\let\iff\Leftrightarrow
\let\epsilon\varepsilon
\let\nothing\varnothing

% Add \contra symbol to denote contradiction
\usepackage{stmaryrd} % for \lightning
\newcommand\contra{\scalebox{1.5}{$\lightning$}}

 \let\phi\varphi

% Command for short corrections
% Usage: 1+1=\correct{3}{2}

\definecolor{correct}{HTML}{009900}
\newcommand\correct[2]{\ensuremath{\:}{\color{red}{#1}}\ensuremath{\to }{\color{correct}{#2}}\ensuremath{\:}}
\newcommand\green[1]{{\color{correct}{#1}}}

% horizontal rule
\newcommand\hr{
    \noindent\rule[0.5ex]{\linewidth}{0.5pt}
}

% hide parts
\newcommand\hide[1]{}

% Environments
\makeatother
% For box around Definition, Theorem, \ldots
\usepackage{mdframed}
\mdfsetup{skipabove=1em,skipbelow=0em}
\theoremstyle{definition}
\newmdtheoremenv[nobreak=true]{definition}{Definition}
\newmdtheoremenv[nobreak=true]{eg}{Example}
\newmdtheoremenv[nobreak=true]{corollary}{Corollary}
\newmdtheoremenv[nobreak=true]{lemma}{Lemma}[section]
\newmdtheoremenv[nobreak=true]{proposition}{Proposition}
\newmdtheoremenv[nobreak=true]{theorem}{Theorem}[section]
\newmdtheoremenv[nobreak=true]{law}{Law}
\newmdtheoremenv[nobreak=true]{postulate}{Postulate}
\newmdtheoremenv{conclusion}{Conclusion}
\newmdtheoremenv{bonus}{Bonus}
\newmdtheoremenv{presumption}{Presumption}
\newtheorem*{recall}{Recall}
\newtheorem*{previouslyseen}{As Previously Seen}
\newtheorem*{interlude}{Interlude}
\newtheorem*{notation}{Notation}
\newtheorem*{observation}{Observation}
\newtheorem*{exercise}{Exercise}
\newtheorem*{comment}{Comment}
\newtheorem*{practice}{Practice}
\newtheorem*{remark}{Remark}
\newtheorem*{problem}{Problem}
\newtheorem*{solution}{Solution}
\newtheorem*{terminology}{Terminology}
\newtheorem*{application}{Application}
\newtheorem*{instance}{Instance}
\newtheorem*{question}{Question}
\newtheorem*{intuition}{Intuition}
\newtheorem*{property}{Property}
\newtheorem*{example}{Example}
\numberwithin{equation}{section}
\numberwithin{definition}{section}
\numberwithin{proposition}{section}

% End example and intermezzo environments with a small diamond (just like proof
% environments end with a small square)
\usepackage{etoolbox}
\AtEndEnvironment{example}{\null\hfill$\diamond$}%
\AtEndEnvironment{interlude}{\null\hfill$\diamond$}%

\AtEndEnvironment{solution}{\null\hfill$\blacksquare$}%
% Fix some spacing
% http://tex.stackexchange.com/questions/22119/how-can-i-change-the-spacing-before-theorems-with-amsthm
\makeatletter
\def\thm@space@setup{%
  \thm@preskip=\parskip \thm@postskip=0pt
}


% \lecture starts a new lecture (les in dutch)
%
% Usage:
% \lecture{1}{di 12 feb 2019 16:00}{Inleiding}
%
% This adds a section heading with the number / title of the lecture and a
% margin paragraph with the date.

% I use \dateparts here to hide the year (2019). This way, I can easily parse
% the date of each lecture unambiguously while still having a human-friendly
% short format printed to the pdf.

\usepackage{xifthen}
\def\testdateparts#1{\dateparts#1\relax}
\def\dateparts#1 #2 #3 #4 #5\relax{
    \marginpar{\small\textsf{\mbox{#1 #2 #3 #5}}}
}

\def\@lecture{}%
\newcommand{\lecture}[3]{
    \ifthenelse{\isempty{#3}}{%
        \def\@lecture{Lecture #1}%
    }{%
        \def\@lecture{Lecture #1: #3}%
    }%
    \subsection*{\@lecture}
    \marginpar{\small\textsf{\mbox{#2}}}
}



% These are the fancy headers
\usepackage{fancyhdr}
\pagestyle{fancy}

% LE: left even
% RO: right odd
% CE, CO: center even, center odd
% My name for when I print my lecture notes to use for an open book exam.
% \fancyhead[LE,RO]{Gilles Castel}

\fancyhead[RO,LE]{\@lecture} % Right odd,  Left even
\fancyhead[RE,LO]{}          % Right even, Left odd

\fancyfoot[RO,LE]{\thepage}  % Right odd,  Left even
\fancyfoot[RE,LO]{}          % Right even, Left odd
\fancyfoot[C]{\leftmark}     % Center

\makeatother




% Todonotes and inline notes in fancy boxes
\usepackage{todonotes}
\usepackage{tcolorbox}

% Make boxes breakable
\tcbuselibrary{breakable}

% Verbetering is correction in Dutch
% Usage:
% \begin{verbetering}
%     Lorem ipsum dolor sit amet, consetetur sadipscing elitr, sed diam nonumy eirmod
%     tempor invidunt ut labore et dolore magna aliquyam erat, sed diam voluptua. At
%     vero eos et accusam et justo duo dolores et ea rebum. Stet clita kasd gubergren,
%     no sea takimata sanctus est Lorem ipsum dolor sit amet.
% \end{verbetering}
\newenvironment{correction}{\begin{tcolorbox}[
    arc=0mm,
    colback=white,
    colframe=green!60!black,
    title=Opmerking,
    fonttitle=\sffamily,
    breakable
]}{\end{tcolorbox}}

% Noot is note in Dutch. Same as 'verbetering' but color of box is different
\newenvironment{note}[1]{\begin{tcolorbox}[
    arc=0mm,
    colback=white,
    colframe=white!60!black,
    title=#1,
    fonttitle=\sffamily,
    breakable
]}{\end{tcolorbox}}


% Figure support as explained in my blog post.
\usepackage{import}
\usepackage{xifthen}
\usepackage{pdfpages}
\usepackage{transparent}
\newcommand{\incfig}[2][1]{%
    \def\svgwidth{#1\columnwidth}
    \import{./figures/}{#2.pdf_tex}
}

% Fix some stuff
% %http://tex.stackexchange.com/questions/76273/multiple-pdfs-with-page-group-included-in-a-single-page-warning
\pdfsuppresswarningpagegroup=1
\binoppenalty=9999
\relpenalty=9999

% My name
\author{Thomas Fleming}

\usepackage{pdfpages}
\title{Algebraic Theory I: Homework IV}
\date{Thu 18 Nov 2021 03:10}
\DeclareMathOperator{\SRG}{SRG}
\DeclareMathOperator{\cut}{Cut}
\DeclareMathOperator{\GF}{GF}
\DeclareMathOperator{\V}{V}
\DeclareMathOperator{\E}{E}
\DeclareMathOperator{\edg}{e}
\DeclareMathOperator{\vtx}{v}
\DeclareMathOperator{\diam}{diam}

\DeclareMathOperator{\tr}{tr}
\DeclareMathOperator{\A}{A}

\DeclareMathOperator{\Adj}{Adj}
\DeclareMathOperator{\mcd}{mcd}

\begin{document}
\maketitle
\begin{problem}[1]
	Let \(\left( X, \subseteq \right) \) be the set of all ideals not containing \(I\) partially ordered by inclusion.\\
	It suffices to show that for a totally ordered set \(\left( I_{\alpha} \right)_{\alpha \in \Omega} \) with ordered set \(\Omega\), and ideals \(I_{\alpha} \in X\) there is an upper bound. Take \(U = \bigcup_{\alpha \in \Omega} I_{\alpha}  \). \(U\) is the union of ideals so it is clearly an ideal. Suppose \(U \not\in X\), that is \(U \supseteq I\). Then, there is a subsequence \(\alpha_1, \ldots, \alpha_{n}\) and a permutation \(\pi\)  so that \[\left( x_{\pi\left( 1 \right) } \right) \subseteq I_{\alpha_1}, \left( x_{\pi\left( 1 \right) }, x_{\pi\left( 2 \right) } \right) \subseteq I_{\alpha_2},  \ldots, \left( x_{\pi \left( 1 \right) }, \ldots, x_{\pi(n)} \right) = I \subseteq I_{\alpha_{n}} \ \lightning .\] Hence, \(U \in X\), and \(U\) contains all \(I_{\alpha}\) so it is an upper bound. Hence, there is a maximal element \(M \in X\) by Zorn's Lemma.
\end{problem}
\newpage
\begin{problem}[2]
	\item First note that \(2, \sqrt{-D}, 1 + \sqrt{-D} \) are all non-units in \(R\) as their respective inverses in \(\C\) all have noninteger coefficients.\\
		 Then define \begin{align*}
			N: \Z\left[ \sqrt{-D}  \right]   &\longrightarrow \Z \\
			\left( a+bi \right) = x &\longmapsto  x \overline{x} = a^2 + b^2
		.\end{align*}
	Then
	\begin{align*}
		N\left( \left( a+bi \right) \left( c+di \right)  \right) &= \left[ \left( ac-bd \right) + \left( bc+ad \right) i \right] \left[ \left( ac-bd \right) - \left( bc+ad \right) i \right] \\
&= \left( ac-bd \right) ^2 + \left( bc+ad \right) ^2 \\
\text{ and } N\left( a+bi \right) N\left( c+di \right) &= \left( a^2+b^2 \right) \left( c^2 + d^2  \right)   \\
&= \left( ac \right) ^2 -2ac bd + \left( bd \right) ^2 + \left( ad \right) ^2 + 2ad bc  + \left( bc \right) ^2 \\
&= \left( ac-bd \right) ^2 + \left( bc + ad \right) ^2 \\
&= N\left( \left( a+bi \right) \left( c+di \right)  \right)  \\
	.\end{align*}
	In particular \(N\) is a ring homomorphism of \(R\).
	Next, suppose \(2\) is not irreducible in \(R\). Then, there are non-units \(x = a + b\sqrt{-D}, y=  c + d\sqrt{-D} \in R \) so that \( \left( a+b\sqrt{-D}  \right) \left( c+d\sqrt{-D}  \right) = 2 \). Passing to \(N\),
	\begin{align*}
		N\left( 2 \right) = 4 &=  N\left( x \right)N\left( y \right)   \\
		&= \left( a^2 + b^2 D \right) \left( c ^2 + d^2 D \right) \in \Z  \\
	.\end{align*}
	Since units pull back to units under homomorphisms, we can assume both of these quantities to be non-units. Hence the only possibility is \[
	a^2 + b^2 D = c ^2 + d^2 D = 2 \text{ (up to units)}
	.\]
	In this case \(D > 2\)  so we see \(b = d = 0\), hence either \(a=2\), \(c =1\) or \(a=1\), \(c=2\). In either case \(\lightning\), as \(x, y\) were assumed nonunits. Hence \(2\) is irreducible in \(R\).\\
	Now assume \(\sqrt{-D} \) non-irreducible in \(R\). Then, we find non-units \(x = a + b\sqrt{-D} , y = c + d\sqrt{-D}  \in R\) so that \(\sqrt{-D} = xy\). Passing to \(N\), we find
	\begin{align*}
		N\left( \sqrt{-D}  \right) = D &=  \left( a^2 + b^2D \right) \left( c^2 + d^2D \right)
	.\end{align*}
	If WLOG \(b =1\) , then we see \(a = d = 0\) and \(c = 1\)  \(\lightning\) as \(y\) is not a unit. If \(b > 1\) or \(d > 1\) , then \(b^2D > D\) so \(\lightning\). Hence \(b = d = 0\). Hence, \(D = a^2 c ^2\), but \(D\) was square-free \(\lightning\).\\
	Lastly, suppose \(1 + \sqrt{-D} \) is irreducible in \(R\). Then, we find non-units \(x = a+b\sqrt{-D} , y =  c + d\sqrt{-D}  \in R\) so that \(xy = 1 + \sqrt{-D} \). Hence \[
	N\left( 1 + \sqrt{-D}  \right) = 1+D = \left( a^2 + b^2D \right) \left( c^2 + d^2D \right)
	.\]
	If WLOG \( b = 1\), then \(d = 0\) otherwise \(1 + D > 2D^2\) \(\lightning\), and similarly \(c = 1\). Hence \(y\) is a unit \(\lightning\). So, \(1 + \sqrt{-D} \) is an irreducible.\\
	Now, note that the element \(D^2 + D\) has two distinct factorizations. First, it is again clear that \(D \pm \sqrt{-D} \) is a non-unit as its complex inverse has nonintegral coefficients. Then, we note \(D\left( D + 1 \right) = D^2 + D = \left( D + \sqrt{-D}  \right)\left( D - \sqrt{-D}  \right)  \). We see \(D, \left( D+1 \right) \) are not units and likewise  for \(\left( D \pm \sqrt{-D}   \right) \). Moreover, the factorizations are not pairwise associate, hence there are two factorizations for \(D^2 + D\), so \(Z\left[ \sqrt{-D}  \right] \) is not a UFD.
\end{problem}
\newpage
\begin{problem}[3]
Let \(I, J \subseteq R\) be ideals of a commutative ring \(R\). Then, let \(x, y \in IJ\) with \(x = \sum_{i= 1}^{n} a_{i}b_{i}\) and \(y = \sum_{i=n+1}^{m} a_{i} b_{i}\) for \(a_{i} \in I\) \(b_{i} \in J\), \(1 \le i \le m\). Then, we see \(x + y = \sum_{i= 1}^{m} a_{i} b_{i} \in IJ\).\\
Next, if \(r \in R\), and \(x \in I\) with \(x = \sum_{i= 1}^{n} a_{i} b_{i}\) for some \(a_{i} \in I\) \(b_{i} \in J\) , then \(rx = r \sum_{i= 1}^{n} a_{i} b_{i} = \sum_{i= 1}^{n} r a_{i} b_{i}\) with \(r a_{i} \in I\) by absorption property and \(b_{i} \in J\) by assumption for \(1 \le i \le n\)  . Hence \(rx \in IJ\), so \(IJ\) is an ideal.
\end{problem}
\newpage
\begin{problem}[4]
Let \(I\) be an ideal in \(R\) and \(I_i = \{x_i :  x\in I\} \subseteq R_i\) for each \(1 \le i \le n\). First, fix \(i\) and let \(r_{i} \in R_i\). Then, there is an \(\textbf{r}\in R\) so that \(\textbf{r}\) has \(r_{i}\) in its \(i\)'th coordinate. Hence, we see \(\textbf{r}\textbf{x} \in I\) for all \(\textbf{x} \in I\), so \(r_ix_i \in I_i\) for all \(x_i \in I_i\) by the pointwise multiplication. Similairly, fix \(i\) and let \(x_{i}, y_{i} \in I_{i}\). Then there are \(\textbf{x}, \textbf{y} \in I\) having \(x_{i}, y_{i}\) in their \(i\)'th coordinates respectively and \(\textbf{x} + \textbf{y}\in I\). Hence, \(x_{i} + y_{i} \in I_{i}\). So, each \(I_{i}\) is an ideal. Now, we show \(I\) to be the product of the \(I_{i}\)'s.\\
As each \(I_{i}\) is simply the projection of \(I\) into its \(i\)'th coordinate it is clear \(I \subseteq \prod_{i= 1}^{n} I_{i} \). Hence, let \(\textbf{x} = \left( x_1, \ldots, x_{n} \right) \in \prod_{i= 1}^{n} I_{i} \). Then, we see there are vectors \(\textbf{x}_1, \textbf{x}_2, \ldots, \textbf{x}_{n} \in I\) each having \(x_{i}\) in their \(i\)'th coordinates respectively and \(\textbf{x}_i \cdot j_{i} = \left( 0, \ldots, \underbrace{x_{i}}_{\text{position } i}, \ldots, 0  \right)  \in I \) for \(j_{i} \in R\) being the indicator vector in the \(i\)'th coordinate. Hence the sum \(\textbf{x} = \sum_{i= 1}^{n} \textbf{x}_{i} j_{i} \in I\) by closure of addition. So, equality holds.
\end{problem}
\newpage
\begin{problem}[5]
	\begin{enumerate}
		\item It is trivial that \(I \subseteq \sqrt{I} \) (taking \(n = 1\) for all \(x \in I\)). To show \(\sqrt{I} \) an ideal, let \(x_1, x_2 \in \sqrt{I} \) with \(x_1^{p} \in I\) and \(x_2^{p_2} \in I\). Then, there are \(a_0, a_1, \ldots, a_{p+q} \in R\) so that
			\begin{align*}
				\left( x_1 + x_2 \right) ^{p + q} &=  a_{p + q}x_1^{p+q} + a_{p+q-1}x_1^{p + q - 1} x_2^{1}  + \ldots + a_{p} x_1^{p}x_2^{q} + \ldots a_1 x_1^{1}x_2^{p + q - 1} + a_0 x_2^{p + q}
			.\end{align*} We know each term of this sum to be in \(I\) by the absorbition property of \(x_1^{p}\) and \(x_2^{q}\), hence the sum is in \(I\), so \(x_1 + x_2 \in \sqrt{I} \). Next,    let \(x \in R, a \in \sqrt{I} \) with \(a^{n} \in I\) , then \(\left( xa \right) ^{n} = x^{n}a^{n} \in I\) by absorption, so \(xa \in \sqrt{I} \), so \(\sqrt{I} \) is an ideal.
		\item Suppose \(\sqrt{I}  = R\). Then, \(1 \in \sqrt{I} \), hence \(1^{n} =1 \in I\), so \(I=  R\). Conversely, \(I = R \subseteq \sqrt{I} \) so the claim holds.
		\item Let \(M\) be a maximal ideal among inclusion and \(n \ge 1\). Then \(M \subseteq \sqrt{M} \) with \(\sqrt{M} \) being an ideal so either \(\sqrt{M} = R \) or \(\sqrt{M}  = M\)  if \(\sqrt{M}  = R\), \(\lightning\) by previous part, so \(\sqrt{M} = M\). Moreover, as \(M^{n}\subseteq M\) , we see \(\sqrt{M^{n}} \subseteq \sqrt{M} \). Hence, we need only show the reverse inclusion. Let \(x \in \sqrt{M} \). Then, \(x^{m} \in M\) for some \(m \ge 1\). Then, we see \(x^{mn} = \underbrace{x^{m} \cdot x^{m} \cdot \ldots \cdot x^{m}}_{n \text{ times}} \in M^{n}\), so \(x \in \sqrt{M^{n}\). Hence equality holds.
	\end{enumerate}
\end{problem}
\end{document}
