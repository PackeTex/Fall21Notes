\lecture{2}{Wed 25 Aug 2021 11:31}{Review of Group Theory Continued}
Let $\alpha: G_1 \to G_2$ be a homomorphism and $\beta: G_2 \to G_3$ be another homomorphism. Now, we define the map $\beta \alpha : G_1 \to G_3$ to be the homomorphism induced by the composition of $\alpha$ and $\beta$ so that $\left(\beta \alpha \left( x \right)\right) = \beta \left( \alpha \left( x \right)  \right) $. In the special case where $G_1 = G_2 = G_3$, we see $\alpha, \beta, \alpha\beta \in \aut\left( G \right) $.
\begin{proposition}
	If $G$ is a group, $H \le G$ and $ \phi: G \to G^{\prime}$, then the image $\phi \left( H \right) \le G^{\prime}$.
\end{proposition}
\begin{definition}[Cosets]
	The \textbf{left $H$-coset} is the set of the form \\$xH = \{xh : h \in H\} $. Similairly, the \textbf{right $H$-coset} is the set of the form $Hx = \{hx: h \in H\} $. We call the number of $H$-cosets of a group $G$ (this can be left or right cosets as the number is always equal) to be the \textbf{index of $H$ in $G$}. We denote this by $\left| G : H \right| = \frac{\left| G \right| }{\left| H \right| }$.
\end{definition}
\begin{remark}
	The left $H$-cosets partition $G$, that being, two cosets are either equal or disjoint and the union of all unique $H$-cosets covers $G$. Similairly for the right $H$-cosets. Hence, we have either $xH = yH$ or $xH \cap yH = \O$. We call $x$ a \textbf{representative} for the coset of $H$ and any element $xh \in xH$ is also a representative.
\end{remark}
\begin{definition}[Normal Groups]
	A subgroup $H\le G$ is called a \textbf{normal subgroup} of $G$ when $xHx^{-1}= H$ for all $x \in G$. This is equivalent to the statement $xH = Hx$ for all $x \in G$. We denote this relation by $H \trianglelefteq G$.
\end{definition}
\begin{remark}
	It is important to know this does not imply commutativity, simply that the sets themselves are equal, but there is not necessarily element-wise equality.
\end{remark}
\begin{definition}[Conjugation Map]
	For each $x \in G$ we can define the \textbf{conjugation map} by $x$ as $ d_{x}: G \to G$, $x \mapsto d_{x}(x) = xyx^{-1} $. This is an automorphism of $G$.
\end{definition}
\begin{remark}[Why are normal subgroups important?]
	If  $ \phi: G \to G$ is a homomorphism, then $\ker \left( \phi \right) = \{x \in G : \phi \left( x \right) = 1\} \trianglelefteq G$.
\end{remark}
\begin{definition}[Quotient Groups]
	We define the \textbf{quotient group} \\$G / H = \{xH : x \in G\} $. Normal groups allow us to define multiplication for this groups as the left and right cosets are equivalent. Thus, presuming $H \trianglelefteq G$ we have $\left( xH \right) \left( yH \right) \coloneqq \left( xyH \right) \in G / H$. We can think of the quotient $G / H$ as sending all elements of $H$ to the identity, or "modding" out by $H$.
\end{definition}
\begin{definition}[Normalizer]
	If $S \subseteq G$, then $N_{G}\left( S \right) = \{x \in G : xSx^{-1} = S\} \le G $. This is called the \textbf{normalizer subgroup} of $S$ in $G$. Generally, we assume $S$ is a subgroup. If $S$ is a subgroup, then $N_{G}\left( S \right) $ is the largest subgroup of $G$ in which $S$ is normal (though it is not necessarily normal in $G$). That is,  $H \trianglelefteq N_{G}\left( H \right) \le G$.
\end{definition}
\begin{definition}[Centralizer]
	We define the \textbf{centralizer subgroup} of $H$ in $G$ to be $Z_{G}\left( H \right) = \{x \in G: xh = hx \ \forall \ h \in H\}$. As this requires commuting element-wise instead of set-wise, we see $Z_{G}\left( H \right) \le N_{G}\left( H \right) \le G$. We call $Z_{G}\left( H \right) $ the \textbf{center} of $G$.
\end{definition}
\begin{notation}
	Sometimes $Z_{G}\left( H \right) = C_{G}\left( H \right) $ is used alternatively for the centralizer.
\end{notation}
\begin{definition}[Subgroup Generated by a Subset]
For $X \subseteq G$ we define $\left< X \right> \le G$ to be the \textbf{subgroup generated by X}. This is simply the smallest subgroup generated by $X$. It is clear to see $\left<X \right> = \{x_1\cdot x_2\cdot \ldots \cdot x_{n} : x_1, x_2, \ldots, x_{n} \in X \cup X^{-1}, n\ge 9\} $ where $X^{-1} = \{x: x^{-1} \in X\} $.
\end{definition}
\begin{definition}[Commutator]
	We define the \textbf{commutator subgroup} of $G$ to be $G^{\prime} = \left[ G : G \right] = \left<X \right> $ where $X = \{ghg^{-1}h^{-1} : g, h \in G\} $.
\end{definition}
\begin{remark}
	We call this the commutator because $G / G ^{\prime}$ is abelian. Furthermore, if $G / H$ is abelian for a subgroup $H \le G$, then $G^{\prime}\le H$. Hence, $G^{\prime}$ is the smallest subgroup which must be quotiented to induce an abelian group.
\end{remark}
With all of these definitions taken care of we may finally state the most powerful theorems of group theory, the $3$ isomoprhism theorems.
\begin{theorem}[The 3 (4) Isomorphism Theorems]
	\begin{enumerate}
		\item Let $ \phi: G \to G^{\prime}$ be a surjective homomoprhism, then $\ker \left( \phi \right) \trianglelefteq G$ and $G^{\prime}= \phi \left( G \right) \simeq G / \ker \left( \phi \right) $.
		\item Suppose $H, K \trianglelefteq G$ and $K \le H$. Then, we have $G / H \simeq \left( G / K \right) / \left( H / K \right) $.
		\item Let $H, K \le G$ and $H \le N_{G}\left( K \right) $. \\Then, $HK = \{hk: h \in H, k \in K\} \le G$. Moreover, $HK / K \simeq H / \left( H \cap K \right) $ (Presuming all terms are well defined, hence $K \trianglelefteq HK$ and $H \cap K \trianglelefteq H$).
		\item (Lattice Theorem) Suppose $ \phi: G \to G^{\prime}$ is a surjective homomorphism with \(\ker \left( \phi \right) = K\), then there is a bijective correspondance between subgroups of $G^{\prime}$ and subgroups of $G$ which contain $\ker \left( \phi \right) $. That is, if $K = \ker\left( \phi \right)$, then $H \mapsto H / K = \phi \left( H \right) $ and if $H \le G^{\prime}$ has $H \mapsto \phi^{-1} \left( H \right) \le G$ where $\ker\left( \phi \right) \subseteq \phi^{-1}\left( H \right) $. Furthermore, if we use  the first isomorphism theorem to write $G^{\prime} \simeq G / K$, then the subgroups of $G / K$ are $H / K$ with $K \le H \le G$. Finally, this correspondance preserves normality.
	\end{enumerate}
\end{theorem}
