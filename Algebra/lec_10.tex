\lecture{10}{Wed 15 Sep 2021 11:26}{Semidirect Products (2)}
\begin{recall}
	We introduced the semidirect product \(G \rtimes_{\alpha} H\) with \(\left( x, h \right) \left( y, g \right)  = \left( xy^{h}, hg \right) \).
\end{recall}
\begin{theorem}
	Let \(G\) be a group with \(H, N \le G\) and \(H \le N_{G}\left( N \right) \) and \(H \cap N = \{1\} \).. Then, \(NH \simeq N\rtimes_{\alpha} H\) is a group when \begin{align*}
		\alpha: H &\longrightarrow \aut \left( N \right)  \\
		h &\longmapsto \alpha(h) = \text{conjugation by \(h\)}
	.\end{align*}
\end{theorem}
\begin{proof}
	Since \(H \le N_{G}\left( N \right) \) this implies \(NH \le G\) with \(N \trianglelefteq NH\) (by the 2nd isomorphism theorem). Furthermore, \(\frac{NH}{N} \simeq \frac{H}{N \cap H}\). As the intersection is trivial, we see \(\left| NH : N \right|  = \frac{\left| NH \right| }{\left| N \right| } = \left| H \right| \), hence \(\left| NH \right|  = \left| N \right| \left| H \right| \). So, there are \(\left| H \right| \) \(N\)-cosets in \(NH\).\\
	But \(NH = \{xh : x \in N h \in H\}  = \bigcup_{h \in H} Nh \) and as there are \(\left| H \right| \) \(N\)-cosets, we see each \(Nh\) is distinct. Hence, every element has a unique representation of the form \(xh\) with \(x \in N\) and \(h \in H\). Thus, the map \(\phi : NH \to N \rtimes_{\alpha} H\), with \(xh \mapsto \left( x, h \right) \) is well defined (as there is only \(1\) way to represent each element) and bijective. Last, we must show it is a homomorphism. Let \(x_1h_1, x_2h_2 \in NH\)  be arbitrary elements with \(x_1, x_2 \in N\) and \(h_1, h_2 \in H\).\\
	Then
	\begin{align*}
		x_1h_1x_2h_2 &= x_1h_1x_2h_1^{-1}h_1h_2 \\
			     &= x_1 x_2^{h_1} \left( h_1h_2 \right)  \\
		\text{where } x^{h} &\coloneqq h x h^{-1} = \alpha \left( h \right) \left( x \right)  \\
		\text{furthermore, } x_1x_2^{h_1} &\in N \text{ and } h_1h_2 \in H\\
		\text{so, } x_1x_2h_1h_2 &= x_1x_2^{h_1}h_1h_2 \in NH.\\
		\text{Hence } x_1h_1x_2h_2 \mapsto \phi\left( x_1h_1x_2h_2 \right) &=  \left( x_1x_2^{h_1}, h_1h_2 \right) \\
										   &=  \left( x_1, h_1 \right) \left( x_2, h_2  \right)\\
										   &= \phi\left( x_1h_1 \right) \phi \left( x_2h_2 \right) .
	\end{align*}
We know \(G\) can act on itself by conjugation with \begin{align*}
	\alpha: G  &\longrightarrow  \aut \left( G \right) \\
	g &\longmapsto \alpha(g) = \text{ conjugation by \(g\)}
.\end{align*}
So, \(\alpha : H \to \aut \left( G \right) \) is also a homomorphism as each \(\alpha \left( h \right) \mid_{N}\) is an automorphism of \(N\) as \(N \trianglelefteq HN\) and \(H \le NH\) as \(H \le N_{G}\left( N \right) \).\\
Hence our original bijective map \(\phi\) is also a homomorphism, hence \(NH \simeq N \rtimes_{\alpha} H\).
\end{proof}
This implies the semidirect product, \(N \rtimes_{\alpha} H\) is completely characterized by
\begin{itemize}
	\item What is \(N\) isomorphic to?
		\item What is \(H\) isomorphic to?
		\item What possibilities for a homomorphism \(\alpha : H \to \aut \left( N \right) \) exist?
\end{itemize}
Hence semidirect products are a robust way to construct new nonabelian groups from a given \(N, H\).
\begin{example}
	\(D_{2n} \simeq C_{n} \rtimes_{\alpha} C_2\)
\end{example}
\begin{definition}[Simple Groups]
	A group \(G\) is \textbf{simple} if the only normal subgroups are \(\{1\} \) and \(G\) itself (It has no proper nontrivial normal subgroups).
\end{definition}
This definition clearly implies there are no nontrivial quotients of a simple group. The main use of simple groups is as a sort of "prime" group which allows us to decompose arbitrary groups by decomposition into simple groups by the quotient of a normal subgroup.
\begin{example}
	Finite groups of prime order \(\Z_{p}\) are simple. Furthermore, there are many families of finite simple groups as well as some particular sporadic groups which form the complete classification of finite simple groups.
\end{example}
