\lecture{26}{Mon 25 Oct 2021 11:31}{Ring Theory}
\begin{recall}
	A ring is a set, an abelian addition and an associative multiplication with identity.
\end{recall}
\begin{definition}[Subring]
	A \textbf{subring},  \(R^{\prime}\) of \(R\) is a subset \(R^{\prime} \subseteq R\) such that \(R^{\prime}\) is closed under its operations and \(1 \in R^{\prime}\).
\end{definition}
This object turns out to be mostly uninteresting, so we introduce the following concept.
\begin{definition}[Ideal]
	A \textbf{left ideal} of the ring \(R\) is a nonempty subset \(I \subseteq R\) so that \(I \le R\) under addition and \(rI \subseteq I\) for all \(r \in R\). This second condition is equivalent to for all \(x \in I\), \(r \in R	\)  \(\implies rx \in I\).\\
	\textbf{Right ideals} follow the same first condition and for the second condition we have \(Ir \subseteq I\) for all \(r \in R\).\
	A \textbf{(two-sided) ideal} is a set \(I\) which is both a left and a right ideal.
\end{definition}
\begin{example}
	\(I = p\Z\) is an ideal of \(\Z\).\\
\end{example}
Ideals will play a similar role as that of normal subgroups.
\begin{definition}[Ring Homomorphisms]
If \(R, R^{\prime}\) 	are rings and \(\psi: R \to R^{\prime}\) is a map. \(\psi\) is a \textbf{ring homomorphism} if
\begin{itemize}
	\item \(\psi\left( x + y \right) = \psi (x) + \psi (y) \) for all \(x, y \in R\),
	\item \(\psi\left( xy \right)  = \psi\left( x \right) \psi\left( y \right) \) for all \(x, y \in R\),
	\item \(\psi\left( 1_{R} \right) = 1_{R^{\prime}}\)  (if \(R, R^{\prime}\) are rings with identities).
\end{itemize}
A ring homomorphism which is a bijection is a \textbf{ring isomorphism}.
\end{definition}
\begin{example}
	If \(R = \Z / 6\Z\). Consider the map \(f: \Z / 6\Z \to \Z / 6\Z, \ x \mapsto 3x\). We see the first two conditions hold under standard modular arithmetic, but the identity condition clearly fails, so we would consider this a ring homomorphism of rings without identity, but it is not a homomorphism of rings with identity.
\end{example}
\begin{definition}
	If \(R\) is a ring and \(I \subseteq R\) is an ideal. \\ Then, we define \(R / I = \{x + I : x \in R\} \), with \(\left( x + I \right)  + \left( y + I \right) \coloneqq \left( x + y \right) + I\) and \(\left( x + I \right) \left( y + I \right)  \coloneqq xy + I\),   to be the \textbf{quotient ring} of \(R \mod I\).
\end{definition}
We see this operation to be well defined as \(x^{\prime} + I = x + I\) and \(y^{\prime} + I = x + I\) implies \(x^{\prime} + a = x\) and \(y^{\prime} + b = y\) for some \(a,b \in I\), so we find \(xy + I = \left( x^{\prime} + a \right) \left( y^{\prime} + b \right) + I = x^{\prime}y^{\prime} + x^{\prime} b + ay^{\prime} + ab + I = x^{\prime} y^{\prime} + I \) by the absorption property.
\begin{theorem}[1st Isomorphism Theorem for Rings]
If \(\psi: R \to R^{\prime}\) 	is a surjective ring homomorphism, then \(\ker \left( \psi \right) \) is a two-sided ideal in \(R\) and \(R / \ker \left( \psi \right) \simeq R^{\prime} \).
\end{theorem}
\begin{proof}
	First, we verify \(\ker \left( \psi \right) \) 	is an ideal. It is clearly an additive subgroup as \(\psi\) is an additive group homomorphism. Also, if \(x \in \ker \left( \psi \right) \) and \(r \in R\), we see \(\psi\left( x \right)  = 0\), hence
	\begin{align*}
		\psi\left( rx \right)  = \psi\left( r \right) \psi\left( x \right) &=  0 \\
		\psi\left( xr \right) = \psi\left( x \right) \psi\left( r \right) &= 0 \\
		\implies rx, xr &\in \ker \left( \psi \right)
	.\end{align*}
Hence, we find \(\ker \left( \psi \right) = I\) is an ideal. Now, take the map \(\begin{align*}
	R^{\prime}:  &\longrightarrow R / I  \\
	\psi\left( x \right)  &\longmapsto x + I
.\end{align*}\)
We wish to show this is well-defined, so we must show that \(\psi\left( x \right)  = \psi\left( x^{\prime} \right) \) produces the same coset. As it turns out, this is in fact well defined, so we need only show there is a bijective homomorphism. Clearly the map is surjective and
\begin{align*}
	xy &\mapsto xy + I\\
	x &\mapsto x + I\\
	y &\mapsto y + I\\
	\text{ and } \left( x + I \right) \left( y + I \right) = xy + I &\mapsto xy + I
.\end{align*}
Hence it is a homomorphism. Lastly, as this is an injective map at the group theory level, it is trivial to show injection holds. Hence \(R^{\prime} \simeq R / \ker \left( \psi \right) \).
\end{proof}
\begin{remark}
	It has yet to be formally stated, but \(0 \cdot x = 0\) for all \(x \in R\) as \(ax = ax\), hence \(\left( a - a \right) x = 0\) , so \(0 \cdot x = 0\) (and \(x \cdot 0 = 0\)).
\end{remark}
\begin{definition}
	If \(R\) is a ring with \(X \subseteq R\), then \(\left( X \right) \) is the smallest ideal containing \(X\). In other words, \[
		\left( X \right) = \bigcap_{\underset{I \text{ is an ideal}}{X \subseteq I \subseteq R}} I
	.\]
	Elements of \(\left( X \right) \) have the form \(\sum_{i=1}^{n} \prod_{j=1}^{m} x_{j_{i}}  \) for \(x_{i} \in X\) . That is, linear combinations of monomials with terms from \(X\).
\end{definition}
\begin{remark}
	The intersection of (right/left/two-sided) ideals is itself a (right/left/two-sided) ideal.
\end{remark}
