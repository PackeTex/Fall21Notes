\section{Nilpotent Groups}
\lecture{14}{Fri 24 Sep 2021 11:30}{Nilpotent Groups}
Let \(G\) be a group, and \(Z_0 \left( G \right)  = \{1\} \) with \(Z_1 \left( G \right)  = Z\left( G \right) \). Thus, \( G / Z_1\left( G \right) \) is a group which has \(Z\left( G / Z_1\left( G \right)  \right)  = \frac{Z_2 \left( G \right) }{Z_1 \left( G \right) }\) where \(Z_2\left( G \right) \) is the preimage of \( Z(G  / Z_1\left( G \right)) \), that being the subgroup of \(G\) containing \(Z_1 \left( G \right) \). We see we may continue
\begin{align*}
	Z_2 \left( G \right) / Z_1\left( G \right)  &=  Z\left( G / Z_1 \left( G \right)  \right)  \\
	\text{ then, } \left( G / Z_1\left( G \right)  \right)  / \left( Z_2 \left( G \right)  / Z_1 \left( G \right) \right) &\simeq G / Z_2\left( G \right)\\
	\text{which has a center } Z\left( G / Z_2\left( G \right)  \right) &=  Z_3 \left( G \right) / Z_2\left( G \right)
.\end{align*}
\begin{definition}[Nilpotence]
	We recursively define \(Z_{i} \left( G \right) \) to be the subgroup such that \(Z\left( G / Z_i \left( G \right)  \right) = Z_{i}\left( G \right)  / Z_{i-1} \left( G \right)  \). This yields a growing sequence \(Z_0 \left( G \right)  \trianglelefteq Z_1 \left( G \right)  \trianglelefteq Z_2\left( G \right)  \trianglelefteq \ldots\). We say a group \(G\) is \textbf{nilpotent} if \(G = Z_{n} \left( G \right) \) for some \(n\ge 0\). The minimal \(n\ge 0\) for which this is the case is called the \textbf{nilpotent class} of \(G\).
\end{definition}
\begin{example}
	The trivial group \(\{1\} \) is nilpotent with class \(c = 0\).\\
	A nontrivial abelian group is nilpotent with class \(c=1\).\\
\end{example}
\begin{theorem}
	Every finite \(p\)-group is nilpotent.
\end{theorem}
\begin{proof}
	We know the center of a nontrivial \(p\)-group to be nontrivial and its subgroups and quotient groups will also be \(p\)-groups. Hence \(Z_1\left( G \right) \) is nontrivial except in the case \(G\) is trivial. Hence we have that \(Z_2\left( G \right)  / Z_1\left( G \right) \) is nontrivial unless \(Z_2 \left( G \right)  = G\). Hence either \(Z_1 < Z_2\) or \(Z_2 = G\). Now, denote \(\left| G \right|  = n\). Then either \(1 = \left| Z_0 \right|  < \left| Z_1 \right|  < \ldots < \left| Z_{n} \right| \) hence \(Z_{n} = G\) or \(Z_{i} = G\) for some \(i < n\), so \(Z_{n} = G\). Hence, \(G\) is nilpotent.
\end{proof}
\begin{definition}
	A subgroup \(H \le G\) is \textbf{characteristic} if for every automorphism of \(G\), we have \(\alpha \left( H \right) = H  \). This is equivalent to \(\alpha\left( H \right)  \le H\) for all automorphisms as \(\alpha^{-1}: G \to G\) is also an automorphism, hence \(H \le \alpha \left( H \right) \), so equality holds. Since conjugation is always an automorphism, being characteristic implies normality.
\end{definition}
\begin{note}{Proving vs. Using Characteristicness}
	This means that in order to show that something is characteristic we need only show \(\alpha\left( H \right) \le H\), but when we use that something is characteristic we will often use the full equality.
\end{note}
\begin{lemma}
	As we know \(K \trianglelefteq H\) and \( H \trianglelefteq G\) does not imply \(K \trianglelefteq G\). On the other hand, \(K\) being characteristic in \(H\) and \(H\trianglelefteq G\) does yield \(K \trianglelefteq G\).
\end{lemma}
\begin{proof}
	Let \(\alpha_{x}: G \to G\) be the conjugation by \(x\) map. We know this to be an automorphism of \(G\), hence as \(H\) is normal, we have \(\alpha_{x}\mid_{H} : H \to H\) is an automorphism of \(H\), and since \(K\) is characteristic in \(H\), we see an automorphism of \(H\) fixed \(K\), hence \(\alpha_{x}\left( K \right) = xKx^{-1} = K \) for all \(x \in G\), hence \(K \trianglelefteq G\) .
\end{proof}
\begin{lemma}
	Let \(G\) be a finite group with \(p\) being prime and \(P\) being a sylow \(p\)-group in \(G\). Then, the following are equivalent
	\begin{enumerate}
		\item \(P\) is the unique sylow \(p\)-group in \(G\).
			\item \(P \trianglelefteq G\).
				\item \(P\) is characteristic in \(G\).
	\item Any subgroup generated by elements whose orders are each powers of \(p\) is itself a \(p\)-group.
	\end{enumerate}
\end{lemma}
\begin{proof}
	\begin{enumerate}
		\item We have already shown \(1 \iff 2\).
		\item As conjugation is always an automorphism, we see \(2 \impliedby 3\) is trivial.
		\item We show \(1 \implies 3\). Let \(\alpha: G \to G\) be an arbitrary automorphism of \(G\). Then, \(\alpha\left( P \right) \le G\) and \(\left| P \right|  = \left| \alpha\left( P \right)  \right| \). As \(P\) is the unique sylow \(p\)-group, we see there is no distinct group of cardinality \(\left| P \right| \), hence \(\alpha\left( P \right)  = P\).
		\item Now we show \(1 \implies 4\). Let \(X\) be a set satisfying \(\ord \left( x \right) = p^{n}\) for each \(x \in X\). Then each \(\left<x \right>\) is contained in a \(p\)-group, and as there is a unique maximal \(p\)-group, we have that \(\left<x \right> \subseteq P\) for each \( x\in X\). Hence, \(\left<X \right> \subseteq P\) and as \(X \) is a \(p\)-group we have that \(X = P\).
			\item \(4 \implies 1\). Let \(X\) to be the union of all sylow \(p\)-groups in \(G\). By hypothesis, \(\left<X \right> \) is a \(p\)-group and thus it is contained in some sylow \(p\)-group so WLOG, we have \(\left<X \right> \subseteq P\). But if there were distinct \(p\)-groups, \(P^{\prime} \neq P\) then \(P ^{\prime} \subseteq X\) and \(P \subset \left<P^{\prime} \cup P \right>  \subseteq X \subseteq P\). \(\lightning\). Hence \(P\) is the unique sylow \(p\)-group.
	\end{enumerate}
\end{proof}
