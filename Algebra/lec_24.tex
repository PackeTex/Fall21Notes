\lecture{24}{Mon 18 Oct 2021 18:06}{Summary of Group Theory}
\section{Summary of Group Theory}
This is a study guide for the midterm and not an actual lecture.
\subsection{Basic Group Theory}
\begin{theorem}[Isomorphism Theorems]
	The isomorphism theorems go roughly as follows:
\begin{itemize}
	\item Kernel's of surjective homomorphisms are normal subgroups.
	\item Quotients behave like division: \(\frac{G}{H} = \frac{\frac{G}{K}}{\frac{H}{K}}\) (if \(K \le H\)).
		\item Quotients "cancel" into simpler quotients: \(\frac{HK}{K} = \frac{H}{H\cap K}\).
			\item Quotients perserve group structure: Bijecetion between \(H \trianglelefteq G\) and \(\frac{H}{K}\trianglelefteq \frac{G}{K}\) if \(\ker \left( \phi \right) \trianglelefteq H\).
\end{itemize}
\end{theorem}

\begin{definition}
	We denote the following sets
	\begin{align*}
		G_{x} &= \{g \in G :  x^{g} = x  \} \\
		G_{X} &= \{g \in G : x^{g} = x \forall x \in X\}  \\
		N_{G}\left( X \right) &= \{y\in G : yXy^{-1} = X \}  \\
		Z_{G}\left( X \right) &= \{y \in G : yxy^{-1} = x  \forall \ x \in X\}  \\
		\left[ X, Y \right] &= \{xyx^{-1}y^{-1} : x \in X, y \in Y\}  \\
		\mathscr{O}_{X} = \{x^{g} : x \in X, g \in G\}
	.\end{align*}

\end{definition}
\begin{definition}[Group Action]
	A group \(G\) acts on \(\Omega\) by permuting its elements. Formally \(\alpha : G \to \perm \left( \Omega \right) \) such that each \(g\) permutes \(\Omega\). A special group action is the conjugation map \(x \mapsto yxy^{-1}\).
\end{definition}
\begin{remark}
	We need only check \(\left( x^{g} \right)^{h} = x^{hg} \) and \(x^{1} = 1\).
\end{remark}
\begin{definition}
	A group action is faithful if it has trivial kernel.
\end{definition}
\begin{theorem}
	\(G_{x^{g}} = gG_{x}g^{-1} \).
\end{theorem}
\begin{proof}
	Allude to definitions and take a change of variables to the conjugation.
\end{proof}
\begin{theorem}
	\(x^{g} = x^{h}\) if and only if \(x, y\) are in a common left \(G_{x}\)-coset.
\end{theorem}
\begin{proof}
	Show \(g \in hG_{x}\) by definitions.
\end{proof}
\begin{theorem}[Orbit-Stabilizer]
\(\left| \mathscr{O}_{x} \right|  = \left| G : G_{x} \right| \) .\\
\(\left| \Omega \right|  = \left| Z_{G}\left( G \right)  \right|  + \sum_{x \in C^{\prime}}^{} \left| G : Z_{G}\left( x \right)  \right| \).
\end{theorem}
\begin{proof}
	Take the map \(f: \{ gG_{x}: g \in G\} \to \Omega, \ x\mapsto f\left( gG_{x} \right) = x^{g}  \) and show its a bijection. For the second equation let the orbit be the whole set and peel of the first term of the summation.
\end{proof}
\subsection{P-groups}
\begin{definition}
	\(H\) and \(K\) are conjugate if \(K = gHg^{-1}\) for some \(g\). Note that the number of subgroups conjugate to \(H\) is \(\left| G : N_{G}\left( H \right)  \right| \) by appealing to definitions.
\end{definition}
\begin{theorem}
	A subgroup of index \(2\) is normal.
\end{theorem}
\begin{proof}
	Let \(G\) act on all conjugate subgroups by conjugation. It is trivial that \(N_{G}\left( H \right)  = H\)  or \(G\). \(G\) is proof and if it is \(H\) we see there are two conjugate subgroups \(\Omega = \{H, K\} \)  so there is a homomorphism into \(S_2\) and its kernel is \(H\).
\end{proof}
\begin{remark}
A subgroup of index of the smallest prime divisor of \(G\) is normal by the same argument.
\end{remark}
\begin{definition}
	A group is a \(p\)-group if the order of every element is \(p^{n}\). A subgroup is a sylow \(p\)-group if its order is the highest prime power of \(p\)  in \(\left| G \right| \).
\end{definition}
\begin{theorem}[Cauchy's Theorem]
	If \(p \mid \left| G \right| \) 	then there is a \(\ord \left( g \right)  = p\) (hence a subgroup of order \(p\)).
\end{theorem}
\begin{proof}
	There are two cases, the abelian and nonabelian.
	\begin{itemize}
		\item For the abelian case we proceed as follows:
		\item Let \(H = \left<x \right> \) and note  that if \(p \mid H\), then \(\ord \left( x^{\left| H \right|/p} \right) = p \), so such an element exists.
\item If \(p \nmid \left| H \right| \) , then appeal to the quotient group so \(p \mid \left| G / H \right| \) and define a homomorphism to the quotient where the IH guaranteed an element of order \(p\) which we can pullback.
\item For the nonabelian case we cite the class equation. If \(p \mid \left| Z\left( G \right)  \right| \), then appeal to the abelian case. Else, we find atleast one \(p \nmid \left| G : Z_{G}\left( x \right)  \right| \) by appealing to the class equation mod p. Then, we see \(p \mid \left| Z_{G}\left( x \right)  \right| \). If \(Z_{G}\left( x \right) \) is smaller than \(G\) we apply IH else we see if a point centralizer is \(G\) this implies that element is in \(Z\left( G \right) \), a contradiction.
	\end{itemize}
\end{proof}
\begin{theorem}
	A \(p\) group acting on a finite set has a number of fixed points congruent to \(\left| \Omega \right| \)  mod p.
\end{theorem}
\begin{proof}
	Seperate out all orbits of index \(\ge 2\) and note that \(\left| G : G_{x} \right|  = p^{m}\), and the congruency follows.
\end{proof}
\begin{theorem}
	A sylow \(p\)-group has \(H \le N_{G}\left( P \right) \implies H \le P\).
\end{theorem}
\begin{proof}
	Appeal to the 3rd isomorphism theorem to see \(\left| HP \right|/ \left| P \right|  = \left| H \right| / \left| H \cap P \right| \). Then, we sandwich \(\left| HP \right| \) between \(\left| P \right| \)  to induce the result.
\end{proof}
\begin{theorem}[Sylow's Theorem]
	\begin{itemize}
		\item  \(n_{p} \ge 1\).
			\item A \(p\)-group is contained in a sylow \(p\)-group.
				\item \(p\)-groups are conjugate.
				\item \(n_{p} \equiv 1 \mod p\)
				\item \(n_{p} = \left| G : N_{G}\left( P \right)  \right| \) hence \(n_{p}\mid \frac{\left| G \right| }{n^{p}}\)
	\end{itemize}
\end{theorem}
\begin{proof}
	\begin{itemize}
		\item 1 is already shown
		\item Let \(\Omega\) be the set of subgroups conjugate to \(P\) and \(G\) act by conjugation. \(G\) acts transitively, hence \(\left| \Omega \right|  = \left| G : G_{P} \right| \) Then, \(p \nmid \left| G : N_{G}\left( P \right)  \right| \). THen, restricting the action to \(H\) yields by an earlier lemma the number of fixed points a multiple of \(p\). Hence, there is some fixed point \(P^{\prime}\) which is conjugate to \(P\) and \(H \le P^{\prime}\).
		\item We find a \(P^{\prime}\) conjugate to \(P\) and we see \(P^{\prime} \le P\) but \(\left| P \right|  = \left| P^{\prime} \right| \) , so equality holds and we see the claim holds.
		\item As all \(p\)groups are conjuagte applying orbit stabilizer yields \(n_{p}= \left| \Omega \right|  = \left| G : G_{P} \right| = N_{G}\left( P \right) \)  hence \(n_{p}\equiv \left| \Omega \right| \mod p\). Letting \(P^{\prime}\)  be another \(P\) group which is fixed we see \(P^{\prime} = P\) and \(P \subseteq N_{G}\left( P^{\prime} \right) \) and \(P^{\prime} = P\) is the only fixed point so \(n_{p} \equiv 1 \mod p\).
	\end{itemize}
\end{proof}
\begin{theorem}
	A group of order \(p^2\) is abelian.
\end{theorem}
\begin{theorem}
	A nontrivial \(p\)-group admits a nontrivial \(Z\left( G \right) \).
\end{theorem}
\begin{proof}
	Appeal to the class equation to see \(p \mid \left| Z\left( G \right)  \right| \). As the center is nontrivial wee it has order \(p\) or \(p^2\). If \(\left| Z\left( G \right)  \right|  = p\) hence cylic hence \(G = Z\left( G \right) \cup G / Z\left( G \right) \). Then, we see generators \(x, Z\left( G \right) \) which commute, so \(G\) is abelian.
\end{proof}
\begin{theorem}
	If \(\left| G \right|  = pq\) \(p < q\)  and \(p \nmid q-1\) , then \(G\) is abelian.
\end{theorem}
\begin{proof}
	We see \(n_{p} = 1 = n_{q}\) by sylow's theorem, Hence every \(g \in G\) fixes \(P, Q\) by conjugation. Then, we see \(pq || \left| PQ \right| \) , so \(\left| PQ \right|  = G\)  THen appealing to the size of the subgroups and normality yields \(xy = yx^{\prime} = x^{\prime}y^{\prime} = xy \implies xy = yx\).
\end{proof}
\subsection{Semidirect products}
\begin{definition}
	\(\left( x, y \right) \left( a, b \right)  = (xa^{y}, b)\)
\end{definition}
\begin{remark}
	\(\left( x, y \right) ^{-1} = \left( \left( x^{-1} \right) ^{h^{-1}} , h^{-1}\right) \)
\end{remark}
\begin{theorem}
	If \(H \trianglelefteq N \rtimes_{\alpha} H\) , then \(\alpha = 1\)
\end{theorem}
\begin{proof}
	Examine \(\left( x, 1 \right) \left( 1, h \right) \left( x^{-1}, 1 \right) \) and we find \(\left( x^{-1} \right)^{h} = x^{-1}\)
\end{proof}
\begin{theorem}
	\(NH \simeq N\rtimes_{\alpha} H\) if \(\alpha: h \mapsto hxh^{-1}\).
\end{theorem}
\begin{proof}
	Appeal to 2nd isomorphism theorem and we see \(\frac{NH}{N} \simeq H\). So, we see there are \(\left| H \right| \) \(N\)-cosets in \(NH\). So every \(Nh\) is distinct. So, \(\alpha: xh \mapsto \left( x, h \right) \) is a bijective homomorphism. So they are isomorphic.
\end{proof}
\subsection{Simple Groups}
\begin{definition}[Simple Groups]
\(G\) is simple if it has no nontrivial proper normal subgroups.
\end{definition}
\begin{remark}
	Methods for Determining if a group is simple
	\begin{itemize}
		\item Counting elements of \(p\)-groups of power \(1\).
			\item Permutation representations.
				\item Small index subgroups.
					\item Playing \(p\)-groups off each other.
	\end{itemize}
\end{remark}
\begin{remark}
	Counting elements of \(p\)-groups of order \(1\) consists of finding sylow \(p\)-groups of order \(p^{1}\) and then it is clear all elements of the sylow \(p\)-groups must be distinct (except identity). Adding these up for all \(p\) yields a contradiction.
\end{remark}
\begin{remark}
	For small index subgroups we know a subgroup of index \(k\) implies \(G \simeq H \le S_{k}\). Hence, \(\left| G \right| \mid \left| S_{k} \right| \). Then, we know if \(k\) is the smallest integer such that \(\left| G \right| \mid k!\), then \(k\) is also the minimal index over all proper subgroups. From here we can induce a contradiction by appealing to sylows theorem.
\end{remark}
\begin{remark}
	For Permutation Representations we appeal to one of the following facts. If \(G\) has an element of order of \(k\) , then so does \(S_{k}\)  and if \(P\) is a sylow \(p\)-group of \(G\), then \(\left| N_{G}\left( P \right)  \right| \mid \left| N_{S_{k}}\left( P \right)  \right|  \). Then, we see the number of \(p\)-groups in \(S_{k}\) is \(\frac{\prod_{i=k-p+1}^{k} i}{p\left( p-1 \right) } \). Hence \(\left|N_{S_{k}}\left( p \right) \right| = p\left( p-1 \right)  \), so \(\left|N_{G}\left( P \right) \right| \mid p\left( p-1 \right) \).
\end{remark}
\begin{remark}
	For playing \(p\)-groups off of each other. Take a \(p\)-group in a \(p\)-group, for example \(P \le Q\) and force it to be normal. Then, it is eithere a \(P\)-group in \(G\) or its contained in one, \(P^{*}\)  (which is contained in \(N_{G}\left( P \right) \)). Hence, we  find \(\left<N_{G}(Q), P^{*} \right> \le N_{G}\left( P \right)  \), so \(\left| N_{G}\left( Q \right)  \right| \left| P^{*} \right|  \mid \left| N_{G}\left( P \right)  \right|  \). We can induce a contradiction from here.
\end{remark}
\subsection{Nilpotent Groups}
\begin{definition}
	The upper central series is \(Z_1\left( G \right)  = Z\left( G \right) \) , and \(Z_{n}\left( G \right) / Z_{n-1}\left( G \right)  = Z\left( G / Z_{n}\left( G \right)  \right) \). If this is \(G\) eventually, then \(G\) is nilpotent.\\
	Equivalently the lower central series is \(G^{1} = \left[ G, G \right] \), \(G^{n} = \left\{ G, G^{n-1} \right\} \). If this is trivial eventually, then \(G\) is nilpotent.
\end{definition}
\begin{theorem}
	Every finite \(p\)-group is nilpotent.
\end{theorem}
\begin{proof}
	We know the center of a \(p\)-grop is nontrivial. From here we show \(Z_1 < Z_2\) and induce up to the size of the group.
\end{proof}
\begin{definition}
	A subgroup \(H\)  is characteristic if every automorphism has \(\alpha\left( H \right)  \le H\).
\end{definition}
\begin{remark}
	\(K\trianglelefteq H\) and \(H\) characteristic is \(G\)  yields \(K\trianglelefteq G\).
\end{remark}
\begin{theorem}
	TFAE
	\begin{itemize}
		\item \(P\) is the unique sylow \(p\)-group in \(G\).
			\item \(P\trianglelefteq G\)
				\item \(P\) characteristic in \(G\).
					\item A subgroup generated by elements of order \(p^{i}\) is a \(p\)-group.
	\end{itemize}
\end{theorem}
\begin{proof}
	\begin{itemize}
		\item \(1 \iff 2\) is already shown and \(1 \implies 3\) follows as \(\alpha\left( P \right) \) is also a sylwo \(p\)-group.
\item \(1 \implies 4\) If \(X\) is such a group \(\left<x \right> \subseteq P\)  for all \(x\) so \(X \subseteq p\) is a \(p\)-group.
	\item \(4 \implies 1\) if they were not unique we have that such a group \(X\) would be \( P \subseteq \left< P \cup P^{\prime} \right>  \subseteq X \subseteq P\) so contradiction.
	\end{itemize}
\end{proof}
\begin{remark}
	If \(H, K\) are groups then \(Z\left( H \times K \right)  = Z\left( H \right) \times Z\left( K \right) \)
\end{remark}
\begin{proof}
	Appeal to definitions.
\end{proof}
\begin{theorem}
	For a homomorphism with \(\ker \left( \alpha \right)  = K \le H\), then \(N_{G}\left( H \right)  = f^{-1}\left( N_{G^{\prime}}\left( \phi\left( H \right)  \right)  \right) \).
\end{theorem}
\begin{proof}
	Appeal to homomorphism properties in both directions with \(x \in N_{G}\left( H \right) \) \(xHx^{-1}\)
\end{proof}
\begin{theorem}
	TFAE
	\begin{itemize}
		\item G is nilpotent
		\item Proper subgroups are proper in their normalizers
			\item All \(p\)-groups are normal
				\item \(G\) is the direct product of its sylow \(p\)-groups.
	\end{itemize}
\end{theorem}
\begin{proof}
	\begin{itemize}
		\item \(2\implies 3\) \(G\) must be abelian with a \(P\) not normal. Then as \(P\) is characteristic in \(N_{G}\left( P \right) \), we see its normal in \(N_{G}\left( N_{G}\left( P \right)  \right) \) so by definition the normalizers are equal. Hence we have a non normal \(P\)-group implies there is a subgroup not in its normalizer contradiction.
	\end{itemize}
\end{proof}
\begin{theorem}
If \(G\) has \(n \mid \left| G \right| \) with at most \(n\) \(x\) , \(x^{n}= 1\) , then \(G\) is cyclic.
\end{theorem}
\begin{proof}
	First, we see there are at most \(\left| P \right| = p^{\alpha}\) elements with \(x^{p^{\alpha}}=1\), so \(P\) must be distinct. So, all \(P\)-groups are normal \(G\) is the product of the \(P\)-groups. Then, we can show each \(P_{i}\) group is cyclic and the product of their generators is a generator of \(G\) as the primes are distinct.
\end{proof}
\begin{theorem}[Frattini Argument]
	If \(H \trianglelefteq G\) 	and \(P \le H\) is a sylow group of \(H\) , then \(G = HN_{G}\left( P \right) \).
\end{theorem}
\begin{proof}
	\(HN_{G}\left( P \right)  \le G\) by an earlier lemma so letting \(G\) act by conjugation yields \(P^{g} \le H\) so \(P^{g}\)  is a sylow \(p\)-group which is conjugate to \(P\), so there is a \(P^{h} = P^{g}\) and we find \(h^{-1}g \in N_{G}\left( P \right) \), so \(g \in hN_{G}\left( P \right) \). Appealing to third isomorphismtheorem yields \(\left| G:H \right|  \mid \left| N_{G}\left( P \right)  \right| \).
\end{proof}
\begin{theorem}
	\(G\) is nilpotent iff every maximal subgroup is normal.
\end{theorem}
\begin{proof}
	\(\implies\) If \(M\) is maximal  then \( M = N_{G}\left( M \right) \) or \(M\) is normal. If \(M = N_{G}\left( M \right) \) this is contradiction as nilpotent groups do not admit proper subgroups equal to their normalizer.
	\(\impliedby\) We need only show all sylow groups are normal. Take a maximal subgroup containing \(N_{G}\left( P \right) \). Applying frattini argument yields \(G = N_G\left( P \right) M\), so \(G \subseteq MM = M < G\) contradiction.
\end{proof}
\subsection{Solvable Groups}
\begin{definition}
	A group is solvable if it admite a normal chain \(H_0 \trianglelefteq H_1 \ldots \trianglelefteq H_{n} = G\) with the quotient of consecutive \(H_{i}\) being abelian.\\
	An equivalent characterization is the iterated commutator \(G^{\left( 1 \right) } = \left[ G, G \right] \) and \(G^{\left( n \right) } = \left[ G^{\left( n-1 \right) }, G^{\left( n-1 \right) } \right] \). If this is trivial at some point then \(G\) is solvable.
\end{definition}
\begin{proof}
	\(\implies\)  We show each \(G^{\left( i \right) } \le H_{i}\). Induce \(G^{\left( i \right) } \le H_{n-i} \) on \(i\) and the base case is trivial. For the i case note \(G^{\left( i \right) }\le \left[ H_{n-\left( i-1 \right) }, H_{n-\left( i-1 \right) } \right] \) and we get \(G^{\left( n \right)} \le H_{n-n} = \{1\}  \).\\
	\(\impliedby\). Let \(H_{i} = G^{\left( n-i \right) }\) and induce on \(i\) to show the quotient \(H_{i} / H_{i-1}\) is abelian as it is the quotient of a commutator..
\end{proof}
\begin{theorem}
	A subgroup of a solvable group is solvable.
\end{theorem}
\begin{proof}
	Induce to show \(H^{\left( n \right) } \le G^{\left( n \right) }\).
\end{proof}
\begin{theorem}
	Homomorphisms preserve solvability.
\end{theorem}
\begin{proof}
	Induce on \(G^{\left( i \right) }\) to show \(\phi\left( G^{\left( i \right) } \right) = \phi\left( G \right) ^{\left( i \right) } \)
\end{proof}
\begin{theorem}
	Let \(G\) and \(H \trianglelefteq G\) then \(G\) solvable iff \(H\) and \(G / H\) are solvable.
\end{theorem}
\begin{proof}
	\(\implies\) Already shown.
	\(\impliedby\). Take normal chains of \(H\) and \(G / H\) and append then to each other.
\end{proof}
\subsection{Free Groups}
\begin{definition}
	\(X\) is an alphabet, then \(F\left( X \right) \) is the free group on \(X\).
\end{definition}
\begin{theorem}[Universal Mapping Property]
	\(F\left( X \right) \) is a group \(F\) with an injection \(\sigma : X \xhookrightarrow F\) so that for any \(\alpha : X \to G\) there is a \(\beta : F\to G\) such that \(\beta\left( \sigma \right) = \alpha\).
\end{theorem}
\begin{theorem}
Use universal mapping property to induce bijective homomorphisms from \(F_1 \to F_2\) which is an extension of the assymed bijection \(\alpha: X_1 \to X_2\).
\end{theorem}
\begin{theorem}
	For \(\alpha: F\to H\) and \(\beta: G\to H\) , we find a \(\gamma : F \to G\) so that \(\beta \gamma = \alpha\).
\end{theorem}
\begin{proof}
	Let \(\beta\left( g_{x} \right)  = \alpha\left( x \right) \) for some \(g_{x}\) , then we find a homomorphism \(x \mapsto g_{x}\).
\end{proof}
\begin{definition}[Group Presentations]
	A group presentation is a set \(X\) and a set of relators \(Y\) such that \(\bigcap_{H \trianglelefteq G, H \ge Y} H = N\) yields a group \(F\left( X \right)  / N\) following the relations.
\end{definition}
\begin{remark}
	\(\{\prod_{i= 1}^{\ell} \left( g_{i}x_{i}g_{i}^{-1} \right): g_{i} \in G, g\times \in X \cup X^{-1}  \} \)
\end{remark}
\begin{theorem}
	If \(G = \left<X : R \right> \) and \(H=\left<X : R^{\prime} \right> \) with all relations in \(R\) being relations in \(R^{\prime}\) , then \(\alpha\left( G \right) = H\) for some \(\alpha\) homomorphism.
\end{theorem}
\begin{proof}
	\(N \le N^{\prime}\) so appealing to isomorphism theorems yields \(F\left( X \right) / N^{\prime} = G / \left( N^{\prime} / N \right) \).
\end{proof}
\begin{theorem}
	Every word is equivalent to a uniqe reduced word.
\end{theorem}
\begin{proof}
	For each letter define a map multiplying elements by \(m\) on the left. It is a permutation on the set of redued words hence each letter corresponds to a symmetry of \(R\) via a homormophism. Then for any two reduced words which are equivalent we find their representation in the symmetry group is the same, hence the words are the same.
\end{proof}
\begin{definition}
	\(V_{X}\left( w \right) = \) the sum of total powers of a letter in a word.
\end{definition}
\begin{definition}
	\(\rank\left( F\left( X \right)  \right) = \left| X \right| \).
\end{definition}
\begin{theorem}
	If \(F\left( X \right) \simeq F\left( Y \right) \) , then \(\left| X \right| = \left| Y \right|  \)
\end{theorem}
\begin{proof}
	Take a subgroup generated by squares and remark that it is characteristic hence normal. Then, we see \(G / H \simeq \phi\left( G \right) / \phi\left( F\left( X \right)  \right)  \simeq G^{\prime} / H^{\prime}\). Then as every elements square is \(1\) in \(G / H\) , so it is an abelian \(2\)-group. Then, we see all products of cosets are unique by multiplying any two and noting the multiplicity of elements versus the multiplicity of their generators.\\
	Hence, we find \(G / H = \bigoplus _{x \in X}\left<x \right> = \left( \Z / 2\Z \right)^{\left| X \right| }  \). This is a vector space over \(\mathbb{F}_2\) with elements corresponding to the power \(1\) or \(0\) of some \(\overline{x} \in X\). Then, we find the dimensions of \(G/ H\) and \(G^{\prime} / H^{\prime}\) are equal and as the dimensions are simply \(\left| X \right| \),   \(\left| X^{\prime} \right| \) this completes the proof.
\end{proof}
\begin{theorem}
	Subgroups of free groups are free.
	A subgroup of finite index, \(m\),  has \(\rank\left( H \right)  = \rank\left( F \right) m +1 - m\).
\end{theorem}
