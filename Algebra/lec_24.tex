\lecture{24}{Mon 18 Oct 2021 18:06}{Summary of Lectures Thus Far}
This is a study guide for the midterm and not an actual lecture.
\begin{theorem}[Isomorphism Theorems]
	The isomorphism theorems go roughly as follows:
\begin{itemize}
	\item Kernel's of surjective homomorphisms are normal subgroups.
	\item Quotients behave like division: \(\frac{G}{H} = \frac{\frac{G}{K}}{\frac{H}{K}}\) (if \(K \le H\)).
		\item Quotients "cancel" into simpler quotients: \(\frac{HK}{K} = \frac{H}{H\cap K}\).
			\item Quotients perserve group structure: Bijecetion between \(H \trianglelefteq G\) and \(\frac{H}{K}\trianglelefteq \frac{G}{K}\) if \(\ker \left( \phi \right) \trianglelefteq H\).
\end{itemize}
\end{theorem}

\begin{definition}
	We denote the following sets
	\begin{align*}
		G_{x} &= \{g \in G :  x^{g} = x  \} \\
		G_{X} &= \{g \in G : x^{g} = x \forall x \in X\}  \\
		N_{G}\left( X \right) &= \{y\in G : yXy^{-1} = X \}  \\
		Z_{G}\left( X \right) &= \{y \in G : yxy^{-1} = x  \forall \ x \in X\}  \\
		\left[ X, Y \right] &= \{xyx^{-1}y^{-1} : x \in X, y \in Y\}  \\
		\mathscr{O}_{X} = \{x^{g} : x \in X, g \in G\}
	.\end{align*}

\end{definition}
\begin{definition}[Group Action]
	A group \(G\) acts on \(\Omega\) by permuting its elements. Formally \(\alpha : G \to \perm \left( \Omega \right) \) such that each \(g\) permutes \(\Omega\). A special group action is the conjugation map \(x \mapsto yxy^{-1}\).
\end{definition}
\begin{remark}
	We need only check \(\left( x^{g} \right)^{h} = x^{hg} \) and \(x^{1} = 1\).
\end{remark}
\begin{definition}
	A group action is faithful if it has trivial kernel.
\end{definition}
\begin{theorem}
	\(G_{x^{g}} = gG_{x}g^{-1} \).
\end{theorem}
\begin{proof}
	Allude to definitions and take a change of variables to the conjugation.
\end{proof}
\begin{theorem}
	\(x^{g} = x^{h}\) if and only if \(x, y\) are in a common left \(G_{x}\)-coset.
\end{theorem}
\begin{proof}
	Show \(g \in hG_{x}\) by definitions.
\end{proof}
\begin{theorem}[Orbit-Stabilizer]
\(\left| \mathscr{O}_{x} \right|  = \left| G : G_{x} \right| \) .\\
\(\left| \Omega \right|  = \left| Z_{G}\left( G \right)  \right|  + \sum_{x \in C^{\prime}}^{} \left| G : Z_{G}\left( x \right)  \right| \).
\end{theorem}
\begin{proof}
	Take the map \(f: \{ gG_{x}: g \in G\} \to \Omega, \ x\mapsto f\left( gG_{x} \right) = x^{g}  \) and show its a bijection. For the second equation let the orbit be the whole set and peel of the first term of the summation.
\end{proof}
\begin{definition}
	\(H\) and \(K\) are conjugate if \(K = gHg^{-1}\) for some \(g\). Note that the number of subgroups conjugate to \(H\) is \(\left| G : N_{G}\left( H \right)  \right| \) by appealing to definitions.
\end{definition}
\begin{theorem}
	A subgroup of index \(2\) is normal.
\end{theorem}
\begin{proof}
	Let \(G\) act on all conjugate subgroups by conjugation. It is trivial that \(N_{G}\left( H \right)  = H\)  or \(G\). \(G\) is proof and if it is \(H\) we see there are two conjugate subgroups \(\Omega = \{H, K\} \)  so there is a homomorphism into \(S_2\) and its kernel is \(H\).
\end{proof}
\begin{remark}
A subgroup of index of the smallest prime divisor of \(G\) is normal by the same argument.
\end{remark}
\begin{definitions}
	A group is a \(p\)-group if the order of every element is \(p^{n}\). A subgroup is a sylow \(p\)-group if its order is the highest prime power of \(p\)  in \(\left| G \right| \).
\end{definitions}
\begin{theorem}[Cauchy's Theorem]
	If \(p \mid \left| G \right| \) 	then there is a \(\ord \left( g \right)  = p\) (hence a subgroup of order \(p\)).
\end{theorem}
\begin{proof}
	There are two cases, the abelian and nonabelian.
	\begin{itemize}
		\item For the abelian case we proceed as follows:
		\item Let \(H = \left<x \right> \) and note  that if \(p \mid H\), then \(\ord \left( x^{\left| H \right|/p} \right) = p \), so such an element exists.
\item If \(p \nmid \left| H \right| \) , then appeal to the quotient group so \(p \mid \left| G / H \right| \) and define a homomorphism to the quotient where the IH guaranteed an element of order \(p\) which we can pullback.
\item For the nonabelian case we cite the class equation. If \(p \mid \left| Z\left( G \right)  \right| \), then appeal to the abelian case. Else, we find atleast one \(p \nmid \left| G : Z_{G}\left( x \right)  \right| \) by appealing to the class equation mod p. Then, we see \(p \mid \left| Z_{G}\left( x \right)  \right| \). If \(Z_{G}\left( x \right) \) is smaller than \(G\) we apply IH else we see if a point centralizer is \(G\) this implies that element is in \(Z\left( G \right) \), a contradiction.
	\end{itemize}
\end{proof}
\begin{theorem}
	A \(p\) group acting on a finite set has a number of fixed points congruent to \(\left| \Omega \right| \)  mod p.
\end{theorem}
\begin{proof}
	Seperate out all orbits of index \(\ge 2\) and note that \(\left| G : G_{x} \right|  = p^{m}\), and the congruency follows.
\end{proof}
\begin{theorem}
	A sylow \(p\)-group has \(H \le N_{G}\left( P \right) \implies H \le P\).
\end{theorem}
\begin{proof}
	Appeal to the 3rd isomorphism theorem to see \(\left| HP \right|/ \left| P \right|  = \left| H \right| / \left| H \cap P \right| \). Then, we sandwich \(\left| HP \right| \) between \(\left| P \right| \)  to induce the result.
\end{proof}
\begin{theorem}[Sylow's Theorem]
	\begin{itemize}
		\item  \(n_{p} \ge 1\).
			\item A \(p\)-group is contained in a sylow \(p\)-group.
				\item \(p\)-groups are conjugate.
				\item \(n_{p} \equiv 1 \mod p\)
				\item \(n_{p} = \left| G : N_{G}\left( P \right)  \right| \) hence \(n_{p}\mid \frac{\left| G \right| }{n^{p}}\)
	\end{itemize}
\end{theorem}
\begin{proof}
	\begin{itemize}
		\item 1 is already shown
		\item Let \(\Omega\) be the set of subgroups conjugate to \(P\) and \(G\) act by conjugation. \(G\) acts transitively, hence \(\left| \Omega \right|  = \left| G : G_{P} \right| \) Then, \(p \nmid \left| G : N_{G}\left( P \right)  \right| \). THen, restricting the action to \(H\) yields by an earlier lemma the number of fixed points a multiple of \(p\). Hence, there is some fixed point \(P^{\prime}\) which is conjugate to \(P\) and \(H \le P^{\prime}\).
		\item We find a \(P^{\prime}\) conjugate to \(P\) and we see \(P^{\prime} \le P\) but \(\left| P \right|  = \left| P^{\prime} \right| \) , so equality holds and we see the claim holds.
		\item As all \(p\)groups are conjuagte applying orbit stabilizer yields \(n_{p}= \left| \Omega \right|  = \left| G : G_{P} \right| = N_{G}\left( P \right) \)  hence \(n_{p}\equiv \left| \Omega \right| \mod p\). Letting \(P^{\prime}\)  be another \(P\) group which is fixed we see \(P^{\prime} = P\) and \(P \subseteq N_{G}\left( P^{\prime} \right) \) and \(P^{\prime} = P\) is the only fixed point so \(n_{p} \equiv 1 \mod p\).
	\end{itemize}
\end{proof}
\begin{theorem}
	A group of order \(p^2\) is abelian.
\end{theorem}
\begin{theorem}
	A nontrivial \(p\)-group admits a nontrivial \(Z\left( G \right) \).
\end{theorem}
\begin{proof}
	Appeal to the class equation to see \(p \mid \left| Z\left( G \right)  \right| \). As the center is nontrivial wee it has order \(p\) or \(p^2\). If \(\left| Z\left( G \right)  \right|  = p\) hence cylic hence \(G = Z\left( G \right) \cup G / Z\left( G \right) \). Then, we see generators \(x, Z\left( G \right) \) which commute, so \(G\) is abelian.
\end{proof}
\begin{theorem}
	If \(\left| G \right|  = pq\) \(p < q\)  and \(p \nmid q-1\) , then \(G\) is abelian.
\end{theorem}
\begin{proof}
	We see \(n_{p} = 1 = n_{q}\) by sylow's theorem, Hence every \(g \in G\) fixes \(P, Q\) by conjugation. Then, we see \(pq || \left| PQ \right| \) , so \(\left| PQ \right|  = G\)  THen appealing to the size of the subgroups and normality yields \(xy = yx^{\prime} = x^{\prime}y^{\prime} = xy \implies xy = yx\).
\end{proof}
\begin{definition}
	\(\left( x, y \right) \left( a, b \right)  = (xa^{y}, b)\)
\end{definition}
\begin{remark}
	\(\left( x, y \right) ^{-1} = \left( \left( x^{-1} \right) ^{h^{-1}} , h^{-1}\right) \)
\end{remark}
\begin{theorem}
	If \(H \trianglelefteq N \rtimes_{\alpha} H\) , then \(\alpha = 1\)
\end{theorem}
\begin{proof}
	Examine \(\left( x, 1 \right) \left( 1, h \right) \left( x^{-1}, 1 \right) \) and we find \(\left( x^{-1} \right)^{h} = x^{-1}\)
\end{proof}
\begin{theorem}
	\(NH \simeq N\rtimes_{\alpha} H\) if \(\alpha: h \mapsto hxh^{-1}\).
\end{theorem}
