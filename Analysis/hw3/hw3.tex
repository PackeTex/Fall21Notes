\documentclass[a4paper]{article}
% Some basic packages
\usepackage[utf8]{inputenc}
\usepackage[T1]{fontenc}
\usepackage{textcomp}
\usepackage{url}
\usepackage{graphicx}
\usepackage{float}
\usepackage{booktabs}
\usepackage{enumitem}

\pdfminorversion=7

% Don't indent paragraphs, leave some space between them
\usepackage{parskip}

% Hide page number when page is empty
\usepackage{emptypage}
\usepackage{subcaption}
\usepackage{multicol}
\usepackage{xcolor}

% Other font I sometimes use.
% \usepackage{cmbright}

% Math stuff
\usepackage{amsmath, amsfonts, mathtools, amsthm, amssymb}
% Fancy script capitals
\usepackage{mathrsfs}
\usepackage{cancel}
% Bold math
\usepackage{bm}
% Some shortcuts
\newcommand\N{\ensuremath{\mathbb{N}}}
\newcommand\R{\ensuremath{\mathbb{R}}}
\newcommand\Z{\ensuremath{\mathbb{Z}}}
\renewcommand\O{\ensuremath{\varnothing}}
\newcommand\Q{\ensuremath{\mathbb{Q}}}
\newcommand\C{\ensuremath{\mathbb{C}}}
% Easily typeset systems of equations (French package)

% Put x \to \infty below \lim
\let\svlim\lim\def\lim{\svlim\limits}

%Make implies and impliedby shorter
\let\implies\Rightarrow
\let\impliedby\Leftarrow
\let\iff\Leftrightarrow
\let\epsilon\varepsilon
\let\nothing\varnothing

% Add \contra symbol to denote contradiction
\usepackage{stmaryrd} % for \lightning
\newcommand\contra{\scalebox{1.5}{$\lightning$}}

 \let\phi\varphi

% Command for short corrections
% Usage: 1+1=\correct{3}{2}

\definecolor{correct}{HTML}{009900}
\newcommand\correct[2]{\ensuremath{\:}{\color{red}{#1}}\ensuremath{\to }{\color{correct}{#2}}\ensuremath{\:}}
\newcommand\green[1]{{\color{correct}{#1}}}

% horizontal rule
\newcommand\hr{
    \noindent\rule[0.5ex]{\linewidth}{0.5pt}
}

% hide parts
\newcommand\hide[1]{}

% Environments
\makeatother
% For box around Definition, Theorem, \ldots
\usepackage{mdframed}
\mdfsetup{skipabove=1em,skipbelow=0em}
\theoremstyle{definition}
\newmdtheoremenv[nobreak=true]{definition}{Definition}
\newmdtheoremenv[nobreak=true]{eg}{Example}
\newmdtheoremenv[nobreak=true]{corollary}{Corollary}
\newmdtheoremenv[nobreak=true]{lemma}{Lemma}[section]
\newmdtheoremenv[nobreak=true]{proposition}{Proposition}
\newmdtheoremenv[nobreak=true]{theorem}{Theorem}[section]
\newmdtheoremenv[nobreak=true]{law}{Law}
\newmdtheoremenv[nobreak=true]{postulate}{Postulate}
\newmdtheoremenv{conclusion}{Conclusion}
\newmdtheoremenv{bonus}{Bonus}
\newmdtheoremenv{presumption}{Presumption}
\newtheorem*{recall}{Recall}
\newtheorem*{previouslyseen}{As Previously Seen}
\newtheorem*{interlude}{Interlude}
\newtheorem*{notation}{Notation}
\newtheorem*{observation}{Observation}
\newtheorem*{exercise}{Exercise}
\newtheorem*{comment}{Comment}
\newtheorem*{practice}{Practice}
\newtheorem*{remark}{Remark}
\newtheorem*{problem}{Problem}
\newtheorem*{solution}{Solution}
\newtheorem*{terminology}{Terminology}
\newtheorem*{application}{Application}
\newtheorem*{instance}{Instance}
\newtheorem*{question}{Question}
\newtheorem*{intuition}{Intuition}
\newtheorem*{property}{Property}
\newtheorem*{example}{Example}
\numberwithin{equation}{section}
\numberwithin{definition}{section}
\numberwithin{proposition}{section}

% End example and intermezzo environments with a small diamond (just like proof
% environments end with a small square)
\usepackage{etoolbox}
\AtEndEnvironment{example}{\null\hfill$\diamond$}%
\AtEndEnvironment{interlude}{\null\hfill$\diamond$}%

\AtEndEnvironment{solution}{\null\hfill$\blacksquare$}%
% Fix some spacing
% http://tex.stackexchange.com/questions/22119/how-can-i-change-the-spacing-before-theorems-with-amsthm
\makeatletter
\def\thm@space@setup{%
  \thm@preskip=\parskip \thm@postskip=0pt
}


% \lecture starts a new lecture (les in dutch)
%
% Usage:
% \lecture{1}{di 12 feb 2019 16:00}{Inleiding}
%
% This adds a section heading with the number / title of the lecture and a
% margin paragraph with the date.

% I use \dateparts here to hide the year (2019). This way, I can easily parse
% the date of each lecture unambiguously while still having a human-friendly
% short format printed to the pdf.

\usepackage{xifthen}
\def\testdateparts#1{\dateparts#1\relax}
\def\dateparts#1 #2 #3 #4 #5\relax{
    \marginpar{\small\textsf{\mbox{#1 #2 #3 #5}}}
}

\def\@lecture{}%
\newcommand{\lecture}[3]{
    \ifthenelse{\isempty{#3}}{%
        \def\@lecture{Lecture #1}%
    }{%
        \def\@lecture{Lecture #1: #3}%
    }%
    \subsection*{\@lecture}
    \marginpar{\small\textsf{\mbox{#2}}}
}



% These are the fancy headers
\usepackage{fancyhdr}
\pagestyle{fancy}

% LE: left even
% RO: right odd
% CE, CO: center even, center odd
% My name for when I print my lecture notes to use for an open book exam.
% \fancyhead[LE,RO]{Gilles Castel}

\fancyhead[RO,LE]{\@lecture} % Right odd,  Left even
\fancyhead[RE,LO]{}          % Right even, Left odd

\fancyfoot[RO,LE]{\thepage}  % Right odd,  Left even
\fancyfoot[RE,LO]{}          % Right even, Left odd
\fancyfoot[C]{\leftmark}     % Center

\makeatother




% Todonotes and inline notes in fancy boxes
\usepackage{todonotes}
\usepackage{tcolorbox}

% Make boxes breakable
\tcbuselibrary{breakable}

% Verbetering is correction in Dutch
% Usage:
% \begin{verbetering}
%     Lorem ipsum dolor sit amet, consetetur sadipscing elitr, sed diam nonumy eirmod
%     tempor invidunt ut labore et dolore magna aliquyam erat, sed diam voluptua. At
%     vero eos et accusam et justo duo dolores et ea rebum. Stet clita kasd gubergren,
%     no sea takimata sanctus est Lorem ipsum dolor sit amet.
% \end{verbetering}
\newenvironment{correction}{\begin{tcolorbox}[
    arc=0mm,
    colback=white,
    colframe=green!60!black,
    title=Opmerking,
    fonttitle=\sffamily,
    breakable
]}{\end{tcolorbox}}

% Noot is note in Dutch. Same as 'verbetering' but color of box is different
\newenvironment{note}[1]{\begin{tcolorbox}[
    arc=0mm,
    colback=white,
    colframe=white!60!black,
    title=#1,
    fonttitle=\sffamily,
    breakable
]}{\end{tcolorbox}}


% Figure support as explained in my blog post.
\usepackage{import}
\usepackage{xifthen}
\usepackage{pdfpages}
\usepackage{transparent}
\newcommand{\incfig}[2][1]{%
    \def\svgwidth{#1\columnwidth}
    \import{./figures/}{#2.pdf_tex}
}

% Fix some stuff
% %http://tex.stackexchange.com/questions/76273/multiple-pdfs-with-page-group-included-in-a-single-page-warning
\pdfsuppresswarningpagegroup=1
\binoppenalty=9999
\relpenalty=9999

% My name
\author{Thomas Fleming}

\usepackage{pdfpages}
\title{Analysis I: Homework III}
\date{Fri 10 Sep 2021 12:58}
\DeclareMathOperator{\SRG}{SRG}
\DeclareMathOperator{\cut}{Cut}
\DeclareMathOperator{\GF}{GF}
\DeclareMathOperator{\V}{V}
\DeclareMathOperator{\E}{E}
\DeclareMathOperator{\edg}{e}
\DeclareMathOperator{\vtx}{v}
\DeclareMathOperator{\diam}{diam}

\DeclareMathOperator{\tr}{tr}
\DeclareMathOperator{\A}{A}

\DeclareMathOperator{\Adj}{Adj}
\DeclareMathOperator{\mcd}{mcd}

\begin{document}
\maketitle
\begin{problem}[14]
	Let \(\left( x_{n} \right) \)	 be a sequence. A point \(x^{*}\) is called an accumulation point of \(\left( x_{n} \right) \) if for each \(\epsilon ? 0\) and each \(N \in \N\) there is a \(n \in \N\) with \(n \ge N\) such that \(\left| x_{n} - x^{*} \right| <\epsilon \). Show the set of all accumulation points is closed.
\end{problem}
\begin{solution}
	Denote the set of all accumulation points \(X\) of \(\left( x_{n} \right) \) and let \(x \in \overline{X}\). Then, for all \(\epsilon > 0\), we have \(X \cap \left( x-\epsilon, x+\epsilon \right) \neq \O\). Hence, for every \(\frac{\epsilon}{2} > 0\) there is an accumulation point \(x^{*}\in X\) such that \(x^{*} \in \left( x- \frac{\epsilon}{2}, x + \frac{\epsilon}{2} \right) \). Thus, \(\left| x - x^{*} \right|  < \frac{\epsilon}{2}\). Furthermore, for each \(\frac{\epsilon}{2} > 0\) and \(N \in \N\) there is an \(n \in \N\) such that \(\left| x^{*} - x_{n} \right| < \frac{\epsilon}{2} \). Combining these yields for each \(\epsilon > 0\) and \(N \in \N\), a \(n \in \N\) with \(n \ge N\) such that
	\begin{align*}
		\left| x_{n} - x\right| &= \left| x_{n} - x^{*} - \left( x - x^{*} \right)  \right|  \\
		&\le \left| x_{n} - x^{*} \right| + \left| x - x^{*} \right|   \\
		&< \frac{\epsilon}{2} + \frac{\epsilon}{2} = \epsilon
	.\end{align*}
Hence, \(x\) is an accumulation point, so \(X \subseteq \overline{X} \subseteq X\), so \(X = \overline{X}\) and \(X\) is closed.
\end{solution}
\newpage
\begin{problem}[15]
	Let \(S\) be a set of nonnegative real numbers. Define \(\sum_{x \in S}^{}x = \sup \{\sum_{x \in S_0}^{} x : S_0 \subseteq S \text{ is finite}\} \). Prove if \(\sum_{x \in S}^{}x < \infty\), then \(S\) is countable.
\end{problem}
\begin{solution}
	We induce a countable covering of \(S\) by finite sets. Note that for each \(n \in \N\) we must have at most finitely many \(x \in S\) such that \(x \ge \frac{1}{n}\). Otherwise, there would be a family of sets \(\hat{S}_{i}\) where \(\hat{S}_{i}\) contains \(i\) elements \(x \ge \frac{1}{n}\), hence \(\sum_{x \in S_{i}}^{} x \ge \frac{i}{n}\) for all \(i \in \N\), hence we would have \[
	\sup \{\sum_{ x\in S_0}^{} x : S_0 \subseteq S \text{ is finite}\} \ge \sup \{\sum_{ x \in \hat{S}_{i}}^{} x : i \in \N\}  \ge \sup \{\frac{i}{n} :  i \in \N\} > M
	\]
for all \(M \in \R\), hence our sum would be unbounded, so \(\sum_{x \in S}^{} x \not < \infty\) \(\lightning\).\\
Thus, the set \(\{x \in S : x \ge \frac{1}{n}\} \) is finite for all \(n \in \N\). Then, we have that \[
	\bigcup_{n \in \N} \{x \in S : x \ge \frac{1}{n}\}  = \left( 0, \infty \right) \cap S =  S \setminus \{0\} \text{ by nonnegative assumption}
.\]
Hence, we have a countable covering of \(S \setminus \{0\} \) by finite sets, so \(S \setminus \{0\} \) is countable. Thus, \(S\) is countable.
\end{solution}
\newpage
\begin{problem}[16]
	For a collection \(\mathscr{S}\) of subsets of \(X\), denote the smallest \(\sigma\)-algebra containing \(\mathscr{S}\) by \(\sigma\left( \mathscr{S} \right) \). Let \(\mathscr{C}\) be a collection of subsets of \(X\) and let \(\mathscr{U}\) be the collection of all countable subcollections \(\mathscr{F}\subseteq \mathscr{C}\). Hence, each subcollection \(\mathscr{F}\) contains only countable many subsets of \(X\). Prove \(\bigcup_{\mathscr{F} \in \mathscr{U}}  \sigma \left( \mathscr{F} \right) \) is a \(\sigma\)-algebra which is equal with \(\sigma\left( \mathscr{C} \right) \).
\end{problem}
\begin{solution}
	First, we show \(\bigcup_{\mathscr{F}\in \mathscr{U}} \sigma \left( \mathscr{F} \right) \) is a \(\sigma\)-algebra. As each \(\sigma \left( \mathscr{F} \right) \) is a \(\sigma\)-algebra, we have that \(X \in \sigma\left( \mathscr{F} \right) \) so \(X \in \bigcup_{\mathscr{F}\in \mathscr{U}} \sigma \left( \mathscr{F} \right) \).\\
	Next, let \(A \in \bigcup_{\mathscr{F}\in \mathscr{U}} \sigma \left( \mathscr{F} \right) \). Then, \(A \in \sigma\left( \mathscr{F} \right) \) for some \(\mathscr{F} \in \mathscr{U}\), hence \(A^{c} \in \sigma ( \mathscr{F})\), so \(A^{c} \in \bigcup_{\mathscr{F}\in \mathscr{U}}  \sigma \left( \mathscr{F} \right) \).\\
	Lastly, let \(\left( A_{k} \right)_{k \in \N} \) be a countable collection of elements \(A_{k} \in \bigcup_{\mathscr{F} \in \mathscr{U}} \sigma \left( \mathscr{F} \right) \). Then, each \(A_{k} \in \sigma \left( \mathscr{F}_{k} \right) \) for some \(\mathscr{F}_{k} \in \mathscr{U}\). As each \(\mathscr{F}_{k}\) is countable, then \(\bigcup_{k \in \N} \mathscr{F}_{k}\) is countable, hence \(\bigcup_{ k\in \N} \mathscr{F}_{k} \in \mathscr{U} \) by definition of \(\mathscr{U}\). Thus, \(\sigma \left( \bigcup_{ k \in \N} \mathscr{F} _{k} \right) \subseteq \bigcup_{\mathscr{F}\in \mathscr{U}} \sigma\left( \mathscr{F} \right) \) and as \(\bigcup_{k \in \N}A_{k} \in \sigma \left( \bigcup_{k \in \N} \mathscr{F}_{k} \right)  \), we see \(\bigcup_{k \in \N}A_{k} \in \bigcup_{\mathscr{F} \in \mathscr{U}} \sigma \left( \mathscr{F} \right)  \).\\

	Note that it is clear as each \(\mathscr{F}\subseteq \mathscr{C}\) that each \(\sigma \left( \mathscr{F} \right) \subseteq \sigma \left( \mathscr{C} \right) \) hence \(\bigcup_{ \mathscr{F} \in \mathscr{U}} \sigma \left( \mathscr{F} \right) \subseteq \sigma \left( \mathscr{C} \right) \).\\
	Now, we show equality. Let \(A \in \mathscr{C}\), then \(A \in \mathscr{F}\) for some \(\mathscr{F}\in \mathscr{U}\), hence \(A \in \sigma \left( \mathscr{F} \right)  \) and \(A \in \bigcup_{\mathscr{F}\in \mathscr{U}} \sigma\left( \mathscr{F} \right) \). Hence, \( \mathscr{C} \subseteq  \bigcup_{\mathscr{F}\in \mathscr{U}} \sigma \left( \mathscr{F} \right)  \). As \(\sigma \left( \mathscr{C} \right) \) is the smallest \(\sigma\)=algebra containing \(\mathscr{C}\) and \(\bigcup_{\mathscr{F}\in \mathscr{U}} \sigma \left( \mathscr{F} \right) \) is a \(\sigma\)-algebra containing \(\mathscr{C}\), then \(\sigma \left( \mathscr{C} \right) \subseteq \bigcup_{\mathscr{F}\in \mathscr{U}}\sigma \left( \mathscr{F} \right)  \). Hence, equality holds.
\end{solution}
\end{document}
