\documentclass[a4paper]{article}
% Some basic packages
\usepackage[utf8]{inputenc}
\usepackage[T1]{fontenc}
\usepackage{textcomp}
\usepackage{url}
\usepackage{graphicx}
\usepackage{float}
\usepackage{booktabs}
\usepackage{enumitem}

\pdfminorversion=7

% Don't indent paragraphs, leave some space between them
\usepackage{parskip}

% Hide page number when page is empty
\usepackage{emptypage}
\usepackage{subcaption}
\usepackage{multicol}
\usepackage{xcolor}

% Other font I sometimes use.
% \usepackage{cmbright}

% Math stuff
\usepackage{amsmath, amsfonts, mathtools, amsthm, amssymb}
% Fancy script capitals
\usepackage{mathrsfs}
\usepackage{cancel}
% Bold math
\usepackage{bm}
% Some shortcuts
\newcommand\N{\ensuremath{\mathbb{N}}}
\newcommand\R{\ensuremath{\mathbb{R}}}
\newcommand\Z{\ensuremath{\mathbb{Z}}}
\renewcommand\O{\ensuremath{\varnothing}}
\newcommand\Q{\ensuremath{\mathbb{Q}}}
\newcommand\C{\ensuremath{\mathbb{C}}}
% Easily typeset systems of equations (French package)

% Put x \to \infty below \lim
\let\svlim\lim\def\lim{\svlim\limits}

%Make implies and impliedby shorter
\let\implies\Rightarrow
\let\impliedby\Leftarrow
\let\iff\Leftrightarrow
\let\epsilon\varepsilon
\let\nothing\varnothing

% Add \contra symbol to denote contradiction
\usepackage{stmaryrd} % for \lightning
\newcommand\contra{\scalebox{1.5}{$\lightning$}}

 \let\phi\varphi

% Command for short corrections
% Usage: 1+1=\correct{3}{2}

\definecolor{correct}{HTML}{009900}
\newcommand\correct[2]{\ensuremath{\:}{\color{red}{#1}}\ensuremath{\to }{\color{correct}{#2}}\ensuremath{\:}}
\newcommand\green[1]{{\color{correct}{#1}}}

% horizontal rule
\newcommand\hr{
    \noindent\rule[0.5ex]{\linewidth}{0.5pt}
}

% hide parts
\newcommand\hide[1]{}

% Environments
\makeatother
% For box around Definition, Theorem, \ldots
\usepackage{mdframed}
\mdfsetup{skipabove=1em,skipbelow=0em}
\theoremstyle{definition}
\newmdtheoremenv[nobreak=true]{definition}{Definition}
\newmdtheoremenv[nobreak=true]{eg}{Example}
\newmdtheoremenv[nobreak=true]{corollary}{Corollary}
\newmdtheoremenv[nobreak=true]{lemma}{Lemma}[section]
\newmdtheoremenv[nobreak=true]{proposition}{Proposition}
\newmdtheoremenv[nobreak=true]{theorem}{Theorem}[section]
\newmdtheoremenv[nobreak=true]{law}{Law}
\newmdtheoremenv[nobreak=true]{postulate}{Postulate}
\newmdtheoremenv{conclusion}{Conclusion}
\newmdtheoremenv{bonus}{Bonus}
\newmdtheoremenv{presumption}{Presumption}
\newtheorem*{recall}{Recall}
\newtheorem*{previouslyseen}{As Previously Seen}
\newtheorem*{interlude}{Interlude}
\newtheorem*{notation}{Notation}
\newtheorem*{observation}{Observation}
\newtheorem*{exercise}{Exercise}
\newtheorem*{comment}{Comment}
\newtheorem*{practice}{Practice}
\newtheorem*{remark}{Remark}
\newtheorem*{problem}{Problem}
\newtheorem*{solution}{Solution}
\newtheorem*{terminology}{Terminology}
\newtheorem*{application}{Application}
\newtheorem*{instance}{Instance}
\newtheorem*{question}{Question}
\newtheorem*{intuition}{Intuition}
\newtheorem*{property}{Property}
\newtheorem*{example}{Example}
\numberwithin{equation}{section}
\numberwithin{definition}{section}
\numberwithin{proposition}{section}

% End example and intermezzo environments with a small diamond (just like proof
% environments end with a small square)
\usepackage{etoolbox}
\AtEndEnvironment{example}{\null\hfill$\diamond$}%
\AtEndEnvironment{interlude}{\null\hfill$\diamond$}%

\AtEndEnvironment{solution}{\null\hfill$\blacksquare$}%
% Fix some spacing
% http://tex.stackexchange.com/questions/22119/how-can-i-change-the-spacing-before-theorems-with-amsthm
\makeatletter
\def\thm@space@setup{%
  \thm@preskip=\parskip \thm@postskip=0pt
}


% \lecture starts a new lecture (les in dutch)
%
% Usage:
% \lecture{1}{di 12 feb 2019 16:00}{Inleiding}
%
% This adds a section heading with the number / title of the lecture and a
% margin paragraph with the date.

% I use \dateparts here to hide the year (2019). This way, I can easily parse
% the date of each lecture unambiguously while still having a human-friendly
% short format printed to the pdf.

\usepackage{xifthen}
\def\testdateparts#1{\dateparts#1\relax}
\def\dateparts#1 #2 #3 #4 #5\relax{
    \marginpar{\small\textsf{\mbox{#1 #2 #3 #5}}}
}

\def\@lecture{}%
\newcommand{\lecture}[3]{
    \ifthenelse{\isempty{#3}}{%
        \def\@lecture{Lecture #1}%
    }{%
        \def\@lecture{Lecture #1: #3}%
    }%
    \subsection*{\@lecture}
    \marginpar{\small\textsf{\mbox{#2}}}
}



% These are the fancy headers
\usepackage{fancyhdr}
\pagestyle{fancy}

% LE: left even
% RO: right odd
% CE, CO: center even, center odd
% My name for when I print my lecture notes to use for an open book exam.
% \fancyhead[LE,RO]{Gilles Castel}

\fancyhead[RO,LE]{\@lecture} % Right odd,  Left even
\fancyhead[RE,LO]{}          % Right even, Left odd

\fancyfoot[RO,LE]{\thepage}  % Right odd,  Left even
\fancyfoot[RE,LO]{}          % Right even, Left odd
\fancyfoot[C]{\leftmark}     % Center

\makeatother




% Todonotes and inline notes in fancy boxes
\usepackage{todonotes}
\usepackage{tcolorbox}

% Make boxes breakable
\tcbuselibrary{breakable}

% Verbetering is correction in Dutch
% Usage:
% \begin{verbetering}
%     Lorem ipsum dolor sit amet, consetetur sadipscing elitr, sed diam nonumy eirmod
%     tempor invidunt ut labore et dolore magna aliquyam erat, sed diam voluptua. At
%     vero eos et accusam et justo duo dolores et ea rebum. Stet clita kasd gubergren,
%     no sea takimata sanctus est Lorem ipsum dolor sit amet.
% \end{verbetering}
\newenvironment{correction}{\begin{tcolorbox}[
    arc=0mm,
    colback=white,
    colframe=green!60!black,
    title=Opmerking,
    fonttitle=\sffamily,
    breakable
]}{\end{tcolorbox}}

% Noot is note in Dutch. Same as 'verbetering' but color of box is different
\newenvironment{note}[1]{\begin{tcolorbox}[
    arc=0mm,
    colback=white,
    colframe=white!60!black,
    title=#1,
    fonttitle=\sffamily,
    breakable
]}{\end{tcolorbox}}


% Figure support as explained in my blog post.
\usepackage{import}
\usepackage{xifthen}
\usepackage{pdfpages}
\usepackage{transparent}
\newcommand{\incfig}[2][1]{%
    \def\svgwidth{#1\columnwidth}
    \import{./figures/}{#2.pdf_tex}
}

% Fix some stuff
% %http://tex.stackexchange.com/questions/76273/multiple-pdfs-with-page-group-included-in-a-single-page-warning
\pdfsuppresswarningpagegroup=1
\binoppenalty=9999
\relpenalty=9999

% My name
\author{Thomas Fleming}

\usepackage{pdfpages}
\title{Analysis I: Homework III}
\date{Fri 10 Sep 2021 12:58}
\DeclareMathOperator{\SRG}{SRG}
\DeclareMathOperator{\cut}{Cut}
\DeclareMathOperator{\GF}{GF}
\DeclareMathOperator{\V}{V}
\DeclareMathOperator{\E}{E}
\DeclareMathOperator{\edg}{e}
\DeclareMathOperator{\vtx}{v}
\DeclareMathOperator{\diam}{diam}

\DeclareMathOperator{\tr}{tr}
\DeclareMathOperator{\A}{A}

\DeclareMathOperator{\Adj}{Adj}
\DeclareMathOperator{\mcd}{mcd}

\begin{document}
\maketitle
\begin{solution}[17]
\end{solution}
\newpage

\begin{solution}[18]
	As \(\left[ a, b \right] \) is compact, we see \(f\) is uniformly continuous. Hence, there is a \(\delta > 0\) such that for all \(\epsilon > 0\) and \(x, y \in \left[ a, b \right] \) we find \(\left| x - y \right| < \delta\)  implies \(\left| f\left( x \right)  - f\left( y \right)  \right|  < \epsilon\).\\
	Define the following sequence. Let \(y_0 = a\) and \(y_{i} = \max \{a + \delta  \cdot i, b\}\) for \(i \ge 0\). Then, we see \(\{\left[ y_{i-1}, y_{i } \right] : i \in \N\} \) is a cover and there is a \(n \ge 0\)  such that \(y_{n} = b\), hence \(y_{m} = b\) for \(m \ge n\) and we see \(\{\left[ y_{i-1}, y_{i} \right] : 1 \le i \le n\} \) is a finite subcover. Define \begin{align*}
		g: \left[ a, b \right]  &\longrightarrow \R \\
		x &\longmapsto g(x) = 	\left \{
			\begin{array}{11}
				f\left( y_{i-1} \right) , & \quad x \in \left[ y_{i-1}, y_{i} \right)  \\
				f\left( b \right) , & \quad x = b
			\end{array}
			\right.
	.\end{align*}
	That is, \(g\) is the piecewise constant (hence linear) interpolation of \(f\) on the \(y_{i}\)'s. Hence, for all \(x \in \left[ a, b \right] \)  there is a \(i \ge 1\) so that  \(x \in \left[ y_{i-1}, y_{i} \right) = \left[ y_{i-1}, y_{i-1} + \delta \right)  \) or \(x = b\) , hence \(\left| y_{i-1} - x \right| < \delta\) implies \(\left| f\left( y_{i-1} \right)  - f\left( x \right)  \right| < \epsilon \). Therefore, if \(g\left( x \right)  = f\left( y_{i-1} \right) \) we see \[
		\left| g\left( x \right) - f\left( x \right)  \right|  < \epsilon
	.\]
	Otherwise, if \(x = b\), we have \(g\left( b	 \right) = f\left( b \right)  \) so the claim holds.
\end{solution}
\newpage
\begin{solution}[19]
	First, we prove the second inequality.	\\If one of \(\limsup_{n \to \infty} x_{n}, \limsup_{n \to \infty} y_{n} = \infty\) (and the other is not \(-\infty\)), we have \[
		\limsup_{n \to \infty} \left( x_{n} + y_{n} \right)  \le \infty
	.\]
	Hence, we may assume neither limit superior to be \(\infty\). Similarly, if \(\liminf_{n \to \infty} x_{n} = -\infty\) we see \(-\infty \le \limsup_{n \to \infty} \left( x_{n} + y_{n} \right) \). Hence, we can assume the limit inferior to not take on \(-\infty\).  Then, we know
	\[
	\inf \{ x_{n} : n \ge K\} +\sup \{ y_{n} : n\ge K \} \le \sup \{ x_{n} + y_{n} : n \ge K \} \le \sup \{ x_{n} : n\ge K \} + \sup \{ y_{n} : n\ge K \}
	.\]  Hence, we have
	\begin{align*}
		\lim_{K \to \infty}\sup \{ x_{n}  + y_{n}: n\ge K \} &\le \lim_{K \to \infty} \left( \sup \{ x_{n} : n\ge K  \}  + \sup \{ y_{n} :n \ge K  \}  \right) \\
								     &= \lim_{K \to \infty}\sup \{ x_{n} : n\ge K \} + \lim_{K \to \infty}\sup \{ y_{n} : n \ge K \} \\
								     &= \limsup_{n \to \infty} x_{n} + \limsup_{n \to \infty} y_{n}
	.\end{align*}
Moreover,
\begin{align*}
	\liminf_{n \to \infty} x_{n} + \limsup_{n \to \infty} y_{n} &= \lim_{K \to \infty}\left( \inf \{ x_{n} : n\ge K \} + \sup \{ y_{n} : n \ge K \}  \right)  \\
								    &\le \lim_{K \to \infty} \left( \sup \{ x_{n} + y_{n} : n \ge K \}  \right)\\
								    &= \limsup_{n \to \infty} (x_{n} + y_{n})
.\end{align*}
Consider the following two sequences
\begin{align*}
	x_{n} &= \left \{
		\begin{array}{11}
			1, & \quad n \equiv 1 \left( \mod 2 \right)  \\
			-1, & \quad n \equiv 0 \left( \mod 2 \right)
		\end{array}
		\right.\\
		y_{n} &=  \left \{
			\begin{array}{11}
				-1, & \quad n \equiv 1 \left( \mod 2 \right)  \\
				1, & \quad n \equiv 0 \left( \mod 2 \right)
			\end{array}
			\right. \\
.\end{align*}
Obviously \(\sup \{ x_{n} : n \ge K \} = \sup \{ y_{n} : n\ge K \} = 1\) for all \(K\). On the other hand, we see \(x_{n} + y_{n} = 0\) for every \(n \in \N\), hence \(\sup \{ x_{n} + y_{n} : n \ge K \}  = 0\) for all \(K\). As these values hold for all \(K\), we see the limit has no effect hence
\begin{align*}
	\limsup_{n \to \infty} x_{n} + y_{n} &= \lim_{k \to \infty}\left( \sup \{ x_{n} + y_{n} : n \ge K \}  \right) \\
&= \lim_{K \to \infty} 0 \\
&= 0 \\
&< 1\\
&= \lim_{K \to \infty}(\sup \{ x_{n} : n\ge K \}  + \sup \{ y_{n} : n\ge K \} ) \\
&= \limsup_{n \to \infty} x_{n} + \limsup_{n \to \infty} y_{n} \\
.\end{align*}
Similairly, define \(x_{n}\) to be the same and \(y_{n} = 0\) for all \(n\). Hence, \(\sup \{ y_{n} :  n\ge k \} = 0\) and \(\inf \{ x_{n} : n\ge K \} = -1\) for all \(K\) with \(\sup \{ x_{n} + y_{n} : n \ge K \} = 1\) for all \(K\). Hence as these hold for all \(K\), we find
\begin{align*}
	\liminf_{n \to \infty} x_{n} + \limsup_{n \to \infty} y_{n} &= -1 + 0 \\
	&= -1 \\
	&\le \limsup_{n \to \infty} \left( x_{n} + y_{n} \right)\\
	&= 1 \\
.\end{align*}
\end{solution}
\newpage
\begin{solution}[20]
	\begin{itemize}
		\item If \(x \in A \triangle B\), then WLOG let \(x \in A \setminus B\). So, \(x \in A \subseteq C\) and \(a \not\in B\). Hence, \(x \in C \setminus B\) and \(x\not\in C \setminus A\), so \(x \in \left( C \setminus B \right) \setminus \left( C \setminus A \right) \). So, \(x \in \left( C \setminus A \right) \triangle \left( C \setminus B \right) \).
		\item If \(x \in \left( C \setminus A\right) \triangle \left( C \setminus B \right)  \), then WLOG let \(x \in (\left( C \setminus A \right) \setminus \left( C \setminus B \right) \). Then, note if \(x \in C\) and \(x \not\in C \setminus B\), then \(x \in B\). So, we see
			\begin{align*}
				x \in \{x \in C\setminus A : x \not\in C \setminus B\} &=  \{x \in C : x \not\in A, x \not\in C \setminus B\}  \\
				&= \{x \in C : x\not\in A, x \in B\} \\
				&= \{x \in C : x \in B \setminus A\}  \\
				&= C \cap (B\setminus A) \\
				&= B \setminus A \\
			.\end{align*}
			Hence, \(x \in B \setminus A\), so \(x \in  A \triangle B\).
	\end{itemize}
	If \(C = \R\), we see \(A \triangle B = \left( \R \setminus A \right) \triangle \left( R \setminus B \right) = A^{c} \triangle B^{c}\).
\end{solution}
\newpage
\begin{solution}[21]
	As \(S\) is measurable and finite, there is an open \(O\) of finite measure such that \(S \subseteq O\) and for all \(\epsilon > 0\), we find \(m\left( O \setminus S \right)  < \frac{\epsilon}{4}\). As \(O\) is the countable disjoint union of intervals \(\{I_{j} : j \in \N\} \) , we see \( m\left( O \right) = \sum_{i= 1}^{\infty} m( I_{j} )  \) , by countable additivity. As this series is finite we see for all \(\epsilon > 0\), there is a \(K\)  such that \[\left| \sum_{j= 1}^{\infty} m (I_{j}) - \sum_{k= 1}^{K} m (I_{k}) \right| = \left| m \left( O \right) - \sum_{k= 1}^{K} m\left( I_{k} \right)  \right| < \frac{\epsilon}{4}  .\] Denote \(U = \bigcup_{i=1} ^{K} I_{j}\). Clearly, \(U\) is measurable and of finite measure and  \[\left| m\left( O \right) - m\left( U \right)  \right| = m\left( O \setminus U \right)  < \frac{\epsilon}{4} . \]	Hence as \(U, S \subseteq O\) , we find \[
		S \triangle U = \left( O \setminus S \right) \triangle \left( O \setminus U \right)
	.\]
	So, as \(\left( O \setminus S \right)  \setminus \left( O \setminus U \right) \) is disjoint from \(\left( O \setminus U \right)  \setminus \left( O \setminus S \right) \) and all measures are finite, we see
	\begin{align*}
		m\left( S \triangle U \right) &=  m\left( \left( O \setminus S \right) \setminus \left( O \setminus U \right)  \right) \right| + \left| m\left( \left( O \setminus U \right) \setminus \left( O\setminus S \right)  \right) \right|  \\
						&\le \left| m\left( O\setminus S \right) - m\left( O\setminus U \right) \right| + \left|m\left( O\setminus U \right)  - m\left( O \setminus S \right) \right|   \\
						&\le 2 m\left( O \setminus S \right)  +  2 m\left( O \setminus U \right) \\
						&< \frac{2\epsilon}{4} + \frac{2\epsilon}{4} = \epsilon \\
	.\end{align*}
\end{solution}
\newpage
\begin{solution}[22]
\begin{enumerate}
	\item It suffices to assume \(m\left( S \right)  < \infty\), because for all sets of infinite measure, we can choose a subset of finite measure and for a set of finite measure \(S^{\prime} \subseteq S\), we have \(S \cap \left( a, b \right)  \supseteq S^{\prime} \cap S\), so \(m\left( S \cap \left( a, b \right)  \right) \ge m\left( S^{\prime} \cap \left( a, b \right)  \right) \).\\
		Then assuming \(m\left( S \right) \) finite, for \(\epsilon = \frac{1}{3} m\left( S \right) \), we find an open \(U\) with \(S \subseteq U\) and \(m\left( U \setminus S \right)  = \epsilon = \frac{1}{3} m\left( S \right) \). Hence, \(m\left( U \right)  = \frac{4}{3} m\left( S \right) \). As \(U\) is open it is the countable union of disjoint intervals \(\left( a_{i}, b_{i} \right) \) and \( m\left( U \right)  = \sum_{i= 1}^{\infty} \left( b_{i} - a_{i} \right) = \frac{4}{3} m\left( S \right)  \). Hence, \[
			\sum_{i= 1}^{\infty} \frac{3}{4}\left( b_{i} - a_{i} \right) < m\left( S \right)
		.\]
		Suppose \(m\left( S \cap \left( a_{i}, b_{i} \right)  \right) \le \frac{3}{4}\left( b_{i}-a_{i} \right) \) for all the intervals \(\left( a_{i}, b_{i} \right) \) . Then,
		\begin{align*}
			m\left( S \right)  &= \sum_{i= 1}^{\infty} m\left( S \cap\left( a_{i}, b_{i} \right)  \right) \\
					   &\le \sum_{i= 1}^{\infty} \frac{3}{4}m\left( a_{i}, b_{i} \right)\\
					   &= \sum_{i= 1}^{\infty} \frac{3}{4} \left( b_{i} - a_{i} \right) \\
					   &< m\left( S \right) \lightning
		.\end{align*}
		Hence, we have atleast one \(\left( a_{i}, b_{i} \right) \) such that \(m\left( S \cap \left( a_{i}, b_{i} \right)  \right) > \frac{3}{4}\left( b_{i}-a_{i} \right)  \).
\item First, note that \(S \cap \left( r + S \right)  = \{s - r \in S : s \in S\} \), and suppose \\\(S \cap \left( r + S \right)  \cap \left( a, b \right)  = \O\). \\That is, for all \(s \in S \cap \left( a, b \right) \), we have \(s + r \not\in S \cap \left( a, b \right) \subseteq \left( a, b \right)  \). Hence, \(s \in \left( b-r, b \right) \subseteq \left( b -  \frac{1}{4}\left( b-a \right) , b \right)  \right) = \left( \frac{1}{4}a + \frac{3}{4}b, b \right)  \). But, we see \(m\left( \left( \frac{1}{4}a + \frac{3}{4}b, b \right)  \right)  = \frac{1}{4} ( b - a) < \frac{3}{4} \left( b-a \right)  \). \\So, we have \(S \cap \left( a, b \right) \subseteq \left( \frac{1}{4}a + \frac{3}{4}b, b \right) \), \(\lightning\). \\ Hence there is a \(s \in S \cap \left( a, a + \frac{3}{4}(b - a) \right) \), so \(s + r \in \left( a, b \right) \), so \[S \cap \left( r + S \right)  \cap \left( a, b \right) \neq \O .\]
	For each \(x \in \left[ -\frac{1}{4}\left( b-a \right) , \frac{1}{4}\left( b-a \right)   \right] \) , note that we have some \(s \in S\) such that \(s + x \in S \) or \(s - x \in S\) since \(S \cap \left( r + S \right) \) is nonempty, \(0 \le r \le \frac{1}{4}\left( b-a \right) \). Denote \(s + x = \overline{s}\) and \(s - x = \hat{s}\). If \(\overline{s} \in S\), then \(\overline{s} - s = x \in S - S\). Otherwise, if \(\hat{s} \in S\), then \(s - \hat{s} = x \in S - S\). Hence, \(\left[ -\frac{1}{4} \left( b-a \right) , \frac{1}{4}\left( b-a \right) \right] \subseteq S - S\).
\end{enumerate}
\end{solution}
\newpage
\begin{solution}[23]
\end{solution}
\newpage
\begin{solution}[24]
\end{solution}
\newpage
\begin{solution}[25]

\end{solution}
\newpage
\begin{solution}[26]
	Let \(S_{i} = \left( i, \infty \right) \) for each \(i \in \N\). Clearly, each \(s_{i}\) is measurable and \(\bigcap_{n \in \N} S_{n} = \O\). However, \(m\left( S_{i} \right) = \infty - i = \infty  \) for all \(i\), so we find \(m\left( \bigcap_{n \in \N} S_{n} \right) = 0 \neq \infty = \lim_{n \to \infty}m\left( S_{n} \right)  \).\\
	For the second claim consider \(M = \Q\). Then, recall by the density of \(\Q\) in \(\R\), we have that for all \(r \in \R\) and some fixed \(\epsilon = \frac{1}{n}> 0\) , we have a \(x \in Q\) so that \(x - \frac{1}{n} < r \le x < x + \frac{1}{n}\), hence   \(r \in \bigcup_{x \in \Q} \left(x-\frac{1}{n}, x + \frac{1}{n} \right) \), so this union is simply \(\R\). Then, we have
	\begin{align*}
		\bigcap_{n \in \N} \bigcup_{x \in \Q} \left( x-\frac{1}{n}, x+\frac{1}{n} \right) &= \bigcap_{n \in \N} \R \\
		&= \R \\
		&\neq \Q
	.\end{align*}

\end{solution}
\newpage
\begin{solution}[27]
\begin{itemize}
	\item Consider the following construction. Let \(A_{i} = \left( a_{i}, b_{i} \right) \) and \(C_{i} = \left( c_{i}, d_{i}  \right) \), where \(\left( a_1, b_1 \right)  = \left( 0, 1 \right) \), \(\left( c_1, d_1 \right) = \left( 1, 2 \right) \)  and \(a_{i} = d_{i-1}\), \(b_{i} = a_{i} + \frac{1}{i^2}\),  \(c_{i} = b_{i}\), \(d_{i} = c_{i} + \frac{1}{i}\) . Define \(A = \bigcup_{i \in \N} A_{i}\) and \(C = \bigcup_{i \in \N} C_{i}\). Then, note all \(A_{i}\) are disjoint and all \(C_{i}\) are disjoint. Then, we see \(m\left( C \right)  = \sum_{i= 1}^{\infty} m\left( C_{i} \right) = \sum_{i= 1}^{\infty} \frac{1}{i} = \infty \) and \(m\left( A \right)  = \sum_{i= 1}^{\infty} m\left( A_{i} \right) = \sum_{i= 1}^{\infty} \frac{1}{n^2} < \infty \), hence \(A\) is of finite measure. However, we have \(a_{i} = d_{i-1} = c_{i-1} + \frac{1}{i}\) and \(c_{i} = b_{i} = a_{i-1} + \frac{1}{i^2} \). Hence, \(a_{i} = a_{i-1} + \frac{1}{i^2} + \frac{1}{i} = \sum_{j= 1}^{i}\frac{1}{j} + \frac{1}{j^2} + a_1\), so for any bounded interval \(I \subseteq \left[ -M, M \right] \) and bound \(M\), we see there is a \(n\) such that \(\sum_{i= 1}^{n} \frac{1}{i} > M\), hence \( a_{n} = \sum_{i= 1}^{n} \frac{1}{i^2} + \frac{1}{i} + a_1 > \sum_{i= 1}^{n} \frac{1}{i} > M\), so \(A \nsubseteq I\).
	\item Recall for a measurable \(E\) there is a finite collection of open intervals \(\{I_{k} : 1 \le k \le K\} \) such that for \(\epsilon > 0\), and \(U = \bigcup_{k=1} ^{K} I_{k} \)  \(m\left( E \triangle U \right) < \epsilon \). Moreover, every \(I_{k}\) is bounded, as if one was of the form \(\left( a, \infty \right) \)  or \((-\infty, a)\) we would find \(m\left( U \right) = \infty\) and \(m\left( E \triangle U \right)  = m\left( E \setminus U \right)  + m\left( U \setminus E \right)  = m\left( E \setminus U \right)  + \infty = \infty\) as \(m\left( E \setminus U \right) \) is finite by assumption. Hence, a finite union of bounded intervals is bounded, so \(U\) is a bounded set with \(m\left( E \triangle U \right) < \epsilon\), but \(m \left( E \triangle U \right)  = m\left( E \setminus U \right)  + m\left( U \setminus E \right) \ge m\left( E \setminus U  \right)  \). Hence, \(m\left( E \setminus U \right)  < \epsilon\).
\end{itemize}
\end{solution}
\newpage
\begin{solution}[28]
	As \(\inf \{ \left| x-y \right|  : x \in A, y \in B \} = p > 0\), we see \(A \cap B = \O\). Then, we see for any coverings \(\{A_{k}\} \), \(\{B_{k}\} \) of \(A\) and \(B\) by open disjoint intervals, with \(A_{k} \cap B_{j} \neq \O\), for some \(k, j \in \N\), we find \(m^{*}\left( A_{k} \setminus B_{j} \right)  \le m^{*}\left( A_{k} \right)  + m^{*}\left( B_{j}^{c} \right) \le m^{*}\left( A_{k}  \right)  \) by subadditivity. Moreover, \(A_{k} \setminus B_{j}\) either empty, an interval, or the union of \(2\) intervals, hence a sufficient reordering of the covers \(\{A_{k}\}, \{B_{k}\}  \) yields a new pair of covers by disjoint open intervals \(\{\overline{A_{k}}\} , \{\overline{B}_{k}\} \) with \(\sum_{i= 1}^{\infty} \overline{A_{k}} \le \sum_{i= 1}^{\infty} A_{k}\) . Hence, every pair of coverings with intersection admits a disjoint pair of smaller cumulative measure, so we can assume all pairs of coverings are disjoint when passing to the infimum .\\
	Now, we see
	\begin{align*}
		m^{*}\left( A \cup B \right) &=  \inf \{ \sum_{k= 1}^{\infty} \ell\left( J_{k} \right)  : \{J_{k} : k \in \N\} \in J\left( A \cup B \right)   \}  \\
					     &= \inf \{ \sum_{k= 1}^{\infty} \ell\left( A_{k} \cup B_{k} \right)  : \{A_{k} : k \in \N\} \in J\left( A \right) , \{B_{k} : k \in \N\} \in J\left( B \right) , A_{k} \cap B_{j} = \O, j, k \in \N  \}  \\
					     &= \inf \{ \sum_{k= 1}^{\infty} \ell\left( A_{k} \right)  + \ell\left( B_{k} \right)  : \{A_{k} : k \in \N\} \in J\left( A \right) , \{B_{k} : k \in \N\} \in B_{k}, A_{k} \cap B_{j} =\O, k, j \in \N  \}  \\
					     &\ge \inf \{ \sum_{k= 1}^{\infty} \ell\left( A_{k} \right)  : \{A_{k} : k \in \N\} \in J\left( A \right)   \}  + \inf \{ \sum_{k=1}^{\infty} \ell\left( B_{k} \right)  : \{B_{k} : k \in \N\} \in J\left( B \right)   \}\\
					     &= m^{*}\left( A \right) + m^{*}\left( B \right)
	.\end{align*}
	But, applying subadditivity implies \(m^{*}\left( A \cup B \right)  \le m^{*}\left( A \right)  + m^{*}\left( B \right) \) for \(A, B\) disjoint. Hence \(m^{*}\left( A \cup B \right)  = m^{*}\left( A \right)  + m^{*}\left( B \right) \).
\end{solution}
\newpage
\begin{solution}[29]

\end{solution}
\newpage
\begin{solution}[30]
	Let \begin{align*}
		f:\left[ 0, 1 \right]   &\longrightarrow \R \\
		x&\longmapsto f\left( x \right)= \left \{
			\begin{array}{11}
				x, & \quad x \in C  \\
				x - 2, & \quad x \not\in C
			\end{array}
			\right.
	.\end{align*}
	Where \(C \subseteq \R\) is a nonmeasurable set. We see \(f\) is injective, so \(f^{-1}\left( \{c\}  \right) = \{ \hat{c}\}  \) for some \(\hat{c} \in \left[ -2, 1 \right] \), hence as all singeltons are measurable, we see all singleton preimages are measurable. However, \(f^{-1}\left( [0, \infty] \right) = C\) and \(C\) is not measurable, so \(f\) is not measurable.
\end{solution}
\end{document}
