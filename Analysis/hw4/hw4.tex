\documentclass[a4paper]{article}
\input{preamble.tex}
\usepackage{pdfpages}
\title{Analysis I: Homework III Corrections}
\date{Fri 10 Sep 2021 12:58}
\DeclareMathOperator{\SRG}{SRG}
\DeclareMathOperator{\GF}{GF}
\DeclareMathOperator{\V}{V}
\DeclareMathOperator{\E}{E}
\DeclareMathOperator{\edg}{e}
\DeclareMathOperator{\vtx}{v}
\DeclareMathOperator{\diam}{diam}

\DeclareMathOperator{\tr}{tr}
\DeclareMathOperator{\A}{A}

\DeclareMathOperator{\Adj}{Adj}
\DeclareMathOperator{\tr}{tr}
\DeclareMathOperator{\mcd}{mcd}

\begin{document}
\maketitle
\begin{solution}[18]
	As \(\left[ a, b \right] \) is compact, we see \(f\) is uniformly continuous. Hence, there is a \(\delta > 0\) such that for all \(\epsilon > 0\) and \(x, y \in \left[ a, b \right] \) we find \(\left| x - y \right| < \delta\)  implies \(\left| f\left( x \right)  - f\left( y \right)  \right|  < \epsilon\).\\
	Fix \(\epsilon > 0\), and  define the following sequence. Let \(y_0 = a\) and \(y_{i} = \max \{a + \delta  \cdot i, b\}\) for \(i \ge 0\). Then, we see \(\{\left[ y_{i-1}, y_{i } \right] : i \in \N\} \) is a cover and there is a \(n \ge 0\)  such that \(y_{n} = b\), hence \(y_{m} = b\) for \(m \ge n\) and we see \(\{\left[ y_{i-1}, y_{i} \right] : 1 \le i \le n\} \) is a finite subcover. Define \begin{align*}
		g: \left[ a, b \right]  &\longrightarrow \R \\
		x &\longmapsto g(x) =
				\frac{f\left( y_{i} \right) - f\left( y_{i-1} \right) }{y_{i} - y_{i-1}} \left( x - y_{i} \right) + f\left( y_{i} \right)   , & \quad \text{ for } x \in \left[ y_{i-1}, y_{i} \right]
	.\end{align*}
	We see \(g\) is simply the piecewise linear interpolation of \(f\) on the \(y_{i}\)'s and it is well defined (the endpoints agree for each closed interval). Hence, for all \(x \in \left[ a, b \right] \) there is an \(i\ge 1\) such that \(x \in \left[ y_{i-1}, y_{i} \right]  = \left[ y_{i-1}, y_{i-1} + \delta \right] = \left[ y_{i} - \delta, y_{i} \right]  \), hence \(\left| y_{i-1} - x \right| < \delta\) and \(\left| y_{i} - x \right| < \delta\)  so we see \(\left| f \left( y_{i-1} \right)  - f\left( x \right)  \right| < \frac{\epsilon}{3} \) and \(\left| f\left( y_{i} \right) - f\left( x \right)  \right| < \frac{\epsilon}{3}\). Then, either \(f\left( y_{i-1} \right) \le g\left( x \right) \le f\left( y_{i} \right) \) or \(f\left( y_{i} \right) \le g\left( x \right) \le f\left( y_{i-1} \right) \) as \(g\) is the linear interpolation between these two points. Then, we see \(\left| f\left( y_{i} \right)  - g\left( x \right)  \right| \le \left| f\left( y_{i} \right)  - f\left( y_{i-1} \right)  \right| \). Hence, we find
	\begin{align*}
		\left| g\left( x \right) - f\left( x \right)  \right| &\le \left| f\left( y_{i} \right) - g\left( x \right)  \right|  + \left| f\left( x \right) - f\left( y_{i} \right)  \right| \\
								      &\le \left| f\left( y_{i} \right) - f\left( y_{i-1} \right)  \right|  + \frac{\epsilon}{3}\\
								      &\le \left| f\left( y_{i} \right) - f\left( x \right)  \right|  + \left| f\left( x \right) - f\left( y_{i-1} \right)  \right|  + \frac{\epsilon}{3}\\
								      &< \frac{\epsilon}{3} + \frac{\epsilon}{3} + \frac{\epsilon}{3}\\
								      &= \epsilon
	.\end{align*}
\end{solution}
\newpage
\begin{solution}[22]
\begin{enumerate}
	\item It suffices to assume \(m\left( S \right)  < \infty\), because for all sets of infinite measure, we can choose a subset of finite measure  \(S^{\prime} \subseteq S\)  and  \(S \cap \left( a, b \right)  \supseteq S^{\prime} \cap \left( a, b \right) \), so \(m\left( S \cap \left( a, b \right)  \right) \ge m\left( S^{\prime} \cap \left( a, b \right)  \right) \).\\
		Then assuming \(m\left( S \right) \) finite, for \(\epsilon = \frac{1}{3} m\left( S \right) \), we find an open \(U\) with \(S \subseteq U\) and \(m\left( U \setminus S \right)  < \epsilon = \frac{1}{3} m\left( S \right) \). Hence, \(m\left( U \right)  < \frac{4}{3} m\left( S \right) \). As \(U\) is open it is the countable union of disjoint intervals \(\left( a_{i}, b_{i} \right) \) and \( m\left( U \right)  = \sum_{i= 1}^{\infty} \left( b_{i} - a_{i} \right) < \frac{4}{3} m\left( S \right)  \). Hence, \[
			\sum_{i= 1}^{\infty} \frac{3}{4}\left( b_{i} - a_{i} \right) < m\left( S \right)
		.\]
		Suppose \(m\left( S \cap \left( a_{i}, b_{i} \right)  \right) \le \frac{3}{4}\left( b_{i}-a_{i} \right) \) for all the intervals \(\left( a_{i}, b_{i} \right) \) . Then,
		\begin{align*}
			m\left( S \right)  &= \sum_{i= 1}^{\infty} m\left( S \cap\left( a_{i}, b_{i} \right)  \right) \\
					   &\le \sum_{i= 1}^{\infty} \frac{3}{4}m\left( a_{i}, b_{i} \right)\\
					   &= \sum_{i= 1}^{\infty} \frac{3}{4} \left( b_{i} - a_{i} \right) \\
					   &< m\left( S \right) \lightning
		.\end{align*}
		Hence, we have atleast one \(\left( a_{i}, b_{i} \right) \) such that \(m\left( S \cap \left( a_{i}, b_{i} \right)  \right) > \frac{3}{4}\left( b_{i}-a_{i} \right)  \).
	\item First, note that \(S \cap \left( r + S \right)  = \{s - r \in S : s \in S\} \), and suppose \\\(S \cap \left( r + S \right)  \cap \left( a, b \right)  = \O\). \\That is, for all \(s \in S \cap \left( a, b \right) \), we have \(s + r \not\in S \cap \left( a, b \right) \subseteq \left( a, b \right)  \). Hence, \(s \in \left( b-r, b \right) \subseteq \left( b -  \frac{1}{4}\left( b-a \right) , b \right)   = \left( \frac{1}{4}a + \frac{3}{4}b, b \right)  \). But, we see \(m\left( \left( \frac{1}{4}a + \frac{3}{4}b, b \right)  \right)  = \frac{1}{4} ( b - a) < \frac{3}{4} \left( b-a \right)  \). \\So, we have \(S \cap \left( a, b \right) \subseteq \left( \frac{1}{4}a + \frac{3}{4}b, b \right) \), \(\lightning\). \\ Hence there is a \(s \in S \cap \left( a, a + \frac{3}{4}(b - a) \right) \), so \(s + r \in \left( a, b \right) \), so \[S \cap \left( r + S \right)  \cap \left( a, b \right) \neq \O .\]
	For each \(x \in \left[ -\frac{1}{4}\left( b-a \right) , \frac{1}{4}\left( b-a \right)   \right] \) , note that we have some \(s \in S\) such that \(s + x \in S \) or \(s - x \in S\) since \(S \cap \left( r + S \right) \) is nonempty, \(0 \le r \le \frac{1}{4}\left( b-a \right) \). Denote \(s + x = \overline{s}\) and \(s - x = \hat{s}\). If \(\overline{s} \in S\), then \(\overline{s} - s = x \in S - S\). Otherwise, if \(\hat{s} \in S\), then \(s - \hat{s} = x \in S - S\). Hence, \(\left[ -\frac{1}{4} \left( b-a \right) , \frac{1}{4}\left( b-a \right) \right] \subseteq S - S\).
\end{enumerate}
\end{solution}
\end{document}
