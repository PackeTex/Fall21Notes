\lecture{11}{Tue 28 Sep 2021 13:00}{Measure Theory (3)}
We prove the final theorem from last lecture.
\begin{proof}
	\begin{itemize}
		\item \(\left( 1 \implies 2 \right) \). There are \(2\) cases, \(S\) being bounded and \(S\) being unbounded.\\
			If \(S\) is bounded, there is an interval \(\left( a, b \right) \supseteq S\), \(a, b \in \R\). Then for any given \(\epsilon > 0\), we find \(\{I_{k} : k \in \N\} \in J\left( S \right)  \) and \(\{J_{k} : k \in \N\}  \in J\left( \left[ a, b \right] \setminus S \right) \) such that \( \mu \left( s \right) \ge \sum_{k=1}^{\infty} \ell\left( I_{k} \right)  - \frac{\epsilon}{3}\) and \( \mu\left( \left[ a, b \right] \setminus S \right)  \ge \sum_{k=1}^{\infty} \ell \left( J_{k} \right) - \frac{\epsilon}{3}\).\\
			Let \(O = \bigcup_{k \in \N} I_{k}\), \(U = \bigcup_{k \in \N} J_{k} \) and \(C = \left[ a, b \right] \setminus U\). Then, \(C  \subseteq S \subseteq O\). Note that \(O, U\) are open and \(C\) is closed. Then, \begin{align*}
				\mu \left( S \right) &\ge \mu\left( O \right) -\frac{\epsilon}{3}\\
				\mu\left( \left[ a, b \right] \setminus S \right) &\ge \mu\left( U \right) - \frac{\epsilon}{3}.
			.
		\end{align*}
		Furthermore, \(U\), \(C\) are disjoint and \(\mu\left( U \right) < \infty\) (as it is an interval minus a measurable set) and \(\left[ a, b \right] \subseteq U \cup C\). Hence,
		\begin{align*}
			\mu\left( C \right) &\ge \mu \left( \left[ a, b \right]  \right)  \setminus \mu \left( U \right) \\
					    &= b - a - \mu\left( U \right)
		.\end{align*}
		Then, since \( \mu\left( C \right) \le \mu\left( S \right)  < \infty\), we have \begin{align*}
			\mu\left( O \setminus C \right) &=  \mu\left( O \right) - \mu\left( C \right)  \\
							&\le \frac{\epsilon}{3} + \underbrace{\mu\left( S \right) - \left( b - a \right)}_{= -\mu\left( \left[ a, b \setminus S \right]  \right) }  + \mu \left( U \right) \\
							&=  \frac{\epsilon}{3}- \mu\left( \left[ a, b \right] \setminus S \right)  + \mu\left( U \right)  \\
							&\le \frac{2\epsilon}{3} \\
							&< \epsilon
		.\end{align*}
		For a general \(S\), let \(S_{n} = S \cap \left[ n, n + 1 \right] \), \( n \in \Z\). Then, there are open \(O_{n}\) and closed \(C_{n}\) such that \(C_{n} \subseteq S_{n} \subseteq O_{n}\) and \( \mu\left( O_{n} \setminus C_{n} \right)  < \frac{\epsilon}{3\cdot 2^{\left| n \right| }}\).\\
		Let \(O = \bigcup_{n \in \Z} O_{n}\) and \(C = \bigcap_{n \in \Z}C_{n} \). Then, \(O\) is open and \(C\) is closed by definition and we see \(O \setminus C = \bigcup_{n \in \Z} \left( O_{n} \setminus C_{n} \right) \) by demorgen and we have \(C \subseteq S \subseteq O\). Then,
		\begin{align*}
			\mu\left( O \setminus C \right) &\le \sum_{n \in \Z}^{} \mu\left( O_{n} \setminus C_{n} \right) \\
							&< \sum_{ n \in \Z}^{} \frac{\epsilon}{3 \cdot 2^{\left| n \right| }} \\
							&=  \epsilon \text{ by geometric summation}
		.\end{align*}
	\item \(\left( 2 \implies 3 \right) \). For each \(n \in \N\), there are closed \(C_{n}\) and open \(O_{n}\) such that \(C_{n} \subseteq S \subseteq O_{n}\) and \( \mu\left( O_{n} \setminus C_{n} \right)  < \frac{1}{n}\). Let \(F = \bigcup_{n \in \N} C_{n}\) and \(G = \bigcap_{ O_{n}} \). Then, \(F\) is a \(F_{\sigma}\) set and \(G\) is a \(G_{\delta}\) set. Then, we have \(F \subseteq S \subseteq G\) and \( \mu\left( G \setminus F \right) \le \mu\left( O_{n} \ C_{n} \right)  < \frac{1}{n}\) for all \(n \in \N\). Hence, \( \mu \left( G \setminus F \right) = 0 \).
	\item \((3 \implies 4)\). This is immediately obvious as \(F_{\sigma}\) and \(G_{\delta}\) sets are measurable.
	\item \(\left( 4 \implies 1 \right) \). Let \(A \subseteq \R\) and \(\epsilon > 0\). Then \(S^{c} \subseteq G \cup \left( G \cap F^{c} \right) \). Then, \(A \cap S^{c} \subseteq \left( A \cap G^{c} \right) \cup \left( G \cap F^{c} \right) \).\\
		Hence, \begin{align*} \mu^{*}\left( A \cap S^{c} \right) &\le \mu^{*}\left( A \cap G^{c} \right) + \underbrace{ \mu^{*}\left(G \cap F^{c}\right)}_{ < \epsilon} \\
			&\le \mu^{*}\left( A \cap G^{c} \right)  + \epsilon
		.\end{align*}
		And, as \(G\) is measurable, we have \[ \mu^{*}\left( A  \right) =  \mu^{*}\left( A \cap G^{} \right)  + \mu^{*}\left( A \cap G^{c} \right)  \ge  \mu^{*}\left( A \cap S \right)  + \mu ^{*}\left( A \cap S^{c} \right)  - \epsilon\]. Hence, in the infimum we have \[
			\mu^{*}\left( A \right) \ge \mu^{*}\left( A \cap S \right)  + \mu^{*}\left( A \cap S^{c} \right)
		.\]
		So, \(S\) is measurable.
	\end{itemize}
\end{proof}
\begin{definition}[Nested Sets]
	A countable collection of sets \(\{S_{k} : k \in \N\} \) is called
\begin{enumerate}
	\item \textbf{ascending} if \(S_{k} \subseteq S_{ k + 1}\) for all \(k\).
	\item \textbf{descending} if \(S_{k + 1} \subseteq S_{k}\) for all \(k\).
\end{enumerate}
\end{definition}
\begin{lemma}
	\begin{enumerate}
		\item If \(\{S_{k} : k \in \N\} \) is an ascending collection of measurable sets, then \( \mu \left( \bigcup_{k \in \N} S_{k} \right) = \lim_{k \to \infty} \mu\left( S_{k} \right) \).
		\item If \(\{S_{k} : k \in \N\} \) is a descending collection of measurable sets and \( \mu\left( S_1 \right)  < \infty\). Then, \( \mu \left( \bigcap_{k \in \N} S_{k}  \right) = \lim_{k \to \infty} \mu\left( S_{k} \right)  \).
	\end{enumerate}
\end{lemma}
\begin{proof}
\begin{enumerate}
	\item It suffices to consider the case \( \mu\left( S_{k} \right) < \infty\) for all \(k\), else the union and limit both trivially have measure \(\infty\). Define \(S_0 = \O\), \(X_{n} = S_{n} \setminus S_{n - 1}\). Then, \(\{X_{k} : k \in \N\} \) is a disjoint collection of measurable sets such that \(\bigcup_{k \in \N} X_{k} = \bigcup_{k \in \N} S_{k} \). Hence, as we know the lebesque measure to be countably additive, we have
		\begin{align*}
			\mu\left( \bigcup_{k \in \N} X_{k} \right) &=  \sum_{k=1}^{\infty} \mu\left( X_{k} \right)  \\
								   &= \lim_{n \to \infty}\sum_{k=1}^{n} \mu\left( X_{k} \right)  \\
								   &=  \lim_{n \to \infty}\sum_{k=1}^{n} \left( \mu\left( S_{k} \right) - \mu\left( S_{ k - 1} \right)  \right)  \\
								   &=  \lim_{k \to \infty} \mu\left( S_{k} \right)
		.\end{align*}
	\item Let \(X_{n} = S_1 \ S_{n}\). Then, \(\{X_{k} : k \in \N\} \) is an ascending collection of measurable sets such that \(\bigcup_{k \in \N} X_{k}= S_1 \setminus \left( \bigcap_{k \in \N}S_{k}  \right)  \). Since \(S_{k} \subseteq S_1\) and \( \mu\left( S_1 \right)  < \infty\) we have by the first lemma that
		\begin{align*}
		\mu\left( S_1 \right) - \mu\left( \bigcap_{k \in \N} S_{k} \right) &=  \mu\left( \bigcup_{k \in \N} X_{k} \right)  \\
			   &= \lim_{k \to \infty} \mu\left( X_{k} \right) \\
	   &= \mu\left( S_1 \right) - \lim_{k \to \infty} \mu\left( S_{k} \right)
		.\end{align*}
		As \( \mu\left( S_1 \right) \) is finite we know this to be well defined, hence
		\[
			\mu\left( \bigcap_{k \in \N}  S_{k} \right)  = \lim_{k \to \infty} \mu\left( S_{k} \right)
		.\]
\end{enumerate}
\end{proof}
\begin{theorem}[Borel-Cantelli Lemma]
	Suppose \(\{S_{k} : k \in \N\} \) is a countable collection of measurable sets such that \(\sum_{k=1}^{\infty} \mu\left( S_{k} \right)  < \infty\). Then, the set of all \(x \in \R\) which belong to an infinite subcollection of \(\{S_{k} : k \in \N\} \) has measure \(0\).
\end{theorem}
\begin{proof}
	Note that \(x\) belongs to an infinite subcollection of \(\{S_{k} : k \in \N\} \) if and only if \(x \in \bigcap_{k \in \N} \bigcup_{n = k} ^{\infty} S_{n}\).\\
	Then, the collection \(\{\bigcup_{n=k} ^{\infty}S_{n} : k \in \N\} \) is descending and  \( \mu\left( \bigcup_{n \in \N} S_{n} \right) \le \sum_{n=1}^{\infty} \mu\left( S_{n} \right) < \infty\). Hence, by the preceding lemma, we have
	\begin{align*}
		\mu\left( \bigcap_{k \in \N} \bigcup_{ n = k} ^{\infty}S_{n} \right) &= \lim_{k \to \infty} \mu\left( \bigcup_{n=k} ^{\infty} S _{n}\right)  \\
										     &\le \lim_{k \to \infty}\sum_{n = k}^{\infty} \mu\left( S_{n} \right)\\
										     &=  0 \\
	.\end{align*}
	This final equality is because for all \(\epsilon > 0\) there is a \(K \in \N\) such that for \(k \ge K\) we have \[
		\left| \sum_{i= 1}^{\infty} \mu\left( S_{i} \right) - \sum_{i= 1}^{k-1} \mu\left( S_{i} \right)  \right| < \epsilon
	.\]
\end{proof}
\begin{problem}
	\begin{enumerate}
		\item Is every set measurable?
		\item Is every set of measure \(0\) countable?
		\item Is every measurable set Borel?
	\end{enumerate}
\end{problem}
