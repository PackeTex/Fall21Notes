\lecture{18}{Tue 26 Oct 2021 13:16}{General Lebesque Integral (2)}
\begin{proposition}
	Let \(f: \R \to \overline{\R}\) be integrable. Then for each \(\epsilon > 0\) there is a \(\delta > 0\) such that each measurable \(S \subseteq \R\) has \(\int_{S} \left| f \right| < \epsilon\) if \(m\left( S < \delta \right) \).
\end{proposition}
\begin{proof}
	Let \(\epsilon > 0\), then there is a \(s \in \mathscr{S}\left( \left| f \right|  \right) \) such that \(\int\left( \left| f \right| -s \right) < \frac{\epsilon}{2}\) . Let \(a\alpha = \sup \{ s\left( x \right)  : x \in \R \} \) and \(\delta = \frac{\epsilon}{2\left( \alpha + \epsilon \right) }\). If \(S\) is measurable and \(m\left( S \right) < \delta\), we find \[
		\int_{S}\left| f \right| \le \int  s + \frac{\epsilon}{2}  \le \alpha m\left( S \right)  + \frac{\epsilon}{2} < \epsilon
	.\]
\end{proof}
\begin{theorem}[Monotone Convergence Theorem]
	Let \(\left( f_{n} \right) \) be a sequence of nonnegative measurable functions with \(f_{n}:\R  \to\overline{\R} \) such that \(\left( f_{n} \left( x \right) \right) \) is increasing for all \(x \in \R\). Then, \(f = \lim_{n \to \infty}f_{n}\) is maesurable with \(\int f = \lim_{n \to \infty}\int f_{n}\).
\end{theorem}
\begin{proof}
	Since \(f = \limsup_{n \to \infty} f_{n} = \liminf_{n \to \infty} f_{n}\), we see \(f\) is measurable. Moreover, the sequence \(\left( \int f_{n} \right) \) is increasing (as the \(f_{n}\)s are increasing). Hence, letting \( L = \lim_{n \to \infty}\int f_{n}\)  exists with \(L \in R^{+}_0\). Since \(\int f_{n} \le \int f\) for all \(n\)  by monotonicity, we find \(L \le \int f\).\\
	Let \(s \in \mathscr{S}\left( f \right) \) and fix \(c \in \left( 0, 1 \right) \) and define \(E_{n} = \{x \in \R : f_{n}\left( x \right) \ge c s\left( x \right) \} \). Then, we find \(\{E_{n} : n \in \N\} \) is an ascending collection (again by monotonicity of \(\left( f_{n} \right) \)) of measurable sets with \(\bigcup_{n \in \N} E_{n} = \R\) as \(cs\left( x \right) < f_{n}\left( x \right) \le f\left( x \right) \). Let \(s = \sum_{k=1}^{K} a_{k} \chi_{S_{k}}\) and we see \(cs \chi_{E_{n}} M= f_{n} \chi_{E_{n}} \le f_{n}\) , with \[L \ge \int f_{n} \ge \int_{E_{n}} f_{n} \ge \int cs \chi_{E_{n}} = c\int_{E_{n}}s = c \sum_{k=1}^{K} a_{k} m\left( S_{k} \cap E_{n} \right) .\]
	Since \(\lim_{n \to \infty} m\left( E_{n} \cap S_{n} \right) = m\left( S \right)  \) for every measurable set \(S\), we find \(L \ge c \sum_{k=1}^{K} a_{k} m\left( S_{k} \right) = c\int s \). Since \(c\) was arbitrary, we see the inequality holds for all \(c \in \left( 0, 1 \right) \), hence we find \(L \ge s\) (by taking supremums), but \(s \in \mathscr{S}\left( f \right) \) , hence \(L \ge \int f\). So, \(L = \int f\).
\end{proof}
\begin{theorem}[Fatou's Lemma]
	If \(\left( f_{n} \right) \) is a sequence of nonnegative measurable functions \(f_{n}:\R  \to\overline{\R} \), then \(\int \liminf_{n \to \infty} f_{n} \le \liminf_{n \to \infty} \int f_{n}\).
\end{theorem}
\begin{proof}
	For \(x \in \R\), define \(g_{n}\left( x \right)  = \inf \{ f_{k}\left( x \right)  :k \ge n  \} \) for \(n \in \N\). Then, we find \(\left( g_{n} \right) \) is a nonnegative measurable sequence of functions with \(\left( g_{n}\left( x \right)  \right) \) increasing for all fixed \(x\) and \(g_{n} \le f_{n}\) for all \(n\). Consequently, \(\int g_{n} \le \int f_{n}\)  and \(\left( \int g_{n} \right) \) is increasing. As \(\lim_{n \to \infty}g_{n} = \liminf_{n \to \infty} f_{n}\) is measurable by an earlier theorem, we find \[
	\liminf_{n \to \infty} \int f_{n} \ge \liminf_{n \to \infty} \int g_{n} = \lim_{n \to \infty} \int g_{n} = \int \lim_{n \to \infty}g_{n} = \int \liminf_{n \to \infty} f_{n}
	.\]
\end{proof}
\begin{proposition}
	For any integral function \(f: \R \to \overline{\R}\), we find \(\left| \int f \right| \le \int \left| f \right| \).
\end{proposition}
\begin{theorem}[Dominated Convergence Theorem]
	Let \(\left( f_{n} \right) \) be a sequence of measurable functions \(f_{n}: \R \to \overline{\R}\). Suppose there is an integrable function \(g\) with \(\left| f_{n} \right|\le g \) for all \(n \in \N\). If \(\left( f_{n} \right) \)  converges pointwise to a function \(f: \R \to \overline{\R}\) almost everywhere, then \(f\) is integrable and \[
	\lim_{n \to \infty}\int\left| f_{n} - f \right|  = 0 \text{ and } \lim_{n \to \infty}\int f_{n} = \int f
	.\]
\end{theorem}
\begin{proof}
	Since \(f\left( x \right)  = \lim_{n \to \infty}f_{n}\left( x \right) \) for almost all \(x \in R\), we find \(f\) is measurable. Moreover, \(\left| f_{n} \right| \le g\) implies \(\left| f \right| < g\) almost everywhere and since \(g\) is integrable (hence finite a.e) we find \(f, f_{n}\) are integrable (hence finite) almost everywhere. Now, define for each \(n \in \N\) \[
		E_{n} = \{x \in \R : \left| f_{n}\left( x \right)  \right| , \left| f\left( x \right)  \right| < \infty, \left| f_{n}\left( x \right) - f\left( x \right)  \right| \le 2g\left( x \right) \}
	.\]
	Since \(R \setminus \bigcup_{n \in \N} E_{n}\) is a set of measure \(0\), we can assume \(\left| f_{n}\left( x \right)  \right| , \left| f\left( x \right)  \right| < \infty\) and \(\left| f_{n}\left(  \right) -f\left( x \right)  \right| \le 2g\left( x \right) \) for all \(x \in \R\). Then, Fatou's lemma applies to the sequence on nonnegative measurable functions \(\left( 2g - \left| f_{n} - f \right|  \right) \)  yielding
	\begin{align*}
		\int 2g &\le \liminf_{n \to \infty} \left( 2g - \left| f_{n} - f \right|  \right) \\
			&= \int 2g + \liminf_{n \to \infty} \left( - \int \left| f_{n} - f \right|  \right)  \\
			&= \int 2g - \limsup_{n \to \infty}  \int \left| f_{n} - f \right|  \\
			&\implies \limsup_{n \to \infty} \int \left| f_{n} - f  \right| \le 0\\
			&\implies \lim_{n \to \infty} \int \left| f_{n} - f \right| = 0
	.\end{align*}
	Hence, \(\lim_{n \to \infty}\left| \int \left( f_{n} - f \right)  \right|  = 0\) by the earlier lemma. So, \(\lim_{n \to \infty} \int f_{n} = \int f_{n}\).
\end{proof}
\begin{definition}[Convergence in Measure]
	Let \(\left( f_{n} \right) \) be a sequence of measurable functions \(f_{n}: \R \to \overline{\R}\) and \(f: \R \to \overline{\R}\) also be measurable. The sequence \(\left( f_{n} \right) \) \textbf{converges in measure} to \(f\) (\(f_{n} \to f\) by measure) if each \(f_{n}\)is finite almost everywhere and for each \(\epsilon > 0\) there is a \(N \in \N\) so that \[
		m\left( \{x \in \R: \left| f_{n}\left( x \right) - f\left( x \right) \right| > \epsilon\}  \right) < \epsilon
	\] for \(n \ge N\).
\end{definition}
\begin{theorem}[Riesz]
	Let \(\left( f_{n} \right) \) be a sequence of measurable functions \(f_{n}: \R \to \overline{\R}\) and \(f: \R \to \overline{\R}\) also being measurable. If \(\left( f_{n} \right) \to f\) in measure, then there is a subsequence \(\left( f_{n_{k}} \right) \) which converges pointwise almost everywhere to \(f\).
\end{theorem}
\begin{proof}
	First, we find a strictly increasing sequence of numbers \(\left( n_{k} \right) \) such that \(m\left( \{x \in \R : \left| f_{j}\left( x \right)  - f\left( x \right)  \right| > 2^{-k}\}  \right) < 2^{-k} \) if \(j \ge n_{k}\). For \(k \in \N\) denote \[
		S_{k} = \{x \in \R : \left| f_{n_{k}} - f\left( x \right)  \right| > 2^{-k}\}
	.\]
	Then, \(\sum_{k=1}^{\infty} m\left( S_{k} \right)  \le \sum_{k=1}^{\infty} 2^{-k} < \infty\). Applying the Borel-Cantelli Lemma yields that almost every \(x\in R\) does not belong to any infinite subcollections of \(\left( S_{k} \right) \). For such \(x\), we find a \(K \in \N\) such that \(\left| f_{n_{k}}\left( x \right)  - f\left( x \right)  \right| \le 2^{-k}\) for \(k \ge K\). Hence, \(f_{n_{k}}\) converges pointwise to \(f\) for all \(x\) not belonging to an infinite subcollection of \(\left( S_{k} \right) \), hence almost everywhere.
\end{proof}
