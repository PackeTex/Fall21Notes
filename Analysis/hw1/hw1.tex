\documentclass[a4paper]{article}
\input{../../preamble.tex}
\usepackage{pdfpages}
\title{Real Variables I: Homework I}

\date{Fri 03 Sep 2021 09:05}
\begin{document}
\maketitle
\begin{problem}[1]
	Let \(f:X \to Y\).
	\begin{enumerate}
		\item Show that for \(A\subseteq X\), \(B\subseteq Y\), \(f\left( f^{-1}\left( B \right)  \right) \subseteq B\) and \(A \subseteq f^{-1}\left( f\left( A \right)  \right) \).
			\item Give examples to show that the set inclusions can be proper.
	\end{enumerate}
\end{problem}
\begin{solution}
	\begin{enumerate}
	\item Let \(b \in f\left( f^{-1} \left( B \right)  \right) \) and note that, as \(b\) is in the image of \(f^{-1}\left( B \right) \), there is \(a \in f^{-1}\left( B \right) \) such that \(f\left( a \right) = b\). As \(a \in f^{-1} \left( B \right) \), we see \(f\left( a \right) \in B\). As \(f\left( a \right) = b \in B\) this completes the proof.\\
\\ Now, let \(a \in A\). We see \(f\left( a \right) \in f\left( A \right) \) by definition, and as \(f\left( a \right) \in f\left( A \right) \) we see that for all \(b \in A\) such that \(f\left( b \right) = f\left( a \right) \in f\left( A \right) \), we have \(b \in f^{-1} \left( f\left( A \right)  \right) \). It is clear that \(a\) is one such element, so \(a \in f^{-1} \left( f\left( A \right)  \right) \). This completes the proof.
	\item Let \(f: \R \to \R, \ x \mapsto f(x) = x^2\) and denote \(B = [-1,1]\). We see \(f^{-1} \left( B \right) = [-1,1]\) and \(f\left( [-1, 1] \right) = [0, 1]\). Hence, \(f\left( f^{-1} \left( B \right) \right) = [0, 1] \subset [-1, 1] = B\).
	\\\\
	Now, let \(f: \R \to \R, \ x \mapsto f(x) = 0\) and denote \(A = [0, 1]\). We see \(f\left( A \right) = \{0\} \) and \(f^{-1}\left( \{0\}  \right) = \R \) as the function is zero everywhere. Hence \(f^{-1} \left( f\left( A \right)  \right) = \R \supset [0, 1] = A\).
	\end{enumerate}
\end{solution}
\newpage
\begin{problem}[2]
	Let \(A, B \subseteq X\). Prove or disprove
	\begin{enumerate}
		\item \(A \triangle B = \O \iff A = B\).
			\item \(A \triangle B = X \iff A = B^{c}\).
	\end{enumerate}
\end{problem}
\begin{solution}
	\begin{enumerate}
		\item Suppose \(A \triangle B = \O\) and let \(a \in A\), \(b \in B\). Then, we see \(a \not\in B \setminus A\) by definition. Furthermore, as \(A \triangle B = \left( A\setminus B \right) \cup \left( B \setminus A \right)  = \O\), we see \(a \not\in A \setminus B\), but as \(a \in A\) this implies \(a \in B\). Hence \(a \subseteq B\). Again, notice \(b \not\in A \setminus B\) by definition. Furthermore, \(b \not\in B \setminus A\) as this would make \(A\triangle B\) nonempty, so \(b \in A\). Hence, \(A=B\).\\
			Conversely, suppose \(A = B\). Then, \[
				A \triangle B = A \triangle A = \left( A \setminus A \right) \cup \left( A \setminus A \right)  = \O \cup \O = \O
			.\]
		\item Suppose \(A\triangle B = X\) and let \(a \in A\). Then, we see \(a \not\in B \setminus A\) by definition, but \(a \in X\), so \( a \in A \setminus B\). Hence \(a \not\in B\). As every \(a \in A\) has \(a \not\in B\), we see \(A \subseteq B^{c}\). Now, let \(b \in B^{c}\). We see \(b \not\in B\) by definition, hence \(b \not\in B \setminus A\). As \(b \in X\), we must then have that \(b \in A \setminus B\), hence \(b \in A\). Thus, \(B^{c} = A\).\\
			Conversely, suppose \(B^{c} = A\). Then, \[
				A \triangle B = B^{c} \triangle B = \left( B^{c} \setminus B \right)  \cup \left( B \setminus B^{c} \right) = B^{c} \cup B = X
			\] by definition of complements.
	\end{enumerate}
\end{solution}
\newpage
\begin{problem}[3]
	Suppose \(f:X \to Y\) and \(g: Y\to Z\)  are functions.
	\begin{enumerate}
		\item Show that \(f:X \to Y\) is injective if and only if there is a map \(g: Y \to X\) such that \(g \circ f\) is the identity on \(X\). If such a map \(g\) exists is it necessarily unique, injective, or surjective.
			\item Show that \(f\) is onto if and only if there is a map \(g:Y \to X\) such that \(f \circ g\) is the identity on \(Y\).
	\end{enumerate}
\end{problem}
\begin{solution}
	\begin{enumerate}
		\item Let \(g: X \to Y\) be a map such that \(g \circ f\) is the identity on \(X\). Then, suppose \(f\) is not injective. Let \(x , y \in X\) such that \(x\neq y\) and \(f\left( x \right)  = f\left( y \right) \). Then \(g\left( f\left( x \right)  \right)  = x \) element. WLOG, suppose \(g\left( f\left( x \right)  \right) = g\left( f\left( y \right)  \right) = x \). Then, \(g\left( f\left( y \right)  \right) =x\) contradicts the assumption that \(g \circ f\) was the identity. \\
		\\	Now, suppose \(f\) is injective. Then, for each \(x \in X\) there is a unique \(f\left( x \right) \in Y\). Hence, let us define the map \(g: Y \to X\) such that \(g\left( f\left( x \right)  \right) = x\) for all \(x \in X\). We see this is a function as each \(f\left( x \right) \in Y \) originates from only \(1\) \(x \in X\) by injectivity. Hence, this implies \(g \circ f\) is the identity by this definition. This completes the proof.
		\item We wish to see if this map is necessarily unique. Suppose \(h: Y \to Z\) is another map such that \(h \circ f\) is the identity on \(X\). Then, for every \(x \in X\), we have \(\left( h \circ f \right) \left( x \right)  = x\), that is \(h\left( f\left( x \right)  \right) = x\), hence \(h: f\left( x  \right) \mapsto x\), just as \(g\), hence \(h = g\) are the same maps.\\
		\\	Furthermore, we wish to see if the map is necessarily a bijection. As \(g\) is the unique inverse map, let us now it denote \(f^{-1}\) as it is equivalent to a pointwise preimage. We see \(X \subseteq f^{-1} \left( f\left( X \right)  \right) \) by problem \(1\), and as the universe under consideration is \(X\), this implies \(f^{-1} : Y \to X\) is in fact a surjection (its image is \(X\)). As for  injectivity , we know the function \(f\) must map \(x \mapsto f\left( x \right) \) uniquely, that is, \(\left| f\left( \{x\}  \right)  \right| = 1\). Hence, suppose \(f^{-1} \left( y \right) = f^{-1} \left( z \right) \) for some \(y \neq z\). Then, by our earlier observation, \(f\left( x \right) = y \text{ and } z \), hence \(\left| f\left( \{x\}  \right)  \right| > 1\). \(\lightning\). So, \(g\) is necessarily a bijection.
\item Suppose \(g:Y \to X\) is a map such that \(f \circ g\) is the identity on \(Y\). Then, we have for each \(y \in Y\), \(f\left( g\left( y \right)  \right) = y\), so, we see \[\bigcup_{y \in Y} f\left( g\left( \{y\}  \right)  \right) = f\left( g\left( \bigcup_{y \in Y} \{y\} \right)   \right) =   f\left( g\left( Y \right)  \right) = \bigcup_{y \in Y} \{y\} = Y.\] That is, \(f\left( g\left( Y \right)  \right) = Y\), hence the image of \(f\) is \(Y\), so \(f\) is a surjection.\\
	\\		Now, suppose \(f\) is onto. Then, for each \(y \in Y\), there is a \(x \in X\) such that \(f\left( x \right)  =y\). Hence, define \(g: Y \to X, \ y \mapsto g(y) = x\) where \(x\) is the afformentioned element such that \(f\left( x \right) =y\) for this particular \(y\). Then, we see \(\left( f \circ g \right) \left( y \right) = f\left( g\left( y \right)  \right) = f\left( x \right) = y\) for arbitrary \(y\), so \(\left( f \circ g \right) \) is the identity on \(Y\),
	\end{enumerate}
\end{solution}
\newpage
\begin{problem}[4]
	Prove or disprove the following. If \(\mathscr{A}\) is a \(\sigma\)-Algebra of subsets of \(Y\) and \(f: X \to Y\) is a function, then the collection \(\{f^{-1} \left( A \right) : A \in \mathscr{A}\} \) is a \(\sigma\)-Algebra of subsets of \(X\).
\end{problem}
\begin{solution}
	First, denote the collection \(\{f^{-1}\left( A \right) : A \in \mathscr{A}\} = \mathscr{B}\). We show all three conditions:
	\begin{enumerate}
		\item As \(Y \in \mathscr{A}\) and \(f\left( X \right) \subseteq Y\) necessarily, we see \(X \subseteq f^{-1} \left( Y \right) \) (as \(X\) is the whole of the domain, we can even say \(X = f^{-1}\left( Y \right) \)). Hence, \(f^{-1}\left( Y \right) = X \in \mathscr{B}\).
		\item Let \(A \in \mathscr{A}\), then \(A^{c} \in \mathscr{A}\). As \(f^{-1}\left( A^{c} \right) = \left[ f^{-1}\left( A \right)  \right] ^{c} \), we see \(\left[ f^{-1}\left( A \right)  \right] ^{c} \in \mathscr{B}\) (for all \(f^{-1}\left( A \right) \in \mathscr{B}\)).
		\item Lastly, let \(B_1, B_2, \ldots \in \mathscr{B}\) be a countable collection of elements with each \(B_{i} = f^{-1}\left( A_{i} \right) \) for \(A_{i} \in \mathscr{A}\) and define \(\bigcup_{n \in \N} B_{n} = B\). We see \(\bigcup_{n \in \N} A_{n} \in \mathscr{A}\) by hypothesis, hence
			\begin{align*}
				\mathscr{B} &\ni f^{-1}\left( A \right) \\
				&= f^{-1}\left( \bigcup_{n \in \N} A_{n} \right) \text{ by construction of \(\mathscr{B}\)}\\
					    &= \bigcup_{n \in \N} f^{-1}\left( A_{n} \right)  \\
					    &= \bigcup_{n\in \N} B_{n} \\
					    &= B \in \mathscr{B}
			.\end{align*}
	\end{enumerate}
	Hence \(\mathscr{B}\) is a \(\sigma\)-Algebra.
\end{solution}
\newpage
\begin{problem}[5]
	Prove the set of all polynomials with rational coefficients is rational.
\end{problem}
\begin{solution}
	Let \(f = \sum_{i= 0}^{n} a_{i} x^{i}\) be an arbitrary polynomial and define the finite sequence \(\left( f_{k} \right)_{i=0}^{n}\) such that \(f_{k} = a_{k}\) for each \(k\) and each polynomial \(f\). Next, define \(\mathscr{F} = \{\left( f_{k} \right)_{k=1}^{n}: n \text{ is finite, }f \in P_{\Q}\left( n \right)  \} \) where \(P_{\Q}\left( n \right) \) is the set of all rational polynomials of degree at most \(n\). We see \(\mathscr{F}\) contains a sequence corresponding to each finite polynomial with rational coefficients, hence as \(\Q\) is countable, and \(\mathscr{F}\) is a subset of the set of all finite sequences from \(\Q\) (which is countable by a proposition in class), we see \(\mathscr{F}\) is countable. As each rational polynomial of finite length, \(f\), has a corresponding sequence \(\left( f_{k} \right) \in \mathscr{F}\), we see the set \(\{f: f \text{ is a rational polynmomial of finite length}\}  \subseteq \mathscr{F}\). Hence, this set is also countable.
\end{solution}
\newpage
\begin{problem}[6]
	Prove the set of all infinite sequences \(\left( x_{k} \right) \) with \(x_{k} \in \{0, 1\} \) is uncountable.
\end{problem}
\begin{solution}
	Assume indirectly that such a set is countable. Let \(f: \N \to \{\left( x_{k} \right)_{k \in \N} : x_{k} \in \{0, 1\}  \} , \ n \mapsto f(n) = \left( x_{n, k} \right) _{k}\). Now define a sequence \(\left( y_{k} \right) \) such that \[y_{k} = \left \{
		\begin{array}{11}
			0 , & \quad  x_{k, k}= 1 \\
			1, & \quad x_{k, k} = 0
		\end{array}
		\right.\]
		We see \(\left( y_{k} \right) \) differs from each sequence , \(f\left( n \right) = \left( x_{n, k} \right) _{k}\) in the \(n\)-th position. Hence, \(f\) is not surjective, so there is no bijection from \(\{\left( x_{k} \right) : x_{k} \in \{0, 1\} \} \to \N\), so the set is not countable.
\end{solution}
\newpage
\begin{problem}[7]
	Let \(A\) be a set and \(B = \{0, 1\} \). Prove there exists a bijection from \(\mathscr{P}\left( A \right) \) to the set of all functions from \(A\) to \(B\).
\end{problem}
\begin{solution}
	Define the set of all functions from \(A\) to \(B\) as \(\mathscr{F}\left( A, B \right) \). Define a function \(f: \mathscr{P}\left( A \right) \to \mathscr{F}\left( A, B \right) \) such that for \(X \in \mathscr{P}\left( A \right) \), \(f\left( X \right) = g: A \to B\) such that for \(a \in A\), \[
		g\left( a \right)  = \left \{
			\begin{array}{11}
				0, & \quad a \in X \\
				1, & \quad a \not\in X
			\end{array}
			\right.
	.\]
	This is clearly a function as each element \(X \in \mathscr{P}\left( A \right) \) has either \(a \in X\) or \(a \not\in X\) for every \(a \in A\). Now, we check that it it bijective.\\
	Suppose \(X, Y \in \mathscr{P}\left( A \right) \) such that \(f\left( X \right) = g = f\left( Y \right) \). Then, we see for each element \(a \in A\), \(f\left( X \right) \left( a \right) = g\left( a \right) = f\left( Y \right) \left( a \right) \), hence if \(a \in X\), then \(a \in Y\). Similarly, if \(a \not\in X\), then \(a \not\in Y\). Hence, as every \(a \in X\) has \(a \in Y \) and every \(a \not\in X\) has \(a \not\in Y\), we see \(X = Y\), so \(f\) is an injection. Now, we wish to show that \(f\left( \mathscr{P}\left( A \right)  \right) = \mathscr{F}\left( A, B \right) \). As we already know \(f\left( \mathscr{P}\left( A \right)  \right) \subseteq \mathscr{F}\left( A, B \right) \), we must only show the reverse containment holds. Let \(g \in \mathscr{F}\left( A, B \right) \). Then, for each element \(a \in A\), \(g\left( a \right) = 0 \text{ or } 1\). Define a new set \(J\) such that \[
	\left \{
		\begin{array}{11}
			a \in J, & \quad g\left( a \right) =1  \\
			a \not\in J, & \quad g\left( a \right) \neq 1
		\end{array}
		\right.
	.\]
	We see \(J \subseteq A\), hence \(J \in \mathscr{P}\left( A \right) \) as it contains some (perhaps all) of the elements of \(A\), and \begin{align*}
		f\left( J \right) \left( a \right) &= \left \{
		\begin{array}{11}
			1, & \quad a \in J \\
			0, & \quad a \not\in J
	\end{array}
\right \\
						   &= g\left( a \right) \end{align*}
						   so \(g \in f\left( \mathscr{P}\left( A \right)  \right) \). Hence \(f\left( \mathscr{P}\left( A \right)  \right) = \mathscr{F}\left( A, B \right) \), so \(f\) is a bijection.
\end{solution}
\newpage
\begin{problem}
	Let \(X = Z \times \left( Z \setminus \{0\}  \right) \). Define a relation \(\sim\) on \(X\) such that \(\left( p, q \right) \sim \left( u, v \right) \) if \(pv = qu\).
	\begin{enumerate}
		\item Show that \(\sim\) is an equivalence relation on \(X\).\\
		\item Show that there exists a bijection \(f: \left( X / \sim \right) \to \Q\).
	\end{enumerate}
\end{problem}
\begin{solution}
\begin{enumerate}
	\item First we show \(~\) is reflexive. Note that \(pq = pq\), hence \(\left( p, q \right) \sim \left( p, q \right) \).\\
		Now, we show it is symmetric. Note that if \(pv = qu\), then \(uq=vp\), hence \(\left( p, q \right) \sim \left( u,  v \right) \implies \left( u, v \right) \sim \left( p, q \right)  \).\\
		Lastly, we show transitivity. Suppose \(\left( a, b \right) \sim \left( c, d \right) \) and \(\left( c, d \right) \sim \left( e, f \right) \). Then this implies \(ad=bc\) and \(cf=de\) dividing through by the (guaranteed) nonzero term in both equations yields \(\frac{a}{b} = \frac{c}{d}\) and \(\frac{c}{d}= \frac{e}{f}\), hence \(\frac{a}{b} = \frac{e}{f}\) so \(af = eb\), so \(\left( a, b \right) \sim \left( e, f \right) \).\\
		Hence the relation is an equivalence relation.
	\item Now, we wish to induce a bijection between \(\left( X / \sim \right) \) and \(\Q\), this will follow directly from the proof of transitivity. For each equivalence class \( \left[ \left( a, b \right)  \right] \in \left( X / \sim \right) \) define \(f \left( \left[ \left( a, b \right)  \right]  \right) = \frac{a}{b}\) (we know this is well defined as the second element is guaranteed to be nonzero). Now, we wish to show that the choice of representative is unimportant, so let \(\left( a, b \right) , \left( c, d \right) \in \left[ \left( a, b \right)  \right] \) (hence \(\left( a, b \right) \sim \left( c, d \right) \)). From the previous proof, we see that dividing by the nonzero term yields \(\frac{a}{b} = f\left( \left[ \left( a, b \right)  \right]  \right)  =  \frac{c}{d} = f\left( \left[ \left( c, d \right)  \right]  \right) \) hence the choice of representative produces the same rational.\\
		Now, we wish to show this mapping is injective. Suppose two different equivalence classes, \(x, y \in \left( X / \sim \right) \) have \(f\left( x \right) = f\left( y \right) \). Let \(\left( x_1, x_2 \right) \in x\) and \(\left( y_1, y_2 \right) = y\) be representatives of each equivalence class. Then, this implies \(f\left( x \right)  = \frac{x_1}{x_2}= \frac{y_1}{y_2} = f\left( y \right) \). Multiplying through by the denominators yields \(x_1 y_2 = y_1x_2\), hence \(\left( x_1, x_2 \right)\sim \left( y_1, y_2 \right)  \), so \(x= y\).
		Lastly, we wish to show this is a surjection. Let \(\frac{p}{q}\in \Q\) be a rational. Then, by definition, \(p \in \Z\) and \(q \in Z \setminus \{0\} \), so \(\left[ \left( p, q \right)  \right] \in \left( X / \sim \right) \) and \(f\left( \left[ \left( p, q \right)  \right]  \right) = \frac{p}{q}\), so there is an equivalence class that produces each rational.\\
		Hence, the mapping \(f:\left( X / \sim \right) \to \Q : \left[ x_1, x_2 \right]  \mapsto f\left( \left[ \left( x_1, x_2 \right)  \right]  \right) = \frac{x_1}{x_2}\) is a bijection.
\end{enumerate}
\end{solution}
\end{document}
