\lecture{3}{Tue 31 Aug 2021 12:17}{Construction of the Reals}
\section{Construction of the Reals}
\begin{definition}[Rational Cauchy Sequences]
	We define a \textbf{rational cauchy sequence} to be \((x_n)_{n\in\N}\) with \(x_{n} \in \Q\), such that for all \(\epsilon > 0\), there is a \(N \in \N\) such that \(\left| a_{n} - a_{m} \right| < \epsilon\) for \(n, m \ge N\). We denote the set of all rational cauchy sequences to be \(\CS \left( \Q \right) \).
\end{definition}
Just as with standard cauchy sequences, we have that \(\left( a_{k} \right) , \left( b_{k} \right) \in \CS\left( \Q \right) \) implies \(\left( a_{k} \pm b_{k} \right) , \left( a_{k} \cdot b_{k} \right) \in \CS\left( Q \right)  \).
\\ Now, we define an equivalence relation on \(\CS\left( \Q \right) \) such that \(\left( a_{k} \right) \sim \left( b_{k} \right) \) if for all \(\epsilon > 0\) there is \(N \in \N\) such that \(\left| a_{m} - b_{m} \right| < \epsilons\) if \(m\ge N\).\\
\begin{remark}
	It is of note that this essentially identifies all sequences which want to converge to the same point (though they may not converge to a rational number, or a number at all).
\end{remark}
Now, we examine the equivalence classes, that being the quotient space \(\CS \left( \Q \right) / \sim\), and we note that if \(q \in \Q\), then \(\left( q \right) _{k} \in \left[ \left( q \right) _{k} \right] \). That is, we can identify \(\Q\) with a subset of the quotient space.
\\ Next, we aim to extend the total ordering of \(\Q\) to \(\CS\left( \Q \right) / \sim \). Let us introduce the relation \(\leqslant\) on \(\CS\left( Q \right) \) such that for \(x, y \in \CS\left( Q \right) / \sim\), we say \(x \leqslant y\) if for all \(\left( x_{k} \right) \in x\) and \(\left( y_{k} \right) \in y\) and for every rational \(\epsilon > 0\) there exists \(N \in \N\) such that \(x_{m} \leq y_{m} + \epsilon\) for \(m \ge N\).
\begin{proposition}
	Suppose \(x, y \in \CS \left( \Q \right) / \sim\) and there are \(\left( x_{k} \right) \in x\), \(\left( y_{k} \right) \in y\) with the property that for all rational \(\epsilon > 0\) there is a \(N \in \N\) such that \(x_{m}\le y_{m} + \epsilon\) if \(m\ge N\), then \(x \leqslant y\). Essentialy, we need not show the property holds for all members of a given equivalence class, but merely a representative of a given equivalence class.
\end{proposition}
\begin{proof}
	Let \(\epsilon > 0\) be rational, then there is \(N \in \N\) such that \(x_{m} \le y_{m} + \frac{\epsilon}{3}\) if \(m\ge N\). For \(\left( \hat{x_{k}} \right) \in x\) and \(\left( \hat{y_{k}} \right) \in y\), then there is \(K \in \N\) such that \(\left| \hat{x_{m}}- x_{m} \right| < \frac{\epsilon}{3}\) and \(\left| \hat{y_{m}}- y_{m} \right| < \frac{\epsilon}{3}\) if \(m \ge K\).\\
	Hence, if \(m\ge \max \{K, N\} \), then \[
		\hat{x_{m}}< x_{m} + \frac{\epsilon}{3}\le y_{m} < \frac{2\epsilon}{3}< \hat{y_{m}}+ \epsilon
	.\]
Hence \(\hat{x_{m}} < \hat{y_{m}} + \epsilon\).
\end{proof}
\begin{lemma}
	The relation \(\leqslant\) is a total ordering on \(\CS \left( \Q \right)  / \sim\) .
\end{lemma}
\begin{proof}
	First, let us show it is a partial ordering. It is clear that \(\leqslant\) is reflexive and transitive as it identifies equivalence classes. Now, suppose \(x, y \in \CS \left( \Q \right) / \sim\) such that \(x\leqslant y\) and \(y \leqslant x\). Given \(\left( x_{k} \right) \in x\) and \(\left( y_{k} \right) \in y\), for a given rational \(\epsilon > -\) we can find \(N \in \N\) such that \(x_{m} \le y_{m} + \frac{\epsilon}{2}\) and \(y_{m} \le x_{m} + \frac{\epsilon}{2}\) for \(m \ge N\).\\
	Hence, \(\left| x_{m} - y_{m} \right|\le \frac{\epsilon}{2}< \epsilon \) for \(m \ge N\), hence \[\underbrace{\left[ \left( x_{m} \right)  \right] }_{= x} = \underbrace{\left[ \left( y_{m} \right)  \right] }_{=y}.\] Thus, \(x=y\), so \(\leqslant\) is a partial ordering.\\
	\newpage
	Now, let us show any two elements are comparably to obtain a total ordering. Let \(x, y \in \CS\left( \Q \right) / \sim\) and suppose \(x\nleqslant y\), then we must show \(y\leqslant x\). For \(\left( x_{k} \right) \in x\), \(y_{k} \in y\) we have a rational \(\epsilon > 0\) such that \(x_{m} > y_{m} + \epsilon\) for infinitely many \(m \in \N\). Also, there is \(K \in N\) such that for \(k, m \ge k\), we have \(\left| x_{k} - x_{m}  \right| < \frac{\epsilon}{2}\) and \(\left| y_{k} - y_{m} \right| < \frac{\epsilon}{2}\). Hence, if \(m\ge K\) such that \(x_{m} > y_{m} + \epsilon\) (there are infinitely many so we must only chooses a sufficiently large one), we obtain for \(k \ge K\),
	\begin{align*}
	x_{k} + \frac{\epsilon}{2} &> x_{k} + \left( x_{m} - x_{k} \right)  \\
				   &= x_{m} \\
				   &> y_{m} + \epsilon \\
				   &= y_{k} + \left( y_{m} - y_{k} \right) + \epsilon \\
				   &> y_{k} + \epsilon - \frac{\epsilon}{2}\\
				   &= y_{k} + \frac{\epsilon}{2}
	.\end{align*}
	Hence, \(x_{k} > y_{k}\), so \(y_{k} < x_{k} + \epsilon\) for \(\epsilon > 0\) and \(k \ge K\). Hence \(y \leqslant x\), so \(\leqslant\) is a total ordering.
\end{proof}
\begin{definition}[Operations on \(\CS (\Q) / \sim\)]
	Now, we wish to introduce addition of \(\CS\left( \Q \right) / \sim\).\\
	Let \(x, y \in \CS\left( \Q \right) / \sim\) with \(\left( x_{k} \right) \in x\) and \(\left( y_{k} \right)\in y \), we define \(x \pm y = \left[ \left( x_{k} \pm y_{k} \right)  \right] \) and \(xy = \left[ \left( x_{k}y_{k} \right)  \right] \).
\end{definition}
We note that this is well defined as we already know \(\left( x_{k} \pm y_{k} \right) \) and \(\left( x_{k}y_{k} \right) \) to be valid rational cauchy sequences. Furthermore, these definitions are independent of which representative we choose. The proof of this is trivial by the algebraic properties of \(\Q\).
\begin{definition}[Real Numbers]
	We define the \textbf{real numbers}, \(\R\) to be \(\CS \left( \Q \right) \) together with the total ordering \(\leqslant\), addition, subtraction, and multiplication.
\end{definition}
\begin{convention}
	If \(x\) is rational, integral, or natural, we simply write \(x= x \in \R\) because we can identify \(x\) with its equivalence class. Furthermore, if \(p, q \in \Q\) such that \(p \le q\), then \(p \leqslant q\), hence we may say for \(a, b \in \R\) with \(a \leqslant b\) we also have \(a \le b\), so the notation \(\leqslant\) becomes redundant. Furthermore, the other order symbols \(<, >, \ge\) are also valid on \(\R\), defined in their usual way.
\end{convention}
\begin{lemma}[The Archimidean Property]
	For any two real numbers \(x, y \in \R\) with \(x > 0\), there exists a natural number \(m\) such that \(mx > y\).
\end{lemma}
\begin{proof}
	Suppose the opposite, then \(mx \le y\) for all \(m \in \N\). Let \(\left( x_{k} \right)\in x \) and \(\left( y_{k} \right)\in y \) and let \(M \in \Q\) such that \(y_{m} \le M\) for all \(m \in \N\) (as rational cauchy sequences are always bounded). Since \(\Q\) is archimidean (trivially), for given rational \(\epsilon > 0\), we find there exists \(m \in \N\) such that \(\frac{M+1}{m} < \epsilon\). Also, there is \(N \in \N\) such that \( 0 \le m + \frac{\epsilon}{2}\) for \(m \ge N\) and \(mx_{m} \le y_{m} + 1\) for \(m\ge N\).\\
	Hence, the latter inequality implies \(x_{m} \le \frac{y_{m}+1}{m} \le \frac{M+1}{m}< \epsilon\) while the first inequality shows \(x\ge - \frac{\epsilon}{2} > -\epsilon\) for all \(m \ge N\). Consequently, \(\left| x_{m} \right|< \epsilon \) for \(m \ge N\). Hence \(\left[ \left( x_{m} \right)  \right] = x = 0\). \(\lightning\). So, the claim is true.
\end{proof}
\begin{corollary}
	For any two real \(x, y \in \R\) with \(x<y\) there exists a rational \(q \in \Q\) such that \(x< q < y\).
\end{corollary}
