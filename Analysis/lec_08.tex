\lecture{8}{Thu 16 Sep 2021 13:02}{Continuity (2) and Extended \(\R\)}
We begin with some results on continuity over intervals and inverses.
\begin{recall}
	If \(f : I \to \R\)  is monotone with \(I\) being an interval. Then, \(f\) is continuous if and only if \(f\left( I \right) \) is an interval.
\end{recall}
\begin{corollary}
	A continuous strictly monotone function \(f: I \to \R\), with \(I\) being an interval, has a continuous inverse \(f^{-1}: f\left( I \right) \to \R\).
\end{corollary}
\begin{proposition}
	A strictly monotone function \(f: I \to \R\), with \(I\) being an interval, has a continuous inverse.
\end{proposition}
\begin{theorem}[Heine's Theorem]
	A continuous function \(F : S\to \R\) with \(S\) being compact is uniformly continuous.
\end{theorem}
\begin{proof}
	For \(\epsilon > 0\) and \(x \in S\), there is a \(\delta_{x} > 0\) such that \(\left| f\left( x \right) - f\left( y \right)  \right| < \frac{\epsilon}{2}\) if \(\left| x-y \right|  < \delta_{x}\). Let \(U_{x} = \left( x - \frac{\delta_{x}}{2}, x+ \frac{\delta_{x}}{2} \right) \). Since \(\{U_{x} : x \in S\} \) is an open cover of \(S\), there are \(x_1, x_2, \ldots, x_{n}\) such that \(\{U_{x_{k}} : 1\le k \le n\} \) is a finite open subcover of \(S.\) Let \(\delta = \min \{\frac{1}{2}\delta_{x_1}, \frac{1}{2}\delta_{x_2}, \ldots, \frac{1}{2}\delta_{x_{n}}\} \) and suppose \(x, y \in S\) such that \(\left| x- y \right| < \delta\). Then there is \(x_{k}\) for some \(1 \le k \le n\) such that \(x \in U_{x_{k}}\) and \(\left| x_{k} - y \right| \le \left| x-y \right| + \left| x_{k} - x \right|  < \delta + \frac{\delta_{x_{k}}}{2} \le \delta_{x_{k}}\).\\
	Consequently
	\[
		\left| f\left( x \right)  - f\left( y \right)  \right| \le   \left| f\left( x \right)  - f\left( x_{k} \right)   \right| + \left| f\left( x_{k} \right) - f\left( y \right)    \right| < \epsilon  .
	\]
\end{proof}
\begin{note}{Justification for continuity}
	There is an equivalence between open sets/continuity and measurable sets/measurableness.
\end{note}
\begin{definition}[Convergence of functions]
	Let \(\left( f_{n} \right) \)	 be a sequence of functions \(f_{n} : S \to \R\). Then
	\begin{enumerate}
		\item \( \left( f_{n} \right) \) \textbf{converges pointwise} if \(\left( f_{n}\left( x \right)  \right) \) is convergent for every \(x \in S\). The limit is defined pointwise for every \(x \in S\) with \(f\left( x \right)  = \lim_{n \to \infty}f_{n} \left( x \right) \) with \(f : S \to \R\) being a function.
		\item \(\left( f_{n} \right) \) \textbf{converges uniformly} to the function \(f: S\to \R\) if for each \(\epsilon > 0\), there \( N \in \N\) such that for all \(x \in S\), \(\left| f\left( x \right)  - f_{n} \left( s \right)  \right| < \epsilon \) if \(n \ge N\).
	\end{enumerate}
\end{definition}
\begin{theorem}
	Suppose \(\left( f_{n} \right) \)	 is a sequence of continuous functions \(f_{n} : S \to \R\) which converges uniformly to \(f: S \to \R\). Then, \(f\) is continuous.
\end{theorem}
\begin{proof}
	Let \(x \in S\) and \(\epsilon > 0\). Then, there is \(k \in \N\) such that \(\left| f\left( y \right)  - f_{k} \left( y \right)  \right| < \frac{\epsilon}{3} \) for all \(y \in S\). Consequently, for any \(y \in S\)
	\begin{align*}
		\left| f\left( x \right) - f\left( y \right)  \right| &\le \left| f\left( x \right)  - f_{k}\left( x \right)  \right|  + \left| f_{k} \left( x \right) - f_{k}\left( y \right)  \right|  + \left| f\left( y \right)  - f_{k}\left( y \right)  \right| \\
								      &< \frac{2\epsilon}{3} + \left| f_{k} \left( x \right)  - f_{k}\left( y \right)  \right|
	.\end{align*}
Since \(f_{k}\) is continuous, we can pick a sufficient \(\delta >0 \) such that this completes the proof.
\end{proof}
\section{Extended \(\R\)}
Recall that many objects such as the \(\limsup\) and \(\liminf\) required a boundedness assumption. We wish to discard this assumption when possible. Hence we introduce the following system.
\begin{definition}[Extending Functions]
	A function \(h: \R \to \R\) is \textbf{extending} if \(h\) is strictly inreasing and \(h\left( \R \right) = \left( -1, 1 \right) \). Note that every extending function is continuous by these assumptions and has a continuous inverse.
\end{definition}
Now, we introduce two external elements \(-\infty,  + \infty\) and we define \(\infty = H^{-1} \left( 1 \right) \) and \(-\infty = H^{-1} \left( -1 \right) \) and we extend the ordering \(\le\) such that \(-\infty < \infty\) and \(-\infty < x < \infty\) for every \(x \in \R\).
\begin{definition}[Extended Real Numbers]
	We denote \(R \cup \{-\infty ,\infty\}  = \overline{\R} = \left[ -\infty,\infty \right] \) to be the \textbf{extended real numbers} for use with extending functions.
\end{definition}
In this way, the extending function \(h\) extends from \(R\) to \(\overline{R}\) and it retains its strictly increasing and the image requirement \(h\left( \overline{\R} \right) = \left[ -1, 1 \right] \).\\
\begin{notation}
	\begin{itemize}
		\item \(\left( a, \infty \right] = \{x \in \overline{\R} : x > a\}  \)
			\item \(\left[ -\infty, a \right]  = \{x \in \overline{\R} : x \le a\} \)
			\item \(\left( a, \infty \right)_{\overline{\R}} = \{x \in \overline{\R} : a < x < \infty\} \)
				\item and so on.
	\end{itemize}
\end{notation}
It is of note that the interval \(\left( a, \infty \right) \) is \(\R\) is still defined as normal, it is only when it is chosen as part of \(\overline{\R}\).\\
Now we examine the topology on \(\overline{\R}\).
\begin{definition}[Topology on \( \overline{\R} \)]
\begin{enumerate}
	\item \(S \subseteq \overline{\R}\) is open/closed if \(H\left( S \right) \) is relatively open/closed in \(\left[ -1, 1 \right] \) for any extending function \(H\).
	\item \(S \subseteq \overline{\R}\) has \(\sup \left( S \right)  = H^{-1}\left(\sup\left( H\left( S \right)  \right) \right)\).
	\item A sequence \(\left( x_{n} \right) \) in \(\overline{\R}\) is convergent (in \(\overline{\R}\)) if for any extending function \(H\), \(\left( H\left( x_{n} \right)  \right) \) isconvergent. In this case we define \[\lim_{n \to \infty}x_{n} = H^{-1} \left( \lim_{n \to \infty} H \left( x_{n} \right)  \right) \].
	\item A point \(x_0 \in \overline{\R}\) is an accumulation or cluster point of the sequence \(\left( x_{n} \right) \) in \(\overline{\R}\) if for any extending function \(H\) we have \(H\left( x_0 \right) \) is an accumulation point of \(\left( H\left( x_{n} \right)  \right) \).
	\item Let \(\left( x_{n} \right) \) in \(\overline{\R}\) . Then, \begin{align*}
			\limsup_{n \to \infty}x_{n} &= H^{-1}\left( \limsup_{n \to \infty}H\left( x_{n} \right)  \right) \\
			\liminf_{n \to \infty}x_{n} &= H^{-1}\left( \liminf_{n \to \infty}H\left( x_{n} \right)  \right)
		\end{align*}
\end{enumerate}
\end{definition}
\begin{example}
	\begin{itemize}
\item	\(\overline{\R}\) is open and closed in \(\overline{\R}\).
	\item \(\R\) is open but not closed in \(\overline{\R}\).
	\item \(\left( 7, \infty \right] \mapsto \left( H\left( 7 \right) , 1 \right]  \), hence it is open.
	\end{itemize}

\end{example}
\begin{proposition}
	If \(\left( x_{n} \right) \) is a sequence with \(x_{n} \in \overline{\R}\). Then \[\limsup_{n \to \infty}x_{n}, \liminf_{x \to \infty} x_{n} \in \overline{\R}\] with
	\begin{align*}
		\limsup_{n \to \infty}x_{n} &= \inf \left( \sup \{x_{k} : k \in \N, k \ge n\} : n \in \N  \right)\\
		&= \lim_{n \to \infty} \sup \{x_{k} : k\in \N, k\ge n\} \text{ and} \\
		\liminf_{n \to \infty}x_{n} &=  \sup \left( \inf \{x_{k} : k \in \N, k\le n\} : n \in\N  \right)  \\
		&= \lim_{n \to \infty} \inf \{x_{k} k\in \N , k \le n\}  \\
	\end{align*}
\end{proposition}

\begin{remark}
	A sequence \(\left( x_{n} \right) \) in \(\R\) is said to converge to \(\infty\) if it is convergent in \(\overline{R}\) with \(\lim_{n \to \infty}x_{n} = \infty\).
\end{remark}
\begin{definition}[]
	\begin{enumerate}
		\item If \(a \in \left( -\infty, \infty \right]\), then \(a + \infty = \infty + a = \infty \).
		\item If \(a \in \left[ -\infty, \infty \right) \) then \(a + \left( -\infty \right) = \left( -\infty \right) + a = -\infty  \).
		\item If \(a \in \left( 0, \infty \right] \) then \(a \cdot \infty = \infty \cdot a = \infty\).
			\item If \(a \in \left[ -\infty, a \right) \) then \(a \cdot \infty = \infty \cdot a = -\infty\).
			\item If \(a \in \left( -\infty, \infty \right) \setminus \{0\}  \) then \(\frac{\infty}{a} = \frac{1}{a} \cdot \infty\).
			\item If \(a \in \left( -\infty, \infty \right) \) then \(\frac{a}{\infty} = \frac{a}{-\infty} = 0\).
			\item If \(a \in \left[ -\infty, \infty \right] \setminus \{0\}  \) then \(\left| \frac{a}{0} \right| = \infty\) (though \(\frac{a}{0}\) is left undefined).
			\item \(\left| \infty \right|=  \left| -\infty \right| = \infty \) and \(\infty^{p} = \infty\), \(\infty^{-p} = 0\) for \(p > 0 \).
			\item \(0 \cdot \infty = \infty \cdot 0 = 0 \cdot \left( -\infty \right) = \left( -\infty \right) \cdot 0 \coloneqq 0  \).
				\item \(\frac{\infty}{\infty} = \frac{-\infty}{\infty} = \frac{\infty}{-\infty} = \frac{-\infty}{-\infty} \coloneqq 0\)
	\end{enumerate}
	These last definitions go against our conventional logic involving \(\infty\), but they are simply definitions which will be useful for measure theoretic results later on.

\end{definition}
These conventions do have the unfortunate consequence that \(\lim_{n \to \infty} \frac{x_{n}}{y_{n}} \neq \frac{\lim_{n \to \infty}x_{n}}{\lim_{n \to \infty}y_{n}}\) in general for sequences \(\left( x_{n} \right), \left( y_{n} \right), \left( \frac{x_{n}}{y_{n}} \right)  \) in \(\overline{\R}\). These facts still hold in sequences which converge in \(\R\) (in \(\overline{\R}\)), it is simply when a sequence converges only in \(\overline{\R}\) for which we have issues.
\begin{remark}
	We left undefined \(\infty - \infty\), \(-\infty + \infty\), and \(\frac{x}{0}\) for \(x \in \overline{\R}\). Furthermore, we have \(\frac{x}{y} = x \cdot \frac{1}{y}\) only if \(x \in \overline{R}\), \(y \in \overline{\R}\setminus \{0\} \).
\end{remark}
