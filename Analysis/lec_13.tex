\lecture{13}{Tue 05 Oct 2021 13:02}{}
We construct a cantor set.\\
First, suppose the interval \(\left[ 0, 1 \right] \) and a series of sets \(C_0, C_1 ,\ldots\) where \(C_{i} = C_{i - 1} \setminus D_{i}\) where \(D_{i}\) is just the set consisting of the middle thirds of each interval of \(C_{i-1}\). Then, we let \(C = \bigcap_{k \in \N} C_{k} \). We then define the \(n\)th partition of \(\left[ 0, 1 \right] \setminus C_{k} \) to be \(J_{k, n}\). We define \(\mathscr{O} = \bigcup_{k, n \in \N} J_{k, n} \) and \(\xi: \mathscr{O}\to \R, \  x \in J_{k, n} \mapsto \frac{n}{2^{k}}\). We see this is well defined by an inductive argument.
\begin{definition}[Cantor-Lebesque Function]
	We define \begin{align*}
		\phi: \left[ 0, 1 \right]  &\longrightarrow \R \\
		x &\longmapsto \phi(x) = \left \{
			\begin{array}{11}
				0, & \quad x=0 \\
				\xi\left( x \right) , & \quad x \in \mathscr{O} \\
				\sup \{ \xi\left( y \right)  : y \in \mathscr{O} \cap \left[ 0,x \right)  \}, & \quad \(x \in C \setminus \{0\} \)
			\end{array}
			\right.
	\end{align*}
	to be the \textbf{Cantor-Lebesque Function}
\end{definition}
\begin{proposition}
	\(\phi\) is a continuous increasing function such that \(\phi\left( \left[ 0, 1 \right]  \right) = \left[ 0, 1 \right] \).
\end{proposition}
\begin{proof}
	It is clear \(\xi\) is and this guarantees \(\phi\) to be increasing.\\
	Next, note \(\phi\left( 0 \right) = 0\) and \(\phi\left( 1 \right) = 1\). Hence, we have the intermediate value theorem guaranteeing the image is \(\left[ 0, 1 \right] \) if \(\phi\) is continuous.\\
	We see \(\phi\) is continuous on \(\mathscr{O}\) since it is constant on each interval \(J_{k, n}\). Now, we consider \(x \in C \setminus \{0, 1\} \). For a given \(\epsilon\), let \(k \in \N\) such that \(\frac{1}{2^{k}} < \epsilon\). Then, there is \(n \in \N\) such that \(1 \le  n \le 2^{k} -2\) such that for all \(u \in J_{k, n}\), \(v \in J_{k, n+1}\) such that for all \(u, v\) we find \(u < x < v\). Let \(a_{k} \in J_{k, n}\) \(b_{k} \in J_{k, n+1}\) then by monotinicity of \(\phi\), for all \(y \in \left[ 0, 1 \right] \) with \(\left| x-y \right|  < \delta = \min \{x-a_{k}, x+b_{k}\} \) we find
	\begin{align*}
		\left| \phi\left( x \right) - \phi\left( y \right)  \right|  &\le \phi\left( b_{k} \right) - \phi\left( a_{k} \right) \\
		&= \frac{n+1}{2^{k}} - \frac{n}{2^{k}} \\
		&= \frac{1}{2^{k}} \\
		&< \epsilon
	.\end{align*}
	Finally, given \(\epsilon > 0\), we take \(k \in \N\) such that \(\frac{1}{2^{k}} < \epsilon\) and let \(c_{k} \in I_{k, 1}\), \(d_{k} \in I_{k, 2^{k} - 1}\). Then, for \(o \le y \le c_{k}\), we find \begin{align*}
		\left| \phi\left( 0 \right)  - \phi\left( y \right)  \right| &= \left| \phi\left( y \right)  \right| \\
									     &\le \phi\left( c_{k} \right) \\
&= \frac{1}{2^{k}} \\
& < \epsilon
.\end{align*}
Similairly, for \(d_{k} < y \le 1\), we find
\begin{align*}
	\left| \phi\left( 1 \right) - \phi\left( y \right)  \right| &\le \left| 1 - \phi\left( d_{k} \right)  \right| \\
								    &= 1-\frac{2^{k} - 1}{2^{k}} \\
								    &= \frac{1}{2^{k}} \\
								    &< \epsilon
.\end{align*}
\end{proof}
\begin{definition}[Modified Cantor-Lebesque Function]
	Let \(\psi = x + \phi\left( x \right) \) be the \textbf{modified Cantor-Lebesque Function}. It is clear \(\psi\) is continuous, strictly increasing and has , \(\psi\left( \left[ 0, 2 \right]  \right)  = \left[ 0, 2 \right] \).
\end{definition}
\begin{proposition}
	The function \(\psi\)  has the following properties
	\begin{enumerate}
		\item \(\psi\left( C \right)\) is measurable with \( \mu\left( \psi \left( C \right)  \right)  = 1\).
		\item There is a measurable set \(S \subseteq C\) such that \(\psi\left( S \right) \) is not measurable.
	\end{enumerate}
\end{proposition}
\begin{proof}
	\begin{itemize}
		\item Note that \(\left[ 0, 1 \right]  = C \cup \mathscr{O}\) and \(\psi\) is injective and continuous. Hence, we have \(\left[ 0, 2 \right] = \psi\left( C \right)  \cup \psi \left( \mathscr{O} \right) \) with \(\psi\left( C \right)  \cap \psi\left( \mathscr{O} \right)  = \O\). Since \(\psi\) is strictly increasing, we know \(\psi^{-1}\) is well-defined and continuous. Hence, \(\psi\) is an open map and we see \(\psi\left( \mathscr{O} \right) \) is open in \(\left[ 0, 2 \right] \), hence \(\psi\left( C \right) \) is closed. Hence, both sets are measurable. We see \(\psi\left( \mathscr{O} \right) \) is the  union of a countable collection of open disjoint intervals, \(\{I_{i} : i \in \N\} \) such that \(\phi \mid J_{i}\) is constant by construction. Hence, we hve for each \( i \in \N\) we find \(\psi\left( I_{n} \right) = x_{i} + I_{i}\) where \(x_{i} \in \left[ 0, 1 \right] \) is a constant. Since \(\psi\) is injective, we find it preserves disjointness, hence the collection \(\{\psi \left( I_{i} \right) : i \in \N\} \) is disjoint.\\
		Then, by countable additivity and translation invariance of \(\mu\) we find
		\begin{align*}
			\mu \left( \psi\left( \mathscr{O} \right)  \right) &= \mu\left( \bigcup_{i \in \N} I_{i} \right)  \\
									   &=  \bigcup_{ i \in \N} \psi\left( I_{i} \right)  \\
									   &= \sum_{i= 1}^{\infty} \mu\left( \psi\left( I_{i} \right)  \right)  \\
									   &= \sum_{i= 1}^{\infty} \ell\left( x_{i} + I_{i} \right)  \\
									   &= \sum_{i=1}^{\infty} \ell\left( I_{i} \right)  \\
									   &= \mu\left( \mathscr{O} \right)
		.\end{align*}
		Since, \( \mu\left( C \right) = 0\), we find \[
			\mu \left( \mathscr{O} \right)  = \mu\left( \left[ 0, 1 \right] \setminus C \right)  = \mu\left( \left[ 0, 1 \right]  \right)  = 1
		.\]
	\end{itemize}
	Consequently, \( \mu\left( \psi\left( \mathscr{O} \right)  \right)  = 1 = \mu\left( \mathscr{O} \right) \). Hence, we find \( \mu\left( \psi\left( C \right)  \right)  = 1\).
\item Since \(\psi\left( C \right) \) has positive measure, it contains a nonmeasurable subset \(T\), however, we see \(S = \psi^{-1}\left( T \right) \) is measurable as \(S \subseteq C\) and \(\mu\left( C \right)  = 0\).
\end{proof}
\begin{corollary}
	There is a measurable set \(S \subseteq C\) such that \(S\) is not borel.
\end{corollary}
\begin{proof}
	Since \(\psi\) has a continuous inverse, we see it maps borel sets to borel sets. Let \(S\) be a subset of \(C\) such that \(\psi\left( S \right) \) is not measurable. Since \(\psi\left( S \right) \) is not measurable, it is not a borel set. Hence \(S\) is not borel, but it was measurable with measure \(0\).
\end{proof}
