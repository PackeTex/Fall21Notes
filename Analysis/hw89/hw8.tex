\documentclass[a4paper]{article}
% Some basic packages
\usepackage[utf8]{inputenc}
\usepackage[T1]{fontenc}
\usepackage{textcomp}
\usepackage{url}
\usepackage{graphicx}
\usepackage{float}
\usepackage{booktabs}
\usepackage{enumitem}

\pdfminorversion=7

% Don't indent paragraphs, leave some space between them
\usepackage{parskip}

% Hide page number when page is empty
\usepackage{emptypage}
\usepackage{subcaption}
\usepackage{multicol}
\usepackage{xcolor}

% Other font I sometimes use.
% \usepackage{cmbright}

% Math stuff
\usepackage{amsmath, amsfonts, mathtools, amsthm, amssymb}
% Fancy script capitals
\usepackage{mathrsfs}
\usepackage{cancel}
% Bold math
\usepackage{bm}
% Some shortcuts
\newcommand\N{\ensuremath{\mathbb{N}}}
\newcommand\R{\ensuremath{\mathbb{R}}}
\newcommand\Z{\ensuremath{\mathbb{Z}}}
\renewcommand\O{\ensuremath{\varnothing}}
\newcommand\Q{\ensuremath{\mathbb{Q}}}
\newcommand\C{\ensuremath{\mathbb{C}}}
% Easily typeset systems of equations (French package)

% Put x \to \infty below \lim
\let\svlim\lim\def\lim{\svlim\limits}

%Make implies and impliedby shorter
\let\implies\Rightarrow
\let\impliedby\Leftarrow
\let\iff\Leftrightarrow
\let\epsilon\varepsilon
\let\nothing\varnothing

% Add \contra symbol to denote contradiction
\usepackage{stmaryrd} % for \lightning
\newcommand\contra{\scalebox{1.5}{$\lightning$}}

 \let\phi\varphi

% Command for short corrections
% Usage: 1+1=\correct{3}{2}

\definecolor{correct}{HTML}{009900}
\newcommand\correct[2]{\ensuremath{\:}{\color{red}{#1}}\ensuremath{\to }{\color{correct}{#2}}\ensuremath{\:}}
\newcommand\green[1]{{\color{correct}{#1}}}

% horizontal rule
\newcommand\hr{
    \noindent\rule[0.5ex]{\linewidth}{0.5pt}
}

% hide parts
\newcommand\hide[1]{}

% Environments
\makeatother
% For box around Definition, Theorem, \ldots
\usepackage{mdframed}
\mdfsetup{skipabove=1em,skipbelow=0em}
\theoremstyle{definition}
\newmdtheoremenv[nobreak=true]{definition}{Definition}
\newmdtheoremenv[nobreak=true]{eg}{Example}
\newmdtheoremenv[nobreak=true]{corollary}{Corollary}
\newmdtheoremenv[nobreak=true]{lemma}{Lemma}[section]
\newmdtheoremenv[nobreak=true]{proposition}{Proposition}
\newmdtheoremenv[nobreak=true]{theorem}{Theorem}[section]
\newmdtheoremenv[nobreak=true]{law}{Law}
\newmdtheoremenv[nobreak=true]{postulate}{Postulate}
\newmdtheoremenv{conclusion}{Conclusion}
\newmdtheoremenv{bonus}{Bonus}
\newmdtheoremenv{presumption}{Presumption}
\newtheorem*{recall}{Recall}
\newtheorem*{previouslyseen}{As Previously Seen}
\newtheorem*{interlude}{Interlude}
\newtheorem*{notation}{Notation}
\newtheorem*{observation}{Observation}
\newtheorem*{exercise}{Exercise}
\newtheorem*{comment}{Comment}
\newtheorem*{practice}{Practice}
\newtheorem*{remark}{Remark}
\newtheorem*{problem}{Problem}
\newtheorem*{solution}{Solution}
\newtheorem*{terminology}{Terminology}
\newtheorem*{application}{Application}
\newtheorem*{instance}{Instance}
\newtheorem*{question}{Question}
\newtheorem*{intuition}{Intuition}
\newtheorem*{property}{Property}
\newtheorem*{example}{Example}
\numberwithin{equation}{section}
\numberwithin{definition}{section}
\numberwithin{proposition}{section}

% End example and intermezzo environments with a small diamond (just like proof
% environments end with a small square)
\usepackage{etoolbox}
\AtEndEnvironment{example}{\null\hfill$\diamond$}%
\AtEndEnvironment{interlude}{\null\hfill$\diamond$}%

\AtEndEnvironment{solution}{\null\hfill$\blacksquare$}%
% Fix some spacing
% http://tex.stackexchange.com/questions/22119/how-can-i-change-the-spacing-before-theorems-with-amsthm
\makeatletter
\def\thm@space@setup{%
  \thm@preskip=\parskip \thm@postskip=0pt
}


% \lecture starts a new lecture (les in dutch)
%
% Usage:
% \lecture{1}{di 12 feb 2019 16:00}{Inleiding}
%
% This adds a section heading with the number / title of the lecture and a
% margin paragraph with the date.

% I use \dateparts here to hide the year (2019). This way, I can easily parse
% the date of each lecture unambiguously while still having a human-friendly
% short format printed to the pdf.

\usepackage{xifthen}
\def\testdateparts#1{\dateparts#1\relax}
\def\dateparts#1 #2 #3 #4 #5\relax{
    \marginpar{\small\textsf{\mbox{#1 #2 #3 #5}}}
}

\def\@lecture{}%
\newcommand{\lecture}[3]{
    \ifthenelse{\isempty{#3}}{%
        \def\@lecture{Lecture #1}%
    }{%
        \def\@lecture{Lecture #1: #3}%
    }%
    \subsection*{\@lecture}
    \marginpar{\small\textsf{\mbox{#2}}}
}



% These are the fancy headers
\usepackage{fancyhdr}
\pagestyle{fancy}

% LE: left even
% RO: right odd
% CE, CO: center even, center odd
% My name for when I print my lecture notes to use for an open book exam.
% \fancyhead[LE,RO]{Gilles Castel}

\fancyhead[RO,LE]{\@lecture} % Right odd,  Left even
\fancyhead[RE,LO]{}          % Right even, Left odd

\fancyfoot[RO,LE]{\thepage}  % Right odd,  Left even
\fancyfoot[RE,LO]{}          % Right even, Left odd
\fancyfoot[C]{\leftmark}     % Center

\makeatother




% Todonotes and inline notes in fancy boxes
\usepackage{todonotes}
\usepackage{tcolorbox}

% Make boxes breakable
\tcbuselibrary{breakable}

% Verbetering is correction in Dutch
% Usage:
% \begin{verbetering}
%     Lorem ipsum dolor sit amet, consetetur sadipscing elitr, sed diam nonumy eirmod
%     tempor invidunt ut labore et dolore magna aliquyam erat, sed diam voluptua. At
%     vero eos et accusam et justo duo dolores et ea rebum. Stet clita kasd gubergren,
%     no sea takimata sanctus est Lorem ipsum dolor sit amet.
% \end{verbetering}
\newenvironment{correction}{\begin{tcolorbox}[
    arc=0mm,
    colback=white,
    colframe=green!60!black,
    title=Opmerking,
    fonttitle=\sffamily,
    breakable
]}{\end{tcolorbox}}

% Noot is note in Dutch. Same as 'verbetering' but color of box is different
\newenvironment{note}[1]{\begin{tcolorbox}[
    arc=0mm,
    colback=white,
    colframe=white!60!black,
    title=#1,
    fonttitle=\sffamily,
    breakable
]}{\end{tcolorbox}}


% Figure support as explained in my blog post.
\usepackage{import}
\usepackage{xifthen}
\usepackage{pdfpages}
\usepackage{transparent}
\newcommand{\incfig}[2][1]{%
    \def\svgwidth{#1\columnwidth}
    \import{./figures/}{#2.pdf_tex}
}

% Fix some stuff
% %http://tex.stackexchange.com/questions/76273/multiple-pdfs-with-page-group-included-in-a-single-page-warning
\pdfsuppresswarningpagegroup=1
\binoppenalty=9999
\relpenalty=9999

% My name
\author{Thomas Fleming}

\usepackage{pdfpages}
\title{Analysis I: Homework 8 and 9 Corrections}
\date{Sun 05 Dec 2021 17:17}
\DeclareMathOperator{\SRG}{SRG}
\DeclareMathOperator{\cut}{Cut}
\DeclareMathOperator{\GF}{GF}
\DeclareMathOperator{\V}{V}
\DeclareMathOperator{\E}{E}
\DeclareMathOperator{\edg}{e}
\DeclareMathOperator{\vtx}{v}
\DeclareMathOperator{\diam}{diam}

\DeclareMathOperator{\tr}{tr}
\DeclareMathOperator{\A}{A}

\DeclareMathOperator{\Adj}{Adj}
\DeclareMathOperator{\mcd}{mcd}

\begin{document}
\maketitle
\begin{problem}[39]
	\begin{itemize}
		\item Assume \(\left( f_{n} \right) \) does not converge to \(f\) in measure. That is, there is an \(\epsilon > 0\) so that for all \(N \in \N\) \[
		m\left( \{x \in \R : \left| f_{n_{N}}\left( x \right) - f\left( x \right)  \right| > \epsilon\}  \right) > \epsilon
		\] for some \(n_{N} \ge N\). Denote this set \(A_{N}\).  Then, we see \[
		\int \left| f_{n_{N}} - f \right| \ge \int_{A_{N}} \left| f_{n_{N}} - f \right|   \ge \int \epsilon \chi_{A_{N}} = \epsilon m\left( A_{N} \right) \ge \epsilon^2
		.\]
		That is, for some \(\epsilon ^{\prime} = \epsilon ^2 > 0\),  and all \(N \in \N\) we find an \(n_N \ge N\), so that \(\int \left| f_{n} - f \right| \ge \epsilon^{\prime}\), so \(f_{n}\) does not converge to \(f\) in mean.
	\item First, note that if \(x = 0\) or \(1\), then \(f_{n}\left( x \right)  = x\) for all \(n \in \N\). Then, if \(x \in \left( 0, 1 \right) \), the ratio test proves \(\sum_{i= 1}^{\infty} nx^{n} < \infty\), hence \(\lim_{n \to \infty} f_{n}\left( x \right)  = \lim_{n \to \infty}nx^{n} = 0\).  \\
		To see that \(f_{n}\) converges to \(0\) in measure denote \(E_{\epsilon;n} = \{x \in \left[ 0, 1 \right] : nx^{n} < \epsilon\} \). Then, suppose \(c \in E_{\epsilon;n}\), then either \(c = 1\) or \(\lim_{n \to \infty} f_{n}\left( c \right) = 0\). We can exclude the first case as this happens only on a set of measure \(0\). Hence, fixing \(\epsilon > 0\) and assuming \(c\in \left[ 0, 1 - \frac{\epsilon}{2} \right) \) we see there is a \(N \in \N\) so that \(f_{n}\left( c \right)  < \epsilon\) for all \(n \ge N\). So, we have \(m\left( E_{\epsilon ; n} \right)  \le m \left(\left[ 1-\frac{\epsilon}{2}, 1 \right]\right)< \epsilon\) for all \(n \ge N\) , so \(f_{n}\) converges to \(0\) in measure.\\
	\item Finally, to show that \(f\) does not converge in measure take \(\epsilon = \frac{1}{100}\). Then, we define \(a_{n} = 1 - \left( \frac{1}{100} \right) ^{\frac{1}{n+1}}\) we define \(s_{n} = f_{n}\left( a_{n} \right)  \chi_{\left[ a_{n}, 1 \right] }\) . Then, we find \(f_{n}\) dominates \(s_{n}\) for every \(n\), hence \[\int f_{n} \ge \int s_{n} = n\left( \frac{1}{100}^{n+1} - \frac{1}{100} \right) \ge n\left( 100^{-2} - \frac{1}{100} \right) = n\delta  \] for all \(n \ge 1\) . Since this grows linearly with \(n\), we find for sufficiently large \(n\), \(n\delta > \epsilon\). Hence, it is shown.
\end{problem}
\newpage
\begin{problem}[42]
\begin{enumerate}
	\item First, we prove the case \(s < \infty\). Let \(f \in L^{s}\left( S \right) \). Then, we define \(r\) so that \(\frac{1}{s} + \frac{1}{r} = \frac{1}{p}\) (hence \(\frac{s}{p}\) and \(\frac{r}{p}\) are conjugate). Then, as we aim to show \(\|f\|_{p}\) finite, we see it suffices to show \(\|f\|_{p}^{p} = \int_{S} \left| f \right| ^{p} = \| f^{p}\|_{1}\) finite. We see
		\begin{align*}
			\|f\|_{p}^{p} &= \| 1f\|_{p}^{p}\\
&= \| 1^{p} f^{p}\|_{1} \\
&\le \|1\|_{\frac{r}{p}} \|f^{p}\|_{\frac{s}{p}}\\
&= (\int_{S} 1^{\frac{r}{p}} )^{\frac{p}{r}} \left( \int_{S} \left| f^{p} \right| ^{\frac{s}{p}} \right) ^{\frac{p}{s}}  \\
&= \|1\|^{\frac{1}{r}}_{p} \|f\|_{s}^{p}\\
&= m\left( S \right) ^{\frac{1}{r}} \|f\|_{s}^{p} \\
&< \infty
		.\end{align*}
		We find this finite by assumption, hence \(f \in L^{p}\left( S \right) \), so the claim is shown. \\
		Next, we show the case \(s = \infty\). In this case \(f\in L^{p}\left( S \right) \) is bounded almost everywhere (else its \(\ess\) would be infinite). Then, we see for \(p < \infty\) \(\int_{S} \left| f \right| ^{p} \le \int_{S}\ess\left( f \right)^{p} = S \|f\|_{\infty}^{p} < \infty\) by assumption so the claim holds.\\
		It is clear that if \(m\left( S \right) = \infty\) this does not hold. For an example, sake \(S = \left[ 0, \infty \right] \) and \(f = \frac{1}{x}\), we see \(\|f\|_{1} = \int_{\left[ 0, \infty \right] } \frac{1}{x} = \infty\), however \(\|f\|_{2} = (\int_{\left[ 0, \infty \right] } \frac{1}{x^2})^{\frac{1}{2}}\). As \(\frac{1}{x^2}\) is integrable on \(\left[ 0, \infty \right] \) we find its root to be finite, hence \(f \in L_2\left( \left[ 0, \infty \right]  \right) \) but \(f \not\in L_1\left( \left[ 0, \infty \right]  \right) \).
	\item Let \(f \in L^{r}\left( S \right) \cap L^{s}\left( S \right) \). Denote the following sets, \(A = \{x  : x \in S, \left| f\left( x \right)  \right| < 1\} \) and \(B = \{x : x \in S, \left| f\left( x \right)  \right| > 1\} \). It is clear \(A \cup B = S\), with \(A, B\) being disjoint. Then, we see if \(s \neq \infty\), we have
		\begin{align*}
			\|f\|_{p}^{p} &= \int_{S} \left| f \right| ^{p}\\
			&= \int_{A} \left| f \right| ^{p} + \int_{B} \left| f \right| ^{p} \\
			&\le \int_{A} \left| f \right| ^{r} + \int_{B} \left| f \right| ^{s} \\
			&\le \int_{S} \left| f \right| ^{r} + \int_{S} \left| f \right| ^{s}\\
			&= \|f\|_{r}^{r} + \|f\|_{s}^{s} \\
			&< \infty
		.\end{align*}
		In the other case where \(s = \infty\) we apply the same logic as in \(41\), that being \(\left| f \right| \le \ess \left( f \right) \) on all but a set of measure \(0\), hence they may be interchanged in the integral:
		\begin{align*}
			\|f\|_{r}^{r} &= \int_{S} \left| f \right| ^{r} \\
				      &= \int_{S} \left| f \right| ^{p} \left| f \right| ^{r-p} \\
				      &\le \int_{S} \left| f \right| ^{p} \underbrace{\left[\ess\left( f \right) \right]^{r-p}}_{\text{constant}}
				\\ &= \|f\|_{\infty}^{r-p} \|f\|_{p}^{p} < \infty \text{ by assumption}
		.\end{align*}
\end{enumerate}
\end{problem}
\newpage
\begin{problem}[43]
	\begin{itemize}
		\item First, note that \(\int_{I} \cos\left( nx \right)  = \int_{I} \cos^{+}\left( nx  \right) + \int_{I} \cos^{-}\left( nx  \right)  \). Since \(I\) is a bounded interval, we see for all but a set of measure \(0\) on its boundary, if \(x \in I\), then there is an \(\epsilon > 0\) so that \(\left( x- \epsilon, x + \epsilon \right) \in I \). Then, \(\cos^{-}\left( nx \right)  = \cos^{+}\left( n\left( x + \frac{\pi}{2n} \right)  \right) \), so for almost every \(x\), we find there is an \(N \in \N\) so that \(x + \frac{\pi}{2n} \in I\) for all \(n \ge N\). Moreover it is bounded by \(g=1\) everywhere, so DCT proves it integrable. Then,
			\begin{align*}
				\lim_{n \to \infty} \int_{I} \cos \left( nx \right)  &=  \lim_{n \to \infty} \int _{I} \cos ^{+}\left( nx \right)  - \lim_{n \to \infty} \int_{I} \cos ^{+}\left( n\left( x + \frac{\pi}{2n} \right)  \right)  \\
				&= \lim_{n \to \infty} \int_{I} \cos ^{+}\left( nx \right)  - \int_{I + \frac{\pi}{2n}} \cos^{+}\left( nx  \right)  \\
				&=  \lim_{n \to \infty}- \int _{\left( I+\frac{\pi}{2n} \right) \setminus I} \cos^{+}\left( nx \right)  \\
				&\ge -  \int _{\left( I + \frac{\pi}{2n} \right) \setminus I} 1\\
				&= - \lim_{n \to \infty}\frac{\pi}{2n} \\
				&= 0
			.\end{align*}
			The same argument shows \(\lim_{n \to \infty} \int_{ I } \cos \left( nx \right)  \le 0\) taking \(\cos ^{-}\) instead. Hence, \(\lim_{n \to \infty} \int_{I}\cos\left( nx \right)  = 0\).
		\item 	First, note that since \(\cos\left( nx \right) \in \left[ -1, 1 \right] \) we have \(\int \left| f \cos\left( nx \right)\right| \le \int \left| f \right| < \infty \), hence it is integrable. Then, we see
			\begin{align*}
				\int f \cos\left( nx \right) &= \int_{\R} f \cos ^{+}\left( nx \right) - \int_{\R} f \cos^{-}\left( nx \right) \\
				&= \int_{\R}f \cos^{+}\left( nx \right) - \int_{\R + \frac{\pi}{2n}} f \cos^{+}\left( nx \right)   \\
				&= \int_{\R} f \cos^{+}\left( nx \right) - \int_{\R} f \cos^{+}\left( nx \right)  \\
				&= 0
			.\end{align*}
	\end{itemize}
\end{problem}
\end{document}
