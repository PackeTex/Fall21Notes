\documentclass[a4paper]{article}
% Some basic packages
\usepackage[utf8]{inputenc}
\usepackage[T1]{fontenc}
\usepackage{textcomp}
\usepackage{url}
\usepackage{graphicx}
\usepackage{float}
\usepackage{booktabs}
\usepackage{enumitem}

\pdfminorversion=7

% Don't indent paragraphs, leave some space between them
\usepackage{parskip}

% Hide page number when page is empty
\usepackage{emptypage}
\usepackage{subcaption}
\usepackage{multicol}
\usepackage{xcolor}

% Other font I sometimes use.
% \usepackage{cmbright}

% Math stuff
\usepackage{amsmath, amsfonts, mathtools, amsthm, amssymb}
% Fancy script capitals
\usepackage{mathrsfs}
\usepackage{cancel}
% Bold math
\usepackage{bm}
% Some shortcuts
\newcommand\N{\ensuremath{\mathbb{N}}}
\newcommand\R{\ensuremath{\mathbb{R}}}
\newcommand\Z{\ensuremath{\mathbb{Z}}}
\renewcommand\O{\ensuremath{\varnothing}}
\newcommand\Q{\ensuremath{\mathbb{Q}}}
\newcommand\C{\ensuremath{\mathbb{C}}}
% Easily typeset systems of equations (French package)

% Put x \to \infty below \lim
\let\svlim\lim\def\lim{\svlim\limits}

%Make implies and impliedby shorter
\let\implies\Rightarrow
\let\impliedby\Leftarrow
\let\iff\Leftrightarrow
\let\epsilon\varepsilon
\let\nothing\varnothing

% Add \contra symbol to denote contradiction
\usepackage{stmaryrd} % for \lightning
\newcommand\contra{\scalebox{1.5}{$\lightning$}}

 \let\phi\varphi

% Command for short corrections
% Usage: 1+1=\correct{3}{2}

\definecolor{correct}{HTML}{009900}
\newcommand\correct[2]{\ensuremath{\:}{\color{red}{#1}}\ensuremath{\to }{\color{correct}{#2}}\ensuremath{\:}}
\newcommand\green[1]{{\color{correct}{#1}}}

% horizontal rule
\newcommand\hr{
    \noindent\rule[0.5ex]{\linewidth}{0.5pt}
}

% hide parts
\newcommand\hide[1]{}

% Environments
\makeatother
% For box around Definition, Theorem, \ldots
\usepackage{mdframed}
\mdfsetup{skipabove=1em,skipbelow=0em}
\theoremstyle{definition}
\newmdtheoremenv[nobreak=true]{definition}{Definition}
\newmdtheoremenv[nobreak=true]{eg}{Example}
\newmdtheoremenv[nobreak=true]{corollary}{Corollary}
\newmdtheoremenv[nobreak=true]{lemma}{Lemma}[section]
\newmdtheoremenv[nobreak=true]{proposition}{Proposition}
\newmdtheoremenv[nobreak=true]{theorem}{Theorem}[section]
\newmdtheoremenv[nobreak=true]{law}{Law}
\newmdtheoremenv[nobreak=true]{postulate}{Postulate}
\newmdtheoremenv{conclusion}{Conclusion}
\newmdtheoremenv{bonus}{Bonus}
\newmdtheoremenv{presumption}{Presumption}
\newtheorem*{recall}{Recall}
\newtheorem*{previouslyseen}{As Previously Seen}
\newtheorem*{interlude}{Interlude}
\newtheorem*{notation}{Notation}
\newtheorem*{observation}{Observation}
\newtheorem*{exercise}{Exercise}
\newtheorem*{comment}{Comment}
\newtheorem*{practice}{Practice}
\newtheorem*{remark}{Remark}
\newtheorem*{problem}{Problem}
\newtheorem*{solution}{Solution}
\newtheorem*{terminology}{Terminology}
\newtheorem*{application}{Application}
\newtheorem*{instance}{Instance}
\newtheorem*{question}{Question}
\newtheorem*{intuition}{Intuition}
\newtheorem*{property}{Property}
\newtheorem*{example}{Example}
\numberwithin{equation}{section}
\numberwithin{definition}{section}
\numberwithin{proposition}{section}

% End example and intermezzo environments with a small diamond (just like proof
% environments end with a small square)
\usepackage{etoolbox}
\AtEndEnvironment{example}{\null\hfill$\diamond$}%
\AtEndEnvironment{interlude}{\null\hfill$\diamond$}%

\AtEndEnvironment{solution}{\null\hfill$\blacksquare$}%
% Fix some spacing
% http://tex.stackexchange.com/questions/22119/how-can-i-change-the-spacing-before-theorems-with-amsthm
\makeatletter
\def\thm@space@setup{%
  \thm@preskip=\parskip \thm@postskip=0pt
}


% \lecture starts a new lecture (les in dutch)
%
% Usage:
% \lecture{1}{di 12 feb 2019 16:00}{Inleiding}
%
% This adds a section heading with the number / title of the lecture and a
% margin paragraph with the date.

% I use \dateparts here to hide the year (2019). This way, I can easily parse
% the date of each lecture unambiguously while still having a human-friendly
% short format printed to the pdf.

\usepackage{xifthen}
\def\testdateparts#1{\dateparts#1\relax}
\def\dateparts#1 #2 #3 #4 #5\relax{
    \marginpar{\small\textsf{\mbox{#1 #2 #3 #5}}}
}

\def\@lecture{}%
\newcommand{\lecture}[3]{
    \ifthenelse{\isempty{#3}}{%
        \def\@lecture{Lecture #1}%
    }{%
        \def\@lecture{Lecture #1: #3}%
    }%
    \subsection*{\@lecture}
    \marginpar{\small\textsf{\mbox{#2}}}
}



% These are the fancy headers
\usepackage{fancyhdr}
\pagestyle{fancy}

% LE: left even
% RO: right odd
% CE, CO: center even, center odd
% My name for when I print my lecture notes to use for an open book exam.
% \fancyhead[LE,RO]{Gilles Castel}

\fancyhead[RO,LE]{\@lecture} % Right odd,  Left even
\fancyhead[RE,LO]{}          % Right even, Left odd

\fancyfoot[RO,LE]{\thepage}  % Right odd,  Left even
\fancyfoot[RE,LO]{}          % Right even, Left odd
\fancyfoot[C]{\leftmark}     % Center

\makeatother




% Todonotes and inline notes in fancy boxes
\usepackage{todonotes}
\usepackage{tcolorbox}

% Make boxes breakable
\tcbuselibrary{breakable}

% Verbetering is correction in Dutch
% Usage:
% \begin{verbetering}
%     Lorem ipsum dolor sit amet, consetetur sadipscing elitr, sed diam nonumy eirmod
%     tempor invidunt ut labore et dolore magna aliquyam erat, sed diam voluptua. At
%     vero eos et accusam et justo duo dolores et ea rebum. Stet clita kasd gubergren,
%     no sea takimata sanctus est Lorem ipsum dolor sit amet.
% \end{verbetering}
\newenvironment{correction}{\begin{tcolorbox}[
    arc=0mm,
    colback=white,
    colframe=green!60!black,
    title=Opmerking,
    fonttitle=\sffamily,
    breakable
]}{\end{tcolorbox}}

% Noot is note in Dutch. Same as 'verbetering' but color of box is different
\newenvironment{note}[1]{\begin{tcolorbox}[
    arc=0mm,
    colback=white,
    colframe=white!60!black,
    title=#1,
    fonttitle=\sffamily,
    breakable
]}{\end{tcolorbox}}


% Figure support as explained in my blog post.
\usepackage{import}
\usepackage{xifthen}
\usepackage{pdfpages}
\usepackage{transparent}
\newcommand{\incfig}[2][1]{%
    \def\svgwidth{#1\columnwidth}
    \import{./figures/}{#2.pdf_tex}
}

% Fix some stuff
% %http://tex.stackexchange.com/questions/76273/multiple-pdfs-with-page-group-included-in-a-single-page-warning
\pdfsuppresswarningpagegroup=1
\binoppenalty=9999
\relpenalty=9999

% My name
\author{Thomas Fleming}

\usepackage{pdfpages}
\title{Analysis I: Homework 7}
\date{Fri 10 Sep 2021 12:58}
\DeclareMathOperator{\SRG}{SRG}
\DeclareMathOperator{\cut}{Cut}
\DeclareMathOperator{\GF}{GF}
\DeclareMathOperator{\V}{V}
\DeclareMathOperator{\E}{E}
\DeclareMathOperator{\edg}{e}
\DeclareMathOperator{\vtx}{v}
\DeclareMathOperator{\diam}{diam}

\DeclareMathOperator{\tr}{tr}
\DeclareMathOperator{\A}{A}

\DeclareMathOperator{\Adj}{Adj}
\DeclareMathOperator{\mcd}{mcd}

\begin{document}
\maketitle
\begin{problem}[31 (Collaborated with Andrea)]
Let \(f = \frac{\sin\left( x \right) }{x}\). We aim to show \(\int f^{+} = \infty\). First, note that \(\int f^{+}  \ge \int_{\left[ 0, \infty \right] } f^{+}\) so it suffices to show this quantity infinite. Moreover, in this interval we find \(f\) positive for \(x \in \left[ 2n\pi, (2n+1)\pi \right] \) for \(n \in \Z^{+}_0\). Hence, defining \(f_{n} = f^{+} \chi_{\left[ 0, \left( 2n+1 \right) \pi \right] }\) we see each \(f_{n}\) is measurable (it is continuous) and non-negative with \(\lim_{n \to \infty}f_{n} \left( x \right) = f^{+}\left( x \right) \) for all \(x \ge 0  \). Moreover, since \(\left[ 0, \left( 2n+1 \right) \pi \right] \subseteq \left[ 0, \left( 2\left( n+1 \right) +1 \right) \pi \right] \) we see \(f_{n} \le f_{n+1}\) for all \(x \in \left[ 0, \infty \right) \).  Hence, applying dominated convergence yields
\begin{align*}
	\int f^{+} &\ge \int _{\left[ 0, \infty \right)} f^{+}\\
	&= \int_{\left[ 0, \infty \right] } \lim_{n \to \infty} f_{n} \\
	&= \lim_{n \to \infty} \int_{\left[ 0, \infty \right] } f_{n} \\
	&= \lim_{n \to \infty} \sum_{i=0}^{n} \int_{\left[ 2i\pi, \left( 2i + 1 \right) \pi \right] } f_{n} \\
	&\ge \lim_{n \to \infty} \sum_{i=0}^{n} \int_{\left[ 2i\pi, \left( 2i+1 \right) \pi \right] } \frac{(\sin ^{+}\left( x \right) \mid_{\left[ 0, 2n\pi \right] })^{*} }{\left( 2i + 1 \right) \pi}\\
	&= \lim_{n \to \infty} \sum_{i=0}^{n} \frac{1}{\left( 2i + 1 \right) \pi } \int_{\left[ 2i\pi, \left( 2i+1 \right) \pi \right] } (\sin^{+}\left( x \right) \mid_{\left[ 0, 2n\pi \right] })^{*}\\
	&\ge \lim_{n \to \infty} \sum_{i=0}^{n} \frac{1}{\left( 2i+1 \right) \pi} \text{ since } \int_{\left[ 2i\pi, \left( 2i+1 \right) \pi \right] } \sin ^{+}\left( x \right) = \int_{\left[ 0, \pi \right] } \sin\left( x \right) = 2 .\\
	&= \frac{1}{\pi} \lim_{n \to \infty} \sum_{i=0}^{n} \frac{1}{\left( 2i+1 \right) }\\
	&= \frac{1}{\pi} \sum_{i= 0}^{\infty} \frac{1}{2i+1} \\
	&=  \infty
.\end{align*}
Hence, \(\int f^{+}\) is not finite, so \(f\) is nonintegrable.
\end{problem}
\newpage
\begin{problem}[32]
	First, note that \(f \coloneqq \frac{1}{\sqrt{x} }\) is measurable (preimage of an interval is an interval) and finite almost everywhere. Then, we define \(A_{n} = \left[ \frac{1}{\left( n+1 \right) ^2}, \frac{1}{n^2} \right] \) and the simple functions \(s_{n} = \sum_{i= 1}^{n} i \chi_{A_{i}}\). As each term is positive, we see \(s_{n}\) is increasing for fixed \(x\). Moreover, \(s = \lim_{n \to \infty}s_{n}\) is integrable by applying DCT
	\begin{align*}
		\int s &= \lim_{n \to \infty} \int s_{n} \\
		&= \lim_{n \to \infty} \sum_{i= 1}^{n} \left(  \frac{1}{i^2} - \frac{1}{\left( i+1 \right) ^2}  \right) i\\
		&= \lim_{n \to \infty} \sum_{i= 1}^{n} \frac{1}{\left( i+1 \right) ^2} + \frac{1}{i\left( i+1 \right) } \\
		&\le \lim_{n \to \infty}\sum_{i= 1}^{n} \frac{2}{i^2}\\
		&= \frac{\pi^2}{3}
	.\end{align*}
	Then, for any \(t \in \mathscr{S}\left( f \right) \) , we see \(t \le s_{n} \le s\) for sufficiently large \(n\) . Hence, since \(s\) is an upper bound of \(\mathscr{S}\left( f \right) \), we find \(\infty > \int s  > \int f\), so \(f\) is measurable.
\end{problem}
\newpage
\begin{problem}[33]
	First, basic limits show \(\lim_{n \to \infty}h_{n}\left( x \right)  = \left \{
		\begin{array}{11}
			3, & \quad x \in \left( -1, 1 \right)  \\
			2, & \quad x = -1 \text{ or } x=1\\
			1, & \quad x \in \left( -\infty, -1 \right)\cup \left( 1, \infty \right)
		\end{array}
		\right.\)
		Moreover, \(h_{n}\left( x \right) \) is continuous for every \(n \in \N\), hence measurable. So, we see \\\(h_{n} \cdot f\) is measurable for every \(n \in \N\). Then, \(\lim_{n \to \infty}\left( h_{n} \cdot f \right) \left( x \right) = \left \{
			\begin{array}{11}
				3f\left( x \right) , & \quad x \in \left( -1, 1 \right)  \\
				2f\left( x \right) , & \quad x= \pm 1\\
				f\left( x \right) , &\quad x\in\left( -\infty, -1 \right)\cup \left( 1, \infty \right)
			\end{array}
		\right\). Hence, we see \(\left| h_{n} \cdot f \right|\le 3\left| f \right|  \) with \(3 \left| f \right| \) being integrable (since \(f\) is integrable). Applying dominated convergence yields \[
			\lim_{n \to \infty}\int h_{n} \cdot f = \int \lim_{n \to \infty} h_{n} \cdot f = \int_{\left[ -\infty, -1 \right] } f + \int_{\left[ -1, 1 \right] } 3f + \int_{\left[ 1, \infty \right] } f = \int f \DX  + 2\int_{ \left[ -1, 1 \right] }f \DX
			.\]
\end{problem}
\newpage
\begin{problem}[34]
	First, basic limits again show \(\lim_{n \to \infty}e^{-\frac{x}{n}} = 1\). Moreover, fixing \(x\), we see \(e^{-\frac{x}{n}} < e^{-\frac{x}{n+1}}\), so we see \(e^{-\frac{x}{n}}\left| f \right| \le e^{-\frac{x}{n+1}}\left| f \right| \). Then, denoting \(e^{-\frac{x}{n}}\left| f \right| = f_{n}\), we see \(\lim_{n \to \infty}f_{n} = \lim_{n \to \infty}e^{-\frac{x}{n}} \lim_{n \to \infty}\left| f \right| = \lim_{n \to \infty}\left| f \right| \) with each \(f_{n}\) being measurable (as it is the product of continous functions) and increasing, hence passing to the \(0\)-extension and applying monotone convergence yields \[
	1 \ge \lim_{n \to \infty}\int_{\left( 0, \infty \right) } f_{n} = \lim_{n \to \infty} \int f_{n}^{*} =  \int_{ } \lim_{n \to \infty}f_{n}^{*} = \int_{} (\left| f \right|)^{*} = \int_{\left( 0, \infty \right) } \left| f \right|
	.\]
	Since \(f\) is continuous, we see it is measurable, and since it is absolutely integrable on \(\left( 0, \infty \right) \), we have \(f\) being integrable on \(\left( 0, \infty \right) \).
\end{problem}
\newpage
\begin{problem}[35]
	First, recall \(\sum_{i= 1}^{\infty} \frac{1}{n^{4} } = \frac{\pi^{4}}{90}\).\\
	Then, define \(g_{n} = \sum_{i= 1}^{n} f_{i}^2\) and note that \(g_{n} \le g_{n+1}\). Moreover \(g_{n}\) is the sum of measurable functions, so it is measurable. Lastly, define \(\lim_{n \to \infty}g_{n}\left( x \right) = g\left( x \right)  = \sum_{i= 1}^{\infty} f_{n}^2\left( x \right) \) Then, monotone convergence yields
	\begin{align*}
		\int_{\left[ 0, 1 \right] } g &= \lim_{n \to \infty} \int_{\left[ 0, 1 \right] }g_{n}\\
		&= \lim_{n \to \infty} \int_{\left[ 0, 1 \right] } \sum_{i= 1}^{n} f_{i}^2 \\
		&= \lim_{n \to \infty} \sum_{i= 1}^{n} \int_{\left[ 0, 1 \right] }f_{i}^2 \\
		&\le \lim_{n \to \infty} \sum_{i= 1}^{n} \frac{1}{i^{4}} \\
		&= \frac{\pi^{4}}{90} \\
	.\end{align*}
	Moreover, \(0 \le \int_{\left[ 0, 1 \right] }f_{n}^2\) as the integrand is always non-negative. Hence, as the sum is bounded and strictly increasing, we see the terms tend to \(0\). That is \( \lim_{n \to \infty} \int _{\left[ 0, 1 \right] }f_{n}^2 = 0\).
\end{problem}
\end{document}
