\lecture{1}{Tue 24 Aug 2021 12:50}{Overview and Basics}
\section{Overview and Basics}
\begin{definition}[Functions]
	A function $f:X \to Y$ consists of a  \textbf{domain}, $X$, a \textbf{codomain} $Y$, and the \textbf{range},  $f\left( X \right) = \{f\left( x \right) \in Y : x \in X\} $. The image of a set $A\subseteq X$ is $f\left( A \right) = \{f\left( x \right) \in Y: x \in A\} $. The \textbf{preimage} of a set $B\subseteq Y$ is the set $f^{-1} \left( B \right) = \{x \in X: f\left( x \right) \in B\} $. The \textbf{restriction} of $f$ to $C \subseteq X$ is $f|_{C}: C \to Y$ such that $f|_{C}\left( x \right) = f\left( x \right) $ for $x \in C$. The \textbf{composition} of $f$ with $g: Y \to Z$ is $\left( g \circ f \right)\left( x \right) = g\left( f\left( x \right)  \right) $.
\end{definition}
\begin{definition}[Properties of Functions]
	A function $f:X \to Y$ is called \textbf{surjective} or \textbf{onto} if $f\left( X \right) = Y$. $f$ is \textbf{one-to-one} or  \textbf{injective} if $f\left( x \right) = f\left( y \right) \implies x=y$ for $x, y \in X$. $f$ is \textbf{bijective} if it is one-to-one and onto.
\end{definition}
\begin{example}
	The function $id_{X}: X \to X : x \mapsto x$ is called the \textbf{identity}.\\
	If $f:X \to Y$ is injective, then there is a function $g: f\left( X \right) \to X$ such that $\left( g \circ f \right) = id_{X}$. We call $g$ the \textbf{inverse} of $X$ and denote it $f^{-1}$.\\
\end{example}
\begin{definition}[Sequences]
	Given $m \in \N$, a function $f: \{k \in \N : 1 \le k \le m\} \to X $ is called a \textbf{finite sequence} of length $m$. This is typically denoted by $f\left( k \right) = \left( x_{k} \right)_{k=1}^{m} = \left( x_{k} \right)_{1\le k\le m}= *\left( x_1, x_2, \ldots, x_{m} \right)  $. \\
	A function $f:\N \to X$ is called a \textbf{sequence} or \textbf{infinite sequence}. This is typically denoted $f\left( k \right) = \left( x_{k} \right) = \left( x_{k} \right)_{k}= \left( x_{k} \right)_{k=1}^{\infty}= \left( x_{k} \right)_{1\le k} $ based on context.\\
	A sequence need not have first index $1$ be it finite or infinite, in this case we denote it, for first index $m$ and last index $n$, $f\left( k \right) = \left( x_{k} \right)_{k=m}^{k=n}= \left( f_{k} \right) _{m\le k \le n} $. Similarly for infinite sequences of first index $m$, we denote $f\left( k \right) = \left( x_{k} \right) _{m\le k}$.
\end{definition}
\begin{definition}[Collections]
	A \textbf{family} or \textbf{collection} of subsets of $X$ is a subset of $\mathscr{P}\left( X \right) = \{A : A\subseteq X\} $.\\
	A collection $\mathscr{C}^{\prime}$ is a \textbf{subcollection} of the collection $\mathscr{C}$ if $\mathscr{C}^{\prime} \subseteq \mathscr{C}$.
\end{definition}
\begin{definition}[Indexed Collection]
	Let $f:\Lambda \to \mathscr{P}\left( X \right)$. $f$ is called an \textbf{indexed collection} of subsets of $X$ for the indexed set $\Lambda$ of subsets of $X$. We generally denote this by $ f\left( \lambda \right) = \left(A_{\lambda} : \lambda \in \Lambda\right) = \left( A_{\lambda}\right) $ based on context.
\end{definition}

\begin{definition}[Set Operations]
	We denote the \textbf{intersection} $A \cap B = \{x: x\in A \text{ and } x \in B\} $,\\
	the \textbf{union} $A \cup B= \{x: x \in A \text{ or } x \in B\} $,\\
	the \textbf{complement} $A^{C}= X \setminus A = \{x: x \not\in A \text{ and } x \in X \} $, \\
	the \textbf{symmetric difference} $A \Delta B = \left( A \setminus B \right) \cup \left( B \setminus A \right) $. \\
	We say sets $A, B$ are \textbf{disjoint} if $A \cap B = \O$.\\
	A collection $\mathscr{C}$ of subsets of $X$ is a \textbf{disjoint collection} if for all $A, B \in \mathscr{C}$, $A$ and $B$ are disjoint unless $A = B$.\\
	Let $\mathscr{C}$ be a collection of subsets of $X$, we define $\bigcap _{A \in \mathscr{C}} A = \{x \in X : x _{n} A \ \forall \ A \in \mathscr{C} \} $.\\
	Similarly, we define $ \bigcup_{A \in \mathscr{C}} A = \{x \in X | \exists \ A \in \mathscr{C} \text{ such that } x \in A\} $.\\
	Lastly, the special case $\mathscr{C} = \{ A_{k} : 1 \le k \le m\} $ or $ \{A_{k} : k \in \N \} $ allows the notation $\bigcap _{k=1}^{m}A_{k}$ or $\bigcap _{k=1}^{\infty}A_{k}$ and $\bigcup _{k=1}^{m} A_{k}$ or $\bigcup _{k=1}^{ \infty} A_{j}$
\end{definition}
\begin{definition}[De Morgen's Laws]
	Let $\mathscr{C} \subseteq \mathscr{P} \left( X \right) $. First we state the \textbf{negation laws}, $\left( \bigcup_{A \in \mathscr{C}} A  \right)^{C}= \bigcap _{A \in \mathscr{C}}A^{C} $ and $\left( \bigcap _{A \in \mathscr{C}} A \right) ^{C}= \bigcup _{A \in \mathscr{C}} A^{C}$
	Now, the \textbf{distributive laws}, let $B \subseteq X$, then $B \cap \left( \bigcup_{A \in \mathscr{C}} A \right) = \bigcup _{A \in \mathscr{C}}\left( A \cap B \right) $ and $f^{-1}\left( \bigcap_{B \in \mathscr{C}^{\prime}} B\right) = \bigcap _{B \in \mathscr{C}^{\prime}}f^{-1}\left( B \right) 	$.
\end{definition}
\begin{proposition}
	Let $\mathscr{C} \subseteq \mathscr{P}\left( X \right) $, $\mathscr{C}^{\prime} \subseteq \mathscr{P} \left( Y \right) $, $A\subseteq X$, $B\subseteq Y$, $f: X \to Y$.\\
	Then, $f\left( \bigcup _{A \in \mathscr{C}} A \right) = \bigcup _{A \in \mathscr{C}} f\left( A \right)   $. However, $f\left( \bigcap _{A \in \mathscr{C}} A \right) \neq \bigcap _{A \in \mathscr{C}} f\left( A \right)   $.\\
	Also, $f^{-1}*\left( \bigcup_{B \in \mathscr{C}^{\prime}} B \right) = \bigcup_{B \in \mathscr{C}^{\prime}} $ and $f^{-1}\left( \bigcap_{B \in \mathscr{C}^{\prime}} B \right) = \bigcap _{B \in \mathscr{C}^{\prime}} f^{-1}\left( B \right) $.\\
	Next, $f^{-1}\left( B^{C} \right)= \left[ f^{-1}\left( B \right)  \right]^{C}  $.\\
	Last, $f\left( f^{-1}\left( B \right)  \right) \subseteq B $ and $A \subseteq f^{-1}\left( f\left( A \right)  \right) $.
\end{proposition}
\begin{definition}[Countability]
	X is \textbf{finite} if $X$ is empty of if there is $m \in \N$	and a bijection function $f : X \to \{k | 1 \le k \le m\} $. If $X$ is not finite, it is called \textbf{infinite}.\\ $X$ is called \textbf{countably infinite} if there is a bijection $g: X \to \N$. $X$ is called \textbf{countable} if it is finite or countably infinite. \\$X$ is called \textbf{uncountable}  if it is not countable.
\end{definition}
\begin{example}
	$\N, \Z$, and $\Q$ are countable, but $\R$ is uncountable.\\
\end{example}
\begin{proposition}
	1. Every subset of a countable collection is countable.\\
	2. The set of all finite sequences from a countable set is countable.\\
	3. The union of a countable collection of countable sets is countable.\\
\end{proposition}
\begin{definition}[Algebras]
	A collection $\mathscr{A}$ of subsets of $X$ is called an \textbf{algebra} if all of the following hold:\\1. $X \in \mathscr{A}$,\\ 2. $A \in \mathscr{A} \implies A^{C} \in \mathscr{A}$,\\ 3. $A \cup B \in \mathscr{A}$ for all $A, B \in \mathscr{A}$.\\
\end{definition}
\begin{definition}[$\sigma$-Algebra]
	A collection $\mathscr{A}$ of subsets of $X$ is called a \textbf{$\sigma$-algebra} if all of the following hold:\\1. $X\in \mathscr{A}$,\\2. $A\in \mathscr{A} \implies A^{C}\in \mathscr{A}$,\\
3. $\bigcup _{k=1}^{\infty}A_{k} \in \mathscr{A}$ for every (countable) collection of subsets $\{A_{k} : k \in \N \} $.
\end{definition}
