\lecture{14}{Thu 07 Oct 2021 12:58}{Measurable Functions (2)}
\begin{recall}
	A function \(f: S \to \R\) was measurable if \(S\) is measurable and \(f^{-1}\left( \left( c, \infty \right]  \right) \) is measurable for all \(c \in \R\). There was an equivalent definition using the extended borel \(\sigma\)-algebra that we will use occasionally.
\end{recall}
\begin{proposition}
	Suppose \(f: S \to \overline{\R}\) is continuous on the measurable set \(S\), then \(f\) is measurable.
\end{proposition}
\begin{proof}
	Let \(H\) be an extending function, then we must show \(H \circ f\) is continuous. We see any subray , \(f\left( X_0 \right) = \left( c , \infty \right] \) will have \((H \circ f)\left( X_0 \right) = \left( \hat{c}, 1 \right] \). We know the preimage of this to be open in \(S\), hence measurable.
\end{proof}
\begin{proposition}
	Let \(S \subseteq \R\)	. Suppose \(f: S \to \R\) is measurable. and let \(g: B \to \R\) with \(B \in \overline{\mathscr{B}}\)  and \(f\left( S \right) \subseteq B\). Then, \(g \circ f: S \to \R\) is measurable.
\end{proposition}
\begin{proof}
	For \(c \in \R\), we note that \(\left( g \circ f \right) ^{-1} \left( \left( c, \infty \right]  \right)  = f^{-1}\left( g^{-1}\left( \left( c, \infty \right]  \right)  \right) \) . By continuity of \(g\), we know \(g^{-1}\left( \left( c, \infty \right]  \right) \in \overline{\mathscr{B}}\) . And, since \(f\) is measurable, we find \(f^{-1}\left( g^{-1}\left( \left( c, \infty \right]  \right)  \right) \).
\end{proof}
\begin{corollary}[]
Let \(S \subseteq \R\)	 and \(f: S \to \R\)	to be a measurable function. Then, for every \(\alpha \in \R\)	and \(0 < \rho < \infty\)	, we find \(\alpha f\)	 and \(\left| f \right| ^{\rho}\)	 are measurable.
\end{corollary}
\begin{proof}
	We see the functions \(g\left( u \right) = \alpha u\)	on \(\overline{\R}\)	and \(h\left( u \right)  = \left| u \right| ^{\rho}\)	 on \(\overline{\R}\)	 to be the corresponding functions. We see the case \(h\)	is clearly continuous and well defined. On the other hand \(g\)	may be poorly defined if \(\alpha = 0\)	 and \(f\left( x \right)  = \infty\)	. Recall, however, we had \(0 \cdot  \pm \infty = 0\)	so \(g\)	 is just the zero functions and we see continuity holds.
\end{proof}
\begin{definition}[Almost-everywhere]
	Let \(S\)	 be measurable, then a property is  said to hold true \textbf{almost everywhere} on \(S\)	 or \textbf{for almost all}	 \(x \in S\)	if there is a set \(T\)	 with \(\mu\left( T \right) = 0\)	 and the property holds on all of \(S \setminus T\)	.
\end{definition}
