\section{Intro to Functional Analysis}
\lecture{22}{Thu 11 Nov 2021 19:29}{\(L^{p}\) spaces}
I skipped a chapter on supporting lines and Jensen's inequality because the material was rather simple and well explained in Hagen's notes.\\
\begin{definition}[Essential Supremum]
	Let \(f: S \to \overline{\R}\) be measurable. Then, we denote the quantity \[
		\ess f = \inf \{ M \in \overline{\R} : m \left( \{x \in S : f\left( x \right) > M\}  \right) = 0   \}
\] is called the \textbf{essentail supremum }of \(f\). Note that \(f \le \ess f\) almost everywhere.
\end{definition}
\begin{definition}[Lp space]
	Let \(f: S \to \overline{\R}\) be measurable ,then
	\begin{itemize}
		\item For \( 1 \le p \le \) we define \(\|f\|_{p} = \left( \int_{S} \left| f \right| ^{p} \right)^{\frac{1}{p}} \) to be the \textbf{\(L^{p}\) norm } of \(f\).
			\item \(\|f\|_{\infty} = \ess \left| f \right| \) is the \textbf{\(L^{\infty}\) norm} of \(f\).
	\end{itemize}
\end{definition}
\begin{definition}[Equivalent functions]
	For \(1 \le p \le \infty\) let \(V_{p}\left( s \right) \) be the set of all measurable functions \(f: S \to \overline{\R}\) so that \(\f\|_{p}< \infty\). Then, functions \(f, g \in V_{p}\left( S \right) \) are \textbf{equivalent}, denoted \(f \sim g\) , if \(f = g\) almost everywhere in \(S\).\\
	The set of all equivalence classes \(V_{p}\left( S \right) / \sim\) is denoted \(L^{p}\left( S \right) \) and called the \textbf{Lebesque space}.
\end{definition}
\begin{remark}
	If \(f \sim g\) in \(L_{P}\left( S \right) \) , then \(f = g\) almost everywhere (on \(S\)) hence \(\|f - g\|_{p}= 0\). Hence the \(L^{p}\) norm can be extended to norms on equivalence classes by simply denoting \(\|\left[ f \right] \|_{p} = \|f\|\) for some equivalence class \(\left[ f \right]  \in L^{p}\left( S \right) \) .
\end{remark}
\begin{theorem}[Minkowski's Inequality]
	Suppose \(f, g \in L^{p}\left( S \right) \) for a \(1 \le p \le \infty\). Then, \(\|f + g\|_{p} \le \|f\|_{p} + \|g\|_{p}\).\\
	Moreover, if \(1 < p < \infty\), then \(\|f + g\|_{p} = \|f\|_{p} + \|g\|_{p}\) if and only if there is a \(c \ge 0\) so that \(f = c g\) almost everywhere.
\end{theorem}
\begin{proof}
	Let \(x = \|f\|_{p}\) 	, \(s = \|g\|_{p}\). Then, we see the claim is trivial true if \(r = 0\), \(s = 0\), or \(p = \infty\). Hence, define \(\lambda = \frac{r}{r+s}\) and we may assume \(f, g\) are finite by definition of \(L^{p}\) space. Since \(t \mapsto \left| t \right| ^{p}\) is convex on \(\R\) and \(\lambda \in \left( 0, 1 \right) \), we see
	\begin{align*}
		\left| f + g \right| ^{p} &=  \left| \lambda \frac{f}{\lambda}  + \left( 1-\lambda \right) \frac{g}{1-\lambda}\right| ^{p}\\
					  &\le \lambda \left| \frac{f}{\lambda} \right| ^{p} + \left( 1-\lambda \right) \left| \frac{g}{1-\lambda} \right| ^{p}  \\
		\implies \|f + g\|_{p} &\le \lambda \|\frac{f}{\lambda}\|_{p}^{p} + \left( 1-\lambda \right) \|\frac{g}{1-\lambda}\|^{p}_{p}\\
					  &= \lambda \left( r + s \right) ^{p} + \left( 1-\lambda \right) \left( r + s \right) ^{p}\\
					  &= \left( \|f\|_{p} + \|g\|_{p} \right)^{p}  \\
	.\end{align*}
	Note that this last step comes from appealing to the definition of lambda and noting \(r^{p} = \int \left| f \right| ^{p}\)  and similarly for \(g\). Now, we note that \(t \mapsto \left| t \right| ^{p}\) is strictly convex for \(1 < p < \infty\) , so equality occurs if and only if \(\frac{f}{\lambda} = \frac{g}{1-\lambda}\) (almost everywhere if \(f, g\) are functions and not equivalence classes) hence \(f\) is a multiple of \(g\).
\end{proof}
\begin{remark}
	Note that this implies \(L^{p}\left( S \right) \) is closed under addition, and constant multiplication (this part is trivial), so it is a linear space.
\end{remark}
\begin{definition}[Normed Linear Space]
A linear space \(V\) is a \textbf{normed linear space} if there is a function \(\|.\|: V \to \R\)  called the \textbf{norm of \(V\) } so that the following hold
\begin{itemize}
	\item \(\|v\|\ge 0\) for all \(v \in V\),
		\item \(\|v\|= 0\) if and only if \(v = 0\),
			\item \(\|\lambda v\| = \left| \lambda \right| \|v\|\) for all \(\lambda \in R\), \(v \in V\),
				\item \(\|v + w\|\le \|v\| + \|w\|\) for all \(v, w \in V\).
\end{itemize}
\end{definition}
\begin{remark}
	\(V_{p}\left( S \right) \) is not itself a normed linear space as the function \(f\left( x \right)  = \left \{
		\begin{array}{11}
			0, & \quad x \not\in \Q \\
			1, & \quad x \in Q
		\end{array}
		\right\) has \(\|f\| = 0\) even though \(f\) is not the zero function. We rule out this possibility by considering only the equivalence classes, in which case \(f \sim 0\), so \(L^{p}\left( S \right) \) is in fact a normed metric space.
\end{remark}
\begin{definition}[Conjugate]
	For \(p \in \left[ 1, \infty \right] \) we define the \textbf{conjugate} of \(p\) to be the extended real number \(q \in \left[ 1, \infty \right] \) so that \(\frac{1}{p} + \frac{1}{q} = 1\).
\end{definition}
\begin{lemma}[Young's Inequality]
	Suppose \(p \in \left( 1, \infty \right) \) with \(q\) its conjugate and \(a, b \in \R\) with \(a, b  \ge 0\). Then, \(ab \le \frac{a^{p}}{p} + \frac{q^{p}}{p} \). Moreover equality holds if and only if \(a^{p} = b^{q}\).\\
	Specifically \(\sqrt{ab} \le \frac{a+b}{2}\), that is the geometric mean is at most the arithmetic mean.
\end{lemma}
\begin{proof}
	It suffices to assume \(a, b\) positive as the \(0\) case is trivial. Then, define \(F\left( t \right) = a^{p\left( 1-t \right) }b^{qt} = a^{p} \left( \frac{b^{q}}{a^{p}} \right)^{t} \). We see \(F\) is convex on \(\R\) as it is exponential. Hence,
	\begin{align*}
		ab &= F\left( \frac{1}{p}\cdot 0 + \left( 1-\frac{1}{p} \right) q\right)  \\
		   &\le \frac{1}{p}F\left( 0 \right) + \left( 1-\frac{1}{p} \right) F\left( 1 \right) \\
&= \frac{a^{p}}{p} + \frac{b^{q}}{q}
	.\end{align*} As \(F\) is strictly convex (except in the case \(\frac{b^{q}}{a^{p}} = 1\)), we see equality will not arrive except in this exceptional case.
\end{proof}
