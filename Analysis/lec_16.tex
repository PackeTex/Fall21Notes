\lecture{16}{Tue 19 Oct 2021 13:02}{Conclusion of Measure Theory and Lebesque Integration}
\begin{recall}
	We stated the theorems behind littlewood's \(3\) principles, now we prove them.
\end{recall}
\begin{proof}
	\begin{enumerate}
		\item \(\left( 2.2 \right) \). Let \(J\) be the collection of all open intervals \(\left( a, b \right) \) with \(a, b \in \Q\)  and \(a < b\). Since \(J\) is countable we can order the intervals \(J = \{J_{k} : k \in \N\} \). Let \(\epsilon > 0\) and first we do the case \(S\) is bounded. For each \(n \in \N\), there is a closed set \(C_{n} \subseteq f^{-1}\left( J_{n} \right) \) and a \(D_{n} = S \setminus f^{-1}\left( I_{n} \right) \) such that \( \mu\left( S \setminus \left( C_{n} \cup D_{n} \right)  \right) < \frac{\epsilon}{2^{n}} \). Since \(S\) is bounded, \(C_{n}\)  and \(D_{n}\) are compact. Let \(K = \bigcap_{n\in \N}\left( C_{n} \cup D_{n} \right)  \) and as \(C_{n}, D_{n} \subseteq S\) , we see \(K \subseteq S\). Furthermore, \(K\) is compact and we find \( \mu\left( S\setminus K \right) \le \sum_{i= 1}^{\infty} \mu\left( S \setminus \left( C_{n} \cup D_{n} \right)  \right) < \epsilon \). Now, we show the restriction is continuous. Let \(\epsilon > 0\), then for \(x \in K\) we find \(a, b \in \Q\) such that \(a < f\left( x \right)  < b\) and \(b - a < \epsilon\). Hence, there is \(n \in \N\) such that \(I_{n} = \left( a, b \right) \). Consequently, \(x \in f^{-1}\left( I_{n} \right) \) and \(x \not\in S \setminus f^{-1}\left( I_{n} \right) \). So, \(x \in \left( S \setminus f^{-1}\left( I_{n} \right)  \right)^{c} \subseteq D_{n}^{c} \). As \(D_{n}\) is closed, \(D_{n}^{c}\) is open, hence there is a \(\delta > 0\) so that \(\left( x - \delta, x + \delta \right) \subseteq D_{n}^{c}\). If \(y \in K \cap D_{n}^{c}\), then \(y \in C_{n}\), thus \(y \in f^{-1}\left( I_{n} \right) \) , hence \(a < f\left( y \right)  < b\). So, \(\left| f\left( x \right)  - f\left( y \right)  \right|  < b-a = \epsilon \) for \(y \in \left( x-\delta, x + \delta \right) \).\\
			Now, we do the unbounded case. As \(S\) is unbounded and \(\epsilon > 0\) , we find \(N \in N\) so that \(S^{\prime} = S \cap \left[ -N, N \right] \)  has the property \( \mu\left( S \setminus S^{\prime} \right) < \frac{\epsilon}{2}\), that is \(S\) is approximated by a bounded function arbitrarily well. Since \(S^{\prime}\) is bounded, there is a compact set \(K \subseteq S^{\prime} \subset S\) so that \(f\mid K\) is continuous and \( \mu\left( S^{\prime} \setminus K \right) < \frac{\epsilon}{2}\). Then, \( \mu\left( S \setminus K \right) = \mu\left( S \setminus S^{\prime} \right)  + \mu\left( S^{\prime} \setminus K \right) < \frac{\epsilon}{2} + \frac{\epsilon}{2} = \epsilon  \).
		\item \(\left( 2.4 \right) \). Let \(E^{*}\) be the set of all \(x \in S\)  such that \(\left( f_{n}\left( x \right)  \right) \) does not converge. By assumption, \( \mu\left( E^{*} \right)  = 0\). Since \(f\left( x \right) = \lim_{n \to \infty}f_{n}\left( x \right) = \limsup_{n \to \infty} f_{n}\left( x \right) \) for all \(x \in S \setminus E^{*}\), then \(f\) is measurable. For \(k, \ell \in \N\) , let \(E_{k, \ell} = \{x \in S : \left| f_{\ell}\left( x \right) - f\left( x \right)  \right| \ge \frac{1}{k}\} \). Then, \(E_{k, \ell}\) is measurable. Fix \(k\). If for each \( n \in \N\) there is a \(\ell \ge n\) so that \(\left| f_{\ell}\left( x \right)  - f\left( x \right)  \right|\ge \frac{1}{k} \) , then \(x \in E^{*}\) as \(f\) does not converge at that point. Hence, \(\bigcap_{n \in \N} \bigcup_{\ell = n} ^{\infty}E_{k, \ell} \subseteq E^{*}\). Since \( \mu\left( \bigcup_{\ell=1} ^{\infty}E_{k, \ell} \right) \le \mu\left( S \right)  \le \infty \) , and the collection \(\{\bigcup_{\ell = n} ^{\infty}E_{k, \ell}\} \) is clearly descending. Hence, \( \mu \left( \bigcap_{n \in \N} \bigcup_{k=n} ^{\infty}E_{k, \ell} \right) = \lim_{n \to \infty} \mu \left( \bigcup_{\ell=n} ^{\infty}E_{k, l} \right) \le \mu\left( E^{*} \right) = 0 \). This holds for all \(k \in \N\). So, for \(\epsilon > 0\) and \(k \in \N\), we have a \(n_{k} \in \N\) such that \( \mu\left( \bigcup_{\ell = n_{k}} ^{\infty} E_{k, \ell} \right) < \frac{\epsilon}{2^{k}}\). Thus, \(E = \bigcup_{k \in \N} \bigcup_{\ell = n_{k}}^{\infty} E_{k, \ell}  \) is measurable and \( \mu\left( E \right)  < \sum_{k=1}^{\infty} \bigcup_{\ell= n_{k}}^{\infty}E_{k, l} = \sum_{k=1}^{\infty} \frac{\epsilon}{2^{k}} = \epsilon \). If \(x \in S \setminus E\), then \(\left| f_{n}\left( x \right) - f\left( x \right)  \right| < \frac{1}{k}\) for \(k \in \N\) if \(n \ge n_{k}\). So, \(\left( f_{n} \right) \) converges uniformly on \(S \setminus E\).

	\end{enumerate}
\end{proof}
This concludes measure theory.
\section{Lebesque Integration}
\begin{definition}[Lebesque Integral: Nonnegative Simple Functions]
Let \(s\) be a	nonnegative simple function of the form \(s = \sum_{k= 1}^{K} a_{k}\chi_{S_{k}}\) where \(\{S_{k} : 1\le k \le K\} \) is a disjoint collection of measurable sets. Then, the \textbf{Lebesque Integral} of \(s\) is defined to be \[
	\int s = \int s\left( x \right) dx = \int s d\mu = \sum_{k=1}^{K} a_{k} \mu\left( S_{k} \right)
.\]
\end{definition}
\begin{proposition}
	If \(s\) is nonnegative and simple with two representations, \(s = \sum_{k=1}^{K} a_{k} \chi _{S_{k}} = \sum_{j=1}^{J} b_{j} \chi _{T_{j}}\) for disjoint collections of measurable sets \(\{S_{k} : 1 \le k \le K\} \) and \(\{T_{j} : 1\le j \le J\} \). Then \[
		\sum_{k=1}^{K} a_{k} \mu\left( S_{k} \right)  = \sum_{j=1}^{J} b_{j} \mu\left( T_{j} \right)
	.\]
	In particular, \(\int s\) is well defined.
\end{proposition}
The proof of this is trivial.
\begin{lemma}
Let \(s, t\) be nonnegative and simple and \(\alpha \ge 0\). Then \[
	\alpha \cdot \int s  = \int \alpha \cdot s \text{ and } \int (s+t) = \int s + \int t

.\]
\end{lemma}
\begin{proof}
	Clearly, multiplying the sum times \(\alpha\) yields \(\alpha \sum_{k=1}^{K} a_{k} \mu\left( S_{k} \right)  = \sum_{k=1}^{K} \alpha a_{k} \mu\left( S_{k} \right) \).\\
	For the second claim. Suppose \(s = \sum_{k=1}^{K} a_{k} \chi_{S_{k}}\) and \(g = \sum_{j=1}^{J} b_{j} \chi_{T_{j}}\) are canonical representations. Then, \(s + t = \sum_{k=1}^{K} \sum_{j=1}^{J} \left( a_{k} + b_{j} \right) \chi_{S_{k} \cap T_{j}}\) with \(\{S_{k} \cap T_{j} : 1 \le k \le K, 1\le j \le J\} \) is a disjoint collection and
	\begin{align*}
		\int\left( s+t \right) &= \sum_{k=1}^{K} \sum_{j=1}^{J} \left( a_{k} + b_{j} \right) \mu\left( S_{k} \cap T_{j} \right)  \\
				       &= \sum_{k=1}^{K} a_{k} \sum_{j=1}^{J} \mu\left( S_{k} \cap T_{j} \right) + \sum_{j=1}^{J} b_{j} \sum_{k=1}^{K} \mu\left( S_{k} \cap T_{j} \right)  \\
				       &= \sum_{k=1}^{K} a_{k} \mu\left( S_{k} \right) + \sum_{j=1}^{J} b_{j} \mu\left( T_{j} \right)   \\
				       &= \int s + \int t.
	\end{align*}
\end{proof}
\begin{lemma}
	Let \(s, t\) be nonnegative and simple such that \(s \le t\). Then, \(\int s \le \int t\).
\end{lemma}
\begin{proof}
	\begin{align*}
		\int t &= \int \left( t-s + s \right) \\
		       &= \int \underbrace{\left( t-s \right)}_{\ge 0}  + \int s \\
		       &\ge \int s
	.\end{align*}
\end{proof}
\begin{definition}
	Let \(f: S \to \overline{\R}\), then the \textbf{zero extension} of \(f\) to \(\R\) is \begin{align*}
		f^{*}: \R &\longrightarrow \overline{\R} \\
		x &\longmapsto f^{*}\left( x \right)  = \left \{
			\begin{array}{11}
				f\left( x \right) , & \quad x \in S \\
				0, & \quad x\not\in S
			\end{array}
			\right.
	.\end{align*}
	Moreover, this function preserves measurability.
\end{definition}
\begin{definition}[Lebesque Integral of a General Nonnegative Function]
	Let \(f: \R \to \overline{\R}\) 	be a nonnegative measurable function and \(\mathscr{S}\left( f \right) \) be the collection of all nonnegative simple functions, \(s\), such that \(s \le f\). Then, the \textbf{Lebesque Integral} of \(f\) over \(\R\) is defined to be
	\begin{align*}
		\int f = \int_{\R} f\left( x \right)dx &= \sup \{ \int s : s \in \mathscr{S}\left( f \right)  \}  \\
	.\end{align*}
	If \(f: S \to \overline{\R}\) is nonnegative and measurable, then
	\begin{align*}
		\int_{S} f = \int_{S}f\left( x \right) dx &= \int_{\R} f^{*} \\
	.\end{align*}
\end{definition}
\begin{theorem}[Chebyshev's Inequality]
	Let \(f: \R \to \overline{\R}\) 	be nonnegative and measurable. Then, for any \(\lambda \in \left( 0, \infty \right) \) , then \[
		\mu\left( \{x \in \R : f\left( x \right)  \ge \lambda\}  \right) \le \frac{1}{\lambda} \int f
	.\]
\end{theorem}
\begin{proof}
	Let \(E = \{x \in \R : f\left( x \right)  \ge \lambda\}\). This is the preimage of an extended borel set, hence measurable. Let \(s = \lambda \chi_{E}\). Then, \(s \in \mathscr{S}\left( f \right) \). Hence, \(\int s = \lambda \mu\left( E \right) \le \int f \). Hence the inequality holds.
\end{proof}
\begin{theorem}
	Let \(f: \R \to \overline{\R}\)  be nonnegative ad measurable. Then \(\int f = 0\) if and only if \(f\left( x \right) = 0\) for almost every \(x \in \R\).
\end{theorem}
\begin{proof}
	Suppose \(\int f = 0\) , then by chebyshev \begin{align*}
		\mu\left( \{x \in \R : f\left( x \right)  > 0\}  \right) &= \mu (\bigcup_{n \in \N} \{x \in \R : f\left( x \ge n \right) \}  ) \\
									 &\le \sum_{n=1}^{\infty} \mu\left( \{ x\in \R : f\left( x \right)  \ge \frac{1}{n}\}  \right) \\
									 &= \sum_{i= 1}^{\infty} \int f \\
									 &= 0
	.\end{align*}
	Conversely, if \(f\left( x \right)  = 0\) almost everywhere, then for every \(s \in \mathscr{S}\left( f \right) \) , we see \(s\) is zero almost everywhere, hence \(\int s = 0\) , so \(\int f = \sup \{ 0 : s \in \mathscr{S}\left( f \right)  \}  = 0\).
\end{proof}
