\documentclass[a4paper]{article}
% Some basic packages
\usepackage[utf8]{inputenc}
\usepackage[T1]{fontenc}
\usepackage{textcomp}
\usepackage{url}
\usepackage{graphicx}
\usepackage{float}
\usepackage{booktabs}
\usepackage{enumitem}

\pdfminorversion=7

% Don't indent paragraphs, leave some space between them
\usepackage{parskip}

% Hide page number when page is empty
\usepackage{emptypage}
\usepackage{subcaption}
\usepackage{multicol}
\usepackage{xcolor}

% Other font I sometimes use.
% \usepackage{cmbright}

% Math stuff
\usepackage{amsmath, amsfonts, mathtools, amsthm, amssymb}
% Fancy script capitals
\usepackage{mathrsfs}
\usepackage{cancel}
% Bold math
\usepackage{bm}
% Some shortcuts
\newcommand\N{\ensuremath{\mathbb{N}}}
\newcommand\R{\ensuremath{\mathbb{R}}}
\newcommand\Z{\ensuremath{\mathbb{Z}}}
\renewcommand\O{\ensuremath{\varnothing}}
\newcommand\Q{\ensuremath{\mathbb{Q}}}
\newcommand\C{\ensuremath{\mathbb{C}}}
% Easily typeset systems of equations (French package)

% Put x \to \infty below \lim
\let\svlim\lim\def\lim{\svlim\limits}

%Make implies and impliedby shorter
\let\implies\Rightarrow
\let\impliedby\Leftarrow
\let\iff\Leftrightarrow
\let\epsilon\varepsilon
\let\nothing\varnothing

% Add \contra symbol to denote contradiction
\usepackage{stmaryrd} % for \lightning
\newcommand\contra{\scalebox{1.5}{$\lightning$}}

 \let\phi\varphi

% Command for short corrections
% Usage: 1+1=\correct{3}{2}

\definecolor{correct}{HTML}{009900}
\newcommand\correct[2]{\ensuremath{\:}{\color{red}{#1}}\ensuremath{\to }{\color{correct}{#2}}\ensuremath{\:}}
\newcommand\green[1]{{\color{correct}{#1}}}

% horizontal rule
\newcommand\hr{
    \noindent\rule[0.5ex]{\linewidth}{0.5pt}
}

% hide parts
\newcommand\hide[1]{}

% Environments
\makeatother
% For box around Definition, Theorem, \ldots
\usepackage{mdframed}
\mdfsetup{skipabove=1em,skipbelow=0em}
\theoremstyle{definition}
\newmdtheoremenv[nobreak=true]{definition}{Definition}
\newmdtheoremenv[nobreak=true]{eg}{Example}
\newmdtheoremenv[nobreak=true]{corollary}{Corollary}
\newmdtheoremenv[nobreak=true]{lemma}{Lemma}[section]
\newmdtheoremenv[nobreak=true]{proposition}{Proposition}
\newmdtheoremenv[nobreak=true]{theorem}{Theorem}[section]
\newmdtheoremenv[nobreak=true]{law}{Law}
\newmdtheoremenv[nobreak=true]{postulate}{Postulate}
\newmdtheoremenv{conclusion}{Conclusion}
\newmdtheoremenv{bonus}{Bonus}
\newmdtheoremenv{presumption}{Presumption}
\newtheorem*{recall}{Recall}
\newtheorem*{previouslyseen}{As Previously Seen}
\newtheorem*{interlude}{Interlude}
\newtheorem*{notation}{Notation}
\newtheorem*{observation}{Observation}
\newtheorem*{exercise}{Exercise}
\newtheorem*{comment}{Comment}
\newtheorem*{practice}{Practice}
\newtheorem*{remark}{Remark}
\newtheorem*{problem}{Problem}
\newtheorem*{solution}{Solution}
\newtheorem*{terminology}{Terminology}
\newtheorem*{application}{Application}
\newtheorem*{instance}{Instance}
\newtheorem*{question}{Question}
\newtheorem*{intuition}{Intuition}
\newtheorem*{property}{Property}
\newtheorem*{example}{Example}
\numberwithin{equation}{section}
\numberwithin{definition}{section}
\numberwithin{proposition}{section}

% End example and intermezzo environments with a small diamond (just like proof
% environments end with a small square)
\usepackage{etoolbox}
\AtEndEnvironment{example}{\null\hfill$\diamond$}%
\AtEndEnvironment{interlude}{\null\hfill$\diamond$}%

\AtEndEnvironment{solution}{\null\hfill$\blacksquare$}%
% Fix some spacing
% http://tex.stackexchange.com/questions/22119/how-can-i-change-the-spacing-before-theorems-with-amsthm
\makeatletter
\def\thm@space@setup{%
  \thm@preskip=\parskip \thm@postskip=0pt
}


% \lecture starts a new lecture (les in dutch)
%
% Usage:
% \lecture{1}{di 12 feb 2019 16:00}{Inleiding}
%
% This adds a section heading with the number / title of the lecture and a
% margin paragraph with the date.

% I use \dateparts here to hide the year (2019). This way, I can easily parse
% the date of each lecture unambiguously while still having a human-friendly
% short format printed to the pdf.

\usepackage{xifthen}
\def\testdateparts#1{\dateparts#1\relax}
\def\dateparts#1 #2 #3 #4 #5\relax{
    \marginpar{\small\textsf{\mbox{#1 #2 #3 #5}}}
}

\def\@lecture{}%
\newcommand{\lecture}[3]{
    \ifthenelse{\isempty{#3}}{%
        \def\@lecture{Lecture #1}%
    }{%
        \def\@lecture{Lecture #1: #3}%
    }%
    \subsection*{\@lecture}
    \marginpar{\small\textsf{\mbox{#2}}}
}



% These are the fancy headers
\usepackage{fancyhdr}
\pagestyle{fancy}

% LE: left even
% RO: right odd
% CE, CO: center even, center odd
% My name for when I print my lecture notes to use for an open book exam.
% \fancyhead[LE,RO]{Gilles Castel}

\fancyhead[RO,LE]{\@lecture} % Right odd,  Left even
\fancyhead[RE,LO]{}          % Right even, Left odd

\fancyfoot[RO,LE]{\thepage}  % Right odd,  Left even
\fancyfoot[RE,LO]{}          % Right even, Left odd
\fancyfoot[C]{\leftmark}     % Center

\makeatother




% Todonotes and inline notes in fancy boxes
\usepackage{todonotes}
\usepackage{tcolorbox}

% Make boxes breakable
\tcbuselibrary{breakable}

% Verbetering is correction in Dutch
% Usage:
% \begin{verbetering}
%     Lorem ipsum dolor sit amet, consetetur sadipscing elitr, sed diam nonumy eirmod
%     tempor invidunt ut labore et dolore magna aliquyam erat, sed diam voluptua. At
%     vero eos et accusam et justo duo dolores et ea rebum. Stet clita kasd gubergren,
%     no sea takimata sanctus est Lorem ipsum dolor sit amet.
% \end{verbetering}
\newenvironment{correction}{\begin{tcolorbox}[
    arc=0mm,
    colback=white,
    colframe=green!60!black,
    title=Opmerking,
    fonttitle=\sffamily,
    breakable
]}{\end{tcolorbox}}

% Noot is note in Dutch. Same as 'verbetering' but color of box is different
\newenvironment{note}[1]{\begin{tcolorbox}[
    arc=0mm,
    colback=white,
    colframe=white!60!black,
    title=#1,
    fonttitle=\sffamily,
    breakable
]}{\end{tcolorbox}}


% Figure support as explained in my blog post.
\usepackage{import}
\usepackage{xifthen}
\usepackage{pdfpages}
\usepackage{transparent}
\newcommand{\incfig}[2][1]{%
    \def\svgwidth{#1\columnwidth}
    \import{./figures/}{#2.pdf_tex}
}

% Fix some stuff
% %http://tex.stackexchange.com/questions/76273/multiple-pdfs-with-page-group-included-in-a-single-page-warning
\pdfsuppresswarningpagegroup=1
\binoppenalty=9999
\relpenalty=9999

% My name
\author{Thomas Fleming}

\usepackage{pdfpages}
\title{Analysis I: Homework III Corrections}
\date{Fri 10 Sep 2021 12:58}
\DeclareMathOperator{\SRG}{SRG}
\DeclareMathOperator{\cut}{Cut}
\DeclareMathOperator{\GF}{GF}
\DeclareMathOperator{\V}{V}
\DeclareMathOperator{\E}{E}
\DeclareMathOperator{\edg}{e}
\DeclareMathOperator{\vtx}{v}
\DeclareMathOperator{\diam}{diam}

\DeclareMathOperator{\tr}{tr}
\DeclareMathOperator{\A}{A}

\DeclareMathOperator{\Adj}{Adj}
\DeclareMathOperator{\mcd}{mcd}

\begin{document}
\maketitle
\begin{solution}[18]
	As \(\left[ a, b \right] \) is compact, we see \(f\) is uniformly continuous. Hence, there is a \(\delta > 0\) such that for all \(\epsilon > 0\) and \(x, y \in \left[ a, b \right] \) we find \(\left| x - y \right| < \delta\)  implies \(\left| f\left( x \right)  - f\left( y \right)  \right|  < \epsilon\).\\
	Fix \(\epsilon > 0\), and  define the following sequence. Let \(y_0 = a\) and \(y_{i} = \max \{a + \delta  \cdot i, b\}\) for \(i \ge 0\). Then, we see \(\{\left[ y_{i-1}, y_{i } \right] : i \in \N\} \) is a cover and there is a \(n \ge 0\)  such that \(y_{n} = b\), hence \(y_{m} = b\) for \(m \ge n\) and we see \(\{\left[ y_{i-1}, y_{i} \right] : 1 \le i \le n\} \) is a finite subcover. Define \begin{align*}
		g: \left[ a, b \right]  &\longrightarrow \R \\
		x &\longmapsto g(x) =
				\frac{f\left( y_{i} \right) - f\left( y_{i-1} \right) }{y_{i} - y_{i-1}} \left( x - y_{i} \right) + f\left( y_{i} \right)   , & \quad \text{ for } x \in \left[ y_{i-1}, y_{i} \right]
	.\end{align*}
	We see \(g\) is simply the piecewise linear interpolation of \(f\) on the \(y_{i}\)'s and it is well defined (the endpoints agree for each closed interval). Hence, for all \(x \in \left[ a, b \right] \) there is an \(i\ge 1\) such that \(x \in \left[ y_{i-1}, y_{i} \right]  = \left[ y_{i-1}, y_{i-1} + \delta \right] = \left[ y_{i} - \delta, y_{i} \right]  \), hence \(\left| y_{i-1} - x \right| < \delta\) and \(\left| y_{i} - x \right| < \delta\)  so we see \(\left| f \left( y_{i-1} \right)  - f\left( x \right)  \right| < \frac{\epsilon}{3} \) and \(\left| f\left( y_{i} \right) - f\left( x \right)  \right| < \frac{\epsilon}{3}\). Then, either \(f\left( y_{i-1} \right) \le g\left( x \right) \le f\left( y_{i} \right) \) or \(f\left( y_{i} \right) \le g\left( x \right) \le f\left( y_{i-1} \right) \) as \(g\) is the linear interpolation between these two points. Then, we see \(\left| f\left( y_{i} \right)  - g\left( x \right)  \right| \le \left| f\left( y_{i} \right)  - f\left( y_{i-1} \right)  \right| \). Hence, we find
	\begin{align*}
		\left| g\left( x \right) - f\left( x \right)  \right| &\le \left| f\left( y_{i} \right) - g\left( x \right)  \right|  + \left| f\left( x \right) - f\left( y_{i} \right)  \right| \\
								      &\le \left| f\left( y_{i} \right) - f\left( y_{i-1} \right)  \right|  + \frac{\epsilon}{3}\\
								      &\le \left| f\left( y_{i} \right) - f\left( x \right)  \right|  + \left| f\left( x \right) - f\left( y_{i-1} \right)  \right|  + \frac{\epsilon}{3}\\
								      &< \frac{\epsilon}{3} + \frac{\epsilon}{3} + \frac{\epsilon}{3}\\
								      &= \epsilon
	.\end{align*}
\end{solution}
\newpage
\begin{solution}[22]
\begin{enumerate}
	\item It suffices to assume \(m\left( S \right)  < \infty\), because for all sets of infinite measure, we can choose a subset of finite measure  \(S^{\prime} \subseteq S\)  and  \(S \cap \left( a, b \right)  \supseteq S^{\prime} \cap \left( a, b \right) \), so \(m\left( S \cap \left( a, b \right)  \right) \ge m\left( S^{\prime} \cap \left( a, b \right)  \right) \).\\
		Then assuming \(m\left( S \right) \) finite, for \(\epsilon = \frac{1}{3} m\left( S \right) \), we find an open \(U\) with \(S \subseteq U\) and \(m\left( U \setminus S \right)  < \epsilon = \frac{1}{3} m\left( S \right) \). Hence, \(m\left( U \right)  < \frac{4}{3} m\left( S \right) \). As \(U\) is open it is the countable union of disjoint intervals \(\left( a_{i}, b_{i} \right) \) and \( m\left( U \right)  = \sum_{i= 1}^{\infty} \left( b_{i} - a_{i} \right) < \frac{4}{3} m\left( S \right)  \). Hence, \[
			\sum_{i= 1}^{\infty} \frac{3}{4}\left( b_{i} - a_{i} \right) < m\left( S \right)
		.\]
		Suppose \(m\left( S \cap \left( a_{i}, b_{i} \right)  \right) \le \frac{3}{4}\left( b_{i}-a_{i} \right) \) for all the intervals \(\left( a_{i}, b_{i} \right) \) . Then,
		\begin{align*}
			m\left( S \right)  &= \sum_{i= 1}^{\infty} m\left( S \cap\left( a_{i}, b_{i} \right)  \right) \\
					   &\le \sum_{i= 1}^{\infty} \frac{3}{4}m\left( a_{i}, b_{i} \right)\\
					   &= \sum_{i= 1}^{\infty} \frac{3}{4} \left( b_{i} - a_{i} \right) \\
					   &< m\left( S \right) \lightning
		.\end{align*}
		Hence, we have atleast one \(\left( a_{i}, b_{i} \right) \) such that \(m\left( S \cap \left( a_{i}, b_{i} \right)  \right) > \frac{3}{4}\left( b_{i}-a_{i} \right)  \).
	\item First, note that \(S \cap \left( r + S \right)  = \{s - r \in S : s \in S\} \), and suppose \\\(S \cap \left( r + S \right)  \cap \left( a, b \right)  = \O\). \\That is, for all \(s \in S \cap \left( a, b \right) \), we have \(s + r \not\in S \cap \left( a, b \right) \subseteq \left( a, b \right)  \). Hence, \(s \in \left( b-r, b \right) \subseteq \left( b -  \frac{1}{4}\left( b-a \right) , b \right)   = \left( \frac{1}{4}a + \frac{3}{4}b, b \right)  \). But, we see \(m\left( \left( \frac{1}{4}a + \frac{3}{4}b, b \right)  \right)  = \frac{1}{4} ( b - a) < \frac{3}{4} \left( b-a \right)  \). \\So, we have \(S \cap \left( a, b \right) \subseteq \left( \frac{1}{4}a + \frac{3}{4}b, b \right) \), \(\lightning\). \\ Hence there is a \(s \in S \cap \left( a, a + \frac{3}{4}(b - a) \right) \), so \(s + r \in \left( a, b \right) \), so \[S \cap \left( r + S \right)  \cap \left( a, b \right) \neq \O .\]
	For each \(x \in \left[ -\frac{1}{4}\left( b-a \right) , \frac{1}{4}\left( b-a \right)   \right] \) , note that we have some \(s \in S\) such that \(s + x \in S \) or \(s - x \in S\) since \(S \cap \left( r + S \right) \) is nonempty, \(0 \le r \le \frac{1}{4}\left( b-a \right) \). Denote \(s + x = \overline{s}\) and \(s - x = \hat{s}\). If \(\overline{s} \in S\), then \(\overline{s} - s = x \in S - S\). Otherwise, if \(\hat{s} \in S\), then \(s - \hat{s} = x \in S - S\). Hence, \(\left[ -\frac{1}{4} \left( b-a \right) , \frac{1}{4}\left( b-a \right) \right] \subseteq S - S\).
\end{enumerate}
\end{solution}
\end{document}
