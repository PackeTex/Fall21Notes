\documentclass[a4paper]{article}
\input{../preamble.tex}
\usepackage{pdfpages}
\title{Real Variables I: Homework I}

\date{Fri 03 Sep 2021 09:05}
\begin{document}
\maketitle
\begin{problem}[1]
	Let \(f:X \to Y\).
	\begin{enumerate}
		\item Show that for \(A\subseteq X\), \(B\subseteq Y\), \(f\left( f^{-1}\left( B \right)  \right) \subseteq B\) and \(A \subseteq f^{-1}\left( f\left( A \right)  \right) \).
			\item Give examples to show that the set inclusions can be proper.
	\end{enumerate}
\end{problem}
\begin{solution}
	\begin{enumerate}
	\item Let \(b \in f\left( f^{-1} \left( B \right)  \right) \) and note that, as \(b\) is in the image of \(f^{-1}\left( B \right) \), there is \(a \in f^{-1}\left( B \right) \) such that \(f\left( a \right) = b\). As \(a \in f^{-1} \left( B \right) \), we see \(f\left( a \right) \in B\). As \(f\left( a \right) = b \in B\) this completes the proof.\\
	\item Now, let \(a \in A\). We see \(f\left( a \right) \in f\left( A \right) \) by definition, and as \(f\left( a \right) \in f\left( A \right) \) we see that for all \(b \in A\) such that \(f\left( b \right) = f\left( a \right) \in f\left( A \right) \), we have \(b \in f^{-1} \left( f\left( A \right)  \right) \). It is clear that \(a\) is one such element, so \(a \in f^{-1} \left( f\left( A \right)  \right) \). This completes the proof.
	\item Let \(f: \R \to \R, \ x \mapsto f(x) = x^2\) and denote \(B = [-1,1]\). We see \(f^{-1} \left( B \right) = [-1,1]\) and \(f\left( [-1, 1] \right) = [0, 1]\). Hence, \(f\left( f^{-1} \left( B \right) \right) = [0, 1] \subset [-1, 1] = B\).
	\item Now, let \(f: \R \to \R, \ x \mapsto f(x) = 0\) and denote \(A = [0, 1]\). We see \(f\left( A \right) = \{0\} \) and \(f^{-1}\left( \{0\}  \right) = \R \) as the function is zero everywhere. Hence \(f^{-1} \left( f\left( A \right)  \right) = \R \supset [0, 1] = A\).
	\end{enumerate}
\end{solution}
\newpage
\begin{problem}[2]
	Let \(A, B \subseteq X\). Prove or disprove
	\begin{enumerate}
		\item \(A \triangle B = \O \iff A = B\).
			\item \(A \triangle B = X \iff A = B^{c}\).
	\end{enumerate}
\end{problem}
\begin{solution}
	\begin{enumerate}
		\item Suppose \(A \triangle B = \O\) and let \(a \in A\), \(b \in B\). Then, we see \(a \not\in B \setminus A\) by definition. Furthermore, As \(A \triangle B = \left( A\setminus B \right) \cup \left( B \setminus A \right)  = \O\), we see \(a \not\in A \setminus B\), but as \(a \in A\) this implies \(a \in B\). Hence \(a \subseteq B\). Again, notice \(b \not\in A \setminus B\) by definition. Furthermore, \(b \not\in B \setminus A\) as this would make \(A\triangle B\) nonempty, so \(b \in A\). Hence, \(A=B\).\\
			Conversely, suppose \(A = B\). Then, \[
				A \triangle B = A \triangle A = \left( A \setminus A \right) \cup \left( A \setminus A \right)  = \O \cup \O = \O
			.\]
		\item Suppose \(A\triangle B = X\) and let \(a \in A\). Then, we see \(a \not\in B \setminus A\) by definition, but \(a \in X\), so \( a \in A \setminus B\). Hence \(a \not\in B\). As every \(a \in A\) has \(a \not\in B\), we see \(A \subseteq B^{c}\). Now, let \(b \in B^{c}\). We see \(b \not\in B\) by definition, hence \(b \not\in B \setminus A\). As \(b \in X\), we must then have that \(b \in A \setminus B\), hence \(b \in A\). Thus, \(B^{c} = A\).\\
			Conversely, suppose \(B^{c} = A\). Then, \[
				A \triangle B = B^{c} \triangle B = \left( B^{c} \setminus B \right)  \cup \left( B \setminus B^{c} \right) = B^{c} \cup B = X
			\] by definition of complements.
	\end{enumerate}
\end{solution}
\begin{problem}[3]
	Suppose \(f:X \to Y\) and \(g: Y\to Z\)  are functions.
	\begin{enumerate}
		\item Show that \(f:X \to Y\) is injective if and only if there is a map \(g: Y \to X\) such that \(g \circ f\) is the identity on \(X\). If such a map \(g\) exists is it necessarily unique, injective, or surjective.
			\item Show that \(f\) is onto if and only if there is a map \(g:Y \to X\) such that \(f \circ g\) is the identity on \(Y\).
	\end{enumerate}
\end{problem}
\begin{solution}
	\begin{enumerate}
		\item Let \(g: X \to Y\) be a map such that \(g \circ f\) is the identity on \(X\). Then, suppose \(f\) is not injective. Let \(x \neq y \in X\) such that \(f\left( x \right)  = f\left( y \right) \). Then \(g\left( f\left( x \right)  \right) = g\left( f\left( y \right)  \right) = x \text{ or } y \text{ or }\) another such element. WLOG, suppose \(g\left( f\left( x \right)  \right) = g\left( f\left( y \right)  \right) = x \). Then, \(g\left( f\left( y \right)  \right) =x\) contradicts the assumption that \(g \circ f\) was the identity. \\
			Now, suppose \(f\) is injective. Then, for each \(x \in X\) there is a unique \(f\left( x \right) \in Y\). Hence, let us define the map \(g: Y \to X\) such that \(g\left( f\left( x \right)  \right) = x\) for all \(x \in X\). We see this is a function as each \(f\left( x \right) \in Y \) originates from only \(1\) \(x \in X\) by injectivity. Hence, this implies \(g \circ f\) is the identity by this definition. This completes the proof.
			\item
	\end{enumerate}
\end{solution}
\end{document}
