\documentclass[a4paper]{article}
% Some basic packages
\usepackage[utf8]{inputenc}
\usepackage[T1]{fontenc}
\usepackage{textcomp}
\usepackage{url}
\usepackage{graphicx}
\usepackage{float}
\usepackage{booktabs}
\usepackage{enumitem}

\pdfminorversion=7

% Don't indent paragraphs, leave some space between them
\usepackage{parskip}

% Hide page number when page is empty
\usepackage{emptypage}
\usepackage{subcaption}
\usepackage{multicol}
\usepackage{xcolor}

% Other font I sometimes use.
% \usepackage{cmbright}

% Math stuff
\usepackage{amsmath, amsfonts, mathtools, amsthm, amssymb}
% Fancy script capitals
\usepackage{mathrsfs}
\usepackage{cancel}
% Bold math
\usepackage{bm}
% Some shortcuts
\newcommand\N{\ensuremath{\mathbb{N}}}
\newcommand\R{\ensuremath{\mathbb{R}}}
\newcommand\Z{\ensuremath{\mathbb{Z}}}
\renewcommand\O{\ensuremath{\varnothing}}
\newcommand\Q{\ensuremath{\mathbb{Q}}}
\newcommand\C{\ensuremath{\mathbb{C}}}
% Easily typeset systems of equations (French package)

% Put x \to \infty below \lim
\let\svlim\lim\def\lim{\svlim\limits}

%Make implies and impliedby shorter
\let\implies\Rightarrow
\let\impliedby\Leftarrow
\let\iff\Leftrightarrow
\let\epsilon\varepsilon
\let\nothing\varnothing

% Add \contra symbol to denote contradiction
\usepackage{stmaryrd} % for \lightning
\newcommand\contra{\scalebox{1.5}{$\lightning$}}

 \let\phi\varphi

% Command for short corrections
% Usage: 1+1=\correct{3}{2}

\definecolor{correct}{HTML}{009900}
\newcommand\correct[2]{\ensuremath{\:}{\color{red}{#1}}\ensuremath{\to }{\color{correct}{#2}}\ensuremath{\:}}
\newcommand\green[1]{{\color{correct}{#1}}}

% horizontal rule
\newcommand\hr{
    \noindent\rule[0.5ex]{\linewidth}{0.5pt}
}

% hide parts
\newcommand\hide[1]{}

% Environments
\makeatother
% For box around Definition, Theorem, \ldots
\usepackage{mdframed}
\mdfsetup{skipabove=1em,skipbelow=0em}
\theoremstyle{definition}
\newmdtheoremenv[nobreak=true]{definition}{Definition}
\newmdtheoremenv[nobreak=true]{eg}{Example}
\newmdtheoremenv[nobreak=true]{corollary}{Corollary}
\newmdtheoremenv[nobreak=true]{lemma}{Lemma}[section]
\newmdtheoremenv[nobreak=true]{proposition}{Proposition}
\newmdtheoremenv[nobreak=true]{theorem}{Theorem}[section]
\newmdtheoremenv[nobreak=true]{law}{Law}
\newmdtheoremenv[nobreak=true]{postulate}{Postulate}
\newmdtheoremenv{conclusion}{Conclusion}
\newmdtheoremenv{bonus}{Bonus}
\newmdtheoremenv{presumption}{Presumption}
\newtheorem*{recall}{Recall}
\newtheorem*{previouslyseen}{As Previously Seen}
\newtheorem*{interlude}{Interlude}
\newtheorem*{notation}{Notation}
\newtheorem*{observation}{Observation}
\newtheorem*{exercise}{Exercise}
\newtheorem*{comment}{Comment}
\newtheorem*{practice}{Practice}
\newtheorem*{remark}{Remark}
\newtheorem*{problem}{Problem}
\newtheorem*{solution}{Solution}
\newtheorem*{terminology}{Terminology}
\newtheorem*{application}{Application}
\newtheorem*{instance}{Instance}
\newtheorem*{question}{Question}
\newtheorem*{intuition}{Intuition}
\newtheorem*{property}{Property}
\newtheorem*{example}{Example}
\numberwithin{equation}{section}
\numberwithin{definition}{section}
\numberwithin{proposition}{section}

% End example and intermezzo environments with a small diamond (just like proof
% environments end with a small square)
\usepackage{etoolbox}
\AtEndEnvironment{example}{\null\hfill$\diamond$}%
\AtEndEnvironment{interlude}{\null\hfill$\diamond$}%

\AtEndEnvironment{solution}{\null\hfill$\blacksquare$}%
% Fix some spacing
% http://tex.stackexchange.com/questions/22119/how-can-i-change-the-spacing-before-theorems-with-amsthm
\makeatletter
\def\thm@space@setup{%
  \thm@preskip=\parskip \thm@postskip=0pt
}


% \lecture starts a new lecture (les in dutch)
%
% Usage:
% \lecture{1}{di 12 feb 2019 16:00}{Inleiding}
%
% This adds a section heading with the number / title of the lecture and a
% margin paragraph with the date.

% I use \dateparts here to hide the year (2019). This way, I can easily parse
% the date of each lecture unambiguously while still having a human-friendly
% short format printed to the pdf.

\usepackage{xifthen}
\def\testdateparts#1{\dateparts#1\relax}
\def\dateparts#1 #2 #3 #4 #5\relax{
    \marginpar{\small\textsf{\mbox{#1 #2 #3 #5}}}
}

\def\@lecture{}%
\newcommand{\lecture}[3]{
    \ifthenelse{\isempty{#3}}{%
        \def\@lecture{Lecture #1}%
    }{%
        \def\@lecture{Lecture #1: #3}%
    }%
    \subsection*{\@lecture}
    \marginpar{\small\textsf{\mbox{#2}}}
}



% These are the fancy headers
\usepackage{fancyhdr}
\pagestyle{fancy}

% LE: left even
% RO: right odd
% CE, CO: center even, center odd
% My name for when I print my lecture notes to use for an open book exam.
% \fancyhead[LE,RO]{Gilles Castel}

\fancyhead[RO,LE]{\@lecture} % Right odd,  Left even
\fancyhead[RE,LO]{}          % Right even, Left odd

\fancyfoot[RO,LE]{\thepage}  % Right odd,  Left even
\fancyfoot[RE,LO]{}          % Right even, Left odd
\fancyfoot[C]{\leftmark}     % Center

\makeatother




% Todonotes and inline notes in fancy boxes
\usepackage{todonotes}
\usepackage{tcolorbox}

% Make boxes breakable
\tcbuselibrary{breakable}

% Verbetering is correction in Dutch
% Usage:
% \begin{verbetering}
%     Lorem ipsum dolor sit amet, consetetur sadipscing elitr, sed diam nonumy eirmod
%     tempor invidunt ut labore et dolore magna aliquyam erat, sed diam voluptua. At
%     vero eos et accusam et justo duo dolores et ea rebum. Stet clita kasd gubergren,
%     no sea takimata sanctus est Lorem ipsum dolor sit amet.
% \end{verbetering}
\newenvironment{correction}{\begin{tcolorbox}[
    arc=0mm,
    colback=white,
    colframe=green!60!black,
    title=Opmerking,
    fonttitle=\sffamily,
    breakable
]}{\end{tcolorbox}}

% Noot is note in Dutch. Same as 'verbetering' but color of box is different
\newenvironment{note}[1]{\begin{tcolorbox}[
    arc=0mm,
    colback=white,
    colframe=white!60!black,
    title=#1,
    fonttitle=\sffamily,
    breakable
]}{\end{tcolorbox}}


% Figure support as explained in my blog post.
\usepackage{import}
\usepackage{xifthen}
\usepackage{pdfpages}
\usepackage{transparent}
\newcommand{\incfig}[2][1]{%
    \def\svgwidth{#1\columnwidth}
    \import{./figures/}{#2.pdf_tex}
}

% Fix some stuff
% %http://tex.stackexchange.com/questions/76273/multiple-pdfs-with-page-group-included-in-a-single-page-warning
\pdfsuppresswarningpagegroup=1
\binoppenalty=9999
\relpenalty=9999

% My name
\author{Thomas Fleming}

\usepackage{pdfpages}
\title{Real Variables I: Homework I}

\date{Fri 03 Sep 2021 09:05}
\begin{document}
\maketitle
\begin{problem}[1]
	Let \(f:X \to Y\).
	\begin{enumerate}
		\item Show that for \(A\subseteq X\), \(B\subseteq Y\), \(f\left( f^{-1}\left( B \right)  \right) \subseteq B\) and \(A \subseteq f^{-1}\left( f\left( A \right)  \right) \).
			\item Give examples to show that the set inclusions can be proper.
	\end{enumerate}
\end{problem}
\begin{solution}
	\begin{enumerate}
	\item Let \(b \in f\left( f^{-1} \left( B \right)  \right) \) and note that, as \(b\) is in the image of \(f^{-1}\left( B \right) \), there is \(a \in f^{-1}\left( B \right) \) such that \(f\left( a \right) = b\). As \(a \in f^{-1} \left( B \right) \), we see \(f\left( a \right) \in B\). As \(f\left( a \right) = b \in B\) this completes the proof.\\
	\item Now, let \(a \in A\). We see \(f\left( a \right) \in f\left( A \right) \) by definition, and as \(f\left( a \right) \in f\left( A \right) \) we see that for all \(b \in A\) such that \(f\left( b \right) = f\left( a \right) \in f\left( A \right) \), we have \(b \in f^{-1} \left( f\left( A \right)  \right) \). It is clear that \(a\) is one such element, so \(a \in f^{-1} \left( f\left( A \right)  \right) \). This completes the proof.
	\item Let \(f: \R \to \R, \ x \mapsto f(x) = x^2\) and denote \(B = [-1,1]\). We see \(f^{-1} \left( B \right) = [-1,1]\) and \(f\left( [-1, 1] \right) = [0, 1]\). Hence, \(f\left( f^{-1} \left( B \right) \right) = [0, 1] \subset [-1, 1] = B\).
	\item Now, let \(f: \R \to \R, \ x \mapsto f(x) = 0\) and denote \(A = [0, 1]\). We see \(f\left( A \right) = \{0\} \) and \(f^{-1}\left( \{0\}  \right) = \R \) as the function is zero everywhere. Hence \(f^{-1} \left( f\left( A \right)  \right) = \R \supset [0, 1] = A\).
	\end{enumerate}
\end{solution}
\newpage
\begin{problem}[2]
	Let \(A, B \subseteq X\). Prove or disprove
	\begin{enumerate}
		\item \(A \triangle B = \O \iff A = B\).
			\item \(A \triangle B = X \iff A = B^{c}\).
	\end{enumerate}
\end{problem}
\begin{solution}
	\begin{enumerate}
		\item Suppose \(A \triangle B = \O\) and let \(a \in A\), \(b \in B\). Then, we see \(a \not\in B \setminus A\) by definition. Furthermore, As \(A \triangle B = \left( A\setminus B \right) \cup \left( B \setminus A \right)  = \O\), we see \(a \not\in A \setminus B\), but as \(a \in A\) this implies \(a \in B\). Hence \(a \subseteq B\). Again, notice \(b \not\in A \setminus B\) by definition. Furthermore, \(b \not\in B \setminus A\) as this would make \(A\triangle B\) nonempty, so \(b \in A\). Hence, \(A=B\).\\
			Conversely, suppose \(A = B\). Then, \[
				A \triangle B = A \triangle A = \left( A \setminus A \right) \cup \left( A \setminus A \right)  = \O \cup \O = \O
			.\]
		\item Suppose \(A\triangle B = X\) and let \(a \in A\). Then, we see \(a \not\in B \setminus A\) by definition, but \(a \in X\), so \( a \in A \setminus B\). Hence \(a \not\in B\). As every \(a \in A\) has \(a \not\in B\), we see \(A \subseteq B^{c}\). Now, let \(b \in B^{c}\). We see \(b \not\in B\) by definition, hence \(b \not\in B \setminus A\). As \(b \in X\), we must then have that \(b \in A \setminus B\), hence \(b \in A\). Thus, \(B^{c} = A\).\\
			Conversely, suppose \(B^{c} = A\). Then, \[
				A \triangle B = B^{c} \triangle B = \left( B^{c} \setminus B \right)  \cup \left( B \setminus B^{c} \right) = B^{c} \cup B = X
			\] by definition of complements.
	\end{enumerate}
\end{solution}
\begin{problem}[3]
	Suppose \(f:X \to Y\) and \(g: Y\to Z\)  are functions.
	\begin{enumerate}
		\item Show that \(f:X \to Y\) is injective if and only if there is a map \(g: Y \to X\) such that \(g \circ f\) is the identity on \(X\). If such a map \(g\) exists is it necessarily unique, injective, or surjective.
			\item Show that \(f\) is onto if and only if there is a map \(g:Y \to X\) such that \(f \circ g\) is the identity on \(Y\).
	\end{enumerate}
\end{problem}
\begin{solution}
	\begin{enumerate}
		\item Let \(g: X \to Y\) be a map such that \(g \circ f\) is the identity on \(X\). Then, suppose \(f\) is not injective. Let \(x \neq y \in X\) such that \(f\left( x \right)  = f\left( y \right) \). Then \(g\left( f\left( x \right)  \right) = g\left( f\left( y \right)  \right) = x \text{ or } y \text{ or }\) another such element. WLOG, suppose \(g\left( f\left( x \right)  \right) = g\left( f\left( y \right)  \right) = x \). Then, \(g\left( f\left( y \right)  \right) =x\) contradicts the assumption that \(g \circ f\) was the identity. \\
			Now, suppose \(f\) is injective. Then, for each \(x \in X\) there is a unique \(f\left( x \right) \in Y\). Hence, let us define the map \(g: Y \to X\) such that \(g\left( f\left( x \right)  \right) = x\) for all \(x \in X\). We see this is a function as each \(f\left( x \right) \in Y \) originates from only \(1\) \(x \in X\) by injectivity. Hence, this implies \(g \circ f\) is the identity by this definition. This completes the proof.
			\item
	\end{enumerate}
\end{solution}
\end{document}
