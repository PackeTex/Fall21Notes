\newpage
\lecture{2}{Thu 26 Aug 2021 13:02}{Algebras}
\section{Algebras and Measure Theory}
For an interval $\left( a, b \right) $ we define the length of the interval to be $b-a$. Similarly, for an interval extending to infinity in either direction, we define the length to be infinite. Again, for the union of multiple intervals, we simply define their length by the sum of the respective interval lengths. We wish to generalize this notion, and for this reason we develop measure theory.
 \begin{remark}
There are sets which are not measurable. The construction of such sets will come later.
\end{remark}
We will begin defining which sets are measurable by constructing $\sigma$-algebras. The simplest $\sigma$-algebra is formed by taking the power set of a given collection.
\begin{proposition}
	 For every collection $\mathscr{C}$ of subsets of $X$, there is a smallest algebra $\mathscr{A} \supseteq \mathscr{C}$. That is, if $\mathscr{D}$ is an algebra such that $\mathscr{D} \supseteq \mathscr{C}$, then $\mathscr{D} \supseteq \mathscr{A}$. Note this applies to $\sigma$-algebras as well. In the case of $\sigma$-algebra, we call this the \textbf{borrel $\sigma$-algebra}.
\end{proposition}
\begin{proof}
	Let $\mathscr{F}$ be the set of all algebras ($\sigma$-algebras) which contain $\mathscr{C}$. Note, as $\mathscr{P}\left( X \right) $ is always an algebra, then we see $\mathscr{P} \left( X \right) \in \mathscr{F}$, hence $\mathscr{F}$ is nonempty. Now, let $\mathscr{A}= \bigcap_{B \in \mathscr{F}} B $. Then, as $\mathscr{C} \subseteq B$ for all $B \in \mathscr{F}$, we have $\mathscr{C} \subseteq \mathscr{A}$. Similarly, $X \in \mathscr{A}$, because $X \in \mathscr{B}$ for all $B \in \mathscr{F}$. If $A \in \mathscr{A}$, then $A \in B$ for all $B \in \mathscr{F}$. Since every $B$ is an algebra ($\sigma$-algebra), then $A^{c} \in B$ for all $B \in \mathscr{F}$, hence $A^{c} \in \bigcap_{B \in \mathscr{F}} B = \mathscr{A}$. If $ \{A_{k}\} $ is a finite (countable) collection with $A_{k} \in \mathscr{A}$, then $A_{k} \in B$ for all $B \in \mathscr{F}$. Since each $B$ is an algebra ($\sigma$-algebra), then $\bigcup_{k} A_{k} \in B$ for all $B \in \mathscr{F}$, then $\bigcup_{k} A_{k} \in \bigcap_{B \in \mathscr{F}} B = \mathscr{A}$. It is clear this is the smallest algebra as it is the smallest set containing every element of $\mathscr{F}$.
\end{proof}
\begin{remark}
	$\sigma$-algebras are the domains of measures. That is, if given a measure, it must be defined over a $\sigma$-algebra.
\end{remark}
\begin{definition}[Choice Function]
	Given a collection $\mathscr{F}$ of nonempty sets, a \textbf{choice function}  $ f: \mathscr{F} \to \bigcup_{F \in \mathscr{F}}F $, $F \mapsto f(F) \in F $. For example, let $\mathscr{F}= \{\{1,2,3\}, \{2, 5\}, \{1\}\} $, then $f\left( \{1, 2, 3\}  \right) = 1, 2, \text{ or } 3$.
\end{definition}
Of course, such a choice function is simple to construct for a finite number of countable sets, but consider $\mathscr{F}= \mathscr{P}\left( \mathscr{P}\left( \mathscr{P}\left( \R \right)  \right)  \right) $, then it is very unclear how one would go about selecting an element from each set.
\begin{law}[Axiom of Choice]
	For all collections $\mathscr{F}$, there is always a choice function (which may be defined non-constructively).
\end{law}
While this axiom is not necessary to construct measure theory, its exclusion makes certain basic theorems unprovable, for instance the proposition from last lecture that the countable union of countable sets is countable. Generally, we work with a weaker version of the axiom of choice:
\begin{law}[Zermelo's Axiom of Choice]
Every collection of nonempty sets has a choice function.
\end{law}
Other weakenings include the axiom of countable choice. The name Zermelo comes from Zermelo-Franklin set theory, known as ZF theory. When Zermelo's AOC is included in the axioms, we denote this model \textbf{ZFC set theory}.
\begin{proposition}
	Let \(\{X_{\lambda}\} \) be a collection of nonempty sets indexed by \(\lambda \in \Lambda\). Let \(X\) be the cartesian product \[\prod_{\lambda \in \Lambda}^{}  X_{\lambda} = \{g : g: \Lambda \to \bigcup_{\lambda \in \Lambda} X_{\lambda}:g\left( \lambda \right) \in X_{\lambda} \text{ for every } \lambda \in \Lambda\} .\] Properly defining this product over arbitrary \(\Lambda\) requires some form of Axiom of Choice.
\end{proposition}
\begin{remark}
In the case of such a cartesian product, taking AOC, we know there is a choice function \(f: \{X_{\lambda}: \lambda \in \Lambda\}  \to \bigcup_{\lambda \in \Lambda} X_{\lambda} ,X_{\lambda} \  \mapsto f(X_{\lambda}) \in X_{\lambda}\). Letting \(g\left( \lambda \right) = f\left( X_{\lambda} \right) \) yields the desired \(g\) for \(\prod_{\lambda \in \Lambda}^{} X_{\lambda}\).
\end{remark}
\begin{definition}[Relation]
	Let \(X\) be nonempty, then a \textbf{relation}, \(R\), is a subset of \(X \times X\). If an element \(x \in X\) is in a relation with \(y \in Y\) we could write \(\left( x, y \right) \in R\), but this is nonstandard. The normal notation is \(xRy\) to mean \(x\) is in relation with \(y\).
\end{definition}
\begin{definition}[Properties of a Relation]
	\begin{enumerate}
		\item Reflexive: \(xRx\) for all \(x \in X\).
		\item Symmetric: \(xRy \implies yRx\).
		\item Transitive: If \(xRy\) and \(yRz\), then \(xRz\).
	\end{enumerate}
\end{definition}
\begin{example}
	The simplest example is equality,\(=\), which is reflexive, symmetric, and transitive.\\
	Another example is the order relation of \(\Q\), \(\le\). This is reflexive, not symmetric, and transitive.
\end{example}
\begin{definition}[Equivalence Relation]
	An \textbf{equivalence relation} is a relation, \(R\),  which is reflexive, symmetric, and transitive. We generally denote an equivalence relation (other than equality) by \(\sim\).
\end{definition}
\newpage
\begin{remark}
	Because of the desired properties of equivalence relations, they allow us to partition a set \(X\) into \textbf{equivalence classes} or \textbf{components}. That is, given two equivalence classes \(\left[ x_1 \right] = \{y \in X : y \sim x_1\}  = \left[ \hat{x_1} \right] \) if and only if \(x_1 \sim \hat{x_1}\). Here \(x_1\) or \(\hat{x_1}\) or any other member of their equivalence class are called the representative. One last property is that given two representatives \(x, y\) either \(\left[ x \right] = \left[ y \right] \) or \(\left[ x \right] \cap \left[ y \right] = \O\).
\end{remark}
\begin{definition}[Partial Order]
\begin{enumerate}
	\item A reflexive transitive relation \(R\) on a nonempty set \(X\) is called a \textbf{partial ordering} if \(xRy\) and \(yRx\) imply \(x=y\).
	\item A partial ordering \(R\) on a nonempty set \(X\) is called a \textbf{total ordering} if for all \(x,y \in X\) we have  at least one of \(xRy\) or \(yRx\) being true. In this case we call \(X\) a \textbf{total ordered} or sometimes just \textbf{ordered}.
	\item For a nonempty \(X\) with partial ordering \(R\) we call \(z \in X\) an \textbf{upper bound} of a subset \(A \subseteq X\), for all \(x \in A\), \(xRz\) is true.
	\item For a nonempty \(X\) and a partial ordering \(R\), we call \(z \in X\) a \textbf{maximal element} if \(zRx\) implies \(z = x\).
\end{enumerate}
\end{definition}
\begin{example}
	\(\le\) is a complete order on \(\R\), but when we extend \(\le\) to \(\mathbb{C}\) such that \(x\le y\) is only well defined for \(x, y \in \R\), it becomes only a partial ordering.
\end{example}
A classical example of a statement about orderings is Zorn's Lemma, which turns out to be equivalent to the Axiom of Choice.
\begin{theorem}[Zorn's Lemma]
	Let \(X\) be a nonempty set with a partial ordering. If every totally ordered subset of \(X\) has an upper bound, then \(X\) has a maximal element.
\end{theorem}
A very similar result to this is Hausdorff's Maximality Theorem.
