\lecture{7}{Tue 14 Sep 2021 13:01}{Sequences (\(2\)), Limits, and Continuity}
\begin{definition}[Limit Superior/Inferior]
	A number \(L\)	 is called the \textbf{limit superior} of \(\left( x_{n} \right) \), denoted by \(\limsup_{n \to \infty}x_{n}\), if for each \(\epsilon > 0\), the set \(\{n \in \N : x_{n} > L + \epsilon\} \) is finite and the set \(\{m \in \N : x_{m} \ge L - \epsilon\} \) is infinite.\\
	Similairly, the \textbf{limit inferior} of \(\left( x_{n} \right) \), denoted \(\liminf _{n \to \infty} x_{n}\), is the number \(\ell\) such that for each \(\epsilon > 0\) \(\{n \in \N : x_{n} < \ell + \epsilon\} \) is finite and \(\{m \in \N : x_{m} \le \ell + \epsilon\} \)
\end{definition}
\begin{remark}
	In order for the limit superior to exist, the sequence must be bounded from above. Furthermore, the limits superior and inferior are unique.
\end{remark}
\begin{proposition}
	Suppose \(\left( x_{n} \right) \) is bounded above, then \(\limsup_{n \to \infty} \left( x_{n} \right) \) exists if and only if \(\left( x_{n} \right) \) has an accumulation point and if so, \(\limsup_{n \to \infty}\left( x_{n} \right) \) is the largest accumulation point of \(\left( x_{n} \right) \).
\end{proposition}
\begin{proof}
	Suppose \(L = \limsup_{n \to \infty}\left( x_{n} \right) \) exists. Then for each \(\epsilon > 0\), the set \(\{{n} \in \N : L - \epsilon \le x_{m} \le L + \epsilon\} \) is infinite hence \(L\) is an accumulation point.\\
	Conversely, let \(A\) be the set of accumulation points of \(\{x_{n}\} \). \(A\) is nonempty and has an upper bound. Hence, \(L = \sup \left( A \right) \) exists. We wish to show \(L = \limsup_{n \to  \infty} \left( x_{n} \right) \). Since \(L\) is the least upper bound of \(A\), for each \(\epsilon > 0\) there is a \(K \in A\) such that \(K > L - \frac{\epsilon}{2}\). Since \(K \in A\), we see for each \(N \in \N\) there is a \(n \ge N\) such that \(\left| K - x_{n} \right|  < \frac{\epsilon}{2}\). Consequently, for such \(n\), we have \(x_{n} > K - \frac{\epsilon}{2} > L - \epsilon\). Hence, \[
	\left| L - x_{n} \right| \le \left| L - K \right|  + \left| K - x_{n} \right| < \epsilon
	.\]
	Hence, \(L \in A\) and the set \(\{n \in \N :  x_{n} > L  - \epsilon\} \) is infinite. Lastly, for every \(\epsilon > 0\), the set \(\{n \in \N : x_{n} > L + \epsilon\} \) is finite for otherwise by Bolzano-Weirstrass, there would be an accumulation point larger than \(L\) contradicting the fact that \(L = \sup \left( A \right) \).
\end{proof}
\begin{remark}
	\(\left( x_{m} \right) \) converges if and only if \(\limsup_{m \to \infty}x_{m}\) and \(\liminf_{m \to \infty x_{m}}\) both exist and are equal. This further guarantees \(\lim_{m \to \infty}x_{m} = \limsup_{m \to \infty}x_{m} = \liminf_{m \to \infty}x_{m}\).
\end{remark}
\begin{proposition}
	Suppose \(\left( x_{n} \right) \) has a limit superior. Then, \[
	\limsup_{n \to \infty}x_{n} = \inf \{\sup \{x_{k} : k \in \N \text{ and } k \ge m\} : m \in \N \}
	.\]
\end{proposition}
\begin{definition}[Series]
	Let \(\left( x_{n} \right) \) to be a sequence, then the sequence \(\left( \sum_{i= 1}^{m} x_{i} \right)_{m} \) is called a \textbf{series}. We often abbreviate this \(\sum_{i= 1}^{\infty} x_{k}\) or \(\sum_{k}^{} x_{k}\). We also sometimes denote the limit of the series as \(\sum_{i= 1}^{\infty} x_{i}\).\\
	The sum of a doubly infinite sequence \(\left( x_{k} \right)_{k \in \Z} \) of nonnegative numbers is defined by \[
		\sum_{k=-\infty}^{\infty} x_{k} = \sum_{ k\in \Z}^{} x_{k} = \sup \{\sum_{k = n}^{m} x_{k} : n, m \in \Z, m \le n\} .
	.\]
\end{definition}
\begin{proposition}
	\begin{itemize}
	\item If \(\sum_{i= 1}^{\infty} x_{i}\) is convergent, then \(\lim_{n \to \infty}x_{n} = 0\).
		\item If \(\sum_{i=1}^{\infty}\left| x_{i} \right| \)  is convergent, then \(\sum_{i= 1}^{\infty} x_{i}\) is convergent.
		\end{itemize}
\end{proposition}
\section{Continuity}
\begin{definition}[ Continuity]
	A function \(f:S \to \R\) is \textbf{continuous at \(x_0\)} if for each \(\epsilon > 0\) there is a \(\delta > 0\) such that \(\left| f\left( x \right)  - f\left( x_0 \right)  \right| < \epsilon\) if \(\left| x - x_0 \right|  < \delta\).\\
	A function \(f:S \to \R\) is \textbf{continuous on \(X\)} if for every \(x_0 \in X \), \(f\) is continuous at \(x_0\).\\
	A function \(f:S \to \R\) is \textbf{continuous} if every \(x \in R\) has \(f\) being continous at \(x\).
	A function \(f: S \to \R\) is \textbf{uniformly continuous} on if for each \(\epsilon > 0\) there is a \(\delta > 0\) such that \(\left| f\left( x \right) - f\left( y \right)  \right| < \epsilon\) for all \(x, y \in S\) with \(\left| x - y \right|  < \delta\).\\
	A function  \(f : S \to \R\) is \textbf{lipschitz continuous} if there exists \(L \ge  0\) such taht \(\left| f\left( x \right) - f\left( y \right)  \right|  \le L \left| x- y \right| \) for all \(x, y \in S\).
\end{definition}
\begin{problem}
	Show lipschitz continuity implies uniform continuity implies continuity.
\end{problem}
\begin{definition}[]
	Let \(S\subseteq \R\). A set \(U \subseteq S\) is \textbf{relatively open}/\textbf{relatively closed} in \(S\) if there exists an open/closed set \(V \subseteq \R\) such that \(U = S \cap V\).
\end{definition}
\begin{proposition}
	\begin{itemize}
\item Let \(S \subseteq \R\). Then, \(f:S \to \R\) is continuous at \(x_0 \in S\) if and only if for every converget sequence \(\left( x_{n} \right) \) in \(S\) with \(\lim_{n \to \infty}x_{n} = x_0\) we have \(\left( f\left( x_{n} \right)  \right) \) is convergent with \(\lim_{n \to \infty}f\left( x_{n} \right)  = f\left( x_0 \right) \).
\item A function \(f: S \to \R\) is continuous if and only if \(f^{-1} \left( U \right) \) is relatively open in \(S\) for every open set \(U\subseteq \R\).
	\end{itemize}
\end{proposition}
\begin{theorem}[Max-Min Value Theorem]
	Suppose \(f:S \to \R\) is continuous, with \(S\) being nonempty and compact. Then, \(f\) takes a maximum and a minimum value.
\end{theorem}
The proof of this is pretty straghtforward so we will not write it in detail here. Essentially one takes an open covering of \(f\left( S \right) \) and notes that the preimage of this cover is also an open cover of \(S\), hence there is a finite subcover which can be pushed back to a finite subcover of \(f\left( S \right) \), As \(f\left( S \right) \) is compact we have that it is closed and bounded, hence maximum and minimum values exist.
\begin{theorem}[Intermediate Value Theorem]
	Suppose \(f: I \to \R\) is continuous on the interval \(I\). Then, \(f\left( I \right) \) is an interval.
\end{theorem}
Again, this theorem makes use of the fact that \(f\) is a continuous map and continuity preserves connectedness, hence I being an interval (connected) implies \(f\left( I \right) \) is an interval (connected).
\begin{definition}[Monotonicity]
	A function \(f: S \to \R\) is \textbf{increasing} if \(x, y \in S\) with \(x < y\) implies \(f\left( x \right)  \le f\left( y \right) \). It is \textbf{strictly increasing} if the inequality is strict.\\
	A function \(f: S \to \R\) is \textbf{decreasing} if \(x, y  \in S\) with \(x < y\) implies \(f\left( x \right) \ge f\left( y \right) \). Again, it is \textbf{strictly decreasing} if the inequality is strict.\\
	A function which is either increasing or decreasing is \textbf{monotone}.
\end{definition}
\begin{proposition}
	A monotone function \(f: I \to \R\) on an interval \(I\) is continuous if and only if \(f\left( I \right) \) is also an interval.
\end{proposition}
