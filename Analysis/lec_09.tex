\lecture{9}{Tue 21 Sep 2021 12:20}{Extended \(\R\) (2) and Intro to Measure Theory}
\begin{definition}
	Let \(S \subseteq \R\) and \(f : S \to \overline{\R}\). Then, we say \(f\) is continuous at \(x_0 \in S\) if \(H \circ f\) is continuous at \(x_0\) on \(S\) for any extending function \(H\). Similarity, we say \(f\) is continuous on \(S\) if \(H \circ f\) is continuous on \(S\) for any extending function \(H\).\\
	Furthermore, we say \(f\) is (strictly) increasing/decreasing/monotone if \(H \circ f\) is (strictly) increasing/decreasing/monotone.\\
	Again, if \(\left( f_{n} \right) \) is a series of functions \(f_{n} : S \to \overline{\R}\), we say \(\left( f_{n} \right) \) converges pointwise/uniformly to \(f:S \to \overline{\R}\) if \(\left( H \circ f_{n} \right) \) converges pointwise/uniformly to \(H \circ f\) for any extending function \(H\).
\end{definition}
\begin{definition}
	Let \(S \subseteq \overline{\R}\) and suppose  \(a \in \overline{R}\) is an accumulation point of a sequence taking values in \(S \setminus \{a\} \).\\
	Then, a function \(f : S \setminus \{a\} \to \overline{R}\) is said to have the limit \(L \in \overline{\R}\) (relative to \(S\)) if for any extending function \(H\) and for each \(\epsilon > 0\) we have an \(\delta > 0\) such that \[
		\left| H\left( f\left( x \right)  \right) - H\left( L  \right)  \right| < \epsilon \text{ for all \(x \in S \setminus \{a\} \) with } \left| H\left( x \right) - H\left( a \right)   \right| < \delta
	.\]
	We denote this by \(\lim_{x \to a} f\left( x \right) = L\) or \(\lim_{\underset{S}{x \to a} } f\left( x \right) = L\)
\end{definition}
\section{Measure Theory}
\begin{definition}[Length]
	Let \(I = \left( a, b \right) \) be an interval, then we define the measure function \(\ell: \mathscr{P}\left( \R \right)  \to \R^{+}_{0}\) with the following properties:
	\begin{align*}
		\ell \left( \O \right) &= 0\\
		\ell \left(  I  \right) &= b - a, a, b \in \R
	.\end{align*}
	In all other cases \(\ell \left( I = \infty \right) \).\\
\end{definition}
We would like to generalize this notion by constructing a set function \(\lambda\) such that
\begin{align*}
	\lambda:  \mathscr{P} \left( \R \right) &\to \left[ 0, \infty \right] \\
	\lambda\left( I \right) &= \ell\left( I \right) \text{ for intervals \(I \subseteq \R\)} \\
	\lambda\left( x + S \right) &=  \lambda\left( S \right) \text{ for \(x \in \R, S \subseteq \R, x + S = \{x + s : s \in S\} \)} \\
	\text{if } \{S_{m} : m \in \N\} &\text{ is a countable disjoint collection of sets in \(\R\), then }\\ \lambda\left( \bigcup_{n=1} ^{\infty} S_{m} \right) &=  \sum_{n=1}^{\infty} \lambda\left( S_{n} \right)  \\
.\end{align*}
It turns out such a function produces contradictions, hence it is poorly posed. Hence, we must alter or remove one of these constraints and as all of the properties are very straight forward it is best to alter the domain of \(\lambda\) itself.
\begin{definition}[Measure]
	Let \(\mathscr{A}\) be a \(\sigma\)-algebra.\\
	\begin{enumerate}
		\item A set function \(\mu: \mathscr{A}\to \left[ 0, \infty \right] \) is called \textbf{countably additive} if for every countable disjoint collection \(\{S_{n} \in \mathscr{A} : n \in \N\} \) we have \[
				\mu \left( \bigcup_{n \in \N} S_{n} \right)  = \sum_{i= 1}^{\infty} \mu \left( S_{i} \right)
		.\]
	\item A countable additive set function \(\mu: \mathscr{A} \to \left[ 0, \infty \right] \) such that \(\mu \left( \O \right) = 0\) is called a \textbf{measure}.
	\end{enumerate}
\end{definition}
\begin{proposition}
	Let \(\mu: \mathscr{A} \to \left[ 0, \infty \right]  \). Then, \( \mu\) is monotone in the sense that if \(A, B \in \mathscr{A}\) with \(A \subseteq B\) , then we have \( \mu\left( A \right)  \le \mu\left( B \right) \).
\end{proposition}
\begin{proof}
	Since \(B = A \cup \left( B \setminus A \right) \)	 and since \( \mu\) is countably additive, then \[
		\mu\left( B \right)  = \mu\left( A \right)  + \mu\left( B \setminus A \right) \ge \mu\left( A \right)
	.\]
\end{proof}
Now, we wish to extend our notion to arbitrary subsets of \(\R\).
\begin{Notation}
	For \(A \in \mathscr{P}\left( \R \right) \), then \(J\left( A \right) \) is defined to be the collection of all countable covers \(\{I_{n} : n \in \N\} \) of \(A\) consisting of open, bounded intervals \(I_{n}\).
\end{Notation}
\begin{definition}[Lebesque Outer Measure]
	Let \(A \in \mathscr{P}\left( \R \right) \)	, then the quantity \( \mu^{*} \left( A \right)  \in \left[ 0, \infty \right] \) is defined by \[
		\mu^{*}\left( A \right) = \inf \{\sum_{i= 1}^{\infty} \ell\left( J_{i} \right) : \{J_{i} : i \in \N\} \in J\left( A \right)  \}
	.\]
	This function \( \mu^{*}: \mathscr{P}\left( \R \right)  \to \left[ 0, \infty \right] \) is called the \textbf{Lebesque outer measure}.
\end{definition}
\begin{lemma}
	\begin{enumerate}
		\item The outer measure is monotone
		\item The outer measure is translation invariant.
		\item The outer measure is countable subadditive, that being for \(\{S_{n} : n \in \N\} \) is a countable collection of sets, then \( \mu^{*}\left( \bigcup_{n \in \N} S_{n} \right) \le \sum_{n= 1}^{\infty} \mu^{*}\left( S_{n} \right) \).  \end{enumerate}
\end{lemma}

\begin{proof}
	\begin{enumerate}
		\item Note that \(J\left( A \right) \subseteq J\left( B \right) \), hence \( \mu^{*} \left( A \right)  \le \mu^{*} \left( B \right) \).
		\item Similarly, as each \(\ell\left( J_{i} \right) \) is translationally invariant, we see \( \mu^{*}\) is translationally invariant.
		\item Let \(\epsilon >0\). Then for  each \(n \in \N\), let \(\{I_{n, k} : k \in \N\} \in J\left( S_{n} \right) \) be a collection of intervals such that \(\sum_{k=1}^{\infty} \ell\left( J_{n, k} \right)  \le \mu^{*}\left( S_{n} \right) + \frac{\epsilon}{2^{n}} \)
	\end{enumerate}
\end{proof}
