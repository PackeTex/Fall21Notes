\lecture{9}{Tue 21 Sep 2021 12:20}{Extended \(\R\) (2) and Intro to Measure Theory}
\begin{definition}
	Let \(S \subseteq \R\) and \(f : S \to \overline{\R}\). Then, we say \(f\) is continuous at \(x_0 \in S\) if \(H \circ f\) is continuous at \(x_0\) on \(S\) for any extending function \(H\). Similarity, we say \(f\) is continuous on \(S\) if \(H \circ f\) is continuous on \(S\) for any extending function \(H\).\\
	Furthermore, we say \(f\) is (strictly) increasing/decreasing/monotone if \(H \circ f\) is (strictly) increasing/decreasing/monotone.\\
	Again, if \(\left( f_{n} \right) \) is a series of functions \(f_{n} : S \to \overline{\R}\), we say \(\left( f_{n} \right) \) converges pointwise/uniformly to \(f:S \to \overline{\R}\) if \(\left( H \circ f_{n} \right) \) converges pointwise/uniformly to \(H \circ f\) for any extending function \(H\).
\end{definition}
\begin{definition}
	Let \(S \subseteq \overline{\R}\) and suppose  \(a \in \overline{R}\) is an accumulation point of a sequence taking values in \(S \setminus \{a\} \).\\
	Then, a function \(f : S \setminus \{a\} \to \overline{R}\) is said to have the limit \(L \in \overline{\R}\) (relative to \(S\)) if for any extending function \(H\) and for each \(\epsilon > 0\) we have an \(\delta > 0\) such that \[
		\left| H\left( f\left( x \right)  \right) - H\left( L  \right)  \right| < \epsilon \text{ for all \(x \in S \setminus \{a\} \) with } \left| H\left( x \right) - H\left( a \right)   \right| < \delta
	.\]
	We denote this by \(\lim_{x \to a} f\left( x \right) = L\) or \(\lim_{\underset{S}{x \to a} } f\left( x \right) = L\)
\end{definition}
