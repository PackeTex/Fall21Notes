\lecture{9}{Tue 21 Sep 2021 12:20}{Extended \(\R\) (2) and Intro to Measure Theory}
\begin{definition}
	Let \(S \subseteq \R\) and \(f : S \to \overline{\R}\). Then, we say \(f\) is continuous at \(x_0 \in S\) if \(H \circ f\) is continuous at \(x_0\) on \(S\) for any extending function \(H\). Similarity, we say \(f\) is continuous on \(S\) if \(H \circ f\) is continuous on \(S\) for any extending function \(H\).\\
	Furthermore, we say \(f\) is (strictly) increasing/decreasing/monotone if \(H \circ f\) is (strictly) increasing/decreasing/monotone.\\
	Again, if \(\left( f_{n} \right) \) is a series of functions \(f_{n} : S \to \overline{\R}\), we say \(\left( f_{n} \right) \) converges pointwise/uniformly to \(f:S \to \overline{\R}\) if \(\left( H \circ f_{n} \right) \) converges pointwise/uniformly to \(H \circ f\) for any extending function \(H\).
\end{definition}
\begin{definition}
	Let \(S \subseteq \overline{\R}\) and suppose  \(a \in \overline{R}\) is an accumulation point of a sequence taking values in \(S \setminus \{a\} \).\\
	Then, a function \(f : S \setminus \{a\} \to \overline{R}\) is said to have the limit \(L \in \overline{\R}\) (relative to \(S\)) if for any extending function \(H\) and for each \(\epsilon > 0\) we have an \(\delta > 0\) such that \[
		\left| H\left( f\left( x \right)  \right) - H\left( L  \right)  \right| < \epsilon \text{ for all \(x \in S \setminus \{a\} \) with } \left| H\left( x \right) - H\left( a \right)   \right| < \delta
	.\]
	We denote this by \(\lim_{x \to a} f\left( x \right) = L\) or \(\lim_{\underset{S}{x \to a} } f\left( x \right) = L\)
\end{definition}
\section{Measure Theory}
\begin{definition}[Length]
	Let \(I = \left( a, b \right) \) be an interval, then we define the measure function \(\ell: \mathscr{P}\left( \R \right)  \to \R^{+}_{0}\) with the following properties:
	\begin{align*}
		\ell \left( \O \right) &= 0\\
		\ell \left(  I  \right) &= b - a, a, b \in \R
	.\end{align*}
	In all other cases \(\ell \left( I = \infty \right) \).\\
\end{definition}
We would like to generalize this notion by constructing a set function \(\lambda\) such that
\begin{align*}
	\lambda:  \mathscr{P} \left( \R \right) &\to \left[ 0, \infty \right] \\
	\lambda\left( I \right) &= \ell\left( I \right) \text{ for intervals \(I \subseteq \R\)} \\
	\lambda\left( x + S \right) &=  \lambda\left( S \right) \text{ for \(x \in \R, S \subseteq \R, x + S = \{x + s : s \in S\} \)} \\
	\text{if } \{S_{m} : m \in \N\} &\text{ is a countable disjoint collection of sets in \(\R\), then }\\ \lambda\left( \bigcup_{n=1} ^{\infty} S_{m} \right) &=  \sum_{n=1}^{\infty} \lambda\left( S_{n} \right)  \\
.\end{align*}
It turns out such a function produces contradictions, hence it is poorly posed. Hence, we must alter or remove one of these constraints and as all of the properties are very straight forward it is best to alter the domain of \(\lambda\) itself.
\begin{definition}[Measure]
	Let \(\mathscr{A}\) be a \(\sigma\)-algebra.\\
	\begin{enumerate}
		\item A set function \(\mu: \mathscr{A}\to \left[ 0, \infty \right] \) is called \textbf{countably additive} if for every countable disjoint collection \(\{S_{n} \in \mathscr{A} : n \in \N\} \) we have \[
				\mu \left( \bigcup_{n \in \N} S_{n} \right)  = \sum_{i= 1}^{\infty} \mu \left( S_{i} \right)
		.\]
	\item A countable additive set function \(\mu: \mathscr{A} \to \left[ 0, \infty \right] \) such that \(\mu \left( \O \right) = 0\) is called a \textbf{measure}.
	\end{enumerate}
\end{definition}
\begin{proposition}
	Let \(\mu: \mathscr{A} \to \left[ 0, \infty \right]  \). Then, \( \mu\) is monotone in the sense that if \(A, B \in \mathscr{A}\) with \(A \subseteq B\) , then we have \( \mu\left( A \right)  \le \mu\left( B \right) \).
\end{proposition}
\begin{proof}
	Since \(B = A \cup \left( B \setminus A \right) \)	 and since \( \mu\) is countably additive, then \[
		\mu\left( B \right)  = \mu\left( A \right)  + \mu\left( B \setminus A \right) \ge \mu\left( A \right)
	.\]
\end{proof}
Now, we wish to extend our notion to arbitrary subsets of \(\R\).
\begin{Notation}
	For \(A \in \mathscr{P}\left( \R \right) \), then \(J\left( A \right) \) is defined to be the collection of all countable covers \(\{I_{n} : n \in \N\} \) of \(A\) consisting of open, bounded intervals \(I_{n}\).
\end{Notation}
\begin{definition}[Lebesque Outer Measure]
	Let \(A \in \mathscr{P}\left( \R \right) \)	, then the quantity \( \mu^{*} \left( A \right)  \in \left[ 0, \infty \right] \) is defined by \[
		\mu^{*}\left( A \right) = \inf \{\sum_{i= 1}^{\infty} \ell\left( J_{i} \right) : \{J_{i} : i \in \N\} \in J\left( A \right)  \}
	.\]
	This function \( \mu^{*}: \mathscr{P}\left( \R \right)  \to \left[ 0, \infty \right] \) is called the \textbf{Lebesque outer measure}.
\end{definition}
\begin{lemma}
	\begin{enumerate}
		\item The outer measure is monotone
		\item The outer measure is translation invariant.
		\item The outer measure is countable subadditive, that being for \(\{S_{n} : n \in \N\} \) is a countable collection of sets, then \( \mu^{*}\left( \bigcup_{n \in \N} S_{n} \right) \le \sum_{n= 1}^{\infty} \mu^{*}\left( S_{n} \right) \).  \end{enumerate}
\end{lemma}

\begin{proof}
	\begin{enumerate}
		\item Note that \(J\left( A \right) \subseteq J\left( B \right) \), hence \( \mu^{*} \left( A \right)  \le \mu^{*} \left( B \right) \).
		\item Similarly, as each \(\ell\left( J_{i} \right) \) is translationally invariant, we see \( \mu^{*}\) is translationally invariant.
		\item Let \(\epsilon >0\). Then for  each \(n \in \N\), let \(\{I_{n, k} : k \in \N\} \in J\left( S_{n} \right) \) be a collection of intervals such that \(\sum_{k=1}^{\infty} \ell\left( J_{n, k} \right)  \le \mu^{*}\left( S_{n} \right) + \frac{\epsilon}{2^{n}} \).\\
			Since, \(\{I_{n, k} : n, k \in \N\} \in J\left( \bigcup_{n \in \N} S_{n} \right) \), we must have that \begin{align*}
				\mu^{*}\left( \bigcup_{n \in \N} S_{n} \right) &\le \sum_{n \in \N} \sum_{k \in \N}^{} \ell \left( I_{n, k} \right)\\
									       &\le \sum_{n= 1}^{\infty} \mu^{*} \left( S_{n} \right) + \epsilon
			.\end{align*}
Since this holds for all \(\epsilon > 0\), this completes the proof.
	\end{enumerate}
\end{proof}
\begin{lemma}
	For every interval \(I \subseteq \R\), the outer measure is \( \mu^{*} \left( I \right)  = \ell \left( I \right) \).
\end{lemma}
\begin{proof}
	Let \(I\subseteq \R\) be nonempty (if \(I = \O\) it is trivial that \( \mu^{*}\left( \O \right) = 0\)). First, assume \(I = \left[ a, b \right] \)  with \(a\le b \in \R\). Let \(\{J_{n} : n \in \N\} \in J\left( I \right) \), then by Heine-Borel there is a finite subcovering \(\{I_{n} : 1\le n \le N\} \) such that no \(I_{n} = \O\). Note that as we will be taking the infimum, then any infinite collection containing this finite collection will be larger (or equal) hence will not matter in the infimum. Furthermore, we can assume that no interval \(J_{n}\) has \(J_{n} \subseteq J_{m}\) for some \(n \neq m\), and we can assume \(I_{n} = \left( a_{n}, b_{n} \right) \) to be ordered such that \(a_{n} < a_{n+1}\) for \(1\le n \le N-1\). Consequently, \(b_{n} > a_{n + 1}\) for \(1 \le n \le N-1\) as otherwise their would be a gap in the covering, and \(b_{n} > b\), as \(b \in I\), and \(a_1 < a\) by the same reasoning. Hence, we have an overlapping covering of \(\left[ a, b \right] \) by open bounded intervals \(\left( a_{n}, b_{n} \right) \). Hence,
	\begin{align*}
		\ell\left( J \right) &=  b-a \\
				     &\le b_{N}- a_1\\
				     &\le \sum_{i= 1}^{N} \left( b_{i} - a_{i} \right) \\
				     &= \sum_{i= 1}^{N} \ell\left( I_{i} \right)  \\
				     &\le \sum_{i= 1}^{\infty} \ell\left( I_{i} \right)\\
				     &\implies \ell\left( I \right)  \le \mu^{*}\left( I \right)
	.\end{align*}
	Now, we look to obtain the opposite inequality. Let \(\epsilon > 0\), then \(\{\left( a - \epsilon, b + \epsilon \right) \} \in J\left( I \right)  \), hence
	\begin{align*}
		\mu^{*} \left( I \right)  &\le b - a + 2\epsilon\\
		&= \ell\left( I \right)  + 2\epsilon\\
	\text{as \(\epsilon\) is arbitrary} &\text{, we then have }\\
		\mu^{*}\left( I \right) &\le \ell\left( I \right)
	.\end{align*}
	Hence \( \mu^{*} \left( I \right)  = \ell\left( I \right) \) for this case.\\
	Now, assume \(I \in \{\left( a, b \right) : \left[ a, b \right), \left( a, b \right]  \} \) is any bounded interval with \(a < b\). By monotonicity, for every \(\epsilon > 0\), we have
	\begin{align*}
		\ell\left( I \right) - 2\epsilon &= b - a - 2\epsilon \\
						 &\le \ell \left( \left[ a + \epsilon, b - \epsilon \right]  \right) \\
						 &=  \mu^{*}\left( \left[ a + \epsilon, b - \epsilon \right]  \right)  \\
						 &\le \mu^{*}\left( I \right)\\
						 &\le \mu^{*}\left( I \right) \\
						 &\le \mu^{*}\left( \left[ a, b \right]  \right)
						 &= b -a \\
						 &= \ell\left( I \right)
	.\end{align*}
	Hence, for every \(\epsilon > 0\), \(\ell\left( I \right)  - 2\epsilon \le \mu^{*}\left( I \right)  \le \ell \left( I \right) \), hence as \(\epsilon\) is arbitrary \( \mu^{*} \left( I \right)  = \ell \left( I \right) = b-a \). This covers all the bounded cases, hence only the unbounded case remains.\\
	If \(I\) is unbounded and \(a \in I\), then \(\left[ a, a+n \right] \subseteq J\) for all \(n \in \N\) or \(\left[ a- n, a \right] \subseteq J\) for all \(n \in \N\). In either case, by the monotonicity of the outer measure,  \( \mu^{*}\left( I \right)  \ge n\) for all \(n \in \N\), hence \( \mu^{*}\left( I \right)  = \infty = \ell \left( I \right) \). This completes the proof.
\end{proof}
Hence, we have that \( \mu^{*}\) conforms to all of our desired properties with the notable exception of countable additivity. This, of course, means \( \mu^{*}\) is not in fact a measure, so we will again modify our measure function in order to induce a countably additive measure. This construction will come next lecture and will consist of again restricting the domain to a subset of \(\mathscr{P}\left( \R \right) \), the Lebesque measurable sets, a collection which will be introduced and formalized next lecture.
