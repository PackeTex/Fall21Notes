\lecture{10}{Thu 23 Sep 2021 12:58}{}
\begin{definition}[]
	A set \(S \subseteq R\) is \textbf{measureable/Lebesque mesaurable} if for every \( A \subseteq \R\), \[
		\mu^{*}\left( A \right)  = \mu^{*}\left( A \cap S \right)  + \mu^{*}\left( A \cap S^{c} \right)
	.\] It actually suffices to show only
	\[
		\mu^{*}\left( A \right)  \ge \mu^{*}\left( A \cap S \right)  + \mu^{*}\left( A \cap S^{c} \right)
	.\]
\end{definition}
\begin{proposition}
	Every set \(S \subseteq \R\) with \( \mu^{*}\left( S \right)  = 0\) is measurable.
\end{proposition}
\begin{proof}
	For every \(A \subseteq \R\), \( \mu^{*}\left( A \cap S \right)  \le \mu^{*}\left( S \right)  = 0\). Similairly, \( \mu^{*}\left( A \cap S^{c} \right) = 0\).
\end{proof}
\begin{definition}
	A set \(S \subseteq \R\) with \( \mu^{*}\left( S \right)  = 0\) is said to have measure \(0\).
\end{definition}
\begin{lemma}
	For each \(a \in \R\), \(\left( a, \infty \right) \) is measurable.
\end{lemma}
\begin{proof}[]
	Given \(A \subseteq \R\)	 and \(\epsilon > 0\), we fine \(\{I_{n} : n \in N\} \in J\left( A \right)  \) such that \[
		\mu^{*}\left( A \right) \ge \sum_{n=1}^{\infty}\ell\left( I_{n} \right)- \epsilon
	.\]
	Since \(A \cap \left( a, \infty \right) \subseteq \bigcup_{n \in \N}\left( I_{n} \cap \left( a, \infty \right)  \right)  \) and \[A \cap \left( a, \infty \right)^{c} \subseteq \left( \bigcup_{n \in \N} \left( I_{n} \cap \left( -\infty, a \right)  \right)   \right) \cup \left( a - \epsilon, a + \epsilon \right) . \]
	It follows that \( \mu^{*}\left( A\cap \left( a, \infty \right)  \right) \le \sum_{n=1}^{\infty}\ell\left( I_{n} \cap \left( a, \infty \right)  \right) \) and \( \mu^{*}\left( A \cap \left( a, \infty \right)^{c}  \right) \le \sum_{n=1}^{\infty} \ell\left( I_{n} \cap \left( -\infty, a \right)  \right) + 2\epsilon  \).
	As \(\ell \left( I_{n} \right) = \ell \left( I_{n} \cap \left( a, \infty \right)  \right) + \ell \left( I_{n} \cap \left( -\infty, a \right)  \right)   \) as the singular point \(a\) will not change the length. Hence, \begin{align*}
		\mu^{*}\left( A \right)  &\ge \sum_{n=1}^{\infty} \ell\left( I_{n} \cap \left( a, \infty \right)  \right)  + \sum_{n=1}^{\infty} \ell\left( I_{n} \cap \left( -\infty \right) , a \right) - \epsilon \\
					 &\ge \mu^{*}\left( A \cap \left( a, \infty \right)  \right) + \mu^{*}\left( A \cap \left( a, \infty \right)^{c}  \right) - 3\epsilon
	.\end{align*}
\end{proof}
\begin{proposition}
	The collection of Lebesque measurable sets in \(\R\) is a \(\sigma\)-algebra \(\mathscr{L}\) containing all Borel sets.
\end{proposition}
\begin{proof}
	If the measurable sets form of \(\sigma\)-algebra \(\mathscr{L}\), then \(\mathscr{L}\) must contain all open and closed subsets of \(\R\), since it contains all intervals of the form \(\left( a, \infty \right) \). To show that the measurable sets form a \(\sigma\)-algebra \(\mathscr{L}\) we first note that \(\left( a, a \right)  = \O\) and the complement of each measurable set are both measurable sets. This is due to the symmetry in the definition of measurbility
	\[
		\mu^{*}\left( A \right)  \ge \mu^{*}\left( A \cap S \right) + \mu^{*}\left( A \cap S^{c} \right)
	.\]
Now, suppose \(\{S_{n} : n \in \cap\} \) is a countable collection of measurable sets. Let \(S = \bigcup_{n \in \N} S_{n}\), then we need only show \(S\) is measurable.\\
Given \(A \subseteq \R\), we define a sequence with \(A_1 = A\), \(A_{n + 1} = A \cap \left( \bigcap_{k=1}^{n} S_{k}^{c} \right) \). Hence, \(A_2 = A \cap S_1^{c}\), \(A_3 = A \cap \left( S_1 ^{c} \cap S_2^{c} \right) \). Now, note that \(A_{n+1} = A_{n} \cap S_{n}^{c}\), hence the sequence is decreasing in size. And \(A \cap S = \bigcup_{k \in \N} \left( A_{k} \cap S_{k} \right) \). We present a short proof of this claim.\\
Note that for \(x \in A \cap S\), there is a smallest positive integer \(k\) such that \(x \in S_{k}\). If \(k=1\), then \(x \in A_1 \cap S_1\), if \(k > 1\), then \(x\not\in S_{n}\) for any \( n < k\), consequently \(x \in A_{k}\) by construction. Hence, \(x \in A_{k} \cap S_{k}\), so \(A \cap S \subseteq \bigcup_{k \in \N} \left( A_{k} \cap S_{k} \right) \).\\
Now, \(\bigcup_{k \in \N} \left( A_{k} \cap S_{k} \right) \subseteq A \cap S \), as each \(A_{k} \in A\) and \(S_{k} \in S\), hence their intersection and subsequent union are also contained. Hence the equality is shown \[
	A \cap S = \bigcup_{ k \in \N}  \left( A_{k} \cap S_{k} \right)
.\]
By measurability of \(S_{n}\), we know any set \(A\) has \[ \mu^{*}\left( A_{n} \right)  = \mu^{*}\left( A_{n} \cap S_{n} \right)  + \underbrace{\mu^{*}\left( A_{n} \cap S_{n}^{c} \right)}_{A_{n + 1}} \].\\
Hence, by induction, we have \( \mu^{*}\left( A \right) = \mu^{*}\left( A_1 \right) = \sum_{k= 1}^{n}  \mu^{*}\left( A_{k} \cap  S_{k} \right) + \mu^{*}\left( A_{n+1} \right) \). Since \(A \cap \left( \bigcap_{k \in \N} S_{k}^{c}  \right) = A\cap S^{c}	\subseteq A_{n + 1} \) for any \(n\).\\
Hence, \[ \mu^{*}\left( A \right) \ge \sum_{k= 1}^{n} \mu^{*}\left( A_{i} \cap S_{i} \right) + \mu^{*}\left( A \cap S^{c} \right)   .\] Finally, as \(\bigcup_{k \in \N} \left( A_{k} \cap S_{k} \right)= A \cap S \) and since \( \mu^{*}\) is contably subadditive, we obtain \begin{align*}
\mu^{*}\left( A \right) &\ge \sum_{k=1}^{\infty} \mu^{*}\left( A_{k} \cap S_{k} \right)  + \mu^{*}\left( A \cap S^{c} \right) \\
			&\ge \mu^{*}\left( \bigcup_{ k \in \N} \left( A_{k} \cap S_{k} \right)  \right)  + \mu^{*}\left( A \cap S^{c} \right) \\
			&=  \mu^{*}\left( A \cap S \right) + \mu^{*}\left( A \cap S^{c} \right).
\end{align*}
\end{proof}
\begin{definition}[Lebesque Measure]
	The \textbf{Lebseque Measure} of a measurable set \(S \subseteq \R\)	, denoted by \( \mu^{*}\left( S \right) \) is defined by \( \mu\left( S \right)  = \mu^{*}\left(S  \right) \). The set function \( \mu: \mathscr{L} \to \left[ 0, \infty \right] \) is called the \textbf{Lebesque Measure}.
\end{definition}
\begin{theorem}
	The Lebesque measure \( \mu\) is a measure on \(\mathscr{L}\) such that
	\begin{itemize}
		\item \( \mu \left( I \right) = \ell\left( I \right) \) for every interval \(I \subseteq \R\).
			\item \( \mu\) is translation invariant.
				\item \( \mu\) is countablly additive.
	\end{itemize}

\end{theorem}
\begin{proof}
	\begin{enumerate}
		\item As \( \mu^{*}\) has the interval property, \( \mu\) trivially inherits this,
			\item Similarly, as \( \mu^{*}\) was translationaly invariant, we see \( \mu\) inherits this.
				\item Let \(\{S_{k} : k \in \N\} \) be a countable, disjoint collection of measurable sets and define \(T_{n} = \bigcup_{k = n} ^{\infty}S_{k}\) for \(n \in \N\).\\
					Since, \(T_{n  + 1} = T_{n} \cap S_{n}^{c}\) we have \[
					\mu \left( T_{n} \right) = \mu \left( T_{n} \cap S_{n} \right) + \mu \left( \underbrace{T_{n} + S_{n}^{c}}_{= T_{n + 1}}	 \right)
				\] by measurability of \(S_{n}\). \\
				Consequently, \( \mu \left( T_1 \right) = \sum_{k=1}^{n} \mu \left( \underbrace{T_{k} \cap S_{k}}_{= S_{n}} \right) + \mu \left( T_{n + 1} \right) \ge \sum_{_{k}=1}^{n} \mu \left( S_{k} \right)  \) for every \( n \in \N\).
				Thus \(T_1 = \bigcup_{k \in \N} S_{k}\) gives \( \mu\left( \bigcup_{k \in \N} S_{k} \right)  \ge \sum_{k= 1}^{n} \mu\left( S_{k} \right) \). And as we already know the inequality goes in the other direction by subadditivity of \( \mu^{*}\), we see equality holds.
	\end{enumerate}
\end{proof}
\begin{corollary}
	Every countable set of real numbers is measurable with measure \(0\).
\end{corollary}
\begin{proof}
	Let \(C\) be our countable sets and note that \(C = \bigcup_{k \in \N} \{x_{k}\} \) with \(x_{k} \neq x_{m}\) for \(k \neq m\). Then, we see that \[
		\mu \left( \bigcup_{k \in \N} \{x_{k}\}  \right)  = \sum_{k=1}^{\infty} \mu \left( \{x_{k}\}  \right)  = 0
	.\]
\end{proof}
\begin{theorem}[Properties of Lebesque Measure]
	Let \(S \subseteq \R\), the following are equivalent
	\begin{enumerate}
		\item \(S\) is measurable.
		\item For each \(\epsilon > 0\), there is an open set \(O\) and a closed set \(C\) such that \(C \subseteq S \subseteq O\) and \( \mu \left( O \setminus C \right)  < \epsilon\).
		\item There is a \(G_{\delta}\) set \(G\) and a \(F_{\sigma}\) set \(F\) such that \(F \subseteq S \subseteq G\) and \( \mu \left( G \setminus F \right)  = 0\).
		\item For each \(\epsilon > 0\), there are measurable sets \(G\) and \(F\) such that \(F \subseteq S \subseteq G\) and \( \mu \left( G \setminus F \right) < \epsilon\).
	\end{enumerate}
	We will prove this result next time, though it is completely trivial that \(3 \ge 4\), so we will primarily focus on proving \(1 \implies 2\) and \(4 \implies 1\).
\end{theorem}
