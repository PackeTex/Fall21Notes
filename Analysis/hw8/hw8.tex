\documentclass[a4paper]{article}
% Some basic packages
\usepackage[utf8]{inputenc}
\usepackage[T1]{fontenc}
\usepackage{textcomp}
\usepackage{url}
\usepackage{graphicx}
\usepackage{float}
\usepackage{booktabs}
\usepackage{enumitem}

\pdfminorversion=7

% Don't indent paragraphs, leave some space between them
\usepackage{parskip}

% Hide page number when page is empty
\usepackage{emptypage}
\usepackage{subcaption}
\usepackage{multicol}
\usepackage{xcolor}

% Other font I sometimes use.
% \usepackage{cmbright}

% Math stuff
\usepackage{amsmath, amsfonts, mathtools, amsthm, amssymb}
% Fancy script capitals
\usepackage{mathrsfs}
\usepackage{cancel}
% Bold math
\usepackage{bm}
% Some shortcuts
\newcommand\N{\ensuremath{\mathbb{N}}}
\newcommand\R{\ensuremath{\mathbb{R}}}
\newcommand\Z{\ensuremath{\mathbb{Z}}}
\renewcommand\O{\ensuremath{\varnothing}}
\newcommand\Q{\ensuremath{\mathbb{Q}}}
\newcommand\C{\ensuremath{\mathbb{C}}}
% Easily typeset systems of equations (French package)

% Put x \to \infty below \lim
\let\svlim\lim\def\lim{\svlim\limits}

%Make implies and impliedby shorter
\let\implies\Rightarrow
\let\impliedby\Leftarrow
\let\iff\Leftrightarrow
\let\epsilon\varepsilon
\let\nothing\varnothing

% Add \contra symbol to denote contradiction
\usepackage{stmaryrd} % for \lightning
\newcommand\contra{\scalebox{1.5}{$\lightning$}}

 \let\phi\varphi

% Command for short corrections
% Usage: 1+1=\correct{3}{2}

\definecolor{correct}{HTML}{009900}
\newcommand\correct[2]{\ensuremath{\:}{\color{red}{#1}}\ensuremath{\to }{\color{correct}{#2}}\ensuremath{\:}}
\newcommand\green[1]{{\color{correct}{#1}}}

% horizontal rule
\newcommand\hr{
    \noindent\rule[0.5ex]{\linewidth}{0.5pt}
}

% hide parts
\newcommand\hide[1]{}

% Environments
\makeatother
% For box around Definition, Theorem, \ldots
\usepackage{mdframed}
\mdfsetup{skipabove=1em,skipbelow=0em}
\theoremstyle{definition}
\newmdtheoremenv[nobreak=true]{definition}{Definition}
\newmdtheoremenv[nobreak=true]{eg}{Example}
\newmdtheoremenv[nobreak=true]{corollary}{Corollary}
\newmdtheoremenv[nobreak=true]{lemma}{Lemma}[section]
\newmdtheoremenv[nobreak=true]{proposition}{Proposition}
\newmdtheoremenv[nobreak=true]{theorem}{Theorem}[section]
\newmdtheoremenv[nobreak=true]{law}{Law}
\newmdtheoremenv[nobreak=true]{postulate}{Postulate}
\newmdtheoremenv{conclusion}{Conclusion}
\newmdtheoremenv{bonus}{Bonus}
\newmdtheoremenv{presumption}{Presumption}
\newtheorem*{recall}{Recall}
\newtheorem*{previouslyseen}{As Previously Seen}
\newtheorem*{interlude}{Interlude}
\newtheorem*{notation}{Notation}
\newtheorem*{observation}{Observation}
\newtheorem*{exercise}{Exercise}
\newtheorem*{comment}{Comment}
\newtheorem*{practice}{Practice}
\newtheorem*{remark}{Remark}
\newtheorem*{problem}{Problem}
\newtheorem*{solution}{Solution}
\newtheorem*{terminology}{Terminology}
\newtheorem*{application}{Application}
\newtheorem*{instance}{Instance}
\newtheorem*{question}{Question}
\newtheorem*{intuition}{Intuition}
\newtheorem*{property}{Property}
\newtheorem*{example}{Example}
\numberwithin{equation}{section}
\numberwithin{definition}{section}
\numberwithin{proposition}{section}

% End example and intermezzo environments with a small diamond (just like proof
% environments end with a small square)
\usepackage{etoolbox}
\AtEndEnvironment{example}{\null\hfill$\diamond$}%
\AtEndEnvironment{interlude}{\null\hfill$\diamond$}%

\AtEndEnvironment{solution}{\null\hfill$\blacksquare$}%
% Fix some spacing
% http://tex.stackexchange.com/questions/22119/how-can-i-change-the-spacing-before-theorems-with-amsthm
\makeatletter
\def\thm@space@setup{%
  \thm@preskip=\parskip \thm@postskip=0pt
}


% \lecture starts a new lecture (les in dutch)
%
% Usage:
% \lecture{1}{di 12 feb 2019 16:00}{Inleiding}
%
% This adds a section heading with the number / title of the lecture and a
% margin paragraph with the date.

% I use \dateparts here to hide the year (2019). This way, I can easily parse
% the date of each lecture unambiguously while still having a human-friendly
% short format printed to the pdf.

\usepackage{xifthen}
\def\testdateparts#1{\dateparts#1\relax}
\def\dateparts#1 #2 #3 #4 #5\relax{
    \marginpar{\small\textsf{\mbox{#1 #2 #3 #5}}}
}

\def\@lecture{}%
\newcommand{\lecture}[3]{
    \ifthenelse{\isempty{#3}}{%
        \def\@lecture{Lecture #1}%
    }{%
        \def\@lecture{Lecture #1: #3}%
    }%
    \subsection*{\@lecture}
    \marginpar{\small\textsf{\mbox{#2}}}
}



% These are the fancy headers
\usepackage{fancyhdr}
\pagestyle{fancy}

% LE: left even
% RO: right odd
% CE, CO: center even, center odd
% My name for when I print my lecture notes to use for an open book exam.
% \fancyhead[LE,RO]{Gilles Castel}

\fancyhead[RO,LE]{\@lecture} % Right odd,  Left even
\fancyhead[RE,LO]{}          % Right even, Left odd

\fancyfoot[RO,LE]{\thepage}  % Right odd,  Left even
\fancyfoot[RE,LO]{}          % Right even, Left odd
\fancyfoot[C]{\leftmark}     % Center

\makeatother




% Todonotes and inline notes in fancy boxes
\usepackage{todonotes}
\usepackage{tcolorbox}

% Make boxes breakable
\tcbuselibrary{breakable}

% Verbetering is correction in Dutch
% Usage:
% \begin{verbetering}
%     Lorem ipsum dolor sit amet, consetetur sadipscing elitr, sed diam nonumy eirmod
%     tempor invidunt ut labore et dolore magna aliquyam erat, sed diam voluptua. At
%     vero eos et accusam et justo duo dolores et ea rebum. Stet clita kasd gubergren,
%     no sea takimata sanctus est Lorem ipsum dolor sit amet.
% \end{verbetering}
\newenvironment{correction}{\begin{tcolorbox}[
    arc=0mm,
    colback=white,
    colframe=green!60!black,
    title=Opmerking,
    fonttitle=\sffamily,
    breakable
]}{\end{tcolorbox}}

% Noot is note in Dutch. Same as 'verbetering' but color of box is different
\newenvironment{note}[1]{\begin{tcolorbox}[
    arc=0mm,
    colback=white,
    colframe=white!60!black,
    title=#1,
    fonttitle=\sffamily,
    breakable
]}{\end{tcolorbox}}


% Figure support as explained in my blog post.
\usepackage{import}
\usepackage{xifthen}
\usepackage{pdfpages}
\usepackage{transparent}
\newcommand{\incfig}[2][1]{%
    \def\svgwidth{#1\columnwidth}
    \import{./figures/}{#2.pdf_tex}
}

% Fix some stuff
% %http://tex.stackexchange.com/questions/76273/multiple-pdfs-with-page-group-included-in-a-single-page-warning
\pdfsuppresswarningpagegroup=1
\binoppenalty=9999
\relpenalty=9999

% My name
\author{Thomas Fleming}

\usepackage{pdfpages}
\title{Analysis I: Homework 8 and 9}
\date{Fri 10 Sep 2021 12:58}
\DeclareMathOperator{\SRG}{SRG}
\DeclareMathOperator{\cut}{Cut}
\DeclareMathOperator{\GF}{GF}
\DeclareMathOperator{\V}{V}
\DeclareMathOperator{\E}{E}
\DeclareMathOperator{\edg}{e}
\DeclareMathOperator{\vtx}{v}
\DeclareMathOperator{\diam}{diam}

\DeclareMathOperator{\tr}{tr}
\DeclareMathOperator{\A}{A}

\DeclareMathOperator{\Adj}{Adj}
\DeclareMathOperator{\mcd}{mcd}

\begin{document}
\maketitle
\begin{problem}[36]
	Our function will be \(\phi\), the cantor-lebesque function. We have already shown it to be continuous and increasing with \(\phi\left( 1 \right) = 1, \phi\left( 0 \right) = 0\). Moreover, letting \(C\) be the cantor set, we see \(\left[ 0, 1 \right] \setminus C \coloneqq C ^{c}\) is open in \(\left[ 0, 1 \right] \) so for all \(x \in C ^{c} \), there is an \(\epsilon > 0\) so that \(\left( x-\epsilon, x+\epsilon \right) \subseteq C ^{c}\) . Then, since for all intervals \(I \)  in the \(\left[ 0, 1 \right] \) complement of the cantor set, we find  \(I \subseteq J_{n, k}\) for some \(n, k \in \N\)  , we have \(\xi(I) = \{\frac{n}{2^{k}}\} \), so \[\overline{D}\left( \phi\left( x \right)  \right)  = \lim_{r \to 0}\sup \{ \frac{ \phi\left( x+h \right) -  \phi\left( x \right) }{h} : 0 < \left| h \right|  < r   \} = \lim_{r \to 0}\sup \{ \frac{0}{h} : 0 < \left| h \right|  <  r\} = 0  .\] Similarly, we find \( \underline{D}  \left( \phi\left( x \right)  \right) = 0\). Hence, \( \phi\) is differentiable at \(x\) and since \( \phi^{\prime} = 0\) almost everywhere, yet \( \phi\) is not constant by the initial claim, we find \( \phi\) is not absolutely continuous.
\end{problem}
\newpage
\begin{problem}[38]
	First, note that \(\phi: \R \to \overline{\R}, \ x\mapsto \sqrt{1 + x^2} \) is convex and since \(h\) is integrable, we see it is finite almost everywhere. Hence, discarding the points for which \(h = \infty\), we see jensens inequality yields \[
	\sqrt{1 + A^2}  \le \int_{\left[ 0, 1 \right] } \sqrt{1 + h^2}
	.\]
	For the second inequality, note that since \(h\) is nonnegative and \(\sqrt{.} \) is an increasing function we have \[
	\int_{\left[ 0, 1 \right] } \sqrt{1 + h^2} \le \int_{\left[ 0, 1 \right] } \sqrt{1 + 2h + h^2} \le \int_{\left[ 0, 1 \right] }1 + h = 1 + A
	.\]
\end{problem}
\newpage

\begin{problem}[39]
	\begin{itemize}
		\item Assume \(\left( f_{n} \right) \) does not converge to \(f\) in measure. That is, there is an \(\epsilon > 0\) so that for all \(N \in \N\) \[
		m\left( \{x \in \R : \left| f_{n_{N}}\left( x \right) - f\left( x \right)  \right| > \epsilon\}  \right) > \epsilon
		\] for some \(n_{N} \ge N\). Denote this set \(A_{N}\).  Then, we see \[
		\int \left| f_{n_{N}} - f \right| \ge \int_{A_{N}} \left| f_{n_{N}} - f \right|   \ge \int \epsilon \chi_{A_{N}} = \epsilon m\left( A_{N} \right) \ge \epsilon^2
		.\]
		That is, for some \(\epsilon ^{\prime} = \epsilon ^2 > 0\),  and all \(N \in \N\) we find an \(n_N \ge N\), so that \(\int \left| f_{n} - f \right| \ge \epsilon^{\prime}\), so \(f_{n}\) does not converge to \(f\) in mean.
	\item First, note that if \(x = 0\) or \(1\), then \(f_{n}\left( x \right)  = x\) for all \(n \in \N\). Then, if \(x \in \left( 0, 1 \right) \), for all \(\epsilon > 0\), there is an \(N \in \N\) so that \(x^{n} < \)
	\end{itemize}
\end{problem}
\newpage
\begin{problem}[40]
	\begin{itemize}
		\item The first function will be \(f_{n} = \chi_{\left( n, \infty \right) }\). We note that for all \(x \), \(x \not\in \left( n, \infty \right) \) for all \(n \ge \left\lceil x \right\rceil \), so \(\left( f_{n} \right) \) converges point wise. On the other hand for \(\epsilon = \frac{1}{2}\), we see \(m\left( \{x \in \R : \left| f_{n}\left( x \right) - f\left( x \right)  \right| > \frac{1}{2}\}  \right)  = m\left( \left( n, \infty \right)  \right) = \infty > \epsilon\), so \(\left( f_{n} \right) \) does not converge in measure (hence not in mean).
		\item For the second function define the following sequence of intervals. \(A_{1} = \left[ 0, 1 \right] \), \(A_{2^{k}} = \left[ 0, \frac{1}{2^{k}} \right] \) and \(A_{2^{k} + c} = \left[ \frac{c}{2^{k}}, \frac{c+1}{2^{k}} \right] \) for \(c < 2^{k}\). This essentially enumerates all partitions with endpoints being a rational with denominators powers of \(2\) and consecutive numerators. Since the collection \(\{A_{2^{k} + c} : 0 \le c < 2^{k}\} \) covers \(\left[ 0, 1 \right] \) for every \(k \in \N\), we see  for all \(N \in \N\) and \(x \in \left[ 0, 1 \right] \) , the function \(f_{n} = \chi_{A_{n}}\) will have \(f_{n}\left( x \right) = 1\) for some (infinitely many) \(n \ge N\), so it will not converge to \(0\) pointwise. On the other hand, we see \(\left| f_{n} - 0 \right|= f_{n} = \chi_{A_{n}}\), so \(\int \left| f_{n} - 0 \right| = m\left( A_{n} \right) \). Moreover, for all \(k \in \N\) we find an \(N = \left\lfloor \log_{2}\left( n \right)  \right\rfloor\) so that \(m\left( A_{n} \right) < \frac{1}{2^{k}} \) for all \(n \ge N\), so \(f_{n}\) does in fact converge in mean and in measure.
		\item For the third function we adopt the same intervals from part \(2\), but we instead define the function \(f_{n} = 2^{n} \chi_{A_{n}}\). Recalling that \(m\left( A_{n} \right)\ge \frac{1}{2^{n}} \) for all \(n\), we see \(\int \left| f_{n} - 0 \right|  = \int 2^{n} \chi_{A_{n}} = 2^{n} m\left( A_{n} \right) \ge \frac{2^{n}}{2^{n}} = 1\) for all \(n \in \N\) . Hence for all \(\epsilon < 1\) we find convergence in mean to fail. Moreover, \(f_{n}\) still fails to converge pointwise. Lastly, recall for all \(k \in \N\) there is a \(N \in \N\) so that \(m\left( A_{n} \right)\le \frac{1}{2^{k}} \) for all \(n \ge N\), hence for all \(\epsilon > \frac{1}{2^{k}}\)  we find the convergence in measure criterion holds. Since there is a \(k \in \N\) so that \(0 < \frac{1}{2^{k}} < \epsilon\) for all \(\epsilon > 0\), we see convergence in measure does in fact hold true.
	\end{itemize}
\end{problem}
\end{document}
