\lecture{10}{Wed 15 Sep 2021 10:20}{Hadamard Matrices (3)}
\begin{recall}
	The tensor product of matrices \(A, B\) is \(A \otimes B\) and this preserves hadamardness.
\end{recall}
\begin{example}
	\[
	\underbrace{	\begin{bmatrix} +&+\\
		+&-\end{bmatrix}}_{=H}  \otimes \begin{bmatrix} +&+\\
			+&-\end{bmatrix} = \begin{bmatrix} +&+&+&+\\
		+&-&+&-\\
	+&+&-&-\\
+&-&-&+\end{bmatrix} = H \otimes H
	.\]
Furthermore, \(H \otimes H \otimes H\) will be an \(8 \times 8\) hadamard matrix. And the arbitrary \(\bigotimes_{i=1}^{n} H\) yields a hadamard matrix of order \(2^{n}\).
\end{example}
A natural question arises, what are the singular values of an arbitrary hadamard \(H\)?
\begin{recall}
	Singular values are the square roots of the eigenvalues of \(A A^{*}\).
\end{recall}
Other definitions also arise, for example the largest singular value of \(A\), denoted \(\sigma_1\) is equal to the operator norm on \(A\). Similarly, we can change the matrix slightly to remove singular value \(\sigma_1\) and this yields \(\sigma_2\) is the operator norm on the modified \(\hat{A}\).\\

For now, we return to the original definition, and we note that as \(H H^{*} = n I\), we have eigenvalues \(\lambda_1, \lambda_2, \ldots, \lambda_{n}\) with \(\lambda_{i} = n\) and corresponding eigenvactor \(e_{i} = \begin{pmatrix} 0 \\ \vdots\\ \underbrace{1}_{\text{position \(i\)}} \\ \vdots \\  0 \end{pmatrix}\).
Hence the singular values of \(A\) are all \(\sqrt{n} \).
\begin{proposition}
	Let \(H = \left[ h_{i, j} \right] \) with \(\left| h _{i, j} \right|  = 1\) for all \(i, j\). Then, the following are equivalent
	\begin{itemize}
		\item \(HH^{*} = nI\)
		\item All singular values are equal to \(\sqrt{n} \)
			\item For singular values \(\sigma_{i}\), \(1\le i \le n\), the sum \(\sum_{i=1}^{n} \sigma_{i} = n\sqrt{n} \).
	\end{itemize}
\end{proposition}
\begin{definition}[Nuclear Norm]
	For a matrix \(A\), we define the \textbf{nuclear norm or trace norm} to be \(\|A\|_{*} = \sum_{i = 1}^{n} \sigma_{i} \left( A \right) \).
\end{definition}
\begin{remark}
	If \(A\) is \(n \times n\) with \(\left| a_{i, j} \right|  \le 1\) then \(\|A\|_{*} \le n\sqrt{n} \). Furthermore, equality holds if and only if \(A\) is hadamard.
\end{remark}
Now, let \(A\) be \(m \times n\) with \(m \le n\). Then, \(\|A\|_{*} \le m\sqrt{n} \). Equality holds if and only if \(A\) is a \textbf{partial hadamard matrix} meaining \(A A^{*} = n I_{m}\).
\begin{definition}[Regular Matrix]
	For a matrix \(A\) we say \(A\) is \textbf{regular} if all row sums are equal.
\end{definition}
We examine the properties of a regular hadamard matrix.\\
It is clear, as we may switch rows and columns and multiply by \(\pm 1\) for each row, that these row sums are fragile, and occasionally we may even induce a regular hadamard matrix from a nonregular one this way.\\
\begin{example}
	\(\begin{bmatrix} +&+&+&-\\
	+&+&-&+\\
+&-&+&+\\
-&+&+&+\end{bmatrix} \) is a regular hadamard matrix induced by the \(4 \times 4\) hadamard from earlier.
\end{example}
\begin{remark}
	A regular matrix need not be symmetric. For example \(\begin{bmatrix} +&+&+&-\\
	+&-&+&+\\
-&+&+&+\\
+&+&-&+\end{bmatrix} \)  is regular and nonsymmetric.
\end{remark}
Note that a real symmetric hadamard matrix has real eigenvalues.
\begin{proposition}
	Suppose \(H\) is a \(n \times n\) symmetric and regular (row sum \(d\)). Then, \(n = d^2\).
\end{proposition}
\begin{proof}
	Let \(\lambda_1, \lambda_2, \ldots, \lambda_{n}\) be the eigenvalues of \(H\) and note that \(\left| \lambda_{i} \right|  = \sqrt{n} \) for \(1\le i \le n\) as the eigenvalues of a real symmetric matrix are precisely the singular values.\\
	Next, we note that there is atleast one \(\lambda_{i} = \sqrt{n} \) and one \(\lambda_{j} = - \sqrt{n} \). Otherwise, suppose WLOG all \(\lambda_{i} = \sqrt{n} \) , then \(\sum_{i= 1}^{n} \lambda_{i} = n\sqrt{n}  = \tr \left( H \right) \), but the trace can be atmost \(n\) by an earlier theorem. Hence, \(\lambda_1 = \sqrt{n} \) and \(\lambda_{n} = -\sqrt{n} \). Then, note that \(H j = dj\) for \(j = \begin{pmatrix} 1 \\ \vdots\\ 1 \end{pmatrix}\). Hence, \(d\) is an eigenvalue with \(j\) as its eigenvector. Hence \(d = -\sqrt{n} \) or \(d = \sqrt{n} \). Hence either case yields \(n = d^2\).
\end{proof}
\begin{definition}[Constant Diagonal]
	A hadamard matrix \(H\) is said to have a \textbf{constant diagonal} if \(h_{1, 1} = h_{2, 2} = \ldots = h_{n, n}\).
\end{definition}
This property can always be ensured for a hadamard matrix with just elementary transformations. Furthermore, if \(H\) is a \(n \times n\) constant diagonal hadamard matrix with \(\delta = h_{1, 1}\). Then, \(\delta H\) has a constant diagonal of \(1\) and we define \(A = \frac{1}{2} \left( J_{n} - \delta H \right) \), hence the diagonal of \(A\) is constant \(0\). Next, note that \(\delta H\) is a hadamard matrix and for an element \(h_{i, j} = 1\), we see \(\delta h_{i, j} = \delta\). Similairly if \(h_{i, j} = -1\) we have \(\delta h_{i, j} = -\delta\). Hence the entries of \(A\) are \(a_{i, j} = 0\) if \(\delta h_{i, j} = 1\) and \(a_{i, j} = 1\) if \(\delta h_{i, j} = -1\). So, this matrix has all entries \(0\) and \(1\), something we call a \textbf{digraph matrix}. Furthermore, if \(H\) is regular, the graph induced by \(A\) is a strongly regular graph.
