\lecture{2}{Wed 25 Aug 2021 10:16}{Strongly Regular Graphs (2)}
\begin{proposition}
	Let $G$ be $\SRG \left( n, k, \lambda, mu \right) $. Then, $\overline{G}$ is $\SRG \left( n, n-k-1, n-2k+\mu -2, n-2k + \lambda \right) $
\end{proposition}
\begin{proof}
	Let $G$ be $\SRG \left( n, k, \lambda, \mu \right) $. It is clear $\overline{G}$ will have $n$ vertices. Furthermore, we have already shown a $k$-regular graph is $\left( n-k-1 \right) $-regular. We will verify the nonadjacent co-regularity condition and the adjacent co-regularity will be left as an exercise. Let $u, v$ be two adjacent vertices in $G$. Then, they have codegree $\lambda$. We see $u, v$ are nonadjacent in $\overline{G}$, hence we must who they have codegree $n-2k-\lambda$. Let us divide the remaining vertices into $4$ groups, the neighbors of only $u$, neighbors of only $v$, neighbors of both, and neighbors of neither. Let us define the set which is adjacent to both (in $G$ ) to be $A$ and the set which is adjacent to neither (in $G$ ) to be $B$. In more precise terms, we have $B = V\left( G \right)  \setminus \left( N\left( u \right)  \cup N\left( v \right)  \right) $. Hence, we see \begin{align*}
	\left| A \right| &= n - \left| N\left( u \right)  \cup N\left( v \right)  \right|\\
			 &= n - \left| N\left( u \right)  \right|  - \left| N\left( v \right)  \right|  + \left| N\left( u \right)  \cap N\left( v \right)  \right| \\
			 &= n -2k + \lambda.
\end{align*}
\begin{figure}[ht]
    \centering
    \incfig{vtxpart}
    \caption{The partitioning of \(G\) into \(4\) sets with \(A\) and \(B\) labeled.}
    \label{fig:vtxpart}
\end{figure}
As $A$ is precisely the set of vertices which have both $u$ or $v$ as common neighbors in $G$, it is the set which has neither $u$ nor $v$ as common neighbors in $\overline{G}$, hence as $u, v$ were arbitrary we see this completes this portion of the proof.
\end{proof}
\begin{problem}
	Complete the proof of the third parameter.
\end{problem}
\begin{proof}

\end{proof}
\begin{remark} It is a general strategy in graph theoretic proofs to take two vertices, $u$ and $v$ and split the remaining vertices into $4$ sets, those adjacent only to $u$, those adjacent only to $v$, those adjacent to both $u$ and $v$, and those adjacent to neither $u$ or $v$.
\end{remark}
\begin{proposition}
	If $G$ is connected and SRG, then $\diam \left( G \right) = 2$.
\end{proposition}
\begin{proof}
	As $G$ is not a complete graph (by SRG assumption), we have that $\diam \left( G \right)  \ge 2$, hence we must only show that $\diam \left( G \right) < 3$. WLOG, assume $u, v$ are two points of distance $3$ (if the diameter is  $>3$ we may always choose 2 points at distance $3$ on the longest path). Now, label the internal vertices of the path from $u$ to $v$ to be $x$ and $y$, that is, $u, x, y, v$ forms a path. We see $u, v$ are nonadjacent and $\hat{d} \left( u, v \right) = 0$, but  $u, y$ are also nonadjacent and $\hat{d} \left( u, y \right) = 1$. $ \lightning$.
\end{proof}
\begin{propostion}
	If $G$ is $r$-regular, of order $n$, and $\diam \left( G \right) = 2$ then $n \le r^2 + 1$.
\end{propostion}
\begin{proof}
	First, let us partition the graph into $3$ parts. The first part will contain $u$, the second part will be $N\left( u \right) $, labeled $A$, and the third part will be the remaining vertices, $V\left( G \right) \setminus \left( N\left( u \right) \cup \{u\}  \right) $, labeled $B$. We see that every point in $B$ will be connected to a point in $A$, as it must have a path of length $2$ to $u$.  Now, let us examine the bipartite graph, $G\left[ A, B \right] $ generated by $A$ and $B$. We see $\left| A \right| = r$ by $r$-regularity and $\left| B \right| = n - r - l$ as this is how many remain. Next, we will double count the number of edges in this graph, denoted by $e\left( G\left[ A, B \right]  \right) = e\left( A, B \right) $. First, let us assume the maximal case, that being that each of the $r$ vertices in $A$ has no internal connections within $A$ and $r-1$ neighbors in $B$. Then, we see the $e\left( A, B \right) \le \max \left( e\left( A, B \right)  \right) = r\left( r-1 \right) $. However, we know $B$ has precisely $n-r-1$ vertices, and as each vertex has degree $1$ within the induced bipartite graph (by construction), then we have $e\left( A, B \right) = n - r - 1$. Hence, this yields the inequality $n - r - 1 \le r \left( r-1 \right) $. Rearranging yields $n \le r^2 + 1$.
\end{proof}

\begin{remark}
	In the case of equality, we have precisely this maximal case, that being each neighbor of $u$ has no common neighbors with $u$ and each nonadjacent vertex to  $u$ has precisely $1$ common neighbor, so $G$ is $\SRG \left( r^2 + 1, r, 0, 1 \right) $.
\end{remark}
\begin{example}
	Graphs with this quality are rare, with only $3$ nontrivial examples known and $1$ theorized but not yet proven, these are:\begin{enumerate*}
		\item $C_5$ ($\SRG\left( 5, 2, 0, 1 \right) $)
		\item The Peterson Graph ($\SRG \left( 10, 3, 0, 1 \right) $ )
		\item The Hoffman-Singelton Graph ($\SRG\left( 50, 7, 0, 1 \right) $ )
		\item The theorized but yet-unproven $\SRG\left( 3250, 57, 0, 1 \right) $.
	\end{enumerate*}

\begin{figure}[ht]
    \centering
    \incfig{c5.svg.2021_08_25_13_14_48.0}
    \caption{$C_5$ (left) and the Peterson Graph (right)}
    \label{fig:c5.svg.2021_08_25_13_14_48.}
\end{figure}
\end{example}
These graphs are essentially the minimally connected $r$-regular graphs for each $r$. All  SRG's have a minimum degree of $\sqrt{n}$ where $n$ is the order.
