\lecture{34}{Fri 19 Nov 2021 10:25}{}
\begin{notation}
	We will begin denoting a matrix of size \(n \times m\) with \(n, m\) being the number of indices in the rows, columns respectively as an \(R \times C\) matrix with set \(R\) of row indices and \(C\) of column indices. It is of note that this definition allows us to consider \(R, C\) to be unordered and hence we can imagine them in any convenient order we want.
\end{notation}
\begin{definition}
	Given an \(R \times C\) matrix , then, for subsets \(S \subseteq R, T \subseteq C, d \in R^{\prime}\), we denote the new \(R \times C\)  matrix \[
	\cut\left( S, T, d \right)  = \left( c_{ij} \right); c_{ij} =  \left \{
		\begin{array}{11}
			d, & \quad i \in S, j \in T \\
			0, & \quad i\not\in S \text{ or } j \not\in T
		\end{array}
		\right.
	\]
	We see this matrix is simply a scaled copy of \(J_{\left| S \right| , \left| T \right| }\) embedded in the zero matrix of size \(R \times C\).
\end{definition}
First, we examine an \(\epsilon\), regular pair \(\left( R, C \right) \) of density \(d = d\left( R, C \right) \). Denote \(A\) to be the biadjacency matrix \(A(R, C)\). Applying \(\epsilon\)-regularity yields the following result,
\begin{proposition}
	\(A\left( R, C \right)  = dJ_{\left| R \right| , \left| C \right| } + W\) for some sufficiently exceptional matrix having \(\| W \|_{\square} \le \epsilon \left| R \right| \left| C \right| \) if and only if \(\left( R, C \right) \) is an \(\epsilon\)-regular pair.
	\end{proposition}
	\begin{proof}
		First the forward implication. Then, denote \(B = A - dJ_{\left| R \right| , \left| C \right| }\). Then, \(\left| b_{ij} \right| \le 1 \) for all \(i,j\). Moreover, \(b_{ij}= \left \{
			\begin{array}{11}
				-d, & \quad a_{ij} = 0 \\
				1-d, & \quad a_{ij} = 1
			\end{array}
			\right.\) .\\
Then, suppose \(S \subseteq R\), \(T \subseteq C\). If \(\left| S \right| \le \epsilon \left| R \right| \) or \(\left| T \right| \le \epsilon \left| C \right| \), then \[
\left| \sum_{S, T}^{} b_{ij} \right|  \le \left| S \right| \left| T \right| \le \epsilon \left| R \right| \left| C \right|
.\]
In this case \(\left( R, C \right) \) is \(\epsilon\)-regular.\\
Otherwise, if \(\left| S \right| > \epsilon \left| R \right| \) and \(\left| T \right| > \epsilon \left| C \right|  \), then \(\left| d\left( S, T \right)  - d\right| < \epsilon\). Expanding terms yields
\begin{align*}
	\left| d\left( S, T \right)   - d\right| &= \left| \frac{e\left( S, T \right) }{\left| S \right| \left| T \right| } - d \right|  \\
	&= \left| e\left( S, T \right) - d \left| S \right| \left| T \right|  \right|  \\
	&< \epsilon \left| S \right| \left| T \right| \\
	&< \epsilon \left| R \right| \left| C \right|
.\end{align*}
Then, note that \(e\left( S, T  \right) - d \left| S \right| \left| T \right| = \sum_{i \in S, j \in T}^{} b_{ij}  \) and the \(\epsilon\)-regularity immediately follows.
	\end{proof}
	Now, we generalize this concept. Suppose \(A\) is an \(R \times C\) matrix. Then, we wish to construct \[
	A = D^{\left( 1 \right) } + \ldots + D^{\left( s \right) } + w
	\] for some \(D^{\left( t \right) } = \cut\left( R_{t}, C_{t}, d_{t} \right) \) for sets \(R_{t}, C_{t}\) and densities \(d_{t}\) and an exceptional set \(W\) with the following conditions holding,
	\begin{itemize}
		\item \(S\) is bounded,
		\item \(\left| d_{t} \right| \) is bounded,
		\item and \(\| W \|_{\square} \) is small.
	\end{itemize}
	More precisely,
	\begin{proposition}[Weak-Regularity Lemma]
		There are real \(c_1 > 0\) , \(c_2 > 0\) so that for every \(\epsilon \in \left( 0, 1 \right) \) with \(A\) being an \(R \times C\) matrix having \(\|A\|_{\infty} \le 1\) we find \[
		A = D^{\left( 1 \right) } + \ldots + D^{\left( s \right) } + w
		\] having
		\begin{itemize}
			\item \(\| W \|_{\square} \le \epsilon \left| R \right|\left| C \right|   \) ,
			\item \(S < \frac{c_1}{\epsilon^2}\),
			\item \(\sup \{ d_{t} :1 \le t\le s  \} \le 2 \).
		\end{itemize}
	\end{proposition}
