\lecture{31}{Fri 12 Nov 2021 16:22}{Blowup Lemma}
Recall an \(\epsilon\)-regular pair \(\left( A, B \right) \) admits an \(\left( \epsilon, \delta \right) \)-super-regular pair \(\left( A^{\prime}, B^{\prime} \right) \) with \(A ^{\prime} \subseteq A\), \(B^{\prime} \subseteq B\). \\
Now, recall that for some \(\epsilon > 0\), if \(n > \epsilon^{-4}\), the the bipartite double of \(P_{n}\) is \(\epsilon\)-regular. We construct a super-regular subpair from this bipartite double, denoted \(B_{n}\) with partite sets \(A, B\). Applying the expander mixing lemma to two subsets \(X \subseteq A\) , \(Y \subseteq B\)  with \(\left| X \right| > \epsilon n\) \(\left| Y \right| > \epsilon n\), we find \[
	\left( \frac{1}{2}- \epsilon \right) \left| X \right| \left| Y \right| < e\left( X, Y \right)  < \left( \frac{1}{2}+ \epsilon \right) \left| X \right| \left| Y \right|
.\] Then, inducing four subsets each of size \(\sim \frac{n}{2}\), denoted \(A_1, A_2, B_1, B_2\) of \(A\) and \(B\) respectively and completing the subgraphs \(\left( A_1, B_1 \right) \) and \(\left( A_2, B_2 \right) \) we see \(d\left( A_1, B_1 \right) = d\left( A_2, B_2 \right) = 1\) and \(d\left( A_1, B_2 \right) \simeq d\left( A_2, B_1 \right) \simeq \frac{1}{2}\). Collecting the densities, we find \(d\left( A^{\prime}, B^{\prime} \right) = \frac{3}{4}\) where \(A^{\prime}, B^{\prime}\) denote the sets \(A, B\) with the extra edges added between \(A_1, B_1\) and \(A_2, B_2\). From this, we can compute the new graph to be \(\left( \epsilon, \frac{1}{2+ \kappa} \right) \)-super regular for \(\kappa > 0\).
\begin{recall}
	We can obtain the blowup of a graph \(G\)  on vertices \(\{v_1, v_2, \ldots, v_{r}\} \) by replacing each vertex with a set \(V_1, V_2, \ldots, V_{r}\) where each \(V_{i}\) is of equal cardinality. We construct the edges such that if \(v_{i} \sim v_{j}\), then \(\left( V_{i}, V_{j} \right) \) is complete otherwise \(\left( V_{i}, V_{j} \right) \) is disconnected. Moreover if \(\left| V_1 \right|  = \ldots = \left| V_{r} \right| = t\), then the blowup of this graph is \(G \otimes J_{t}\).
\end{recall}
\begin{definition}[Generalized Blowup]
	Let \(R\) be a graph with \(V\left( R \right) = \{v_1, \ldots, v_{r}\} \). Then, we replace each vertex \(v_{i}\) with a set \(V_{i}\) of cardinality \(n_{i}\) and connect these sets in the same manner as a normal blowup. The induced graph is denoted \(R\left( n_1, \ldots, n_{r} \right) \) and called the \textbf{generalized blowup}.
\end{definition}
We modify this construction slightly. Let \(\epsilon, \delta \in \left( 0, 1 \right) \). Then we construct a new graph by applying the generalized blowup to \(G\) with numerical vector \((n_1, \ldots, n_{r})\), but rather than each connected pair \(v_{i} \sim v_{j}\) inducing a complete bipartite subgraph, we only connect sufficient edges in order for \(V_{i}, V_{j}\) to form an \(\left( \epsilon, \delta \right) \)-super regular pair. We denote this new graph \(R_{\epsilon, \delta}\left( n_1, \ldots, n_{r} \right) \).
\begin{theorem}[Blowup Lemma]
	Let \(R\) be a graph of order \(r\) with \(\delta > 0\) and \(\Delta \in \N \setminus \{1\} \). Then, there is \(\epsilon > 0\) so that if \(H \subseteq R\left( n_1, n_2, \ldots, n_{r} \right) \) with \(\Delta\left( H \right) \le \Delta\) , then \(H \subseteq R_{\epsilon, \delta}\left( n_1, \ldots, n_{r}\right) \)
\end{theorem}
This lemma is espcially useful because it allows us to efficiently embed binary trees within these modified blowups. It is trivial to embed a binary tree into a complete generalized blowup, and \(\Delta\left( T \right) = 3\) for a binary tree \(T\), hence fixing a \(\delta > 0\) we can find an \(\epsilon\) so that the tree embeds in \(R_{\epsilon, \delta}\left( n_1, \ldots, n_{r} \right) \) as well.
