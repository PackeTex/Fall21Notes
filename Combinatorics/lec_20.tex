\lecture{20}{Fri 08 Oct 2021 10:13}{Quasi-Random Graphs (3)}
We complete the proof from last time.\\
\begin{proof}

Take \(m\)  values \(x_1, x_2, \ldots, x_{m}\)  and let \(\overline{x}\)  be their arithmetic mean. Then, recall that \(\sum_{i= 1}^{m} \left( x_{i} - \overline{x} \right)^2 = \sum_{i= 1}^{m} x_{i}^2 - n\overline{x}^2 \)This is simply the definition of variance.\\
Then, letting \(m = \binom{n}{2}\), \(\hat{d}_{ij}= x_{k}\)  and the mean codegree to be \(\mcd = \frac{1}{\binom{n}{2}}\sum_{1\le i, j \le n}^{} \hat{d}_{ij} = \frac{1}{\binom{n}{2}} \left( \frac{1}{8}n^3 + o\left( n^3 \right) ) = \frac{n}{4} + o\left( n \right)  \right) \) . Then, we have \begin{align*}
	\sum_{1\le i , j \le n}^{} \left( \hat{d}_{ij}- \mcd \right) ^2 &= \sum_{1 \le i, j \le n}^{} \hat{d}_{ij}^2 - \binom{n}{2}\mcd\\
									&= \frac{1}{32} n^{4} + o\left( n^{4} \right)- \frac{1}{32}n^{4} + o\left( n^{4} \right)   \\
									&= o\left( n^{4} \right)
.\end{align*}
Hence, we obtain \(\sum_{1 \le i, j \le n}^{} \left( \hat{d}_{ij} - \mcd \right) ^2 = o\left( n^{4} \right) \). Then, letting \(y_{i} = \left| \hat{d}_{ij} - \mcd  \right| \) we see by cauchy shwartz that \(\frac{1}{m}\sum_{i=1}^{n} y_{i} \le\sqrt{\frac{1}{m} \sum_{i= 1}^{n} y_{i}}  \) , hence \(\sum_{i= 1}^{n} x_{i} \le \sqrt{m \sum_{i= 1}^{n} y_{i}} \). Hence, we have \(\sum_{1 \le i, j \le n}^{} \left| \hat{d}_{ij} - \mcd \right| \le \sqrt{\binom{n}{2} \sum_{1\le i, j \le n}^{} \left( \hat{d}_{ij} - \mcd \right) ^2} = o\left( n^3 \right)  \) . Hence, \[
	\sum_{1\le i, j \le 2}^{} \left| \hat{d}_{ij} - \mcd \right|  = o\left( n^3 \right)
.\]
Then triangle inequality yields
\begin{align*}
	\sum_{1 \le i, j \le n}^{} \left| \hat{d}_{ij} - \frac{n}{4} \right| &\le \sum_{1 \le i, j \le n}^{} \left| \hat{d}_{ij} - \mcd \right| + \left| \mcd - \frac{n}{4} \right| \\
									     &= o\left( n^3 \right)   + o\left( n^3 \right) \\
									     &= o\left( n^3 \right)
.\end{align*}
\end{proof}
Now, we proceed to prove some more implications, but first we state a lemma.
\begin{lemma}
	Let \(x_1, x_2, \ldots, x_{n}\)  be an orthornormal basis with associated eigenvalues \(\lambda_1, \ldots, \lambda_{n}\). Then for \(j = \frac{1}{\sqrt{n} } \begin{pmatrix} 1 &  \hdots & 1 \end{pmatrix}\) , we find \(\left| x_1 -j \right|_{2} = o\left( 1 \right)  \) .
\end{lemma}
\begin{proof}
	\(\left( P_3 \implies P_5 \right) \) . Let \(x_1\)  be a unit eigenvector of \(G\)  corresponding to \(\lambda_1\). Then, let \(j = \frac{1}{\sqrt{n} }\begin{pmatrix} 1 &  \hdots & 1 \end{pmatrix}\), then by lemma we have \(\left| x_1 - j \right|_{2} = o\left( 1 \right) \) .
\end{proof}
