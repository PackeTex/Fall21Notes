\lecture{7}{Wed 08 Sep 2021 10:19}{Conclusion of SRG and Graph Eigenvalues}
\begin{recall}
	The eigenvalues of a connected SRG take only \(3\) distinct values. Let \( G = \SRG\left( n, k, \lambda, \mu \right) \) Then the eigenvalues are \(k\) with multiplicity \(1\), \(r\) with multiplicity \(f\) and \(s< 0\) with multiplicity \(g\). Furthermore, either \(r, s \in \Z\), \( \left( f \neq g \right) \), or  \(G\) is  a conference graph, \(f = g\). The values values of all these parameters are so interconnected that a given triple of parameters almost always completely characterizes the rest of the parameters.
\end{recall}
\begin{theorem}[Seidel]
	\begin{align*}
		n &\le \frac{f\left( f+3 \right) }{2} \\
		  n&\le \frac{g\left( g+3 \right) }{2}
	.\end{align*}
\end{theorem}
From these statements, we can extract the following inequalities:
\begin{align*}
	f&\ge \sqrt{n} \\
	g &\le \sqrt{n}
.\end{align*}
\begin{remark}
	Taking a sufficiently large SRG and extracting a given number of vertices small in proportion to the size, yields an approximately random graph.
\end{remark}
\begin{figure}[ht]
    \centering
    \incfig{payley}
    \caption{All possible Payley graphs of order \(3\)}
    \label{fig:payley}
\end{figure}

\begin{recall}[Triangular Graphs]
	\(T\left( q \right)  = \SRG \left( \binom{q}{2}, 2\left( q-1 \right), q-2, 4  \right) \).
\end{recall}
This yields the following eigenvalues.
\begin{itemize}
	\item \(k = 2\left( q-2 \right) \)\\
		\item \(r = q-4\) with \(f = q-1\)
		\item \(s = -2\) with \(g = \frac{q\left( q-3 \right) }{2}\).
\end{itemize}
 We illustrate these eigenvalues in the following diagram:
\begin{figure}[ht]
    \centering
    \incfig{eigen}
    \caption{In this graph the length is proportional to multiplicity and the height is proportional to eigenvalue}
    \label{fig:eigen}
\end{figure}
These proportions of graph are essentially guaranteed by Seidel's inequalities.
\begin{recall}[Lattice Graphs]
	\(L_2 \left( q \right)  = \SEG\left( q^2, 2\left( q-1 \right) , q-2, 2 \right) \).
\end{recall}
This yields eigenvalues:
\begin{itemize}
	\item \(k=2\left( q-1 \right) \)
	\item \(r = q-2\) with \(f = 2\left( q-1 \right) \)
	\item \(s = -2\) with \(g = \left( q-1 \right) ^2\)
\end{itemize}
The triangular and lattice graphs are the only families of graph with \(s = -2\).
\begin{recall}[Latin Square Graphs]
	\(L_3 \left( q \right)  = \SRG \left( q^2, 3\left( q-1 \right) , q, 6 \right) \).
\end{recall}
This yields eigenvalues:
\begin{itemize}
	\item \(k = 3\left( q-1 \right) \)
	\item \( r = q - 3\) with \(f = 3\left( q-1 \right) \)
		\item \(s = -3\) with \(g = q^2 -3q + 2\)
\end{itemize}
Lastly, recall
\begin{recall}[Conference Graphs]
	A graph \(G\) is conference if it is \( \SRG \left( n, \frac{n-1}{2}, \frac{n-5}{4}, \frac{n-1}{4} \right) \).
\end{recall}
This yields Eigenvalues
\begin{itemize}
	\item  \(k = \frac{n-1}{2}\)
		\item \(r= \frac{\sqrt{n} -1}{2} \) with \(f = \frac{n-1}{2}\)
			\item \(s = -\frac{\sqrt{n} +1}{2}\) with \(g = \frac{n-1}{2} \)
\end{itemize}
A rare counterexample where the proportions of the eigenvalue diagram are not equal is seen in the conference graph
\begin{figure}[ht]
    \centering
    \incfig{coneig}
    \caption{A graph is conference precisely when \(f = g\). Furthermore, any graph with \( f\neq g\) will have \(r \neq \left| s \right| \)}
    \label{fig:coneig}
\end{figure}
\begin{recall}[Quasi-random Graphs]
	A quasi-random graph is one with parameters approximately \( \SRG \left( n, \sim \frac{n}{2}, \sim \frac{n}{4}, \sim \frac{n}{4} \right) \). These graphs have the quality that for a sufficiently large quasi-random graph we can find an induced subgraph equivalent to  any sufficently smaller graph.
\end{recall}
\begin{definition}[Taylor Graphs]
	Let \(q\) be an odd prime power, then the parameters of a \textbf{Taylor SRG} are
	\begin{itemize}
		\item \(n = q^3\)
		\item \(k = \frac{\left( q-1 \right) \left( q^2 + 1 \right)}{2} \)
		\item \(\lambda = \frac{\left( q-1 \right) ^3}{4} -1 \simeq \frac{q^3}{4}\)
		\item \( \mu = \frac{\left( q-1 \right) \left( q^2 + 1 \right)}{4} \simeq \frac{q^3}{4} \)
	\\ \\	These imply
		\item \(r = \frac{\left( q-1 \right) }{2}\) with \(f = \left( q-1 \right) \left( q^2 + 1 \right) \)
		\item \(s = - \frac{q^2 + 1}{2}\) with \( g = q \left( q-1 \right) \)
	\end{itemize}
\end{definition}
This case of \(f \sim q^3\) is nearly the maximal ratio of \(f\) to \(g\).
\section{Hadamard Matrices}
Let \(A\) be a complex matrix. Then the \textbf{adjoint matrix}\(A^{*} = \overline{\left( A^{T} \right) }\). Furthermore, a \textbf{hermitian matrix} is a matrix for which \(A^{*} = A\).
\begin{example}
	\(A = \begin{bmatrix} 1 & 1+ i \\
	1 - i & -1
\end{bmatrix} \) is a hermitian matrix.
\end{example}
Let \(A \in M[\C]\). We find \(A\cdot A^{\star}\) is always hermitian. Moreover, its eigenvalues are real and nonnegative.
\begin{definition}[Singular Values]
	The square roots of the eigenvalues  of \(A A^{*}\) are called \textbf{singular values} of \(A\).
\end{definition}
