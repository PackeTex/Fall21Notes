\lecture{21}{Wed 13 Oct 2021 10:17}{Quasi-Random Graphs (4)}
We complete the proof from last time. Recall our lemma that for orthornormal basis containing \(x_1\) we have \(\left| x_1 - j \right|_{2} = o\left( 1 \right)  \). We proceed
\begin{proof}
	WLOG assume \(G\) to be a random graph of even order and \(\left| S \right|  = \frac{n}{2}\). Then, we define a vector \(\vec{S}\) with \(s_{i} = \left \{
		\begin{array}{11}
			\frac{1}{\sqrt{n} }, & \quad i \in S \\
			-\frac{1}{\sqrt{n} }, & \quad i \in V \setminus S
		\end{array}
		\right.\)
It is clear \(\left| S \right| _{2} = 1\) and we see \[
\left<S, j \right>  = \underbrace{\frac{1}{\sqrt{n}}  \cdot \frac{1}{\sqrt{n} }}_{\frac{n}{2} \text{ times}}  + \underbrace{\frac{1}{-\sqrt{n} } \cdot \frac{1}{\sqrt{n} }}_{\frac{n}{2} \text{ times}}  = 0
.\]
Then, we note \(\left<S, x_1 \right>  = \left<S, j \right>  + \left<S, x_1-j \right> = \left<S, x_1-j \right> \)  and applying cauchy-shwartz yields \[
	\left<S, x_1 \right>  = \left<S, x_1-j \right>  \le \left| S \right| _{2}\left| x_1-j \right| _{2} = 1\cdot o\left( 1 \right)  = o\left( 1 \right)
.\]
Now, define \(Z = S - \left<S, x_1 \right>x_1 )\). Then, we see \[
	\left<Z, x_1 \right> = \left<S, x_1 \right> - \left<S, x_1 \right>\left| x_1 \right| _{2}^2 = 0
.\]
So, \(Z\) is orthogonal to \(x_1\). Hence, there is a \(n-1\) dimensional space, \(M\),  generated by \(x_2, \ldots, x_{n}\) with eigenvalues \(\lambda_2, \ldots, \lambda_{n}\)  with largest eigenvalue \(\max \{\lambda_2, \left| \lambda_{n} \right| \} \). Then, we find by the rayleigh quotient that \(\left| \left<Ay, y \right>  \right|  \le \lambda_1\left( M \right) \left| y \right| _{2}^2 = \sigma_2 \left| y \right| _{2}^2\)  for all \(y \in M\). Similairly, we find \[
\lambda_{n} \left| y \right| _{2}^2 \le \left<Ay, y \right>  \le \lambda_2	 \left| y \right| _{2}^2
\]
for all \(y \in M\). From this we get \(\lambda_{n} \left| Z \right| _{2}^2\left| \left<AZ, Z \right>  \right| \le \lambda_2 \left| Z \right| _{2}^2 \) , and recalling \(\left| Z \right| _{2} \le \left| S \right| _{2} + \left| \left<S, x_1 \right>  \right|\left| x_1 \right| _{2} = 1 + o\left( 1 \right) 1 \le 2\)  \[
	\left| \left<AZ, Z \right>  \right|  \le \sigma_2 \left| Z \right| _{2}^2 \le \sigma_2 \left| 2 \right| _{2}^2 = 4\sigma_2 = o\left( n \right)
.\]
Finally, we see
\begin{align*}
	\left<AS, S \right> &=  \left<A\left( Z + \left<S, x_1 \right> x_1 \right), Z + \left<S, x_1 \right> x_1  \right>  \\
			    &= \underbrace{\left<AZ, Z \right>}_{o\left( n \right) }  + \underbrace{\left<S, x_1 \right> }_{o\left( 1 \right) } \underbrace{\left<AZ, x_1 \right>}_{=0}  + \underbrace{\left<S, x_1 \right>}_{o\left( 1 \right) }  \underbrace{\left<Ax_1, Z \right>}_{0} + \underbrace{\left<S, x_1 \right> ^2}_{o\left( 1 \right) } \left<Ax_1, x_1 \right>   \\
			    &= o\left( n \right)  + \left<S, x_1 \right> ^2 \left<Ax_1,x_1 \right>  \\
			    &= o\left( n \right)  + \lambda_1 \\
			    &= o\left( n^2 \right)  \\
.\end{align*}
Recall we also know \[
	\left<AS, S \right> = 2e\left( S \right)  + 2e\left( G \setminus S \right) - 2e\left( S, G \setminus S  \right)
.\]
and \(2e\left( S \right)  + 2e\left( G \setminus S \right)  + 2e\left( S, G \setminus S \right) = e\left( G \right) \ge \frac{1}{4} n^2 + o\left( n^2 \right)  \) . Then, adding and dividing yields these identities yields  \(e\left( S \right)  + e\left( G \setminus S \right) = \frac{n^2}{8} + o\left( n^2 \right) \) . Furthermore, \(\sum_{i \in S}^{} d_{i} = = \frac{n^2}{4} + o\left( n^2 \right)  2e\left( S \right)  + e\left( S, G \setminus S \right) \) and \(\sum_{i \in G \setminus S}^{} d_{i} = \frac{n^2}{4} + o\left( n^2 \right)  = 2e\left( G \setminus S \right) + e\left( S, G \setminus S \right) \) . Adding all of the identities thus far yields that \(2e\left( S \right)  -2e\left( G \setminus S \right)  = o\left( n^2 \right) \) , hence \(e\left( S \right)  = \frac{1}{16}n^2 + o\left( n^2 \right) \) .
\end{proof}
We are nearing the end of quasi-random graphs, but note we have always assumed a quasi-random graph to have density \(\frac{1}{2}\). These properties are easily generalized to one of density \(p\). We list the generalized properties.
\begin{definition}
	\begin{enumerate}
		\item \(\left( P_2 \right) \). A graph is \(P_2\) if
			\begin{itemize}
				\item \(e\left( G \right) \ge \frac{pn^2}{2} + o\left( n^2 \right) \)
				\item \(\# CW_4 \le p^{4} n^{4} + o\left( n^{4} \right) \) .
			\end{itemize}
		\item \(\left( P_3 \right) \). A graph is \(P_3\) if
			\begin{itemize}
				\item \(e\left( G \right)  \ge \frac{pn^2}{2} + o\left( n^2 \right) \)
				\item \(\lambda_1 \left( G \right)  = pn + o\left( n \right) \)
				\item \(\sigma_2\left( G \right)  = o\left( n \right) \) .
			\end{itemize}
		\item \(\left( P_7 \right) \) . A graph is \(P_7\)  if
			\begin{itemize}
				\item \(\sum_{1 \le i , j \le n}^{} \left| \hat{d}_{ij} - p^2n \right| = o\left( n^2 \right) \) .
			\end{itemize}
	\end{enumerate}
\end{definition}
