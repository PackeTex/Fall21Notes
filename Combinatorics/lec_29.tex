\lecture{29}{Fri 05 Nov 2021 10:21}{}
\begin{note}{}
I skipped taking notes on a few lectures. See the paper in the onedrive for info on the szemerdi lemma.
\end{note}
\begin{lemma}[Triangle Removal Lemma]
	Denote \(K_{3} \left( G \right) \) 	to be the number of triangles in \(G\).\\
	Then, for all \(0 < \epsilon < 1\) there is \(\delta > 0\) so that if \(G\) is a graph of order \(n\) with \(K_{3} \left( G \right) < \delta n^3\) there are edges \(e_1, e_2, \ldots ,e_{p}\) with \(p < \epsilon n^2\) and \(K_{3} \left( G - e_1 -e_2 - \ldots -e_{p} \right) = 0\).
\end{lemma}
For an example, we may take the graph \(K_{n, n} \) and add an edge within one of the partite sets. Then, we see the only triangles are the ones incident to this new edge. Hence, we need only remove one edge to obtain a triangle-less graph.\\
More generally, adding cross edges between two subsets of size \(\epsilon n\) within one of the partite sets will yield \(\frac{\epsilon^2}{2}n^3\) triangles which may be removed by simply deleting all of the edges which we added.

\begin{proof}
	Fix \(\epsilon\) and apply the regularity lemma with \(\frac{\epsilon}{4}\) and \(m > \frac{2}{\epsilon}\). Then, there is \(n_0 \left( \frac{\epsilon}{4}, m \right) \) so that for all \(G\) of order \(n > n_0\)  we find \(V\left( G \right) = \bigcup_{i=0} ^{k} V_{i}\) with \(\left| V_{i} \right|< \frac{\epsilon}{4}n \) and \(\left| V_1 \right|  = \ldots = \left| V_{k} \right| \). Moreover, \(m < k < M\) and all but \(\frac{\epsilon}{4} k^2\) of the pairs \(\left( V_{i}, V_{j} \right) \) , \(1 \le i < j \le k\) are \(\frac{\epsilon}{4}\)-regular.\\
	Then, we delete all edges incident to \(V_0\). We see this is at most \(\sum_{x \in V_0}^{}d\left( x \right) < \frac{\epsilon}{4}n^2\) edges being removed. Now, fix \(1 \le i \le k\) are delete all edges within \(V_{i}\). Again, we see this is at most \( \binom{\left| V_{i} \right| }{2}\) edges, hence doing this for all sets \(V_{i}\) , \(1 \le i \le k\) we see we remove at most \(k \binom{\left| V_{i} \right| }{2} < k \frac{n^2}{2k^2} = \frac{n^2}{2k}< \frac{n^2}{2m} \le \frac{\epsilon}{4}n^2\).\\
Next, we delete all edges within the non \(\frac{\epsilon}{4}\)-regular pairs. Again, we see this is at most \(\left| V_{i} \right|\left| V_{j} \right| < \frac{n^2}{k^2}  \) pairs. Summing over all non \(\frac{\epsilon}{4}\)-regular pairs yields \(\frac{\epsilon}{4}n^2\) more edges to remove.\\
Finally, we delete all edges in the remaining pairs with \(d\left( V_{i}, V_{j} \right) < \frac{\epsilon}{2} \). We see \(e\left( V_{i}, V_{j} \right) \le \frac{\epsilon}{2} \frac{n^2}{k^2} \). Summing over all \(\binom{k}{2}\) pairs \(\left( V_{i}, V_{j} \right) \), we see once again we remove less than \(\frac{\epsilon}{4}n^2\) edges.\\
Hence, in total we have removed strictly less than \(\epsilon n^2\) edges. Now, we verify that we have removed all triangles. Suppose there remains a triangle. Of course, this triangle will not be incident to \(V_0\). Moreover, this triangle will have edges contained in \(3\) distinct \(V_{i}\)s. This is because we removed all edges within each \(V_{i}\). Suppose WLOG, that the edges are in \(V_1, V_2, V_3\). Then, as we have removed edges between irregular and low density pairs, \(d\left( V_1, V_2 \right) , d\left( V_2, V_3 \right) , d\left( V_1, V_3 \right) \ge \frac{\epsilon}{2}\).\\
Now, recall for \(3\) mutually \(\frac{\epsilon}{4}\)-regular pairs \(V_1, V_2, V_3\) of equal cardinality we had \(k_3\left( V_1, V_2, V_3 \right) \ge \left( 1 - 2 \frac{\epsilon}{4} \right) \left( d_{1,2} - \frac{\epsilon}{4} \right) \left( d_{1, 3} - \frac{\epsilon}{4} \right) \left( d_{2, 3} - \frac{\epsilon}{4} \right)\left| V_1 \right| ^3   \). Hece, we see as \(\left| V_1 \right| > \left( 1-\epsilon \right) \frac{n}{k} > \left( 1-\epsilon \right) \frac{n}{M}\), and letting \(\delta = \left( 1 - \frac{\epsilon}{2} \right) \left( \frac{\epsilon}{4} \right) ^3 \frac{\left( 1-\epsilon \right) ^3}{M^3}\), we see
\begin{align*}
	K_3\left( V_1, V_2, V_3 \right) &\ge \left( 1- \frac{\epsilon}{2} \right) \left( \frac{\epsilon}{4} \right) ^3 \left| V_1 \right| ^3\\
					&> \delta n^3
.\end{align*}
Hence, there will only be triangles remaining if the initial assumptions are violated so the claim is shown.
\end{proof}
\begin{remark}
	We have shown this for \(n > n_0\). If instead we wish to show for \(n \le n_0\) we can choose \(\delta = \frac{1}{\binom{n_0}{3}}\) which will exclude the possibilities of any triangles before removing any edges, hence the lemma is trivially true in this case.
\end{remark}
