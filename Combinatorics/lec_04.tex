\lecture{4}{Mon 30 Aug 2021 10:20}{Strongly Regular Graphs (4)}
\begin{recall}
	Recall a Latin square of order \(q\) is a \(q \times q\) matrix with no duplicate entries in any given row or column. The number of Latin squares of size \(q\) grows even faster than \(q!\).
\end{recall}
\begin{definition}[Back-Circulant Latin Square]
	We define a \textbf{back-circulant Latin Square} to be a \( q \times q\) matrix,\( L\), defined such that \(\ell_{i, j}= \left( i + j \right) \mod 4 \left( + 1 \right) \), where the \(+1\) is optional. This yields entries \(0, 1, 2, \ldots, q-1\) or \(1, 2, \ldots, q\) in the case of \(\left( +1 \right) \). It is clear this is symmetric and satisfies the \(q\) symbols requirement of a Latin square. Lastly, we must show that this produces the unique column and row requirement, this will be left as an exercise.
\end{definition}
\begin{problem}
	Check that the back-circulant Latin square satisfies uniqueness of columns/rows.
\end{problem}
The main use of these Latin squares is that a Latin square of order  \(q\) can generate a \(\SRG\left( q^2, 3\left( q-1 \right) , q, 6 \right)  \). This graph we define will be an extension of the lattice graph on \(q^2\). To generate this graph we define the vertices to be the entries of the matrix, \(L\), where  \(L\) is a Latin square (hence there are \(q^2\)). Two vertices, \(\ell_{i, j}\) and \(\ell_{p, q}\), are joined if one of the following holds
 \begin{enumerate}
	\item \(i=p\)
	\item \(j=q\)
	\item \(\ell_{i, j}= \ell_{p, q}\).
\end{enumerate}
We see this identifies elements of the same row, column, and symbol. Let us check that this is indeed a regular graph of order  \(3\left( q-1 \right) \). Let \(\ell_{i, j}\) be an entry in \(L\) and note that as  \(L\) is \(q \times q\) matrix there are \(q-1\) entries in row \(i\) and \(q-1\) entries in column \(j\). Furthermore, by construction of Latin squares, we have that every column contains exactly one entry, \(\ell_{p, q}\) with  \(\ell_{i, j} = \ell_{p, q}\), so excluding \(\ell_{i, j}\) itself yields another \(\left( q-1 \right) \) entries, which must be distinct from the entries sharing a column or row by construction. Hence the graph generated by \(L\) is  \(\left( q-1 \right) \)-regular.
\begin{definition}[Conference Graphs]
	A graph is called a \textbf{conference graph} if it is \(\SRG \left( n, \frac{n-1}{2}, \frac{n-5}{4}, \frac{n-1}{4} \right) \). We see this definition necessitates \(n=1 \mod 4\). Less obvious, we must have that \(n= q^2 + p^2\) for \(p, q \in \Z\). This graph also has the special property that for a conference graph \(G\) which is \(\left( \frac{n-1}{2} \right) \)-regular, \(\overline{G}\) is also \(\left( \frac{n-1}{2} \right) \)-regular.
\end{definition}
\begin{example}
	\(C_5\) is a conference graph.\\
	Paley graphs are conference graphs.\\
\end{example}
\begin{definition}[Paley Graphs]
	Let \(p^{n} = q \cong 1 \mod 4\) for a prime \(p\) and \(n \in Z\). Consider the finite field \(\GF\left( q \right) \). We say a graph \(G\) is a \textbf{paley graph} if \(\V\left( G \right) = \GF\left( q \right) \), the elements of the finite field, and two elements \(x, y \in \GF\left( q \right) \) have \(x \sim y\) if \(x-y = k^2\)  for some \(k \in \GF\left( q \right) \setminus \{0\}  \) (hence \(x \neq y\)). The reason we restrict \(q \cong 1 \mod 4\) is that this implies the relation symmetric ( \(x \sim y \iff y \sim x\)). If we let \(q \cong 3 \mod 4\) this property does not hold and we get a directed graph instead.
\end{definition}
\begin{example}
	\(\GF\left( 5 \right) \) consists of elements \(0, 1, 2, 3, 4\) and  \(1, 4\) are defined as squares. So the graph has edges \(\{4, 3\} ,  \{3, 2\} , \{2, 1\} \{1, 0\} , \ldots\). \\
\end{example}
\begin{remark}
	Take a quadratic nonresidue \(a \in \GF\left( q \right) \) and define \(x \mapsto ax\). We see this is a bijection of \(GF\left( q \right) \) with itself. Furthermore, if \(x-y\) is a square in \(\GF\left( q \right) \), then \(a\left( x-y \right) \) is not a square by the fact that \(a\) is a quadratic nonresidue. Similairly, this maps a nonsquare to a square. Hence, this mapping takes the paley graph of \(\GF\left( q \right) \) to its complement.
\end{remark}
\begin{figure}[ht]
    \centering
    \incfig{payleycomp}
    \caption{We see the payley graph of \(\GF\left( 5 \right) \), \(C_5\),  (solid) and the mapping \(x \mapsto 2x\) yields the complement (dotted)}
    \label{fig:payleycomp}
\end{figure}
\begin{remark}
	If we take a SRG with \(n\) vertices which is \( \sim cn\)-regular for \(0 < c < 1\), then the graph will be approximately \(\SRG\left( n, \sim cn, c^2n, c^2n \right) \). We see this in the definition of the conference graph, which is about \(\frac{n}{2}\)-regular and is approximatly \(\SRG\left( n, \frac{n}{2}, \frac{n}{4}, \frac{n}{4} \right) \). We call such graphs \textbf{quasi-random}.
\end{remark}
\begin{remark}
	Delsarte-Goethals and Turyn have desribed these graphs.
\end{remark}
\begin{definition}[]
	Take a \(2\)-dimensional linear space over \(\GF\left( q \right) \) denoted \(X\). This is simply ordered pairs of elements of \(\GF\left( q \right) \). Then, there are \(q+1\) distinct lines passing through the origin. Taking an arbitrary set \(S\), of these lines allows us to define a graph \(G\) with \(\V\left( G \right) \) being the points in \(X\) and we say \(x \sim y\) if \(x-y\) is a point on an element of \(S\). This graph is  \(SRG\left( q^2, s\left( q-1 \right) , \left( s-1 \right) \left( s-2 \right) +q-2, s\left( s-1 \right)  \right) \) where \(\left| S \right| = s\).
\end{definition}
