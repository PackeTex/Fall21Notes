\lecture{30}{Wed 10 Nov 2021 16:24}{The (6,3) Problem}
\begin{definition}[Hypergraphs]
	A \textbf{\(r\) -uniform graph} or \textbf{\(r\)-graph}, \(G\) is a pair of two sets. \(V\), being a vertex set and \(E\), being a set of edges are subsets of cardinality \(r\). That is, a graph where a single edge connects \(r\) (not necessarily distinct) vertices instead of \(2\) vertices as in a graph (with or without loops).
\end{definition}
\begin{proposition}
	Suppose \(G\) is a \(3\)-uniform graph of order \(n\) such that every set of \(6\) vertices induces at most \(2\) edges. Then \(e\left( G \right)  = o\left( n^2 \right) \).
\end{proposition}
\begin{proof}
	First note that for  a fixed vertex \(v\), we find \(d \left( v \right) \le 4\) else the second assumption is violated. Hence, induce a graph from \(G\) by removing all vertices sharing a common pair. Denote this graph \(G^{\prime}\) of order \(n^{\prime}\). Then, we see \(e\left( G \right) - e\left( G^{\prime} \right) \le 4\left( n - n^{\prime} \right) \) by the restriction on the degrees. \\
	Next, induce a graph from \(G^{\prime}\) by considering only the edges participating in a triple (that is pairs of two triangles sharing a single vertex) and "demote" it to a simple graph by convering every triple into a triangle. Call this \(G^{\prime}_2\) and note that we convered a triple into \(3\) edges, so by the preceding lemma, we see \(e\left( G^{\prime}_{2} \right) = o\left( n^{\prime}^2 \right) \) and \(e\left( G^{\prime} \right) = \frac{1}{3}e\left( G^{\prime} \right)  = o\left( n^2 \right) \). Then, working backwards, we see \(e\left( G \right) \le 4\left( n - n^{\prime} \right)  + o\left( n^{\prime}^2 \right) = n\left( n^2 \right) \).
\end{proof}

\begin{definition}[Super Regular Pair]
	We define a graph \(G\left( A, B \right) \) to be an \textbf{\(\left( \epsilon, \delta \right) \)-super regular pair} for some \(\epsilon, \delta \in \left( 0, 1 \right) \)  if the following conditions hold
	\begin{itemize}
		\item For all \(X \subseteq A\) and \(Y \subseteq B\) with \(\left| X \right| < \epsilon \left| A \right| \) and \(\left| Y \right| < \epsilon\left| B \right| \) we have \(e\left( X, Y \right) > \delta \left| X \right| \left| Y \right| \).
		\item For every vertex \(a \in A\), \(d\left( a \right) > \delta \left| B \right| \) and similairly, for all \(b \in B\), \(d\left( b \right) > \delta \left| A \right| \).
	\end{itemize}
\end{definition}
\begin{proposition}
	If \(\left( A, B \right) \) is an \(\epsilon\) regular pair with \(d\left( A, B \right) > 3\epsilon\), then there are \(A^{\prime}, B^{\prime}\) so that \(\left| A^{\prime} \right| > \left( 1-\epsilon \right) \left| A \right|  \) and \(\left| B^{\prime} \right|> \left( 1-\epsilon \right) \left| B \right|  \)  such that \(\left( A^{\prime}, B^{\prime} \right) \) is an \(\left( 2\epsilon, d - 2\epsilon \right) \)-super regular pair.
\end{proposition}
\begin{proof}
	Define \(A_0 = \{a \in A :  d\left( a \right) \le \left( d - \epsilon \right) \left| B \right| \} \). Then, we see \(\left| A_0 \right| < \epsilon \left| A \right| \) by assumption. Define \(A^{\prime} = A \setminus A_0\). Then, \(d_{B}\left( a \right) > \left( d -\epsilon \right) \left| B \right| \), so once we have removed at most \(\epsilon\) elements from \(B\) as well, we see \(d_{B^{\prime}}\left( a \right) > \left( d - 2\epsilon \right) \left| B \right| \). Similairly, we find \(d_{A^{\prime}}\left( b  \right) > \left( d - 2\epsilon \right) \left| A \right|  \) after removing an exceptional set \(B_0\).\\
	Then, by the slicing lemma, we find an \(\alpha > \epsilon\) so that \(\left| A^{\prime} \right| > \alpha \left| A \right| \) and \(\left| B^{\prime} \right| > \alpha \left| B \right|  \). Then, we see \(\left( A^{\prime}, B^{\prime} \right) \) is an \(\epsilon^{\prime}\) regular pair with \(\epsilon^{\prime} = \sup \{ \frac{\epsilon}{\alpha}, 2\epsilon \} \). If \(\alpha = 1-\epsilon\), we find \(\epsilon^{\prime} = 2\epsilon\).
\end{proof}
We will complete this proof next time.
