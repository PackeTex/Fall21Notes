\lecture{30}{Wed 10 Nov 2021 16:24}{The (6,3) Problem}
\begin{definition}[Hypergraphs]
	A \textbf{\(r\) -uniform graph} or \textbf{\(r\)-graph}, \(G\) is a pair of two sets. \(V\), being a vertex set and \(E\), being a set of edges are subsets of cardinality \(r\). That is, a graph where a single edge connects \(r\) (not necessarily distinct) vertices instead of \(2\) vertices as in a graph (with or without loops).
\end{definition}
\begin{proposition}
	Suppose \(G\) is a \(3\)-uniform graph of order \(n\) such that every set of \(6\) vertices induces at most \(2\) edges. Then \(e\left( G \right)  = o\left( n^2 \right) \).
\end{proposition}
