\lecture{5}{Wed 01 Sep 2021 10:18}{SRG (5) and Graph Matrices}
\begin{recall}[Delsarte-Goethals-Turyn Graphs]
	Let \(n=q^2\) for \(q\) being a prime power. Then, let \(1\le s \le q\). From this, we may  construct a graph \\\(\SRG \left( q^2, s\left( q-1 \right) , \left( s-1 \right) \left( s-2 \right) +q-2 , s\left( s-1 \right) \right) \). Such a graph is part of a family of graphs called psuedo-Latin square graphs.
\end{recall}
\begin{problem}
	Is there a graph which is \(\SRG \left( n, \frac{n}{3}, \sim\frac{n}{9}, \sim\frac{n}{9} \right) \).
\end{problem}
\begin{solution}
	If we let \(n=q^2\) where \(q\) is a prime power and \(s = \left\lfloor \frac{q}{3} \right\rfloor\), we obtain
	\begin{align*}
		n&=q^2\\
		k&= \left\lfloor \frac{q}{3} \right\rfloor\left( q-1 \right) \\
		\lambda &= \left( \left\lfloor \frac{q}{3} \right\rfloor -1 \right) \left( \left\lfloor \frac{q}{3} \right\rfloor-2 \right) +q - 2 = \left( \left\lfloor \frac{q}{3} \right\rfloor \right)^2 - 3\left\lfloor \frac{q}{3} \right\rfloor + q\\
		\mu &= \left\lfloor \frac{q}{3} \right\rfloor \left( \left\lfloor \frac{q}{3} \right\rfloor-1 \right) \simeq \frac{q^2}{9} = \frac{n}{9}
	.\end{align*}
We see this yields such an approximate construction.
\end{solution}
\begin{remark}
	The parameter \(s=\frac{q-1}{2}\) yields a conference graph. This is useful for a problem posted on the website.
\end{remark}
\newpage
\section{Graph Matrices and Eigenvalues}
\begin{definition}[Eigenvalues]
	Recall, if \(A\) is a \(n\times n\) square matrix and \(x\neq \vec{0}\) is a \(n\)-vector, then \(x\) is a \textbf{eigenvector} of \(A\) if \(Ax = \lambda x\) for some \(\lambda \in \R\). \\For symmetric (real) matrices \(A\), we have that all eigenvalues are real. Furthermore, if we arrange eigenvalues \(\lambda_1, \lambda_2	, \ldots, \lambda_{n}\) in descending order (\(\lambda_1\ge \lambda_2 \ge \ldots \ge \lambda_{n}\)), then we have \(\lambda_1 = \max \{ \left<Ax, x \right> : \|x\|_{2}  =1\}\) (the \(\ell_2\) norm) where \(\left<Ax, x \right> = \sum_{i= 1}^{n} \sum_{j=1}^{n} a_{ij}x_{j}x_{i}\).\\ Similarly, \(\lambda_{n} = \min \{ \left<Ax, x \right> : \|x\|_{2}=1 \}\)\\
	Lastly, we also have that \(\sum_{i= 1}^{n} \lambda_{i} = \tr\left( A \right) \).
\end{definition}
\begin{remark}
	It is trivial that \(\lambda^2\) is an eigenvalue to \(A^2\), and generally \(\lambda^{n}\) is an eigenvalue to \(A^{n}\).
\end{remark}

\begin{definition}[Adjacency Matrix]
	Let \(G = \left( V, E \right) \), where \(V = \{1, 2, \ldots, n\} \) WLOG, then the \textbf{adjacency matrix}, \(\A \left( G \right) \) is defined such that the columns and rows are indexed by \(1, 2, \ldots, n\) (so \(A \left( G \right) \) is \( n\times n\)) and the entry \(a_{ij} = \left \{ \begin{array}{11}
			1, & \quad i \sim j\\
			0, & \quad i \nsim j
		\end{array}
		\right
		\)
\end{definition}
\begin{remark}
	It is clear that the entry \(a_{i, i} = 0\) for all \(i\).
\end{remark}
As for the eigenvalues, we can say that \(\lambda_1 \ge 0\) and \(\lambda_1 + \lambda_2 + \ldots + \lambda_{n} = 0\) by the earlier remark on the trace. Lastly, as we need to have \(\lambda_1 > 0\) for nontrivial graphs, then we can say \(\lambda_{n} < 0\), and furthermore we even have that \(\lambda_{n} \le -1\). This can be shown in the following way:
\begin{proof}
	Let \(i, j\) be adjacent vertices, \(x_{i}= -\frac{1}{\sqrt{2} }\) and \(x_{j} = \frac{1}{\sqrt{2} }\) with \(x_{k} = 0 \) for all \(k \neq i, j\). Then, we see \(a_{i, j} = a_{j, i} = 1\) in the adjacency matrix. Hence taking \(\left<Ax, x \right> = a_{i, j}\left( -\frac{1}{\sqrt{2} } \right) \frac{1}{\sqrt{2}} + a_{j, i} \left( -\frac{1}{\sqrt{2} } \right) \frac{1}{\sqrt{2} } = -1 \), hence the smallest eigenvalue must be less than \(-1\).
\end{proof}
Let \(A, B\) be \(n\times n\) with entries \(a_{i, j}\) and \(b_{i, j}\) being their entries respectively. Remember \(AB = C\) is \(n\times n\)  with entries \(c_{i, j}=\sum_{k=1}^{n} a_{i, k}b_{k, j}\).
\begin{remark}
	Let \(A\) be the adjacency matrix of an arbitrary graph, and denote \(A^2 = B\). We see \(\sum_{k=1}^{n} a_{i, k} a_{k, j} = b_{i, j}\) are the entries of \(B\). Furthermore, each product will either be \(0\) or \(1\) as each element was either \(0\) or \(1\). Hence, we have \(b_{i, j}\) is equal to the number of pairs \(a_{i, k}, a_{k, j}\) are both \(1\). We see this implies \(k \sim i\) and \(k \sim j\), in other words, \(b_{i, j}\) is the number of vertices which are adjacent to both \(i\) and \(j\), hence \(b_{i, j} = \hat{d}\left( i, j \right) \). Thus the entries of \(A^2\) are equal to the codegree of the corresponding vertices.
\end{remark}
