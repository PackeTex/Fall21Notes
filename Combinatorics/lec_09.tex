\lecture{9}{Mon 13 Sep 2021 10:22}{Hadamard Matrices (2) and Kroenikker Product}
\begin{recall}
	A Hadamard matrix is a square matrix \(H\) with entries \(\left|h_{i, j} \right| = 1\) and \(HH^{*} = nI_{n}\). Furthermore, its rows are orthogonal.
\end{recall}

Let \(H\), \(H^{T}\) be real symmetric matrices, hence \(H^{T} = \overline{ \left( H^{T} \right)} \). Then, take rows, \(r_{i}\) and \(r_{j}\). We that for an entry \(h_{i, j} \in H H^{*}\) has \(h_{i, j} = \left<r_{i}, r_{j} \right> \).\\
\begin{proposition}
	If \(H\) is hadamard, then \(H^{*}\) is hadamard.
\end{proposition}
\begin{proof}
	We know \(HH^{*} = nI_{n}\), implying \(H \cdot \frac{1}{n}H^{*} = I\), hence \(\frac{1}{n}H^{*} = H^{-1}\) and as inverses commute, we see \(\frac{1}{n}H^{*}H = I\), hence \(H^{*}H= nI_{n}\).
\end{proof}
From this fact, we an also say \(H^{T}\) is hadamard and \(\overline{H}\) is hadamard. This also clearly yields that the columns of \(H\) are orthogonal. Furthermore, any interchanging of rows or columns will also leave a matrix hadamard. Also, negating any rows or columns preserves hadamardness.\\
These simple facts imply that we can always construct a hadamard matrix with all positives in the first row(column). This is the general case we will use becuase of the following theorem.
\begin{proposition}
	For a normalized \(n \times n\) hadamard matrix , \(H\), with all entries in first row being \(+1\), we must have that \(n\) is even and each row after the first will have an equal number of \(+1\)s and \(-1\)s.
\end{proposition}
\begin{proof}
	For the case \(n = 2\) it is clear the matrix must be of the form \(\begin{bmatrix} 1&1\\1&-1 \end{bmatrix} \). For the case \(n\ge 3\) we have that \(\left<r_1, r_2 \right> = \sum_{i= 1}^{n} (1 + a_{2, i}) = 0\)  , hence half  of the \(r_2\) entries are \(+1\) and the other half are \(-1\). The same argument follows for row \(r_{i}\)
\end{proof}
From this, we may see that for row \(r_2, r_3\) we must have exactly \(\frac{n}{2}\) positions in agreement by a similar argument. Let the entries of \(r_2, r_3\) be partitioned into those where both entries are \(+1\), with cardinality denoted \(a\), those where both are \(-1\), with cardinality denoted \(d\), those where \(r_2\) is \(+1\) and \(r_3\) is \(-1\), with cardinality denoted \(b\), and the opposite, with cardinality denoted \(c\).Then, \(a + b = \frac{n}{2}\) and  \(c+d = \frac{n}{2}\) by \(r_1 \perp r_2\). Similarly,  \(a+c = \frac{n}{2}\) and \(b+d = \frac{n}{2}\) by \(r_1 \perp r_3\). This yields \(a + d = \frac{n}{2}\) and \(b+c = \frac{n}{2}\) by \(r_2 \perp r_3\), hence a quick calculation yields \(a=b=c=d=\frac{n}{4}\).\\
\\
Next, we examine the columns or rows of a hadamard. An alternative definition is to note that all possible rows of a \(n \times n\) hadamard matrix is equivalent to \(\{-1, 1\} ^{n}\) yielding \(2^{n}\) possible vectors. Then, a hadamard matrix \(H\) is simply any orthogonal set of these vectors.\\
\\
\begin{definition}[Kroenkker product]
	Let \(A\) be a \(m \times n\) matrix and \(B\) be a \(p \times q\) matrix. The \textbf{kronecker product/tensor product} is denoted by \(A \otimes B\), which is an \(mp \times nq\) matrix such that \(A \otimes B = \begin{bmatrix} a_{1, 1}B&& a_{1, 2}B && \ldots && a_{1, n}B\\
		\vdots && \vdots && \vdots && \vdots \\
a_{n, 1}B && \ldots && \ldots && a_{n, n} B\end{bmatrix} \)
\end{definition}
Generally speaking \(A \otimes B \neq B \otimes A\), but their rows and columns may be rearranged to yields equivalent matrices.\\
\begin{theorem}
	Let \(m = n\) and let \(\lambda_1, \lambda_2, \ldots, \lambda_{n}\) be the eigenvalues of \(A\) and let \(p = q\) and \(\mu_1, \mu_2, \ldots, \mu _{p}\) be the eigenvalues of \(P\). Then, the eigenvalues of \(A \otimes B\) are \(\lambda_{i}\mu_{j}\) where \(1\le i \le n\) and \(1 \le j \le p\).
\end{theorem}
\begin{theorem}
	Let \(A\) be \(m \times n\) and \(B = p \times q\) with \(\lambda_1, \ldots, \lambda_{n}\) being singular values of \(A\) and \(\mu_1, \ldots, \mu_{n}\) be singular values of \(B\), then the singular values of \(A \otimes B\) are \(\lambda_{i} \mu_{j}\) with \(1 \le i \le n\) and \(1 \le j \le p\).
\end{theorem}
\begin{proposition}
	For two hadamard matrices, \(A, B\), we have \(A \otimes B\) is hadamard.
\end{proposition}
\begin{definition}[Blowup of Graph]
	Let \(G\) be a graph and replace each vertex with a group of \(t\) vertices. We define the blowup of \(G\) to have these new vertices after replacement and two edges are connected if their generating vertices were connected. Hence, any two connected vertices from \(G\) yield a bipartite relationship (as all possible combinations of their degenerate vertices are connected).
	\end{definition}
We wonder what is \(A\left( G^{\left( t \right) } \right) \). We see, grouping respective independent sets, then each \(1\) or \(0\) in the adjacency matrix of \(G\) will be a \(t \times t\) block of \(0\) or \(1\) in \(\adj \left( G^{\left( t \right) } \right) \).\\
This yields eigenvalues of \(G^{\left( t \right) } = t\lambda_1, t\lambda_2, \ldots, t\lambda_{n}, \underbrace{0, \ldots, 0}_{n\left( t-1 \right)  } \).\\
This yields \(v\left( G^{\left( t \right) } \right) = tn \), \(e\left( G^{\left( t \right) } \right) = t^2 m \) and \(K_3 \left( G^{\left( t \right) } \right) = t^3 K_{3}\left( G \right) }  \), Many other graph parameters, such as clique numbers, are also preserved in some form.
