\lecture{1}{Mon 16 Aug 2021 11:33}{Basics of graph theory}
\section{Basics of Graph Theory}
It is often useful to be able to mathematically encode the relationship of objects and connections between them. For instance, people and friendships, cities and roads, etc. For this, we use graphs, sets of vertices and edges represented by nodes and lines in space. It is clear that the spatial relationships are not important, this is merely an encoding of the aforementioned relationships.
\begin{definition}[Graph]
	A \textbf{graph} $G$ is an ordered pair $\left( V, E \right) $ where $V = \left\{ v_1, v_2, \ldots, v_{n} \right\}$ is a set of vertices and $E = \left\{ \left\{ v_{i}, v_{k} \right\}, \ldots \right\}  $  is a set of $2$-tuples of edges. We write $V\left( G \right), E\left( G \right)$ to denote the sets of vertices and edges of the prescribed graph $G$. It is customary to assume $V\left( G \right) = \left\{ 1, 2, \ldots, n \right\} $ as these are the simplest indices for most cases. Vertices will always be denoted by lower case letters, and the edge between two vertices $u, v$, while officially denoted $\left\{ u,v \right\} $ will often be abbreviated to $uv$.
\end{definition}
To refer to the existence of an edge between two vertices $u, v$ we will often state $u$ is joined/adjacent/a neighbor to $v$, this relationship is clearly reciprocal.
 \begin{notation}
	 We often want to reference the number of vertices or number of edges. To do so, we write  $ \left| V\left( G \right)  \right| = v\left( G \right) $ and call it the order of $G$. Similarly, $ \left| E\left( G \right)  \right| = e\left( G \right) $ is called the size of $G$.
\end{notation}
\begin{proposition}[Maximal size]
	The maximum size of a graph of order $n$ is $\binom{n}{2} = \frac{n\left( n-1 \right) }{2}$. This is because this is the number of $2$ element subsets of $V$.
\end{proposition}
\begin{definition}[Complete Graph]
	The graph of order $n$ and size $\binom{n}{2}$ is called the  \textbf{Complete Graph} of order $n$ and is denoted $K_{n}$.
\end{definition}
As previously stated, the manner in which we draw graphs is not important. Often, we will be able to draw graphs in the plane without intersecting edges, but this is not always the case and we will later examine theorems that determine when this is possible.
\begin{definition}[Path]
	A \textbf{Path} of order $n$ is a graphs with vertices $v_1, v_2, \ldots, v_{n} \in V$ such that the edges $\left\{ v_1, v_2 \right\}, \left\{ v_2, v_3 \right\} ,\ldots , \left\{ v_{n-1}, v_{n} \right}  \in E$. We denote such a path by $P_{n}$.
\end{definition}
\begin{notation}
	It is clear that $e\left( P_{n} \right) = n-1 $, and we call this the length of the path.
	Furthermore, we denote $v_1, v_{n}$ as the end vertices of $P_{n}$ and the rest of the vertices as internal vertices. Finally, we say that for two end vertices $v_1, v_{n}$, $P_{n}$ joins $v_1, v_{n}$.
\end{notation}
\begin{definition}[Maximal path]
	Let  $G$ be a graph and $P_{k} = u_1, u_2, \ldots, u_{k}$ be a path in $G$. $P$ is called \textbf{maximal} if $G$ has no such path $u_0, u_1, \ldots, u_{k}$ or $u_1, u_2, \ldots, u_{k}, u_{k+1}$. This implies that all vertices joined to $u_1$ and $u_{k}$ are a part of $P$ (else they could be prepended or appended in order to form such a longer path). We should note that this does not imply $P$ is the longest path in $G$, simply that it cannot be extended.
\end{definition}
\begin{definition}[Cycles]
	For $n\ge 3$ we define a  \textbf{Cycle} of order $n$ to be a graph with vertices $v_1, v_2, \ldots, v_{n}$ which contains edges $\left\{ v_1, v_2 \right\} , \left\{ v_2, v_3 \right\}, \ldots, \left\{ v_{n-1}, v_{n} \right\}, \left\{ v_{n}, v_1 \right\}  $. This is clearly also a path where $v_{n}$ and $v_1$ are joined. We denote such a cycle to be $C_{n}$ and we sometimes call it an $n$-cycle.
\end{definition}
\begin{remark}
	It is clear that $C_{n}$ is of size $n$. Just as with paths, we will refer to this as the length of $C_{n}$.
	The special case $ n=3$ is referred to as a triangle.
\end{remark}
\begin{definition}[Stars]
	A  \textbf{Star} of order  $n$ is a graph with vertices $v_1, v_2, \ldots, v_{n}$ which contains edges $\left\{ v_1, v_2 \right\} , \left\{ v_1, v_3 \right\} , \ldots, \left\{ v_1, v_{n} \right\} $. We denote such a star by $K_{1, n-1}$.
\end{definition}
\begin{remark}
	It is clear that the size of $K_{1, n-1}$ is $n-1$. Furthermore, we see $K_{1,1} = K_2 = P_2$ and $K_{1,2} = P_3$.
\end{remark}
Much as in geometry, we do not consider graphs to be equal unless they are precisely identical, but for graphs which are essentially the same we still want a way to refer to this relationship. For this we define graph isomorphism:
\begin{definition}[Graph Isomorphism]
	Two graphs $G_1 = \left( V_1, E_1 \right) $ and $G_2 = \left( V_2, E_2 \right) $ are \textbf{isomorphic} if there is a bijection $\phi : V_1 \to V_2$ such that $\left\{ u,v \right\} \in E_1 \iff \left\{ \phi \left( u \right) , \phi \left( v \right)  \right\} \in E_2$. This bijection $\phi$ is referred to as the isomorphism.
\end{definition}
For all graph theoretic purposes, two isomorphic graphs are identical and will share all of the same properties. Actually determining if such graphs are isomorphic, however, is a difficult algorithmic problem which has not been full deduced as of yet. Next, we would like to determine when a graph has certain symmetries through some sort of mathematical procedure. As in algebra, we will use the concept of automorphisms to formalize this.
\begin{definition}[Graph Automorphism]
	Let  $G = \left( V, E \right) $ be a graph. A permutation $\phi: V \to V$ is called an \textbf{automorphism} of $G$ if $\left\{ u,v \right\}  \in E \iff \left\{ \phi \left( u \right) , \phi \left( v \right)  \right\}  \in E$. We see this is essentially an isomorphism of $G$ with itself, though the function need not be a bijection, where vertices are rearranged, but edges remain in place. The set of all automorphism of $G$ form a group, which is a subgroup of the symmetric group of order $n$ and is denoted $\aut\left( G \right) $.
	The most basic automorphism are the permutations of the complete graph $K_{n}$, of which all are automorphisms, hence $K_{n}$ has $n!$ automorphisms. Further study yields that almost all graphs lack (non-trivial) automorphisms, hence almost all graphs are asymmetric.

\end{definition}
\begin{definition}[Labeled Graph]
	Let $V \neq \O$ be a finite set with elements indexed by unique labels (numbers). A \textbf{labeled graph} with vertex set $V$ is a set of $2$-tuples of $V$. That is, an indexed set of edges. Two labeled graphs can be isomorphic, but are distinct if their edge sets are different. If $V$ has $n$ elements we see there are $2^{\binom{n}{2}}$ labeled graphs with vertex set $V$ (as every pair of vertices of which there are $\binom{n}{2}$ can either be connected or not).
\end{definition}
\begin{remark}
	We note that the vertices of $G$ can be labeled in $n!$ ways, but the resulting graphs may not be distinct. For example, all labelings of $K_{n}$ result in the same labeled graph. As a labeling of a graph is just a permutation of its vertices, group theoretic arguments are particularly powerful when proving theorems concerning labeled graphs.
\end{remark}
\begin{proposition}
	The number of labeled graphs isomorphic to a given graph $G$ of order $n$ is equal to $\frac{n!}{\left| \aut \left( G \right)  \right| }$.
\end{proposition}
Summing over this equality for all labeled graphs of order $n$ yields \[
	\sum_{v\left( G \right) = n}^{} \frac{n!}{\left| \aut \left( G \right)  \right| } = 2 ^{\frac{n\left( n-1 \right) }{2}}
.\]
We note that this is because the right side of the equality is just the number of all such labeled graphs of order $n$. Next we define $\Gamma \left( n \right) = \sum_{v\left( G \right) = n}^{} 1$ to be the number of graphs of order $n$ and we see that as $\left| \aut \left( G \right)  \right| > 1$, $\Gamma \left( n \right)  \ge \sum_{ v\left( G \right) = n}^{} \frac{1}{\left| \aut \left( G \right)  \right| } = \frac{2^{\frac{n\left( n-1 \right) }{2}}}{n!}$. Taking the fact that for almost all graphs, $\left| \aut \left( G \right)  \right| = 1$ yields  $\Gamma \left( n \right) \approx \frac{2^{\frac{n\left( n-1 \right) }{2}}}{n!}$.
\begin{definition}[Degree]
	Let  $G$ be a graph and $v \in V\left( G \right) $. We define the \textbf{degree} of $v$ in $G$ to be the number of edges containing $v$. We denote this by $d_{G}\left( v \right) $.
\end{definition}
\begin{definition}[Neighborhood]
	Similarly, we define the \textbf{neighborhood} of $v$ to e the set of all vertices which share an edge with $v$. Again, this is denoted $N_{G}\left( v \right) $. It is clear that $d_{G}\left( v \right) = \left| N_{G}\left( v \right)  \right| $.
\end{definition}
 \begin{proposition}
	If $G$ is a graph with vertex set $V$ and size $m$, then \[
		\sum_{u \in V}^{} d_{G}\left( u \right)  = 2m
	.\]
\end{proposition}
\begin{proof}
	This equality is relatively trivial, as for a given edge $\left\{ u,v \right\} $, the sum will count this edge once for $d_{G}\left( v \right) $ and once for $d_{G}\left( u \right) $. Hence, every edge will get counted once for its left member and once for its right member, so they will all be counted twice.
\end{proof}
\begin{corollary}
	The number of vertices of odd degree in $G$ is even
\end{corollary}
\begin{proof}
	Precise details of the proof evade me, but you should split up the sum from the previous proposition into the sum of vertices of odd degree and sum of vertices of even degree and show that one is even, and as they add to an even number, the other must be even as well.
\end{proof}
\begin{proposition}
	Every graph of order $2$ or greater has two vertices of the same degree.
\end{proposition}

\begin{definition}
	The \textbf{minimum degree} $\delta \left( G \right) $ of a graph $G$ is defined $\delta\left( G \right) = \min _{u \in V}\left\{ d_{G}\left( u \right)  \right\} $. The  \textbf{maximum degree} $\Delta \left( G \right) $ of $G$ is defined similarly but with maximum.
\end{definition}
\begin{remark}
	It is clear that if $G$ is of order $n$, then $0 \le \delta\left( G \right) \le \Delta \left( G \right) \le n-1$
\end{remark}
\begin{definition}[Isolated Vertex]
	If $u$ is a vertex with $d_{G}\left( u \right)  = 0$ we call it an \textbf{isolated vertex} of $G$.
\end{definition}
\begin{definition}[Dominating Vertex]
	If $u$ is a vertex with $d_{G} \left( u \right) = \left|  V\left( G \right)  \right| - 1$, then $u$ is called a \textbf{dominating vertex} of $G$.
\end{definition}
\begin{remark}
	It is of note that the star $K_{1,n-1}$ has precisely one dominating vertex  ($n >2 $) and the complete graph $K_{n}$ has all vertices being dominant	.
\end{remark}
\begin{definition}[Regular Graphs]
A graph is called \textbf{regular} if all its degrees are the same. A regular graph of degree $r$ is called an \textbf{$r-$regular} graph.
\end{definition}
 \begin{example}
	 Cycles are $2$-regular graphs and the complete graph $K_{n}$ is a $\left( n-1 \right) $-regular graph.
\end{example} \newline
\begin{remark}
	If $G$ is an $r$-regular graph of order $n$, then $G$ has $\frac{nr}{2}$ edges. This is clear as $2e\left( G \right) = \sum_{u \in V\left( G \right) }^{} d\left( u \right) = \sum_{u \in V\left( G \right) }^{} r = nr$ by the earlier proposition.
\end{remark}
