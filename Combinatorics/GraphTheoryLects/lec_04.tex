\lecture{4}{Wed 18 Aug 2021 21:25}{Cliques, Independence, and Matchings}
\section{Cliques, Independence, and Matchings}
 \begin{definition}[Cliques]
 	A subgraph of $G$ that is complete is called a \textbf{clique} of $G$. A clique of order $k$ is usually denoted as a $k$-clique.
 \end{definition}
 \begin{remark}
 	$2$-cliques are simply edges and $3$-cliques are called triangles.
 \end{remark}
 \begin{definition}[Maximal/Maximum Cliques]
 	A \textbf{maximal clique} in a graph is a clique that is not a contained in any clique of higher order. A \textbf{maximum clique} of a graph is the clique of largest order in the graph.
 \end{definition}
 \begin{remark}
 	Clearly a maximum clique is also maximal, but the reverse may not hold. Additionally, a maximum clique need not be unique.
 \end{remark}
 \begin{definition}[Clique Number]
	 The order of the maximum clique in a graph $G$ is called the \textbf{clique number} of $G$ and is denoted by $\omega  \left( G \right) $.
 \end{definition}
 \begin{problem}
	 $\omega \left( K_{m,n} \right) = 2$ as any set of $3$ or more vertices would necessarily have $2$ vertices contained within the same partite set, and hence not connected.\\
	 $\omega \left( P_{n} \right) = 2$ as every vertex is only connected to the vertices immediately before and after itself, and no two vertices share prior/next vertices.
	 $\omega \left( C_{n} \right) = 2 $ for the same reason, except in the special case $C_3$, where $\omega \left( C_3 \right) = 3$ as $C_3$ is a triangle.\\
 \end{problem}
 Determining $\omega \left( G \right) $ for an arbitrary graph is very hard (NP-complete) and is known as the maximum clique problem.
 \begin{definition}[Independent/Stable sets]
	 A set of vertices $X \subseteq V\left( G \right)$ is called an \textbf{independent set} or \textbf{stable set} of $G$ if no two vertices of $X$ are joined in $G$.
 \end{definition}
 \begin{definition}[Maximal/Maximum Independent Sets]
 	A \textbf{maximal independent set} of $G$ is an independent set that is not contained in a larger independent set. A \textbf{maximum independent set} of $G$ is the largest possible independent set in $G$.
 \end{definition}
 \begin{remark}
 	Again, Maximum is a stronger condition and implies maximal.
 \end{remark}
 \begin{definition}[Independence Number]
	 The order of a maximum independent set in a graph $G$ is called the \textbf{independence number} of $G$ and is denoted by $\alpha \left( G \right) $.
 \end{definition}
 \begin{remark}
	 As these properties are essentially opposite of one another, we notice that any clique in $G$ will be an independent set in $\overline{G}$, hence $\alpha \left( G \right) = \omega \left( \overline{G} \right) $ and vice-versa.
 \end{remark}
 \begin{problem}
	 $\alpha \left( K_{m, n} \right) = \max \left( m, n \right)$. This is clear as any set of members of the same partite set will be mutually disconnected from one another, and any set which contains members from both partite sets necessarily has shared edges.\\
	 $\alpha \left( P_{n} \right)= \left\lceil \frac{n}{2} \right\rceil  $. Again, this is clear as for a path of even length, one need simply take every other member and for a path of odd length, one can take the vertices of odd index.
	 $\alpha \left( C_{n} \right)= \left\lfloor \frac{n}{2} \right\rfloor $. Again, this is clear as one can again take every other vertex for a cycle of even length (and hence take only one end vertex), but for a chain of odd length one cannot take all odd numbered indices as both the first and last vertices are of this form, so one must omit one of the end vertices and hence take less than half.
 \end{problem}
\begin{problem}
	Prove that complete graphs are the only graphs with independence number $1$.
\end{problem}
\begin{solution}
	Suppose $G$ is a  graph with $\alpha  \left( G \right) = 1$. Then, by definition, for any pair of vertices $u, v \in V\left( G \right) $, $uv \in E\left( G \right) $ else $\left\{ u, v \right\} $ would be an independent set of $G$ of size $2$. Thus as all vertices have an edge connecting them,  $G$ is complete.
\end{solution}
\begin{problem}
	Prove that if $G$ is a graph of order $n$, then $\alpha \left( G \right) + \delta \left( G \right)  \le n$.
\end{problem}
\begin{solution}
	Let $\alpha \left( G \right) = a$ and $\delta \left( G \right) = d$. Then, let $A =  \{v_1, v_2, \ldots, v_{a}\} $ be a maximal independent set of $G$. As  $d\left( v_{i} \right) \ge \delta \left( G \right)  $ for every $1 \le i \le n$, then by the definition of independent sets, we must have $v_1$ has at least $d$ neighboring vertices which are not a part of $A$. Hence $v\left( G \right) = n \ge   \left| A \right|  + d = \alpha \left( G \right) + \delta \left( G \right) $ vertices.
\end{solution}
\begin{problem}
	Let $n \ge 2 \delta \ge 2$. Construct a graph of order $n$ such that $\alpha \left( G \right)  + \delta \left( G \right)  = n$.
\end{problem}
\begin{solution}
	$K_{\delta, n- \delta}$ satisfies this as $\alpha \left( K_{\delta, n-\delta} \right) = \max\left( \delta, n- \delta \right) $ and $\delta \left( K_{\delta, n- \delta} \right) = \min \left( \delta, n-\delta \right) $ and $\max\left( \delta, n- \delta \right)  + \min \left( \delta, n-\delta \right) = \delta + n-\delta = n$.
\end{solution}
\begin{definition}[Matching]
	A set of vertex disjoint edges in a graph $G$ is called a \textbf{matching} of $G$. A matching of cardinality $k$ is called a $k$-matching of $G$.
\end{definition}
\begin{definition}[Covering]
	If a vertex $v$ belongs to an edge of a matching $M$, we say that $M$ \textbf{covers} $v$.
\end{definition}
\begin{remark}
Note that vertices covered by $M$ may have edges not contained within $M$.
\end{remark}
 \begin{problem}
	Check that $K_4$ has three $2$-matchings.
\end{problem}
\begin{solution}
\begin{figure}[ht]
    \centering
    \incfig{matchingprob}
    \caption{The three 2-matching of $K_4$}
    \label{fig:matchingprob}
\end{figure}
\end{solution}
\begin{problem}
	Prove that the number of $2$-matchings of $C_{n}$ is $\frac{n\left( n-3 \right) }{2}$.
\end{problem}
\begin{solution}
	We know $e \left( C_{n} \right)= n $, so there are $\binom{n}{2}$ pairs of edges is $C_{n}$. As every vertex has $2$ unique edges which share it there are precisely $n$ pairs of edges which share a vertex, while all others are mutually disjoint (hence a $2$-matching). So, the total number of ways is $\binom{n}{2}- n =  \frac{n\left( n-1 \right) }{2} - n = \frac{n \left( n-3 \right) }{2}$.
\end{solution}
\begin{problem}
	Let $n\ge 4$. Prove that the number of $2$-matchings of $K_{n}$ is $3\binom{n}{4}$.
\end{problem}
\begin{solution}
	It is clear that every $2$-matching covers exactly $4$ vertices, hence there are $\binom{n}{4}$ possible sets of vertices which one could make a $2$ matching with. As each of these sets of $4$ vertices forms a clique which is isomorphic to $K_4$ and we know there are $3$ ways in which to form a $2$-matching on $K_4$, we see there are $3\binom{n}{4}$ $2$-matchings on $K_{n}$.
\end{solution}
\begin{definition}[Maximal/Maximum Matchings]
	A \textbf{maximal matching} in a graph is a matching which is not contained within a larger matching. A \textbf{maximum matching} is the largest possible matching in a graph.
\end{definition}
\begin{definition}[Matching Number]
	The \textbf{matching number} of a graph $G$ is the number of edges in a maximum matching of $G$. This is denoted $\beta \left( G \right) $.
\end{definition}
\begin{problem}
	$\beta \left( P_{n} \right) = \left\lfloor \frac{n}{2} \right\rfloor$ as one can take pairs of each odd index and the following even index and the edges between them will have no mutual vertices.\\
	$\beta \left( C_{n} \right) = \left\lfloor \frac{n}{2} \right\rfloor $ by the same reasoning.\\
	$\beta \left( K_{n} \right) = \left\lfloor \frac{n}{2} \right\rfloor$ as we can partition the vertices into two sets of equal or near equal size and take edges which connect a vertex in the first set to one in the second set.\\
	$\beta \left( K_{m, n} \right) = \min \left( m, n \right) $. Let  $A$ and $B$ be the partite sets and WLOG let $\left| A \right| \le \left| B \right| $. For each vertex in $A$, take an edge going to a unique vertex in $B$. As the remaining vertices in $B$ are not connected to each other, but only to those in $A$, we see that there can only be $\left| A \right|$ possible edges in this matching.
\end{problem}
