\lecture{3}{Tue 17 Aug 2021 14:13}{Trees and Connectivity}
\section{Trees and Connectivity}

\begin{definition}[Acyclic]
	A graph is \textbf{acyclic} if it contains no cycle. Being acyclic is a \textbf{monotone property}, hence every subgraph of an acyclic graph is itself acyclic. Other monotone properties are bipartiteness and containing no triangles.
\end{definition}
\begin{definition}[Trees]
	A \textbf{tree} is a connected acyclic graph.
\end{definition}
\begin{definition}[Leaves]
	A vertex of degree one in a tree is called a \textbf{leaf}.
\end{definition}
\begin{theorem}
	Every tree has atleast two leaves.
\end{theorem}
\begin{proof}
	Let $P = u_1, u_2, \ldots, u_{k}$ be a maximal path in the tree $T$. We shall show that  $d\left( u_1 \right) = d\left( u_{k} \right) = 1 $. Suppose $d\left( u_1 \right) > 1$, then it has a neighbor besides $u_2$. Either this neighbor is already a member of $P$, in which case there is a cycle $C = u_1, u_2, \ldots, u_{j}, u_1$, $\contra$. Otherwise, this neighbor is not a member of $P$. Denote the neighbor $u$, and note that $u, u_1, u_2, \ldots, u_{k}$ would be a longer path hence $P$ is not maximal. $\contra$. Therefore $u_1$ and $u_{k}$ are leaves.
\end{proof}
\begin{example}
	Paths and Stars are both trees.
	We may also construct new examples of trees by taking two disjoint trees and joining them with a single new edge. We know the new graph will be connected by a previous theorem, and as they are only joined by one edge, there cannot exist a cycle between them, hence our new construction is in fact a tree.
\end{example}
\begin{proposition}
	For every two vertices of a tree, there is a unique path joining them.
\end{proposition}
\begin{proof}
	Suppose there are two paths $P_1 = u_1, u_2, \ldots, u_{k}$ and $P_2 = u_1, v_2, \ldots, v_{k-1}, u_{k}$ joining $u_1$ and $u_{k}$. Suppose $P_1$ and $P_2$ share an internal vertex $u_{i}$, then $u_1, u_2, \ldots, u_{i}, v_{i-1}, v_{i-2}, \ldots, v_2, u_1$ is a path, $\contra $. Thus, suppose $P_1$ and $P_2$ have disjoint internal vertices. Then, $u_1, u_2, \ldots, u_{k}, v_{k-1}, \ldots, v_2, u_1$ is a path, $\contra$.
\end{proof}
\begin{proposition}
	If a connected graph $G$ has a cycle, then for any edge $e$ from the cycle, the graph $G - e$ is also connected. This is trivial as removing the edge $e$ yields a path as we have proven earlier and it is known that all paths are connected $n\ge 2$.
\end{proposition}
\begin{corollary}
	If $G$ is a connected graph such that deleting any edge of $G$ makes it disconnected, then $G$ does not contain any cycles and hence it is a tree. This is clear by the contrapositive of the previous statement.
\end{corollary}
\begin{proposition}
	Every connected graph has a spanning tree.
\end{proposition}
\begin{proof}
	Let $G$ be connected. If $G$ has no cycles it is a spanning tree of itself. Otherwise, we know deleting any edge $e$ from a cycle of $G$ will leave $G$ connected. Repeat this process until $G$ has no more cycles and we now have a tree,
\end{proof}
\begin{proposition}
	Deleting an edge from a tree makes it a disconnected graph.
\end{proposition}
\begin{proof}
	Suppose $T$ is a tree and $\left\{ u, v \right\} $ is an edge of $T$. Now suppose $T - \left\{ u, v \right\} $ is still connected. Then, we must have that there is a path from $u$ to $v$, denoted $P = u, u_1, u_2, \ldots, u_{k}, v$. Hence adding back the edge we deleted yields a cycle $C = u, u_1, u_2, \ldots, u_{k}, v, u$. $\contra$. Thus,  $T - \left\{ u, v \right\} $ must be disconnected.
\end{proof}
\begin{proposition}
	The removal of a leaf from a tree yields a tree.
\end{proposition}
\begin{proof}
	Let $v$ be a leaf of the tree $T$. As $d\left( v \right) = 1$, it cannot be an internal vertex as its degree would be at least $2$. Hence, any path between members $a, b \in V\left( T \right) $ , $a, b \neq v$, would still be a path. Furthermore, removing a vertex could not possibly create a cycle, hence $T - v$ is a tree.
\end{proof}
\begin{proposition}
	For every tree $T$ we have $e\left( T \right)  = v\left( T \right) - 1$.
\end{proposition}
\begin{proof}
	We induce on $v\left( T \right) $. For the case $v\left( T \right) = 1$ there are trivially no edges, hence the statement holds. Now, assume the statement holds for $v\left( T \right) = n-1$. Let $T$ be a tree of order $n$. Then, removing a leaf, $v$, will produce a tree of order $n-1$ for which $e\left( T \right) = n-2$. As $d\left( v \right) = 1$, we see adding the leaf and its single edge back yields $e\left( T \right)  = n-1 = v\left( T \right) -1$.
\end{proof}
\begin{definition}[Cuts]
	Denote $c\left( G \right) $ to be the number of components of a graph $G$. An edge $e$ is called a  \textbf{cut edge} of $G$ if $c\left( G - e \right) > c\left( G \right) $. Similarly, a vertex $v$ of $G$ is called a \textbf{cut vertex} of $G$ if $c\left( G - v \right) > c\left( G \right) $.
\end{definition}
\begin{example}
	Every edge in a tree is a cut edge.\\
	Every non-leaf in a tree is a cut vertex.\\
	$P_{n}$ has the most cut vertices of all trees of order $n$.\\
	Cycles and complete graphs have no cut vertices.\\
	Complete bipartite graphs (besides stars) have no cut vertices. In a star, only the vertex which is connected to all others is a cut vertex.\\
\end{example}
\begin{definition}[Seperability]
	A connected graph $G$ is called \textbf{nonseparable} if it has no cut vertices. A graph which is not nonseparable is \textbf{separable}.
\end{definition}
\begin{example}
	The only nonseparable graph of order $2$ is $K_2$.\\
	Trees of order $\ge 3$ are separable as any non-leaf is a cut vertex.\\
	Cycles and complete graphs are nonseparable.\\
\end{example}
\begin{theorem}
	A connected graph of order $\ge 3$ is nonseparable if and only if every pair of vertices are joined by two internally disjoint paths.
\end{theorem}
\begin{proof}
	Suppose $G$ is connected, nonseparable, and of order $\ge 3$. Let $u, v \in V\left( G \right) $ and suppose there is only one internally disjoint path $P$ between $u, v$. Clearly, removing an internal vertex, $w$, from this path yields a separation of $G$. Let $U, V$ be the components containing $u$ and $v$ respectively after this cut. Prior to the cut there was a path between $U$ and $V$, so they were subgraphs of the same component. After the separation, there are no paths joining them, so they are different components, hence the number of components has increased by  $1$ at the least. Hence,  $G$ is separable. $\contra$
	Now, suppose $G$ is connected and of order $\ge 3$ and let $u, v \in V\left( G \right) $ and assume for all such $u, v$ there are two internally disjoint paths connecting them. As $u, v$ are connected, let $C$ be the component containing them and suppose we remove a vertex from one of the paths connecting $u$ to $v$. As the paths are internally disjoint, the path which we did not remove an element from is still a path, hence $u, v \in C$ still. Thus, $G$ is nonseparable.
\end{proof}
\begin{corollary}
	A connected graph is nonseparable if and only if any two of its edges lie in a common cycle.
\end{corollary}
\begin{problem}
	If $G$ has a nonseparable spanning subgraph, then $G$ is nonseparable.
\end{problem}
\begin{solution}
	Let $G$ be a graph, $H$ be a nonseparable spanning subgraph of $G$, and $u, v \in G$. Then, $u, v \in H$. Let $C$ be the cycle which $u, v$ are mutually part of and note that this implies $C \in G$. As two arbitrary points have a mutual cycle in $G$, then $G$ is nonseparable.
\end{solution}

 \begin{problem}
	Let $G$ be a connected graph of order $\ge 3$. If $\left\{  u, v \right\} $ is a cut edge of $G$, then either $u$ or $ v$ is a cut vertex of $G$.
\end{problem}
 \begin{solution}
	 Notice that the vertex deletion of $u$ or $v$ will also result in the deletion of $\left\{ u, v \right\} $. We know such a deletion causes a separation of $G$, hence the deletion of $u$ or $v$ causes a separation of $G$, so $u$ and $v$ are cut vertices of $G$. For a slightly more formal argument, one can examine the order of the components generated by the edge deletion. As  $\left\{ u, v \right\} $ is a cut edge, we know $G - uv$ has two components. Denote the one containing $u$ to be $C_{u}$ and the come containing $v$ to be $C_{v}$. Then, as $v\left( C_{u} \right) + v\left( C_{v} \right) = v\left( G \right) \ge 3$ by hypothesis, we know $v\left( C_{u} \right) \ge 2$ or $v\left( C_{v} \right) \ge 2$. WLOG, let $v\left( C_{v} \right) \ge 2$, then we see that $C_{u}$ and $C_{v} - v$ both have at least one vertex and are disjoint by hypothesis. Hence,  $C_{u}$ and $C_{v} - v$ form a separation of $V$ and we see that $v$ or $u$ or both are cut vertices of $G$. The reason the first argument fails is illustrated in the following figure (This is also why $u$ and $v$ may not both be cut vertices):
\begin{figure}[ht]
    \centering
    \incfig{cutvertthm2}
    \caption{$uv$ is a cut edge, $v$ is a cut vertex, but deleting $u$ does not result in a separation.}
    \label{fig:cutvertthm2}
\end{figure}
\end{solution}
\begin{definition}[Vertex Connectivity]
	The \textbf{vertex connectivity} of a noncomplete graph $G$ is the minimum number of vertices that have to be removed to obtain a disconnected graph. We denote this by $\kappa \left( G \right) $. The reason we exclude complete graphs is that $\kappa \left( K_{n} \right) = v\left( K_{n} \right) $, by this definition. For this reason we usually define $\kappa \left( K_{n} \right)  = n-1$ in order to ensure certain theorems hold for complete and noncomplete graphs.
\end{definition}
\begin{definition}[k-Connectivity]
	A graph $G$ is  \textbf{$k$-connected}	if $\kappa \left( G \right)  \ge k$.
\end{definition}
This definition further generalized our terms from earlier, we see
\begin{example}
	A $1$-connected graph is a connected graph.\\
	A $2$-connected graph is a nonseparable graph.\\
\end{example}
\begin{problem}
	If $m\ge n$, then $K_{m, n}$ is $n$-connected.
\end{problem}
\begin{solution}
	WLOG, let $A$, and $B$ be the partite sets of $K_{m,n}$ such that $\left| A \right| = m$ and $\left| B \right| = n$. Now, define $\kappa \left( K_{m,n} \right)= k$ and let $x_1, x_2, \ldots, x_{k}$ be the vertices such that $K_{m,n} - x_1 - x_2 - \ldots - x_{k}$ is disconnected and define $X = \left\{ x_1, x_2, \ldots, x_{k} \right\} $. If $k < n$, then $B \setminus X \neq \O  $ and $A \setminus X \neq \O$, so, there remain elements of $A$ which share a connection to all remaining elements of $B$, and vice-versa, hence $K_{m,n}- X$ is still a complete bipartite graph and is thus connected. Thus, $K_{m, n}$ is at least $n$-connected.
\end{solution}
\begin{theorem}[Menger's Theorem]
	Let $G$ be a graph. If $u, v$ are two nonadjacent vertices of $G$, then there are at least $\kappa \left( G \right) $  pairwise internally disjoint paths joining $u$ and $v$.
\end{theorem}
We see this is essentially a generalization of the earlier theorem concerning nonseparablity and the existence of  $2$ pairwise internally disjoint paths connecting vertices. The proof of this follows from the generalization of such arguments.
\begin{proof}
	Let $G$ be a graph, $u$ and $v$ be two nonadjacent vertices of $G$, and let $\kappa \left( G \right) = k$. Suppose there are $m$ internally disjoint paths joining $u$ and $v$ and suppose $m < k$. Define these paths to be $P_{i} = u, u_{i, 1}, u _{i, 2}, \ldots, u_{i, j_{i}}, v$. Now, let $k_{i} \in \N$ for each $1 \le i \le m$ and suppose we delete  $u_{i, k_{i}}$ from each path. Now, as each path is broken, we must have that there are no paths joining $u, v$, but as we only removed $m$ vertices this implies $\kappa \left( G \right) \le m$ $\contra$. Hence, as the selection of $u$ and $v$ was arbitrary, except that they must not be neighbors, there must be $k$ pairwise internally disjoint paths connecting each non-neighboring pair of vertices.
\end{proof}
\begin{lemma}[Fan Lemma]
	Let $G$ be a $k$-connected graph, and let $x, y_1, \ldots, y_{k}$ be distinct vertices of $G$. Then, there exist $k$ pairwise internally disjoint paths joining $x$ to $y_1, \ldots, y_{k}$.
\end{lemma}
\begin{proof}
	Denote a vertex not already in $G$ to be $v$. Append  $v$ to $G$ and add an edge connecting each $ y_{i}$ to $v$ for $1 \le i \le k$ and denote the new graph $ G^{\prime}$. As $G$ was $k$-connected and we would have to remove $k$ vertices to isolate $v$, we see $G^{\prime}$ is $k$-connected. Then, by Menger's lemma there are $k$ pairwise internally disjoint paths from $x$ to $v$. As $v$ has $k$ edged, each with a $ y_{i}$, then we must have that each path contains exactly and uniquely one $ y_{i}$. Hence, by removing $v$ and examining the paths with $v$ removed from the end, we see each path now has end vertices $x$ and $ y_{i}$ and, as they are already known to be internally disjoint, this completes the proof.
\end{proof}
\begin{lemma}[Equivalent Fan Lemma]
	Let $G$ be a $k$-connected graph, and let $X$, $Y$ be subsets of $V\left( G \right) $ such that $\left| X \right| = \left| Y \right| = k$. Then, there are $k$ pairwise disjoint paths joining a vertex of $X$ to a vertex of $Y$.
\end{lemma}
\begin{proof}
	Again, let us add a vertex $v$ to $G$ such that $v$ has $k$ vertices, one to each $ y_{i}$. Similairly, add a vertex $u$ to $G$ such that $u$ has $k$ edges, one to each $x_{i}$ and denote the new graph $G^{\prime}$. Again, as $G$ was $k$-connected and each of $u$ and $v$ have $k$ edges, it would require $k$ cuts in order to isolate $u$ or $v$, hence $G^{\prime}$ is $k$-connected. Thus, there exist $k$ pairwise internally disjoint paths from $u$ to $v$. Now, removing $u$ and $v$ and examining the paths (with $u$ and $v$ removed from the ends) yields $k$ paths connecting each $x_{i}$ to the corresponding $y_{i}$ and, as each path was internally disjoint from all others, this completes the proof.
\end{proof}

\begin{definition}[Edge Connectivity]
	The \textbf{edge connectivity} of a graph $G$ is the minimum number of edges that must be removed in order to obtain a disconnected graph. This is denoted by  $\kappa^{\prime} \left( G \right) $.
\end{definition}
\begin{example}
	Any tree $T$ of order $\ge 2$ has $\kappa^{\prime} \left( T \right)  = 1$.\\
	Any cycle $C$ has $\kappa^{\prime} \left( C \right) = 2$.\\
	$\kappa^{\prime} \left( K_{n} \right) = n-1 $.\\
	Let $G$ be the graph formed by joining $2$ cycles with a common vertex. It is clear that $\kappa \left( G \right) = 1$ as removing the common vertex separates $G$, bus $\kappa^{\prime} \left( G \right) = 2$ as each cycle has $2$ internally disjoint paths from each element to the common vertex.
\end{example}
 \begin{theorem}
	 For any graph $G$, we have $\kappa \left( G \right)  \le \kappa^{\prime} \left( G \right) \le \delta \left( G \right)$.
\end{theorem}
\begin{proof}
	First let us show that $\kappa \left( G \right)  \le \kappa^{\prime}\left( G \right) $. Suppose $\kappa^{\prime} \left( G \right) = k < \kappa \left( G \right) $ for a graph $G$ and denote the edges which must be removed by $e_1, e_2, \ldots, e_{k}$. Let $u_1, u_2, \ldots, u_{k}$ to be vertices connected to their respective edge $e_{i}$. Then, $E\left(G - u_1 - u_2, - \ldots - u_{k}\right) \subseteq E\left( G - e_1 - e_2 - \ldots - e_{k} \right) $ by definition of vertex deletion. Hence,  $G - u_1 - u_2 - \ldots - u_{k}$ is disconnected (as it contains at most the same edges as $G - e_1 - e_2 - ... - e_{k}$ which is disconnected by hypothesis). Hence, $\kappa\left( G \right) \le \kappa^{\prime}\left( G \right) $ $\contra$. Now, let us show $\kappa^{\prime}\left( G \right) \le \delta \left( G \right) $. Suppose $\kappa^{\prime}\left( G \right) > \delta\left( G \right) = d$. Then, let $u$ be a vertex of $G$ with exactly $\delta\left( G \right) $ adjacent edges, $e_1, e_2, \ldots, e_{d}$. Then, $G - e_1 - e_2 - \ldots - e_{d}$ would leave $u$ with no edges joining to it, and as $v\left( G \right) > 1$ by the assumption that $e\left( G \right) > 0$ as $\kappa^{\prime}\left( G \right) > \delta\left( G \right) \ge 0$, then we see $u$ is an isolated vertex and hence $G - e_1 - e_2 - \ldots - e_{d}$ is disconnected, so $\kapp^{\prime}\left( G \right) <= \delta \left( G \right) $ $\contra$.
\end{proof}
\begin{remark}
	For trees, cycles, and complete graphs, $\kappa \left( G \right)  = \kappa^{\prime} \left( G \right) = \delta \left( G \right) $
\end{remark}
