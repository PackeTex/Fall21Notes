\lecture{2}{Tue 17 Aug 2021 12:16}{More Basics}
\section{More Graph Basics}
Given a graph $G$, there are two basic operations to obtain smaller graphs from $G$, \textbf{vertex deletion} and \textbf{edge deletion}. These names are self explanatory but let us rigorously define them.
\begin{definition}[Vertex Deletion]
	To delete a vertex $v$ from the graph $G$, one removes $v$ from $V$ and removes all edges containing $v$ from $E$. These deletions can be iterated until there are no vertices remaining in the graph. The order of such deletions does not matter and the graph obtained by removing $v$ is denoted $G - v$ or $G - \left\{ v \right\} $.
\end{definition}
\begin{definition}[Edge Deletion]
Similarly, deleting an edge $ \left\{ u,v \right\} $ from $G$ entails removing the edge  $uv$ from $E$, but in this case the vertices $u$ and $v$ are allowed to remain in $V$. Again, this can be iterated as long as their are edges remaining in the graph, and the order of such deletions does not matter. We denote the graph $G$ with edge $e$ removed by $G - e$.
\end{definition}
\begin{example}
Deleting an end-vertex from $P_{n}$ produces $P_{n-1}$, or in the special case of $P_2$ it produces two isolated vertices.\\
Deleting any vertex from $C_{n}$ produces $P_{n-1}$.\\
Deleting any edge from $C_{n}$ produces $P_{n}$.\\
Deleting any vertex from $K_{n}$ produces $K_{n-1}$ provided $n\ge 2$.\\
Deleting any edge from $K_{1,n}$ produces an isolated vertex.\\
\end{example}
\begin{definition}[Subgraphs]
	A graph $H$ is a \textbf{subgraph} of $G$ if $V\left( H \right) \subseteq V \left( G \right) $	 and $E\left( H \right) \subseteq E\left( G \right) $. Equivalently we say $G$ contains $H$ or $H$ is contained in $G$. One way to produce $H$ from $G$ is to delete all vertices and edges that are not in $V\left( H \right) $ and $E\left( H \right) $ respectively, until one obtains $H$.
\end{definition}
\begin{definition}[Induced Subgraph]
	A graph $H$ is called an \textbf{induced subgraph} of $G$ if $V\left( H \right) \subseteq v\left( G \right) $ and $E\left( H \right) $ is the set of all edges in $E\left( G \right) $ which join vertices within $H$. To obtain the induced subgraph $H$, one need only delete the vertices of $G$ which are not in $V\left( H \right) $.
\end{definition}
\begin{example}
	Every graph of order $\le n$ is a subgraph of $K_{n}$.\\
	Every induced subgraph of $K_{n}$ is complete.\\
	The path $P_{n}$ is an induced subgraph of the path $P_{n+1}$.\\
	The path $P_{n}$ is an induced subgraph of the cycle $C_{n+1}$.\\
	The path $P_{n}$ is a subgraph of the cycle $C_{n}$.\\
	The star $K_{1,n}$ is an induced subgraph of the star $K_{1,n+1}$.\\
	Every graph is an (induced) subgraph of itself.\\
\end{example}
\begin{definition}[Spanning Subgraphs]
	If $H$ is a subgraph of $G$ and $V\left( H \right) = V\left( G \right) $, then $H$ is called a \textbf{spanning subgraph} of $G$. Similarly to an induced subgraph, a spanning subgraph can be obtained by deleting all edges which are not a part of $E\left( H \right) $.
\end{definition}
\begin{example}
	$C_{n}$ contains a spanning path $P_{n}$.\\
	$K_{n}$ contains a spanning star $K_{1,n+1}$.\\
	An even cycle $C_{2n}$ contains a spanning graph with $n$ disjoint edges (deleting every other edge).
\end{example}
\begin{definition}[Bipartite Graphs]
	A graph is \textbf{bipartite} if its vertices can be split into two disjoint sets $A$ and $B$ such that all edges of the graph have one vertex in $A$ and one vertex in $B$. We call $A$ and $B$ \textbf{partite sets}. An equivalent definition is that a graph $G$ is bipartite if its vertices can be split into two disjoint sets $A$ and $B$ such that no edge connects two vertices in $A$ or two vertices in $B$. A final way of defining bipartite graphs is with graph coloring. A graph $G$ is bipartite if its vertices can be $2$-colored.
\end{definition}
\begin{example}
	Every path is bipartite (color odd numbered vertices red and even numbered vertices blue).\\
	Every cycle of even order is bipartite by the same coloring (as the last vertex will be even and the first vertex will be odd, so this does not break the coloring).\\
	The star $K_{1,n}$ is bipartite (by definition).
\end{example}
\begin{proposition}
	If $G$ is a bipartite graph and $A,B$ are its partite sets, then \[
		\sum_{u \in A}^{} d\left( u \right)  = e\left( G \right) = \sum_{u \in B}^{} d \left( u \right)
	.\]
	This fact is trivial by the fact that every edge will contain one edge in $A$ (and $B$ ), and so a given sum will count each edge once.
\end{proposition}
\begin{proposition}[Regular Bipartite Graphs]
	If a nonempty graph is bipartite and regular, then its partite sets $A$ and $B$ are of equal size. By the previous proposition, we know that $e \left( G \right) = \sum_{ u \in A}^{} d\left( u \right) = r \left| A \right| $. Similarly, $e \left( G \right) = r \left| B \right| $. Hence, by the counting argument, we have $\left| A \right|  = \left| B \right| $.
\end{proposition}
\begin{corollary}
	Cycles of odd order are nonbipartite. Suppose a cycle of odd order is bipartite, then as every cycle is $2$-regular we would have that its partite sets are of equal size and hence their combined size is even, but this is precisely the order of the cycle.  $\contra$.
\end{corollary}
\begin{definition}[Complete Bipartite Graphs]
	Let $G$ be bipartite with partite sets $A$ and $B$. The graph is \textbf{complete bipartite} if it contains all possible edges $\left\{ u,,v \right\} $ where $u \in A$ and $v \in B$. A complete bipartite graph with partite sets $A$ and $B$ is denoted $K_{a,b}$ where $a = \left| A \right| $ and $b = \left| B \right| $. By a simple counting argument, we see $e\left( G \right) = \left| A \right| \left| B \right| $.
\end{definition}
\begin{example}
	Every star $K_{1,n}$ is trivially complete bipartite.\\
	The $4$-cycle is isomorphic to $K_{2,2}$ mapping evens in $A$ and odds to $B$ but this is the only cycle which is complete bipartite.
\end{example}
\begin{problem}
	What is the maximum number of edges that a bipartite graph of order $n$ can have.
\end{problem}
As there are finitely many graphs of order $n$, we know there must be a finite answer for every $n$. Furthermore, if $G$ is a bipartite graph of order  $n$ which satisfies having the maximum number of edges, then it must be complete bipartite, else we could add edges until it became complete bipartite. So, we must find the maximum number of edges for complete bipartite graphs $K_{a,n-a}$ where $1\le a < n$.
\begin{solution}
	As $\left| A \right| = a$ and $\left| B \right| = n-a$ for such a complete bipartite graph, and we know that $e\left( K_{a, n-a} \right) =\left| A \right| \left| B \right| = a\left( n-a \right) $, we must simply solve this maximization problem for integral $a$. First, solving the continuous case yields a maximum at $\frac{n}{2}$, so we must simply find the integers closest to this analytic maximum. Suppose  $\frac{n}{2}$ is an integer, then this is simply our solution (for $a$) yielding a maximum size of  $\frac{n^{2}}{4}$. Else, the two closest integers will be the one directly beneath $\frac{n}{2}$, $\left\lfloor \frac{n}{2} \right\rfloor$, and the integer directly above, $\left\lceil \frac{n}{2} \right\rceil $. Taking $a = \left\lfloor \frac{n}{2} \right\rfloor$, $n-a = \left\lceil \frac{n}{2} \right\rceil $ yields a maximum size of $\left\lfloor \frac{n^2}{4} \right\rfloor$.
\end{solution}
\begin{definition}[Graph Complement]
	Let $G = \left( V,E \right) $ be a graph. The \textbf{complement} of $G$ is a graph $\overline{G}$ whose vertex set is $V$ and whose edge set is the "complement" of $E$, that being all edges between two members of $V$ which are not  a part of $E\left( G \right) $.
\end{definition}
\begin{example}
	$\overline{\overline{G}}= G$.\\
	$\overline{K_{n}}$ is an edgeless graph of order $n$.\\
	If $v\left( G \right)  = n$ and $e\left( G \right)  = m$, then $v\left( \overline{G} \right) = n$ and $e\left( \overline{G} \right)  = \binom{n}{2}- m$. This is clear as there are $\binom{n}{2}$ possible edges and $\overline{G}$ will have all of those which are not present in $E\left( G \right) $.\\
	For all $u \in V$, $d_{G} \left( u \right)  + d_{\overline{G}}\left( u \right) = n-1$. Again, this is clear as there are $n-1$ other vertices to which $u$ can connect and if it is not connected to one in $G$, it will be connected in $\overline{G}$ and vice-versa.\\
	More generally, $\delta \left( G \right) + \Delta \left( \overline{G} \right) = \Delta \left( G \right)  + \delta \left( \overline{G} \right)  = n-1$ for essentially the same reason.
\end{example}
\begin{definition}[Disconnected Graphs]
	A graph $G$ is \textbf{disconnected} if $V\left( G \right) $ can be split into two disjoint nonempty sets $A, B$ such that there are no members of $E \left( G \right) $ which join a vertex from $A$ to one from $B$. This is essentially the opposite of a bipartite graph.
\end{definition}
\begin{example}
	$\overline{K_{n}}$ for $n\ge 2$ has $n$ isolated vertices and is hence trivially disconnected ($A$ and $B$ can be any two nontrivial subsets of $E$.\\
	$\overline{C_4}$ has two disjoint edges and is disconnected. This is because $C_4$ is isomorphic to $K_{2,2}$ and upon taking the complement this yields two sets of vertices joined to each other which are disconnected from the opposing pair.\\
	$\overline{K_{a,b}}$ consists of disjoint  $K_{a}$ and $K_{b}$ and is disconnected. This is essentially a generalization of the previous argument.
\end{example}
\begin{definition}[Connected Graphs]
	A graph which is not disconnected is called \textbf{connected}.
\end{definition}
\begin{theorem}
	A graph $G$ is connected if and only if for every pair of vertices $u$ and $v$, $G$ contains a path with $u$ and $v$ as end vertices.
\end{theorem}
\begin{proof}
	Suppose the negation. Then, there are two vertices $u, v$ such that there is no path with $u$ and $v$ as end vertices. Let  $U$ be the set of all vertices which share a path with $u$ and likewise $V$ be a path of all vertices which share an endpoint with $V$.  As $u \in U$ and $v \in V$ these sets are nonempty. Now, supposes there is a path with one end vertex in $U$ and one end vertex in $V$, then if the path contains $u$ and $v$, one could delete vertices from both ends to produce a path from $u$ to $v$. Otherwise, if one or both of $u$ and $v$ are not already in the path, one could append the path from the end vertex which is in $U$ to $u$ and similarly for the end vertex of $V$ to $v$ in order to produce a path from $u$ to $v$. Hence, there are no paths with vertices in both $U$ and $V$, hence there are no edges between $U$ and $V$, so $G$ is disconnected.$\contra$\\
	Now, suppose $G$ is connected and let $u, v \in V\left( G \right) $. Suppose there is not a path between $u$ and $v$, then taking $U$ to be the set of all points with paths to $u$ and $V$ to be all points with paths to $v$ yields a seperation of the graph $G$, hence $G$ is disconnected. $\contra$.
\end{proof}
\begin{example}
	Paths, cycles, starts, complete graphs, and complete bipartite graphs are all connected.
\end{example}
\begin{proposition}
	If a graph $G$ contains a connected spanning subgraph, then $G$ is connected.
\end{proposition}
\begin{proof}
	Let $H$ be such a connected spanning subgraph.$E\left( H \right)  \subseteq E\left( G \right) $ and $V\left( H  \right) = V\left( G \right)$. Let $u, v$ be two points in $G$. As $H$ is connected, there exists a path $P$ between $u$ and $v$. As every edge of this path is contained in $E\left( G \right) $, we have that this path is also a subgraph of $G$, and hence there is a path in $G$ which connects $u, v$, so $G$ is connected.
\end{proof}
\begin{theorem}
	If $G$ is disconnected, then its complement is connected.
\end{theorem}
\begin{proof}
	Let $V\left( G \right) $ be seperated into disjoint sets $A$ and $B$ who form a seperation of $G$. Then, we see for every $u \in A$, $v \in B$ we must have $\left\{ u,v \right\}  \in E\left( \overline{G} \right) $. Hence $\overline{G}$contains a complete bipartite graph with partite sets $A$ and $B$ denoted as $H$. However, as $A \cup B = V\left( \overline{G} \right) $, $H$ is a spanning subgraph of $\overline{G}$, hence $\overline{G}$ is connected.\\
	Note that the converse is not necessarilly true, $\overline{G}$ is connected does not imply $G$ is disconnected.
\end{proof}
It is clear that a connected graph of order $\ge 2$ has no isolated vertices.
\begin{theorem}
	If $u \in V\left( G \right) $ such that every vertex $v \in V\left( G \right) $ can be joined to $u$ be a path, then $G$ is connected. This follows directly from the argument of theorem 2.1.
\end{theorem}
\begin{theorem}
	if $G$, $H$ are connected graphs with a common vertex, then their union is also a connected graph. This follows from theorem $2.1$, taking points $g \in V\left( G \right) $, $h \in V\left( H \right) $ and letting $P_{g}$, $P_{h}$ be paths to their common element. Then, joining these paths together creates a new path between any members of $G$ and members of $H$. For pairs of members both in $G$ or both in $H$, the connectedness of $G$ and $H$ respectively satisfies these paths. Hence the union is connected.
\end{theorem}
\begin{corollary}
	Let $G$ be connected. Adding a vertex $v$ to $V\left( G \right) $ and joining it to an existing vertex $u \in V\left( G \right) $ yields a new connected graph. This follows directly from the previous theorem. Let $H$ be a graph such that $V\left( H \right)  = \left\{ v, u \right\} $ and $E\left( H \right)  = \left\{ \left\{ v, u \right\}  \right\} $. Then,  $H$ is connected and it shares a common point with $G$, so their union is connected.
\end{corollary}
\begin{theorem}
	A connected graph of order $n$ has atleast $n-1$ edges.
\end{theorem}
\begin{proof}
	We shall induce on $n$. For the case $n=2$, suppose the graph had less than $1$, edge, then it has $0$ edges but $2$ vertices so it is clearly disconnected, taking each singleton as the partitions forms a seperation.
	Now, we assume that the case  $n -1$ is true. Suppose we have a connected graph of order  $n$, then let $A$ be a connected subgraph of order $\le n-1$ and $B = \overline{A}$. Clearly $A \cup B = G$, WLOG assume $v\left( A \right)  = n-1$, then $e\left( A \right) \ge n-2$ and, by the preceding corollary, we may construct $G$ by taking the union  of $A$ and $B$ and adding one edge. Hence, $e\left( G \right) \ge n-1$.
\end{proof}
\begin{definition}[Distance]
	Let $G$ be connected. For any two vertices, $u, v \in V\left( G \right) $ the length of the shortest path which joins them is called the \textbf{distance} between $u$ and $v$ and this satisfies all the normal properties of a distance function.
\end{definition}
It is clear that $u, v$ are adjacent if and only if $\dist \left( u, v \right) = 1$.
 \begin{definition}[Diameter]
	 Given a connected graph $G$, the \textbf{diameter} of $G$ is defined as $\diam \left( G \right)  = \max _{u, v \in G} \dist \left( u, v \right) $.
\end{definition}
\begin{example}
	$\diam \left( K_{n} \right) = 1$.\\
	$\diam \left( K_{a, b} \right) = 2 $.\\
	$\diam \left( C_{n} \right) = \left\lfloor \frac{n}{2} \right\rfloor$.\\
	$\diam \left( P_{n} \right) = n-1$.\\
\end{example}
\begin{definition}[Components]
	Given a graph $G$, a \textbf{component} of $G$ is a maximal connected subgraph of $G$. That is, an equivalence class under the equivalence relation that $a ~ b$ if there exists a connected subgraph of $G$ containing $a$ and $b$. Equivalently, $a ~ b$ if there exists a subgraph of $G$ which is a path with endpoints $a, b$.
\end{definition}
\begin{remarks}
	A subgraph $H$ of $G$ is a component if there is no connected subgraph $F$ with the property $H \subset F$.\\
	A component $C$ is either equal to the graph $G$ or a proper induced subgraph of $G$ such that no vertex of $C$ is joined to any vertex outside of $C$.\\
	If $e \in E\left( G \right) $ such that both endpoints of $e$ are in $ V\left( C \right) $, then adding $e$ to $E\left( C \right) $ keeps $C$ connected and increases its size. Hence $C$ was not a component, a contradiction. Thus, we may say that all edges in $G$ which join two members of the same component must be a part of that component (and that component only).\\
	Any disconnected graph can be partitioned into connected subgraphs such that there is no edge joining vertices from distinct subgraphs. (These disconnected subgraphs are subgraphs of each component).
\end{remarks}
Many of the definitions from the end of this section are reminiscent of those in topology, simply discretized. This is no coincidence.
