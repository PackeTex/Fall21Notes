\lecture{1}{Mon 23 Aug 2021 09:10}{Strongly Regular Graphs (1)}
\section{Strongly Regular Graphs}
\begin{recall}[Regular Graphs]
	A $k$-regular graph is a graph with all vertices of equal degree (namely $d\left( v \right) = k$ for $v \in G$ implies $G$ is $k$-regular). Examples of $k$-regular graphs are cycles with common degree $2$ and cubes with common degree  $3$.
\end{recall}
\begin{definition}[Codegree]
	We define the \textbf{codegree} of two vertices $u, v$ as the number of common neighbors of $u, v$. We denote this $\hat{d}\left( u, v \right) = \left| N\left( u \right) \cap N\left( v \right)  \right| $ where $N$ is simply the neighborhood of connected vertices to a given vertex. Note this implies $\left| N\left( i \right)\right| = d\left( i \right) $.
\end{definition}
\begin{remark}
	Excluding loops, we see that $u, v$ cannot be in their own common neighborhoods.
\end{remark}
\begin{definition}[Strongly Regular Graph]
	A graph $G$ is  \textbf{SRG (Strongly Regular Graph)} if $G$ is regular, $i \thicksim j$ ($i$ adjacent to $j$ ) implies $\hat{d} \left( i, j \right) $ is equal for all adjacent pairs $i, j \in V\left( G \right) $. Lastly, for all nonadjacent $i \neq j$, $\hat{d} \left( i , j \right)$ is also equal for all nonadjacent pairs $i, j \in V\left( G \right) $. We denote this by $G$ is $\SRG\left( n, k, \lambda, \mu \right) $. Here, $n$ is the order of $G$, $k$ is the degree for which $G$ is $k$-regular, $\lambda$ is the codegree of all pairs of adjacent vertices, $\mu$ is the codegree of all pairs of nonadjacent unique vertices.
\end{definition}
\begin{example}
	The most trivial SRG is the union of two disjoint edges, denoted $2K_2 = SRG\left( 4, 1, 0, 0 \right) $.\\
	$K_{n}$ is an edge case but is not considered a SRG as we cannot meaningfully define $\mu$ as there are no nonadjacent vertices.
	$2K_3= \SRG\left( 6, 2, 1, 0 \right)$.\\
	$2K_{n} = \SRG\left( 2n, n-1, n-2, 0 \right) $.\\
	$C_5 = \SRG\left( 5, 2, 0, 1 \right) $.\\
	$K_{n, n}= \SRG\left( 2n, n, 0, n \right) $.\\
	$K_{m, n}, m\neq n$ is not SRG.\\
	$C_{n}, n>5$ is not SRG.\\
\end{example}
\begin{problem}
	Prove the complete multipartite graph which is regular is strongly regular. Here we define the complete multipartite graph $K_{m_1, m_2, \ldots, m_{n}}$ to be a graph with $n$ partite sets $A_1, A_2, \ldots, A_{n}$ for which all $A_{i}$ are independent and for any $u, v$ such that $u, v \not\in A_{i}$ for a particular $A_{i}$ implies $u \thicksim v$.
\end{problem}
\begin{solution}

\end{solution}
\begin{proposition}
A SRG is disconnected if and only if it is isomorphic to $mK_{r}$, $1 < m, r$.
\end{proposition}
\newpage
\begin{proposition}
	For a given $\SRG\left( n, k, \lambda, \mu \right) $, we have \\$k\left( k - \lambda - 1 \right) = \left( n - k - 1 \right) \mu$.
\end{proposition}
\begin{proof}
	First, note that for a given $u$, the graph induced by $N\left( u \right) $ (simply denoted $N\left( u \right)$) is a $\lambda$-regular graph. Furthermore, the graph induced by $\overline{N\left( u \right) } - u$ (denoted $\overline{N \left( u \right) }$)  will be $\mu$-regular.\\
	Now, let $v \in N\left( u \right) $ and note that this has $k - \lambda - 1$ neighbors in $\overline{N \left( u \right)}$ as $v$ has $k$ neighbors by regularity, $\lambda$ of which are shared with $u$ (hence in $N\left( u \right) $ and $1$ of which is $u$ itself. Now, for a $w \in \overline{N \left( u \right)}$, note that $w$ has $\mu$ neighbors in $N\left( G \right) $ as $\hat{d} \left( w, u \right) = \lambda$ by definition. So, we can partition $G - u$ into a  bipartite graph with sets $A = N\left( u \right) $, $B = \overline{N \left( u \right)} $ and we see each $u \in A$ has $n - \lambda - 1$ neighbors in $B$ and $v \in B$ has $\mu$ neighbors in $A$. We will use double counting on $e\left( G - u \right) $, counting the number of edges between  $A$ and $B$ yields that there are $k$ members in $A$ each with $k - \lambda - 1$ edges, hence $e\left( G - u \right)  = k \left( k - \lambda - 1 \right) $. Similarly, there are $n - k -1$ members in $B$ (as there are  $n$ vertices overall, $k$ of which are neighbors of $u$, hence not in $\overline{N \left( u \right)} $ and $1$ of which is $u$ itself) each with $\mu$ neighbors in $A$, hence $e\left( G - u \right) = (n - k - 1)\mu$. As these two quantities are equal, this completes the proof by double counting.
\end{proof}
\begin{remark}
	This implies that an arbitrary combination of parameters may not yield a proper graph. Hence, it is unknown whether $\SRG\left( 99, 14, 1, 2 \right) $ exists. Furthermore, these parameters do not uniquely define a graph, for a given set of parameters one may be able to find an arbitrarily large number of SRG's with these parameters which are non-isomorphic.
\end{remark}
