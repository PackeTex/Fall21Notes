%\lecture{3}{Thu 15 Jul 2021 12:57}{Lagrange Interpolation, Zero-Sum Sequences, and the q-Dyson Theorem}
\section{Zero-Sum Sequences, and the q-Dyson Theorem}
Let us begin with some notation:
\begin{notation}[Subpower]
	For a given independent variable $q$, define \[
		\left( x \right) _{k} = \left( 1-x \right) \left(  1-xq \right) \left( 1-xq^2 \right) \ldots \left( 1-xq^{k-1} \right)
	.\] 	For simplicity, define $\left( t \right) _{0} = 1$.
\end{notation}
The following theorems, concerned with the constant term of a particular laurent polynomial, serve to further demonstrate the power of our previous results. The special case $q=1$ yields the Dyson conjecture:
\begin{theorem}[Dyson Conjecture]
	\[
		[x_1^{0}\ldots x_{n}^{0}]\left( \prod_{1\le i < j \le n}^{}  \left( 1- \frac{x_{i}}{x_{j}} \right)^{a_{i}}  \right)  = \frac{\left( \sum_{i= 1}^{n} a_{i} \right) ! }{\prod_{i= 1}^{n} \left( a_{i}! \right)}
	.\]
\end{theorem}
This case, originally proven by Gunson and Wilson using lagrange interpolation yields the following generalization:
\begin{theorem}[q-Dyson Conjecture]
	First, let us define the following polynomial:
	\[
		f_{q}\left( \textbf{x} \right) = f_{q}\left( x_1, x_2, \ldots, x_{n} \right) \coloneqq \prod_{1\le i < j \le n}^{} \left( \frac{x_{i}}{x_{j}} \right) _{a_{i}} \left( \frac{qx_{j}}{x_{i}} \right) _{a_{j}}
	.\]
	Then,  \[
		[x_1^{0}\ldots x_{n}^{0}]f_{q}\left( \textbf{x} \right)  = \frac{\left( q \right) _{\sum_{i= 1}^{n} a_{i}}}{\prod_{j= 1}^{n} \left( q \right) _{a_{j}}}
	.\]
\end{theorem}
